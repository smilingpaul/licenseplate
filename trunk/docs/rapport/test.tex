\section{Resultater}

\subsection{Indsamling af testdata}

%Fra start var vi meget opmærksomme på at afgrænsningen af projektet skulle være klar, så det ikke blev for omfattende. Omkring valg af billedemateriale afholdt vi os eksempelvis fra at 

%Til at udarbejde og teste systemet havde vi brug for nogle fotografier af biler med nummerplader. Ved fotograferingen var vi opmærksomme på følgende afgrænsninger:



Vi tog i alt 407 fotografier med to digitalkameraer: 236 med et Canon og 171 med et Olympus. Et eksempel på et fotografi ses i Figur 1.%\ref{b_xc33139}.

FIGUR 1: EKSEMPEL PÅ FOTOGRAFI AF BIL MED NUMMERPLADE

%\begin{figure}[h]
%\begin{center}
%\includegraphics[width=10cm]{illu/B_XC33139.jpg}
%\label{b_xc33139}
%\caption{Fotografieksempel}
%\end{center}
%\end{figure}

% vi holder dem adskilt - ikke noget med histo
Da vi bl.a. havde planer om udarbejdelse af en histogrambaseret metode til identificering af nummerplader (se afsnit \ref{histo}), holdt vi de to fotografisæt adskilte. På denne måde ville det f.eks. være muligt for os at teste om skift fra et kamera til et andet, vil give ændrede resultater. Billederne blev navngivet i vores database, så det, i deres filnavn, indgik om billedet forestillede en bil set forfra eller bagfra samt hvilken nummerplade bilen på nummerpladen havde. Derudover udarbejdede vi et mindre program som hjalp os til at identificere nummerpladens fire hjørnekoordinater og indskrive disse i filnavnet. Denne sidste tilføjelse ville hjælpe os i testfasen, til at undersøge om de nummerpladekandidater vores system ville udvælge er de korrekte.


%\subsection{Fotografering}
%Hvor mange og hvilke billeder har vi taget.
%At billederne er taget i naturligt lys for at undgå kunstige farver og genskin fra pladen som er malet med reflekterende materiale.
%Opdeling af billeder i træning- og testsæt
%til f.eks. histogram metode herunder adskillelse af fotos fra de to forskellige kameraer.


\subsection{Billedbehandling}
\fixme{Søren: ikke god titel}

Eksempel på testtabel for identifikation af nummerplader:

\begin{tabular}{|l|l|l|l|l|l|}
\hline
\multicolumn{6}{|c|}{Metode: Histogram} \\ \hline
Param 1 & Param 2 & Skalering & Overordnet resultat & Sande positiver & Bemærkninger\\ \hline
1 & 1 & 1 & 19 \% & 19 \% & skide godt\\ \hline
2 & 2 & 2 & 19 \% & 21 \% & vildt godt \\ \hline
3 & 3 & 3 & 19 \% & 22 \% & diku niveau \\
\hline
\end{tabular}

\fixme{Noget om usikkerhed ved simpel testmetode af tegnopdel}

\subsection{Mønstergenkendelse}
\fixme{Søren: ikke god titel}