\section{Resultater}
\label{sec:resultater}

I dette afsnit vil vi afprøve vores system og give resultaterne af denne afprøvning.

%\subsection{Indsamling af testdata}

%Fra start var vi meget opmærksomme på at afgrænsningen af projektet skulle være klar, så det ikke blev for omfattende. Omkring valg af billedemateriale afholdt vi os eksempelvis fra at 


% vi holder dem adskilt - ikke noget med histo
%Da vi bl.a. havde planer om udarbejdelse af en histogrambaseret metode til identificering af nummerplader (se afsnit \ref{histo}), holdt vi de to fotografisæt adskilte. På denne måde ville det f.eks. være muligt for os at teste om skift fra et kamera til et andet, vil give ændrede resultater.

%%%%%%%%%%%%%%%%%%%%%%%%%%%%%%%%%%%
%%% LOKALISERING AF NUMMERPLADE %%%
%%%%%%%%%%%%%%%%%%%%%%%%%%%%%%%%%%%

\subsection{Lokalisering af nummerplader}
Lokalisering af nummerplader kan gøres ved brug af en enkelt af systemets metoder eller ved brug af alle metoderne samtidigt. Vi vil i dette afsnit afprøve disse muligheder. Derudover vil vi se på hvordan resultaterne ændrer sig når man ændrer opløsning af billederne. I alle tilfælde vil vi afprøve metoderne på både trænings- og kontrolsæt. 

KUN EN TABEL MED DE SEKS METODER

Hvis vi skal have test med flere forskellige parametre, kan vi jo også køre en gang hvor vi siger at mindst to metoder skal være enige før vi udpeger et område.s
Trainingset
DetectMain = 97.5\%/99.3\% (99.2366)
\begin{tabular}{|l|l|l|l|l|}
\hline
\rowcolor[gray]{0.9} \multicolumn{5}{|>{\columncolor[gray]{0.9}}c|}{\textbf{Træningssæt}} \\ \hline
Param 1 & Param 2 & Skalering & Overordnet resultat & Sande positiver\\ \hline
1 & 1 & 1 & 19 \% & 19 \%\\ \hline
2 & 2 & 2 & 19 \% & 21 \% \\ \hline
3 & 3 & 3 & 19 \% & 22 \% \\
\hline
\end{tabular}

\begin{tabular}{|l|l|l|l|l|}
\hline
\rowcolor[gray]{0.9} \multicolumn{5}{|>{\columncolor[gray]{0.9}}c|}{\textbf{Kontrolsæt}} \\ \hline
Param 1 & Param 2 & Skalering & Overordnet resultat & Sande positiver\\ \hline
1 & 1 & 1 & 19 \% & 19 \%\\ \hline
2 & 2 & 2 & 19 \% & 21 \% \\ \hline
3 & 3 & 3 & 19 \% & 22 \% \\
\hline
\end{tabular}

\begin{comment}
\subsubsection*{Områder domineret af lyse gråtoner}

\begin{tabular}{|l|l|l|l|l|}
\hline
\rowcolor[gray]{0.9} \multicolumn{5}{|>{\columncolor[gray]{0.9}}c|}{\textbf{Træningssæt}} \\ \hline
Param 1 & Param 2 & Skalering & Overordnet resultat & Sande positiver\\ \hline
1 & 1 & 1 & 19 \% & 19 \%\\ \hline
2 & 2 & 2 & 19 \% & 21 \% \\ \hline
3 & 3 & 3 & 19 \% & 22 \% \\
\hline
\end{tabular}

\begin{tabular}{|l|l|l|l|l|}
\hline
\rowcolor[gray]{0.9} \multicolumn{5}{|>{\columncolor[gray]{0.9}}c|}{\textbf{Kontrolsæt}} \\ \hline
Param 1 & Param 2 & Skalering & Overordnet resultat & Sande positiver\\ \hline
1 & 1 & 1 & 19 \% & 19 \%\\ \hline
2 & 2 & 2 & 19 \% & 21 \% \\ \hline
3 & 3 & 3 & 19 \% & 22 \% \\
\hline
\end{tabular}


\subsubsection*{Områder med høj kontrast}

\begin{tabular}{|l|l|l|l|l|}
\hline
\rowcolor[gray]{0.9} \multicolumn{5}{|>{\columncolor[gray]{0.9}}c|}{\textbf{Træningssæt}} \\ \hline
Param 1 & Param 2 & Skalering & Overordnet resultat & Sande positiver\\ \hline
1 & 1 & 1 & 19 \% & 19 \%\\ \hline
2 & 2 & 2 & 19 \% & 21 \% \\ \hline
3 & 3 & 3 & 19 \% & 22 \% \\
\hline
\end{tabular}

\begin{tabular}{|l|l|l|l|l|}
\hline
\rowcolor[gray]{0.9} \multicolumn{5}{|>{\columncolor[gray]{0.9}}c|}{\textbf{Kontrolsæt}} \\ \hline
Param 1 & Param 2 & Skalering & Overordnet resultat & Sande positiver\\ \hline
1 & 1 & 1 & 19 \% & 19 \%\\ \hline
2 & 2 & 2 & 19 \% & 21 \% \\ \hline
3 & 3 & 3 & 19 \% & 22 \% \\
\hline
\end{tabular}

\subsubsection*{Frekvensanalyse}

\begin{tabular}{|l|l|l|l|l|}
\hline
\rowcolor[gray]{0.9} \multicolumn{5}{|>{\columncolor[gray]{0.9}}c|}{\textbf{Træningssæt}} \\ \hline
Param 1 & Param 2 & Skalering & Overordnet resultat & Sande positiver\\ \hline
1 & 1 & 1 & 19 \% & 19 \%\\ \hline
2 & 2 & 2 & 19 \% & 21 \% \\ \hline
3 & 3 & 3 & 19 \% & 22 \% \\
\hline
\end{tabular}

\begin{tabular}{|l|l|l|l|l|}
\hline
\rowcolor[gray]{0.9} \multicolumn{5}{|>{\columncolor[gray]{0.9}}c|}{\textbf{Kontrolsæt}} \\ \hline
Param 1 & Param 2 & Skalering & Overordnet resultat & Sande positiver\\ \hline
1 & 1 & 1 & 19 \% & 19 \%\\ \hline
2 & 2 & 2 & 19 \% & 21 \% \\ \hline
3 & 3 & 3 & 19 \% & 22 \% \\
\hline
\end{tabular}

\subsubsection*{Maksimer lokal kontrast}

\begin{tabular}{|l|l|l|l|l|}
\hline
\rowcolor[gray]{0.9} \multicolumn{5}{|>{\columncolor[gray]{0.9}}c|}{\textbf{Træningssæt}} \\ \hline
Param 1 & Param 2 & Skalering & Overordnet resultat & Sande positiver\\ \hline
1 & 1 & 1 & 19 \% & 19 \%\\ \hline
2 & 2 & 2 & 19 \% & 21 \% \\ \hline
3 & 3 & 3 & 19 \% & 22 \% \\
\hline
\end{tabular}

\begin{tabular}{|l|l|l|l|l|}
\hline
\rowcolor[gray]{0.9} \multicolumn{5}{|>{\columncolor[gray]{0.9}}c|}{\textbf{Kontrolsæt}} \\ \hline
Param 1 & Param 2 & Skalering & Overordnet resultat & Sande positiver\\ \hline
1 & 1 & 1 & 19 \% & 19 \%\\ \hline
2 & 2 & 2 & 19 \% & 21 \% \\ \hline
3 & 3 & 3 & 19 \% & 22 \% \\
\hline
\end{tabular}

\subsubsection*{Kvantifisering}

\begin{tabular}{|l|l|l|l|l|}
\hline
\rowcolor[gray]{0.9} \multicolumn{5}{|>{\columncolor[gray]{0.9}}c|}{\textbf{Træningssæt}} \\ \hline
Param 1 & Param 2 & Skalering & Overordnet resultat & Sande positiver\\ \hline
1 & 1 & 1 & 19 \% & 19 \%\\ \hline
2 & 2 & 2 & 19 \% & 21 \% \\ \hline
3 & 3 & 3 & 19 \% & 22 \% \\
\hline
\end{tabular}

\begin{tabular}{|l|l|l|l|l|}
\hline
\rowcolor[gray]{0.9} \multicolumn{5}{|>{\columncolor[gray]{0.9}}c|}{\textbf{Kontrolsæt}} \\ \hline
Param 1 & Param 2 & Skalering & Overordnet resultat & Sande positiver\\ \hline
1 & 1 & 1 & 19 \% & 19 \%\\ \hline
2 & 2 & 2 & 19 \% & 21 \% \\ \hline
3 & 3 & 3 & 19 \% & 22 \% \\
\hline
\end{tabular}

\subsubsection*{Endeligt valg af nummerpladekandidat}

\begin{tabular}{|l|l|l|l|l|}
\hline
\rowcolor[gray]{0.9} \multicolumn{5}{|>{\columncolor[gray]{0.9}}c|}{\textbf{Træningssæt}} \\ \hline
Param 1 & Min. antal enige & Skalering & Overordnet resultat & Sande positiver\\ \hline
1 & 1 & 1 & 19 \% & 19 \%\\ \hline
2 & 2 & 2 & 19 \% & 21 \% \\ \hline
3 & 3 & 3 & 19 \% & 22 \% \\ \hline
3 & 4 & 3 & 19 \% & 22 \% \\ \hline
2 & 5 & 2 & 19 \% & 21 \% \\ 
\hline
\end{tabular}

\begin{tabular}{|l|l|l|l|l|}
\hline
\rowcolor[gray]{0.9} \multicolumn{5}{|>{\columncolor[gray]{0.9}}c|}{\textbf{Kontrolsæt}} \\ \hline
Param 1 & Min. antal enige & Skalering & Overordnet resultat & Sande positiver\\ \hline
1 & 1 & 1 & 19 \% & 19 \%\\ \hline
2 & 2 & 2 & 19 \% & 21 \% \\ \hline
3 & 3 & 3 & 19 \% & 22 \% \\ \hline
3 & 4 & 3 & 19 \% & 22 \% \\ \hline
2 & 5 & 2 & 19 \% & 21 \% \\ 
\hline
\end{tabular}

\end{comment}



\begin{comment} % udkommenteret da vi vel ikke skal bruge resultater på sættet med 407 billeder?
\subsubsection*{Observeret sæt på 407 billeder}
Scale: 0.25
DetectMain: 96.6/99.24
DetectQuant: 67.8/75.4
DetectSameness: 56.8/95.5
DetectContrastAvg: 62.7/85.0
DetectPlateness: 50.4/65.5
DetectCStretch: 84.0/92.7

Scale: 0.50 (Ekstremt langsomt)
DetectPlateness: 29.7/56.5
DetectCStretch:
\end{comment}

%Confusion matrix: De elementer der ligger udenfor diagonalen er elementer der ikke er nummerplader.

Delkonklusion:

VI SKAL SE HVORDAN DETECTMAIN OPFØRER SIG NÅR F.EKS. ALLE METODER SKAL VÆRE ENIGE.

Vi har skaleret ned for at spare tid.

Hvordan ændrer resultaterne sig når man ændrer opløsning af billederne?
Er der et mønster i de billeder hvot vi ikke finder nummerpladen? Hvile udfordringer møder vi: Mørke plader... 
HVAD SKER DER MED GULE ETC. PLADER??


%%%%%%%%%%%%%%%%%%%%%%%%%%
%%% SEPARATION AF TEGN %%%
%%%%%%%%%%%%%%%%%%%%%%%%%%

\subsection{Separation af tegn}

DETTE AFSNIT SKRIVES SAMMEN MED GENKENDELSE.

I dette afsnit testes de to metoder til separation af tegn. Metoden til rotation vil ikke blive testes, blot kommenteret.

\subsubsection*{Rotation}
%Om en nummerplade står vandret i et billede kan afgøres ved at kantdetektere billedet og udføre en Radon transformation på dette kant-billede. Hvis de stærkeste linier optræder ved $0^{\circ}$ er rotation udført korrekt.

Vi kan ikke teste det fordi det ville være noget med at vi skulle sidde og angive alle rotationer manuelt..

Vi kan ikke teste en matlab-metode..

\subsubsection*{Separation}
De to metoder til separation af tegn testes ved at se på om de finder syv objekter som optræder indenfor pladens koordinater. Denne forholdsvis enkle test mener vi er fyldestgørende, da funktionerne hvori metoderne er implementeret, bør være "skrappe" nok.

Vi vil desuden se på om der er forskel på succesraten i forhold til hvilket område der repræsenterer nummerpladen. Her findes to muligheder: pladen er fundet ved brug af identifikationsmetoderne eller defineret ved hjælp af de manuelt registrerede pladekoordinater (med evt. et yderområde lagt til).

Parametre: størrelse på componenter der sorteres fra, skalering: kan vi tillade os at skalere ned (eller op, hvor vi måske får mere plads mellem komponenterne) eller mister vi for meget information VI KAN IKKE SKALERE

VI VIL KUN SE PÅ TEST AF DETTE NIVEAU, IKKE? 

%OVERVEJ NEDENSTÅENDE SOM SAMLET GRAF I ET AFSNIT

%\subsubsection{Bjerg/Dal}
%HVOR GODE ER VI TIL AT FINDE 7 OMRÅDER INDEN FOR PLADEN I TRÆNINGSSÆT?
%HVOR GODE ER VI TIL AT FINDE 7 OMRÅDER INDEN FOR PLADEN I TESTSÆT
%HVOR GODE ER VI TIL AT FINDE 7 OMRÅDER INDEN FOR PLADEN I TRÆNINGSSÆT?
%HVOR GODE ER VI TIL AT FINDE 7 OMRÅDER INDEN FOR PLADEN I TESTSÆT
%HVAD SKER DER NÅR PLADEOMRÅDET UDVIDES MED 10PIXELS PÅ ALLE SIDER?
%HVAD SKER DER NÅR PLADEOMRÅDET UDVIDES MED 20PIXELS PÅ ALLE SIDER?
%HVAD SKER DER NÅR PLADEOMRÅDET UDVIDES MED 30PIXELS PÅ ALLE SIDER?


\begin{tabular}{|l|c|c|}\hline
\rowcolor[gray]{0.9} \multicolumn{3}{|>{\columncolor[gray]{0.9}}c|}{\textbf{Træningssæt}} \\ \hline
Metode & Sammenhængende komponenter & Bjerg/dal \\\hline
Egne metoder & 96\% & 3\% \\\hline
10 & 96\% & 3\% \\\hline
20 & 96\% & 3\% \\\hline
30 & 96\% & 3\% \\\hline \end{tabular}

\begin{tabular}{|l|c|c|}\hline
\rowcolor[gray]{0.9} \multicolumn{3}{|>{\columncolor[gray]{0.9}}c|}{\textbf{Kontrolsæt}} \\ \hline
Metode & Sammenhængende komponenter & Bjerg/dal \\\hline
Egne metoder & 96\% & 3\% \\\hline
10 & 96\% & 3\% \\\hline
20 & 96\% & 3\% \\\hline
30 & 96\% & 3\% \\\hline \end{tabular}

Delkonklusion:
Sammenhængende komponenter er go og bjerg/dal dårlig?

%%% GENKENDELSE AF TEGN %%%

\subsection{Genkendelse af tegn}
I dette afsnit testes de to metoder til genkendelse af tegn i en nummerplade. Derudover vil syntaksanalysen blive testet.

Noget om at det er i forhold til alle plader, dvs. at procenterne nedenfor er afhængige af hvor gode vi er til separere tegn.

\subsubsection{Middelvektor}

%HVOR MANGE PLADER ER KORREKT LÆST I TRÆNINGSSÆT?
%HVOR MANGE PLADER ER KORREKT LÆST I TETSSÆT?
%HVOR MANGE PLADER ER KORREKT LÆST PÅ 6 POSITIONER I TRÆNINGSSÆT?
%HVOR MANGE PLADER ER KORREKT LÆST PÅ 6 POSITIONER I TETSSÆT?
%HVOR MANGE PLADER ER KORREKT LÆST PÅ 5 POSITIONER I TRÆNINGSSÆT?
%HVOR MANGE PLADER ER KORREKT LÆST PÅ 5 POSITIONER I TETSSÆT?

Følgende tabeller viser resultaterne for at læse en hel plade, seks tegn i pladen osv. (med forskellige vektorlængder).

DE FØLGENDE GRAFER SKAL EKSISTERE FOR MED OG UDEN SYNTAKS ANALYSE

INKL. 1 og 0

\begin{tabular}{|l|c|c|c|c|c|c|}\hline
\rowcolor[gray]{0.9} \multicolumn{7}{|>{\columncolor[gray]{0.9}}c|}{\textbf{Træningssæt}} \\ \hline
Vektorlængde & Hele pladen læst & 6 tegn læst & 5 tegn & 4 tegn & 3 tegn & 2 tegn \\\hline
9 & 0\% & 0\% & 0\% & 0\% & 0\% & 0\% \\\hline
16 & 0\% & 0\% & 0\% & 0\% & 0\% & 0\% \\\hline
25 & 0\% & 0\% & 0\% & 0\% & 0\% & 0\% \\\hline \end{tabular}

\begin{tabular}{|l|c|c|c|c|c|c|}\hline
\rowcolor[gray]{0.9} \multicolumn{7}{|>{\columncolor[gray]{0.9}}c|}{\textbf{Kontrolsæt}} \\ \hline
Vektorlængde & Hele pladen læst & 6 tegn læst & 5 tegn & 4 tegn & 3 tegn & 2 tegn \\\hline
9 & 0\% & 0\% & 0\% & 0\% & 0\% & 0\% \\\hline
16 & 0\% & 0\% & 0\% & 0\% & 0\% & 0\% \\\hline
25 & 0\% & 0\% & 0\% & 0\% & 0\% & 0\% \\\hline \end{tabular}

%HVOR GODE ER VI PÅ TAL I TRÆNINGSSÆT?
%HVOR GODE ER VI PÅ TAL I TESTSÆT?
%HVOR GODE ER VI PÅ BOGSTAVER I TRÆNINGSSÆT?
%HVOR GODE ER VI PÅ BOGSTAVER I TESTSÆT?
%HVAD ER GENKENDELSESPROCENTEN PÅ TEGN PR PLADE I TRÆNINGSÆT?
%HVAD ER GENKENDELSESPROCENTEN PÅ TEGN PR PLADE I TESTSÆT?

Følgende tabeller viser hvor godt systemet er til at læse bogstaver hhv. tal. hhv. alle tegn

\begin{tabular}{|l|c|c|c|}\hline
\rowcolor[gray]{0.9} \multicolumn{4}{|>{\columncolor[gray]{0.9}}c|}{\textbf{Træningssæt}} \\ \hline
Vektorlængde & Alle tegn & Bogstaver & Tal \\\hline
9 & 0\% & 0\% & 0\% \\\hline
16 & 0\% & 0\% & 0\%\\\hline
25 & 0\% & 0\% & 0\%\\\hline \end{tabular}

\begin{tabular}{|l|c|c|c|}\hline
\rowcolor[gray]{0.9} \multicolumn{4}{|>{\columncolor[gray]{0.9}}c|}{\textbf{Kontrolsæt}} \\ \hline
Vektorlængde & Alle tegn & Bogstaver & Tal \\\hline
9 & 0\% & 0\% & 0\% \\\hline
16 & 0\% & 0\% & 0\% \\\hline
25 & 0\% & 0\% & 0\% \\\hline \end{tabular}

%HVILKEN POSITION PÅ HITLISTEN TAGER DEN I GENNEMSNIT PR POSITION - TRÆNING?
%HVILKEN POSITION PÅ HITLISTEN TAGER DEN I GENNEMSNIT PR POSITION - TEST?
%MAXHITNO: HVAD GIVER DET JO LÆNGERE NED AD HITLISTEN VI KAN GÅ?

\begin{comment}
Syntaks analyse: Hvilke hits bliver valgt på hitlisterne af syntaksanalysen (sæt maxhitno højt):

\begin{tabular}{|l|c|}\hline
\multicolumn{2}{|l|}{Træningssæt} \\\hline
Hitnr. & Valgt \\\hline
1 & 95,4\% \\\hline
2 & 3\% \\\hline
3 & 0\% \\\hline
4 & 0\% \\\hline
5 & 0\% \\\hline
6 & 0\% \\\hline \end{tabular}

\begin{tabular}{|l|c|}\hline
\multicolumn{2}{|l|}{Kontrolsæt} \\\hline
Hitnr. & Valgt \\\hline
1 & 92,9\% \\\hline
2 & 0\% \\\hline
3 & 0\% \\\hline
4 & 0\% \\\hline
5 & 0\% \\\hline
6 & 0\% \\\hline \end{tabular}

\end{comment}

UDEN SYNTAKSANALYSE - TRÆNING?
UDEN SYNTAKSANALYSE - TEST?

Evt. en tabel over de enkelte tegn A, B, C osv. Hvor gode er vi til at genkende disse? Tabellen kunne laves udfra den bedste vektorstørrelse? Måske bare en tabel over de dårligste tegn, altså dem der er sværest at genkende? På denne måde bliver det ikke en LANG tabel

\begin{tabular}{|l|c|}\hline
\multicolumn{2}{|l|}{Træningsæt} \\\hline
Tegn & Valgt \\\hline
0 & 92,9\% \\\hline
1 & 0\% \\\hline
2 & 0\% \\\hline
3 & 0\% \\\hline
4 & 0\% \\\hline
5 & 0\% \\\hline
6 & 92,9\% \\\hline
7 & 0\% \\\hline
8 & 0\% \\\hline
9 & 0\% \\\hline
A & 0\% \\\hline
B & 0\% \\\hline
C & 92,9\% \\\hline
D & 0\% \\\hline
E & 0\% \\\hline
H & 0\% \\\hline
J & 0\% \\\hline
K & 0\% \\\hline 
L & 92,9\% \\\hline
M & 0\% \\\hline
N & 0\% \\\hline
O & 0\% \\\hline
P & 0\% \\\hline
R & 0\% \\\hline
S & 0\% \\\hline
T & 0\% \\\hline
U & 0\% \\\hline
V & 0\% \\\hline
X & 0\% \\\hline
Y & 0\% \\\hline
Z & 0\% \\\hline \end{tabular}



\subsubsection{Summerede billeder}
Samme tabeller som ovenfor?

\subsubsection{And-billeder}
Hvert billede laves til et kvadrat. Ved at lave middelvektorer på træningssættet kan vi se at man ikke kan have vektorer af længden 4, da flere bogstavs vektorer så vil være de samme (nemlig 0 0 0 0). Afprøv med andre længder.


%Søren: Ved klassifikation bruges en eller anden grænseværdi som bestemmer hvad et element skal klassificeres som: denne grænseværdi kan reguleres og kan derfor testes.

Delkonklusion:
Billederne af tegnene er ret store i forhold til dem der bruges i litteraturen. Dette giver vores system en væsentlig fordel i forhold til "de andre".

\subsection{Det samlede system}

Her vil vi afprøve det samlede system... eller har vi gjort det "på vejen"?
KUN MED DEN BEDSTE KONFIGURATION:

pcT. AF ALLE TEGN
PCT. AF ALLE PLADER

\subsection{Sammenligning med andres resultater}

MÅSKE SKAL DETTE STÅ I KONKLUSION? ELLERS HER
