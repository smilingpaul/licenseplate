\section{Resultater}
\label{sec:resultater}

I dette afsnit vil vi beskrive afprøvningen af vores system og give resultaterne af denne afprøvning. Vi afprøver alle trin i lokalisering og læsning af nummerplader individuelt før vi afprøver det samlede system. Som det første undersøger vi metoder til lokalisering af nummerplader. Derefter afprøver vi to metoder til at separere tegn i nummerpladerne. Som den sidste individuelle afprøvning, undersøger vi forskellige metoder til gendkendelse af tegn.

På baggrund af resultaterne af de individuelle afprøvninger, vælger vi den optimale konfiguration af systemet. Vi afprøver det samlede system på kontrolsættet med og uden syntaksanalyse af de nummerplader systemet læser.  

Til sidst afprøver vi HVAD på billederne der er taget med blitz samt billederne af biler med gule nummerplader.

%%%%%%%%%%%%%%%%%%%%%%%%%%%%%%%%%%%
%%% LOKALISERING AF NUMMERPLADE %%%
%%%%%%%%%%%%%%%%%%%%%%%%%%%%%%%%%%%

\subsection{Lokalisering af nummerplader}
Vi undersøger vores systems evne til at lokalisere nummerplader. Vi kigger altså ikke på nummerpladens tegn, men blot på om vi kan afgøre dens position i inddata-billedet. Vi har foretager afprøvning af metoderne både individuelt og som et samlet system. Resultaterne er vist i figur \vref{fig:test:lokalisering_traening_kontrol}.

\subsubsection{Individuel lokalisering}
Vi afprøver effektiviteten af de fem metoder til lokalisering af nummerplader.

Alle metoderne giver dårligere resultater på kontrolsættet end på træningsættet. Størst forringelse observeres i metoden der kigger på områder domineret af lyse gråtoner. Dette kan skyldes væsentligt forskellige vejrforhold for de dage hvor billederne er taget. Billederne i kontrolsættet er taget på en dag med meget kraftigt sol hvorimod billederne i træningsættet er taget på dage med . Muligvis er farvesammensætningen på nummerpladerne væsentligt anderledes under disse forhold.  Metoden der anvender frekvensanalyse er den mest robuste, og opnår næsten lige gode resultater på de to sæt. De øvrige metoder oplever et fald på mellem 9.5\% og 6.1\% i effektivitet.

De to bedste metoder er dem der arbejder med kontrast. I træningsættet er metoden der kigger på områder i originalbilledet med høj kontrast den mest effektive med 83\%. I kontrolsættet er den mest effektive metode den der maksimerer lokal kontrast for at skabe et sammenhængende nummerpladeområde. 

HVAD ER HUTIGST?

\begin{figure}[htp]
\centering
\begin{tabular}{|l|l|l|l|}
\hline
\rowcolor[gray]{0.9} \multicolumn{4}{|>{\columncolor[gray]{0.9}}c|}{\textbf{Individuelle metoder til lokalisering}} \\ \hline
Metode & Træningsæt & Kontrolsæt & Forskel \\ \hline
Områder domineret af lyse gråtoner & 60.5\% & 32.2\% & 28.3\% \\ \hline
Områder med høj kontrast           & 83.0\% & 73.5\% &  9.5\% \\ \hline
Frekvensanalyse                    & 59.3\% & 58.3\% &  1.0\% \\ \hline
Maksimer lokal kontrast            & 82.8\% & 76.7\% &  6.1\% \\ \hline
Kvantifisering                     & 69.8\% & 60.5\% &  9.3\% \\ \hline
\end{tabular}
\caption{Resultaterne af afprøvning af de fem metoder til lokalisering af nummerplader på trænings- og kontrolsæt.}
\label{fig:test:lokalisering_traening_kontrol}
\end{figure}


\subsubsection{Samlet lokalisering}
Vi afprøver vi hvor god vores software er til at lokalisere nummerplader når alle fem metoder arbejder sammen. Vi har foretaget afprøvningen med flere forskellige indstillinger for minimal enighed. En enighed på f.eks. tre, vil sige at tre af de underliggende metoder til lokalisering skal have udpeget samme område for at systemet opfatter området som et nummerpladeområde og sender det videre i systemet til separation og genkendelse. Vi undersøger hvor høj enighed systemet skal konfigureres til for at give en korrekthed på 100\%. Resultaterne for afprøvning på træningsættet er vist på figur \vref{fig:test:lokalisering_traening_samlet} og resultaterne for kontrolsætter kan ses på figur \ref{fig:test:lokalisering_kontrol_samlet}.

I træningsættet er korrektheden høj selv ved en minimal enighed på en, men en korrekthed på 100\% opnås først når den minimale enighed sættes til fem. I denne konfiguration kan systemet kun lokalisere 35.3\% af nummerpladerne.

Selvom fire ud af de fem metoder til lokalisering giver dårligere resultater på kontrolsættet i de individuelle afprøvninger, falder den samlede effektivitet ikke tilsvarende når metoderne anvendes samlet. Faldet i effektivitet for den samlede metode er på 3\% mod gennemsnitligt 10.8\% for de individuelle metoder. Den samlede metode er altså forholdsvis robust.

\begin{figure}[htp]
\centering
  \begin{tabular}{|l|l|l|}
    \hline
    \rowcolor[gray]{0.9} \multicolumn{3}{|>{\columncolor[gray]{0.9}}c|}{\textbf{Samlet metode til lokalisering - Træningssæt}} \\
    \hline
    Minimal enighed & Fundne nummerplader & Korrekthed \\ \hline
    1 &  97.5\% &  99.0\% \\ \hline
    2 &  96.0\% &  99.5\% \\ \hline
    3 &  88.0\% &  99.7\% \\ \hline
    4 &  68.2\% &  99.6\% \\ \hline
    5 &  35.3\% & 100.0\% \\ \hline
  \end{tabular}
\caption{Resultaterne af afprøvning af hvor effektivt vi kan lokalisere nummerplader i træningsættet når alle fem metoder arbejde sammen.}
\label{fig:test:lokalisering_traening_samlet}
\end{figure}


\begin{figure}[htp]
\centering
  \begin{tabular}{|l|l|l|}
    \hline
    \rowcolor[gray]{0.9} \multicolumn{3}{|>{\columncolor[gray]{0.9}}c|}{\textbf{Samlet metode til lokalisering - Kontrolsæt}} \\
    \hline
    Minimal enighed & Fundne nummerplader & Korrekthed\\ \hline
    1 &  94.5\% & 96.8\%\\ \hline
    2 &  93.0\% & 97.7\%\\ \hline
	3 &  78.7\% & 99.8\%\\ \hline
    4 &  49.0\% & 100.0\%\\ \hline
  \end{tabular}
\caption{Resultaterne af afprøvning af hvor effektivt vi kan lokalisere nummerplader i kontrolsættet når alle fem metoder arbejde sammen.}
\label{fig:test:lokalisering_kontrol_samlet}
\end{figure}


%%%%%%%%%%%%%%%%%%%%%%%%%%
%%% SEPARATION AF TEGN %%%
%%%%%%%%%%%%%%%%%%%%%%%%%%

\subsection{Separation af tegn, herunder rotation}

%DETTE AFSNIT SKRIVES SAMMEN MED GENKENDELSE.

I dette afsnit afprøves metoderne, der bruges til at separere tegn i en nummerplade.

\subsubsection*{Rotation}
%Om en nummerplade står vandret i et billede kan afgøres ved at kantdetektere billedet og udføre en Radon transformation på dette kant-billede. Hvis de stærkeste linier optræder ved $0^{\circ}$ er rotation udført korrekt.

Da vi har lavet funktionen til rotation primært ved brug af en Matlab funktion (\textit{radon}), vil det ikke være relevant at teste funktionen. Derudover vil en test af funktionen betyde at alle pladernes rotation skulle bestemmes manuelt. Vi har i stedet besluttet at funktionen testes i forbindelse med test af de andre funktioner i dette afsnit, hvliket vil sige at rotation indgår som en del af separation og genkendelse af tegn og ikke er "sin egen" funktion. ER SKREVET TIDLIGER. OMSKRIVES.

\subsubsection*{Separation}
Metoderne til separation af tegn afprøves ved at se på om de finder syv objekter indenfor nummerpladens koordinater. Denne enkle afprøvning kan ikke være fyldestgørende. Det er rettere en indikation af ....

Vi vil desuden afprøve om der er forskel på succesraten i forhold til hvilket område der repræsenterer nummerpladen. Ved at definere forskellige størrelser for nummerpladeområdet kan afprøvningen indikere hvor robust metoderne er i forhold til størrelsen på udklipningen af nummerpladen fra originalbilledet.

%Parametre: størrelse på componenter der sorteres fra, skalering: kan vi tillade os at skalere ned (eller op, hvor vi måske får mere plads mellem komponenterne) eller mister vi for meget information VI KAN IKKE SKALERE

%OVERVEJ NEDENSTÅENDE SOM SAMLET GRAF I ET AFSNIT
Resultaterne fra afprøvningen af separation af tegn ses i figurerne \vref{fig:test:sep-traening-manuel} og \vref{fig:test:sep-kontrol-manuel}.

\begin{figure}[htp]
\centering
\begin{tabular}{|l|c|c|}\hline
\rowcolor[gray]{0.9} \multicolumn{3}{|>{\columncolor[gray]{0.9}}c|}{\textbf{Træningssæt}} \\ \hline
Antal pixels lagt til & Sammenhængende komponenter & Bjerg/dal \\\hline
10 & 96,5\% & 91,8\% \\\hline
20 & 95,8\% & 91,3\% \\\hline
30 & 92,5\% & 83,5\% \\\hline
40 & 87,5\% & 73,4\% \\\hline \end{tabular}
\caption{Afprøvning af metoderne til separation af tegn på træningssættet ved brug af manuelt udpegede plader.}
\label{fig:test:sep-traening-manuel}
\end{figure}

\begin{figure}[htp]
\centering
\begin{tabular}{|l|c|c|}\hline
\rowcolor[gray]{0.9} \multicolumn{3}{|>{\columncolor[gray]{0.9}}c|}{\textbf{Kontrolsæt}} \\ \hline
Antal pixels lagt til & Sammenhængende komponenter & Bjerg/dal \\\hline
10 & 90,0\% & 87,2\% \\\hline
20 & 86,8\% & 81,0\% \\\hline
30 & 84,7\% & 71,2\% \\\hline
40 & 79,0\% & 58,5\% \\\hline \end{tabular}
\caption{Afprøvning af metoderne til separation af tegn på kontrolsættet ved brug af manuelt udpegede plader.}
\label{fig:test:sep-kontrol-manuel}
\end{figure}

%\begin{tabular}{|l|c|c|}\hline
%\rowcolor[gray]{0.9} \multicolumn{3}{|>{\columncolor[gray]{0.9}}c|}{\textbf{Kontrolsæt}} \\ \hline
%Bedste udklip & Sammenhængende komponenter & Bjerg/dal \\\hline
% & 0\% & 0\% \\\hline
%\end{tabular}

Sammenhængende komponenter - Træningssæt

Denne metode giver de bedste resultater af de to præsenterede metoder. Tegnene klippes tæt og hvis nummerpladeområdet ikke er for stort, men derimod stadig indeholder nummerpladens kanter, sådan at en eventuelt rotation kan foretages korrekt, giver metoden sjældent fejl.

Fejlkilderne, når nummerpladeområdet er lille (det vil sige at området er udviddet med 10 eller 20 pixels), er primært skruerne i pladen og skæve plader. Skruerne bevirker enten at tegnene gror sammen med hinanden eller med pladens kant og derfor vælges fra, fordi de sammenhængende komponenter er for store. De skæve plader bevirker, at de komponenter der repræsenterer tegnene ikke står i samme højde og derfor bliver valgt fra.

EVT. EKSEMPEL PÅ PROBLEM MED SKRUER SAMT SKÆV PLADE.

Fejlkilderne når nummerpladeområdet er større (det vil sige udviddet med 30 eller 40 pixels) er at en anden komponent end den der indeholder nummerpladen vælges når billedet skal indskrænkes eller at indskrænkningen ikke er stor nok og at der derfor laves komponenter som er udenfor og er tilpas høje osv.

ETV. EKSEMPEL PÅ FORKERT INDSKRÆNKNING OG DÅRLIG INDSKRÆNKNING

Sammenhængende komponenter - kontrolsæt

Bjerg/dal - Træningssæt

Som tidligere beskrevet er bjerg/dal metoden meget følsom overfor støj, hvilket også ses i afprøvningsresultaterne. Derudover er afprøvningsmetoden et dårligt skøn på om metoden udklipper tegnene korrekt, da mange af de udklippede elementer enten er halve tegn eller store elementer, hvor tegnet kun er en del af.

EVT. EKSEMPEL PÅ SIGNATUR FØLSOM OVERFOR STØJ OG DÅRLIG UDKILP

Bjerg/dal - Kontrolsæt

Støt begrundelse af div. valg i koden med testresultater. Er det rimeligt at klippe de største område ud når vi finder nummerplader? 


\begin{comment}
Problemer ved 10: tegn der er smeltet sammen pga. søm skruer: OE26906 og ----290. Godt udskåret: UX33152, skæve plader et problem. go kontrast: YE39734.

Problemer ved 20: for stort område giver komponenter der ikke klippes bort, fordi kontrastforstærkning giver et andet resultat, her ville dynamisk kontrast-blok størrelse måske hjælpe. Dårlig indskrænk XK29750. TS57793: GOD(?) 

Problemer ve 30: stadig for stort område: giver komponenter udenfor der vælges. dårlig indskrænk, især foran? eksempler.

\end{comment}

%%% GENKENDELSE AF TEGN %%%

\subsection{Genkendelse af tegn}
I dette afsnit afprøves metoderne til genkendelse af tegn i en nummerplade. Syntaksanalysen vil blive afprøvet i afprøvningen af det samlede system, da denne kun arbejder på syv følgende tegn.

Afprøvningen vil blive foretaget på alle tegn fra kontrolsættet, separeret med den bedste metode (sammenhængende komponenter hvor 10 pixels er lagt til). Disse tegn er sorteret så billeder af tegn der er udklippet forkert og hvor tegnet eksempelvis er blevet halveret under separationen er blevet sorteret fra. Afprøvningen vil derfor give en indikation af, hvor gode metoderne er til at genkende tegn i billeder som et menneske kan aflæse.

I træningssættet er der generelt færre bogstaver end der er tal. Vi vurderer at dette vil have indflydelse på afprøvningsresultaterne. Der er for eksempel kun tre billeder af tegnet \textbf{O}. Derudover er der ingen billeder af tegnet \textbf{A} i hverken træningssættet eller kontrolsættet. Derfor vil tabelrækker, hvor afprøvningsresultater for \textbf{A} skulle stå, være udfyldt med "-".

%Nedenstående tests er i forhold til manuelt udklippede plader - ikke automtisk lokalisering.

I figurerne \vref{fig:test:middel_tal}, \vref{fig:test:middel_bogstav} samt \vref{fig:test:middel_alletegn} ses resultaterne af afprøvningen af genkendelse af tegn ved brug af middelvektorer. Afprøvningen er udført med vektorer som er lavet ud fra billeder med størrelsen 3 $\times$ 3 og opefter. Vektorer der laves ud fra mindre billeder vil ikke være fornuftige at afprøve med, da flere af disse vektorer ligner hinanden og man derfor ikke kan skelne mellem genkendelse af to tegn med samme vektor. MANGLER 8 x 8!!! NOGET OM ØVRE GRÆNSE?

\begin{figure}[htp]
\centering
\begin{tabular}{|l|c|c|c|c|c|}\hline
\rowcolor[gray]{0.9} \multicolumn{6}{|>{\columncolor[gray]{0.9}}c|}{\textbf{Middelvektor}} \\ \hline
Tegn & 3 $\times$ 3 & 4 $\times$ 4 & 5 $\times$ 5 & 6 $\times$ 6 & 7 $\times$ 7\\\hline
0 & 21,0 & 77,1 & 74,8 & 56,1 & 91,1\\\hline
1 & 79,3 & 94,1 & 97,5 & 97,5 & 98,0\\\hline
2 & 92,7 & 93,3 & 98,0 & 98,9 & 99,2\\\hline
3 & 86,1 & 99,4 & 98,9 & 100,0 & 100,0\\\hline
4 & 84,1 & 99,1 & 99,1 & 99,7 & 99,7\\\hline
5 & 96,6 & 98,3 & 98,3 & 99,4 & 99,4\\\hline
6 & 42,1 & 99,5 & 100,0 & 99,5 & 100,0\\\hline
7 & 92,4 & 94,2 & 98,7 & 99,6 & 99,6\\\hline
8 & 13,0 & 45,2 & 90,4 & 93,5 & 97,0\\\hline
9 & 82,6 & 94,6 & 96,4 & 98,2 & 97,3\\\hline
\end{tabular}
\caption{Afprøvningsresultater for genkendelse af cifre ved brug af middelvektorer. For hvert ciffer er det angivet hvor mange procent af billederne af dette ciffer, der er korrekt genkendt.}
\label{fig:test:middel_tal}
\end{figure}

\begin{figure}[htp]
\centering
\begin{tabular}{|l|c|c|c|c|c|}\hline
\rowcolor[gray]{0.9} \multicolumn{6}{|>{\columncolor[gray]{0.9}}c|}{\textbf{Middelvektor}} \\ \hline
Tegn & 3 $\times$ 3 & 4 $\times$ 4 & 5 $\times$ 5 & 6 $\times$ 6 & 7 $\times$ 7\\\hline
A & - & - & - & - & -\\\hline
B & 40,9 & 68,2 & 90,9 & 95,5 & 95,5\\\hline
C & 65,6 & 93,8 & 96,9 & 100,0 & 100,0\\\hline
D & 91,4 & 100,0 & 100,0 & 97,1 & 91,4\\\hline
E & 65,0 & 85,0 & 95,0 & 95,0 & 100,0\\\hline
H & 28,6 & 100,0 & 85,7 & 100,0 & 100,0\\\hline
J & 97,2 & 97,2 & 100,0 & 97,2 & 100,0\\\hline
K & 54,8 & 80,6 & 90,3 & 90,3 & 90,3\\\hline
L & 25,0 & 100,0 & 100,0 & 100,0 & 100,0\\\hline
M & 20,5 & 87,2 & 97,4 & 94,9 & 94,9\\\hline
N & 41,7 & 100,0 & 100,0 & 100,0 & 100,0\\\hline
O & 0,0 & 66,7 & 0,0    & 66,7 & 66,7\\\hline
P & 78,8 & 100,0 & 100,0 & 100,0 & 100,0\\\hline
R & 28,2 & 87,2 & 92,3 & 94,9 & 94,9\\\hline
S & 98,6 & 97,1 & 92,7 & 100,0 & 100,0\\\hline
T & 94,4 & 98,9 & 98,9 & 98,9 & 100,0\\\hline
U & 1,3 & 93,3 & 96,0 & 97,3 & 98,7\\\hline
V & 90,6 & 98,1 & 98,1 & 98,1 & 98,1\\\hline
X & 92,2 & 97,4 & 98,7 & 98,7 & 98,7\\\hline
Y & 95,8 & 98,9 & 98,9 & 98,4 & 98,9\\\hline
Z & 77,2 & 100,0 & 96,5 & 100,0 & 100,0\\\hline
\end{tabular}
\caption{Afprøvningsresultater for genkendelse af bogstaver ved brug af middelvektorer. For hvert bogstav er det angivet hvor mange procent af billederne af dette bogstav, der er korrekt genkendt. Rækken med bogstavet \textbf{A} er tom, da hverken trænings- eller kontrolsættet indeholder nogen tegn af denne type.}
\label{fig:test:middel_bogstav}
\end{figure}

\begin{figure}[htp]
\centering
\begin{tabular}{|l|c|c|c|c|c|c|c|}\hline
\rowcolor[gray]{0.9} \multicolumn{8}{|>{\columncolor[gray]{0.9}}c|}{\textbf{Middelvektor}} \\ \hline
Tegn & 3 $\times$ 3 & 4 $\times$ 4 & 5 $\times$ 5 & 6 $\times$ 6 & 7 $\times$ 7 & 8 $\times$ 8\\\hline
Alle tal & 73,3 & 90,8 & 95,9 & 95,3 & 98,4 & 98,8 & 97,0\\\hline
Alle bogstaver & 74,5 & 95,7 & 97,0 & 98,0 & 98,2 & 98,6 & 98,4\\\hline
\end{tabular}
\caption{Afprøvningsresultater for genkendelse af tegn ved brug af middelvektorer, hvor tegnene er opdelt i en gruppe af alle cifre og en gruppe af alle bogstaver. For hver gruppe er det angivet hvor mange procent af tegnene i gruppen der er korrekt genkendt.}
\label{fig:test:middel_alletegn}
\end{figure}

Som det ses har metoden sværest ved at genkende tegnene \textbf{0} og \textbf{O} hviket skyldes, at middelvektorerne for disse tegn ligger tæt på hinanden (selv med det menneskelige øje, kan disse to tegn være svære at skelne fra hinanden). Derudover ses det at resultaterne for vektorer der er lavet på baggrund af 3 $\times$ 3 billeder, giver væsentligt dårlige resultater end de restende vektorer. Dette kommer især til udtryk i tabellen i figur \vref{fig:test:middel_alletegn}, hvor alle vektorer giver resultater over 90 \%, bortset fra den første.

I figurene \vref{fig:test:sum_tal}, \vref{fig:test:sum_bogstav} samt \vref{fig:test:sum_alletegn} vises afprøvningsresultaterne for genkendelse af tegn ved brug af sum-billeder. Som ovenfor vil det ikke give nogen mening at teste på vektorer lavet ud fra 2 $\times$ 2 billeder. NOGET OM ØVRE GRÆNSE?

\begin{figure}[htp]
\centering
\begin{tabular}{|l|c|c|c|c|c|}\hline
\rowcolor[gray]{0.9} \multicolumn{6}{|>{\columncolor[gray]{0.9}}c|}{\textbf{Sum-billeder}} \\ \hline
Tegn & 3 $\times$ 3 & 4 $\times$ 4 & 5 $\times$ 5 & 6 $\times$ 6 & 7 $\times$ 7\\\hline
0 & 0,0 & 64,0 & 73,4 & 54,2 & 81,3\\\hline
1 & 98,5  & 99,0 & 96,1 & 99,0 & 99,5\\\hline
2 & 81,6 & 56,7 & 90,5 & 99,7 & 99,7\\\hline
3 & 3,9 & 45,3 & 98,9 & 97,8 & 98,9\\\hline
4 & 22,8 & 98,8 & 98,8 & 99,4 & 99,4\\\hline
5 & 73,4 & 42,7 & 93,7 & 99,1 & 99,1\\\hline
6 & 31,6 & 57,4 & 96,3 & 98,4 & 97,9\\\hline
7 & 0,0 & 93,8 & 99,6 & 99,6 & 99,6\\\hline
8 & 1,3 & 0,7 & 42,6 & 80,0 & 98,3\\\hline
9 & 47,8 & 62,9 & 94,2 & 98,7 & 96,9\\\hline
\end{tabular}
\caption{Afprøvningsresultater for genkendelse af tal-tegn ved brug af sum-billeder. For hvert tegn er det angivet hvor mange procent af billederne af dette tegn, der er korrekt genkendt.}
\label{fig:test:sum_tal}
\end{figure}

\begin{figure}[htp]
\centering
\begin{tabular}{|l|c|c|c|c|c|}\hline
\rowcolor[gray]{0.9} \multicolumn{6}{|>{\columncolor[gray]{0.9}}c|}{\textbf{Sum-billeder}} \\ \hline
Tegn & 3 $\times$ 3 & 4 $\times$ 4 & 5 $\times$ 5 & 6 $\times$ 6 & 7 $\times$ \\\hline
A & - & - & - & - & -\\\hline
B & 27,3 & 40,9 & 95,5 & 100,0 & 95,5\\\hline
C & 0,0 & 84,4 & 28,1 & 3,1 & 53,1\\\hline
D & 20, & 100,0 & 100,0 & 91,4 & 80,0\\\hline
E & 65,0 & 90,0 & 100,0 & 100,0 & 100,0\\\hline
H & 100,0 & 85,7 & 85,7 & 85,7 & 85,7\\\hline
J & 100,0 & 97,2 & 97,2 & 97,2 & 97,2\\\hline
K & 54,8 & 80,6 & 87,1 & 90,3 & 87,1\\\hline 
L & 15,0 & 0,0 & 0,0 & 0,0 & 0,0\\\hline
M & 25,6 & 89,7 & 97,4 & 97,4 & 89,7\\\hline
N & 0,0 & 100,0 & 100,0 & 100,0 & 100,0\\\hline
O & 0,0 & 33,3 & 0,0 & 66,7 & 33,3\\\hline
P & 39,4 & 42,4 & 51,5 & 100,0 & 100,0\\\hline
R & 10,3 & 82,1 & 94,9 & 92,3 & 94,9\\\hline
S & 0,0 & 0,0 & 27,5 & 56,5 & 23,2\\\hline
T & 82,0 & 3,4 & 98,9 & 98,9 & 97,8\\\hline
U & 2,7 & 92,0 & 93,3 & 82,7 & 98,7\\\hline
V & 82,1 & 80,2 & 97,2 & 96,2 & 95,3\\\hline
X & 0,0 & 98,7 & 96,1 & 98,7 & 98,7\\\hline
Y & 97,9 & 54,5 & 93,1 & 97,9 & 97,4\\\hline
Z & 82,5 & 100,0 & 100,0 & 100,0 & 98,2\\\hline
\end{tabular}
\caption{Afprøvningsresultater for genkendelse af bogstav-tegn ved brug af sum-billeder. For hvert tegn er det angivet hvor mange procent af billederne af dette tegn, der er korrekt genkendt. Rækken med tegnet \textbf{A} er tom, da hverken trænings- eller kontrolsættet indeholder nogen tegn af denne type.}
\label{fig:test:sum_bogstav}
\end{figure}

\begin{figure}[htp]
\centering
\begin{tabular}{|l|c|c|c|c|c|c|c|}\hline
\rowcolor[gray]{0.9} \multicolumn{8}{|>{\columncolor[gray]{0.9}}c|}{\textbf{Sum-billeder}} \\ \hline
Tegn & 3 $\times$ 3 & 4 $\times$ 4 & 5 $\times$ 5 & 6 $\times$ 6 & 7 $\times$ 7 & 8 $\times$ 8 & 9 $\times$ 9 \\\hline
Alle tal & 37,4 & 61,4 & 89,6 & 93,8 & 97,5 & 97,6 & 91,1\\\hline
Alle bogstaver & 47,2 & 67,6 & 86,1 & 89,1 & 88,3 & 95,9 & 91,9\\\hline
\end{tabular}
\caption{Afprøvningsresultater for genkendelse af tegn ved brug af sum-billeder, hvor tegnene er opdelt i en gruppe af alle cifre og en gruppe af alle bogstaver. For hver gruppe er det angivet hvor mange procent af tegnene i gruppen der er korrekt genkendt.}
\label{fig:test:sum_alletegn}
\end{figure}


I \vref{fig:test:sum_alletegn} er det tydeligt at denne metode er generelt dårligere end den forrige. Resultaterne med de mindre vektorer er ikke nær så gode som med middelvektorer. Igen har metoden svært ved at genkende tegnene \textbf{0} og \textbf{O} men også tegnet \textbf{8} har behov for store sum-billeder for at kunne genkendes, hvilket i følge vores observationer er fordi, at det forveksles med tegnet \textbf{B}. Også \textbf{L} og \textbf{S} er svære at genkende, hvilket sandsynligvis skyldes at disse tegns densitet er forholdsvis lav, og at de binære billeder derfor ikke altid har et tilstrækkeligt antal markerede pixels til at tegnene genkendes.

I figurene \vref{fig:test:and_tal}, \vref{fig:test:and_bogstav} samt \vref{fig:test:and_alletegn} vises afprøvningsresultaterne for genkendelse af tegn ved brug af forenings-billeder. Denne metode har brug for billeder med megen information, hvorfor den mindste billedstørrelse er 10 $\times$ 10. NOGET OM ØVRE GRÆNSE?

\begin{figure}[htp]
\centering
\begin{tabular}{|l|c|c|c|c|c|c|}\hline
\rowcolor[gray]{0.9} \multicolumn{7}{|>{\columncolor[gray]{0.9}}c|}{\textbf{Forenings-billeder}} \\ \hline
Tegn & 10 $\times$ 10 & 20 $\times$ 20 & 30 $\times$ 30 & 40 $\times$ 40 & 50 $\times$ 50 & 100 $\times$ 100\\\hline
0 & 2,3 & 0,0 & 8,4 & 15,4 & 0 & 0,0\\\hline
1 & 0,0 & 15,3 & 1,5 & 25,1 & 3,9 & 11,3\\\hline
2 & 99,7 & 98,9 & 99,4 & 99,7 & 99,4 & 99,4\\\hline
3 & 100,0 & 100,0 & 99,7 & 100,0 & 100,0 & 99,7\\\hline
4 & 99,7 & 99,7 & 99,7 & 99,7 & 99,7 & 99,7\\\hline
5 & 97,7 & 91,4 & 99,4 & 99,4 & 99,1 & 99,1\\\hline
6 & 0,0 & 0,0 & 0,0 & 0,0 & 0,0 & 0,0\\\hline
7 & 97,3 & 99,1 & 99,6 & 98,7 & 99,1 & 99,1\\\hline
8 & 93,5 & 98,7 & 98,7 & 99,6 & 99,6 & 99,1\\\hline
9 & 40,2 & 69,6 & 79,9 & 78,1 & 67,0 & 72,8\\\hline
\end{tabular}
\caption{Bla.}
\label{fig:test:and_tal}
\end{figure}

\begin{figure}[htp]
\centering
\begin{tabular}{|l|c|c|c|c|c|c|}\hline
\rowcolor[gray]{0.9} \multicolumn{7}{|>{\columncolor[gray]{0.9}}c|}{\textbf{Forenings-billeder}} \\ \hline
Tegn & 10 $\times$ 10 & 20 $\times$ 20 & 30 $\times$ 30 & 40 $\times$ 40 & 50 $\times$ 50 & 100 $\times$ 100\\\hline
A & - & - & - & - & - & -\\\hline
B & 77,3 & 36,4 & 59,1 & 50,0 & 50,0 & 63,6\\\hline
C & 0,0 & 0,0 & 0,0 & 0,0 & 0,0 & 0,0\\\hline
D & 100,0 & 2,9 & 80,0 & 62,9 & 62,9 & 48,6\\\hline
E & 100,0 & 100,0 & 100,0 & 100,0 & 100,0 & 100,0\\\hline
H & 100,0 & 100,0 & 100,0 & 100,0 & 100,0 & 100,0\\\hline
J & 0,0 & 25,0 & 83,3 & 88,9 & 58,3 & 25,0\\\hline
K & 87,1 & 90,3 & 90,3 & 90,3 & 90,3 & 90,3\\\hline
L & 0,0 & 0,0 & 0,0 & 0,0 & 0,0 & 0,0\\\hline
M & 79,5 & 38,5 & 46,2 & 94,9 & 59,0 & 74,4\\\hline
N & 12,5 & 95,8 & 95,8 & 100,0 & 100,0 & 95,8\\\hline
O & 33,3 & 100,0 & 100,0 & 100,0 & 100,0 & 100,0\\\hline
P & 81,8 & 75,8 & 9,1 & 12,1 & 12,1 & 27,3\\\hline
R & 97,4 & 89,7 & 87,2 & 94,9 & 89,7 & 89,7\\\hline
S & 0,0 & 1,4 & 0,0 & 1,4 & 0,0 & 0,0\\\hline
T & 98,9 & 98,9 & 98,9 & 98,9 & 98,9 & 98,9\\\hline
U & 2,7 & 0,0 & 0,0 & 0,0 & 0,0 & 0,0\\\hline
V & 92,5 & 97,2 & 97,2 & 98,1 & 98,1 & 97,2\\\hline
X & 94,2 & 98,1 & 97,4 & 98,1 & 97,4 & 98,1\\\hline
Y & 87,3 & 97,9 & 96,8 & 97,4 & 97,4 & 97,9\\\hline
Z & 100,0 & 100,0 & 100,0 & 100,0 & 100,0 & 100,0\\\hline
\end{tabular}
\caption{Bla.}
\label{fig:test:and_bogstav}
\end{figure}

\begin{figure}[htp]
\centering
\begin{tabular}{|l|c|c|c|c|c|c|}\hline
\rowcolor[gray]{0.9} \multicolumn{7}{|>{\columncolor[gray]{0.9}}c|}{\textbf{Forenings-billeder}} \\ \hline
Tegn & 10 $\times$ 10 & 20 $\times$ 20 & 30 $\times$ 30 & 40 $\times$ 40 & 50 $\times$ 50 & 100 $\times$ 100\\\hline
Alle tal & 71,6 & 74,7 & 76,3 & 78,5 & 74,7 & 75,7\\\hline
Alle bogstaver & 70,7 & 70,3 & 73,0 & 75 & 72,3 & 72,0\\\hline
\end{tabular}
\caption{Afprøvningsresultater for genkendelse af tegn ved brug af forenings-billeder, hvor tegnene er opdelt i en gruppe af alle cifre og en gruppe af alle bogstaver. For hver gruppe er det angivet hvor mange procent af tegnene i gruppen der er korrekt genkendt.}
\label{fig:test:and_alletegn}
\end{figure}

Generelt viser afprøvningen af metoden med forenings-billeder at metoden er meget dårlig tilgenkendelse af bestemte tegn. det drejer sig i sær om \textbf{0}, \textbf{1}, \textbf{6}, \textbf{C}, \textbf{L}, \textbf{S} og \textbf{U}. Fælles for disse tegn er at tegnenes markerede pixels ligger i yderkanten af billederne.

%HVILKEN POSITION PÅ HITLISTEN TAGER DEN I GENNEMSNIT PR POSITION - TRÆNING?
%HVILKEN POSITION PÅ HITLISTEN TAGER DEN I GENNEMSNIT PR POSITION - TEST?
%MAXHITNO: HVAD GIVER DET JO LÆNGERE NED AD HITLISTEN VI KAN GÅ?

\begin{comment}

\begin{figure}[htp]
\centering
\begin{tabular}{|l|c|c|}\hline
\rowcolor[gray]{0.9} \multicolumn{3}{|>{\columncolor[gray]{0.9}}c|}{\textbf{Automatisk lokalisering}} \\ \hline
Sæt & Sammenhængende komponenter & Bjerg/dal \\\hline
Træningssæt & 0\% & 0\% \\\hline
Kontrolsæt & 0\% & 0\% \\\hline
\end{tabular}
\caption{Afprøvning af metoderne til separation af tegn på trænings- og kontrolsættet ved brug af automatisk lokalisering.}
\label{fig:test:sep-auto}
\end{figure}

\end{comment}

\begin{comment}
Syntaks analyse: Hvilke hits bliver valgt på hitlisterne af syntaksanalysen (sæt maxhitno højt):

\begin{tabular}{|l|c|}\hline
\multicolumn{2}{|l|}{Træningssæt} \\\hline
Hitnr. & Valgt \\\hline
1 & 95,4\% \\\hline
2 & 3\% \\\hline
3 & 0\% \\\hline
4 & 0\% \\\hline
5 & 0\% \\\hline
6 & 0\% \\\hline \end{tabular}

\begin{tabular}{|l|c|}\hline
\multicolumn{2}{|l|}{Kontrolsæt} \\\hline
Hitnr. & Valgt \\\hline
1 & 92,9\% \\\hline
2 & 0\% \\\hline
3 & 0\% \\\hline
4 & 0\% \\\hline
5 & 0\% \\\hline
6 & 0\% \\\hline \end{tabular}

\end{comment} %%%%%%%% COMMENT END


%Ovenstående med og uden syntaksanalyse? kun med? kun uden?

%Søren: Ved klassifikation bruges en eller anden grænseværdi som bestemmer hvad et element skal klassificeres som: denne grænseværdi kan reguleres og kan derfor testes.

%Delkonklusion:

Vi kunne også have testet hvilke tegn der bliver valgt i stedet for, når et forkert tegn vælges. Vælges 8 f.eks. når B burde ha været valgt??

Billederne af tegnene er ret store i forhold til dem der bruges i litteraturen. Dette giver vores system en væsentlig fordel i forhold til "de andre".

\subsection{Det samlede system}

Her vil vi afprøve det samlede system... eller har vi gjort det "på vejen"?
KUN MED DEN BEDSTE KONFIGURATION:

pcT. AF ALLE TEGN
PCT. AF ALLE PLADER


\subsection{Billeder uden for afgrænsningerne}

\begin{figure}[htp]
\centering
\begin{tabular}{|l|l|l|}
\hline
\rowcolor[gray]{0.9} \multicolumn{3}{|>{\columncolor[gray]{0.9}}c|}{\textbf{..}} \\ \hline
Metode & Blitzsæt & Gult sæt\\ \hline
Områder domineret af lyse gråtoner &  24.6\% & 0.0\%\\ \hline
Områder med høj kontrast & 5.4\% & 76.0\%\\ \hline
Frekvensanalyse &  20.8\% & 52.0\%\\ \hline
Maksimer lokal kontrast &  70.6\% & 68.0\%\\ \hline
Kvantifisering &  85.4\% & 48.0\%\\
\hline
\end{tabular}
\caption{Resultaterne af afprøvning af de individuelle metoder til lokalisering af nummerplader på blitzsættet og de gule sæt.}
\label{fig:test:lokalisering_blitz_gul}
\end{figure}

\begin{figure}[htp]
\centering
  \begin{tabular}{|l|l|l|}
    \hline
    \rowcolor[gray]{0.9} \multicolumn{3}{|>{\columncolor[gray]{0.9}}c|}{\textbf{Blitzsæt}} \\
    \hline
    Minimal enighed & Fundne nummerplader & Korrekthed\\ \hline
    1 &  83.1\% & 85.7\%\\ \hline
  \end{tabular}
\caption{Resultaterne af afprøvning af hvor effektivt vi kan lokalisere nummerplader i blitzsættet når alle fem metoder arbejde sammen.}
\label{fig:test:lokalisering_blitz_samlet}
\end{figure}


\begin{figure}[htp]
\centering
  \begin{tabular}{|l|l|l|}
    \hline
    \rowcolor[gray]{0.9} \multicolumn{3}{|>{\columncolor[gray]{0.9}}c|}{\textbf{Gult sæt}} \\
    \hline
    Minimal enighed & Fundne nummerplader & Korrekthed\\ \hline
    1 &  84.0\% & 89.4\%\\ \hline
  \end{tabular}
\caption{Resultaterne af afprøvning af hvor effektivt vi kan lokalisere nummerplader i det gule sæt når alle fem metoder arbejde sammen.}
\label{fig:test:lokalisering_gul_samlet}
\end{figure}




\subsection{Sammenligning med andres resultater}


MÅSKE SKAL DETTE STÅ I KONKLUSION? ELLERS HER
