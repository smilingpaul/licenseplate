\section{Resultater}
\label{sec:resultater}

I dette afsnit vil vi beskrive afprøvningen af vores system og give resultaterne af denne afprøvning. Alle afprøvninger vil blive udført på både trænings- og kontrolsættet.

%%%%%%%%%%%%%%%%%%%%%%%%%%%%%%%%%%%
%%% LOKALISERING AF NUMMERPLADE %%%
%%%%%%%%%%%%%%%%%%%%%%%%%%%%%%%%%%%

\subsection{Lokalisering af nummerplader}
I dette afsnit afprøver vi vores systems evne til at lokalisere nummerplader. Vi kigger altså ikke på nummerpladens tegn, men blot på om vi kan afgøre dens position i inddata-billedet. Vi har foretaget afprøvning af det samlede system som beskrevet i \vref{sec:system:lokalisering} samt individuelle afprøvninger af de fem metoder til lokalisering som beskrevet i samme afsnit.

Hvis vi skal have test med flere forskellige parametre, kan vi jo også køre en gang hvor vi siger at mindst to metoder skal være enige før vi udpeger et område.s
Trainingset


\subsubsection{Individuel afprøvning af metoderne}
Vi afprøver metoderne individuelt. Korrekthed kan ikke bruges til noget da vi ikke opgiver nummerpladekandidater på det individuelle niveau.
\begin{center}
\begin{tabular}{|l|l|l|}
\hline
\rowcolor[gray]{0.9} \multicolumn{3}{|>{\columncolor[gray]{0.9}}c|}{\textbf{..}} \\ \hline
Metode & Træningsæt & Kontrolsæt\\ \hline
Områder domineret af lyse gråtoner &  60.5 \% & 32.2\%\\ \hline
Områder med høj kontrast & 83.0\% & 73.5\%\\ \hline
Frekvensanalyse &  59.3\% & 58.3\%\\ \hline
Maksimer lokal kontrast &  82.8\% & 76.7\%\\ \hline
Kvantifisering &  69.8\% & 60.5\%\\
\hline
\end{tabular}
\end{center}


\begin{center}
\begin{tabular}{|l|l|l|}
\hline
\rowcolor[gray]{0.9} \multicolumn{3}{|>{\columncolor[gray]{0.9}}c|}{\textbf{..}} \\ \hline
Metode & Blitzsæt & Gult sæt\\ \hline
Områder domineret af lyse gråtoner &  24.6\% & 0.0\%\\ \hline
Områder med høj kontrast & 5.4\% & 76.0\%\\ \hline
Frekvensanalyse &  20.8\% & 52.0\%\\ \hline
Maksimer lokal kontrast &  70.6\% & 68.0\%\\ \hline
Kvantifisering &  85.4\% & 48.0\%\\
\hline
\end{tabular}
\end{center}


\subsubsection{Afprøvning af metoderne som et system}
Vi afprøver vi hvor god vores software er til at lokalisere nummerplader når alle fem metoder arbejder sammen. Vi har foretaget afprøvningen med flere forskellige indstillinger for minimal enighed. En enighed på f.eks. vil sige at tre af de underliggende metoder til lokalisering skal have udpeget samme område for at systemet opfatter området som et numerpladeområde og sender det videre i systemet til seperation og genkendelse. Resultaterne for afprøvningerne på træningsættet er vist på figur \vref{fig:test:lokalisering_traening_samlet} og de tilsvarende resultater for kontrolsættet er vist i figur \ref{fig:test:lokalisering_kontrol_samlet}

DER VAR EN FEJL HVIS KUN EN METODE RETURNEREDE EN KANDIDAT. KØR RESULTATER FOR ENIGHED 1 IGEN.

\begin{figure}[htp]
\centering
  \begin{tabular}{|l|l|l|}
    \hline
    \rowcolor[gray]{0.9} \multicolumn{3}{|>{\columncolor[gray]{0.9}}c|}{\textbf{Træningssæt}} \\
    \hline
    Minimal enighed & Fundne nummerplader & Korrekthed\\ \hline
    1 &  97.5\% & 99.2\%\\ \hline
    2 &  95.8\% & 99.5\%\\ \hline
    3 &  87.8\% & 99.4\%\\ \hline
    4 &  67.5\% & 100.0\%\\ \hline
  \end{tabular}
\caption{Resultaterne af afprøvning af hvor effektivt vi kan lokalisere nummerplader i træningsættet når alle fem metoder arbejde sammen.}
\label{fig:test:lokalisering_traening_samlet}
\end{figure}


\begin{figure}[htp]
\centering
  \begin{tabular}{|l|l|l|}
    \hline
    \rowcolor[gray]{0.9} \multicolumn{3}{|>{\columncolor[gray]{0.9}}c|}{\textbf{Kontrolsæt}} \\
    \hline
    Minimal enighed & Fundne nummerplader & Korrekthed\\ \hline
    1 &  93.5\% & 95.6\%\\ \hline
    2 &  92.0\% & 97.7\%\\ \hline
	3 &  78.3\% & 99.6\%\\ \hline
    4 &  49.3\% & 100.0\%\\ \hline
  \end{tabular}
\caption{Resultaterne af afprøvning af hvor effektivt vi kan lokalisere nummerplader i kontrolsættet når alle fem metoder arbejde sammen.}
\label{fig:test:lokalisering_kontrol_samlet}
\end{figure}

KØR MED ERODE!

\begin{figure}[htp]
\centering
  \begin{tabular}{|l|l|l|}
    \hline
    \rowcolor[gray]{0.9} \multicolumn{3}{|>{\columncolor[gray]{0.9}}c|}{\textbf{Blitzsæt}} \\
    \hline
    Minimal enighed & Fundne nummerplader & Korrekthed\\ \hline
    1 &  83.1\% & 85.7\%\\ \hline
  \end{tabular}
\caption{Resultaterne af afprøvning af hvor effektivt vi kan lokalisere nummerplader i blitzsættet når alle fem metoder arbejde sammen.}
\label{fig:test:lokalisering_blitz_samlet}
\end{figure}


\begin{figure}[htp]
\centering
  \begin{tabular}{|l|l|l|}
    \hline
    \rowcolor[gray]{0.9} \multicolumn{3}{|>{\columncolor[gray]{0.9}}c|}{\textbf{Gult sæt}} \\
    \hline
    Minimal enighed & Fundne nummerplader & Korrekthed\\ \hline
    1 &  84.0\% & 89.4\%\\ \hline
  \end{tabular}
\caption{Resultaterne af afprøvning af hvor effektivt vi kan lokalisere nummerplader i det gule sæt når alle fem metoder arbejde sammen.}
\label{fig:test:lokalisering_gul_samlet}
\end{figure}



\begin{comment} % udkommenteret da vi vel ikke skal bruge resultater på sættet med 407 billeder?
\subsubsection*{Observeret sæt på 407 billeder}
Scale: 0.25
DetectMain: 96.6/99.24
DetectQuant: 67.8/75.4
DetectSameness: 56.8/95.5
DetectContrastAvg: 62.7/85.0
DetectPlateness: 50.4/65.5
DetectCStretch: 84.0/92.7

Scale: 0.50 (Ekstremt langsomt)
DetectPlateness: 29.7/56.5
DetectCStretch:
\end{comment}

%Confusion matrix: De elementer der ligger udenfor diagonalen er elementer der ikke er nummerplader.

Delkonklusion:

VI SKAL SE HVORDAN DETECTMAIN OPFØRER SIG NÅR F.EKS. ALLE METODER SKAL VÆRE ENIGE.

Vi har skaleret ned for at spare tid.

Hvordan ændrer resultaterne sig når man ændrer opløsning af billederne?
Er der et mønster i de billeder hvot vi ikke finder nummerpladen? Hvile udfordringer møder vi: Mørke plader... 
HVAD SKER DER MED GULE ETC. PLADER??


%%%%%%%%%%%%%%%%%%%%%%%%%%
%%% SEPARATION AF TEGN %%%
%%%%%%%%%%%%%%%%%%%%%%%%%%

\subsection{Separation}

%DETTE AFSNIT SKRIVES SAMMEN MED GENKENDELSE.

I dette afsnit afprøves metoderne, der bruges til at separere tegn i en nummerplade.

\subsubsection*{Rotation}
%Om en nummerplade står vandret i et billede kan afgøres ved at kantdetektere billedet og udføre en Radon transformation på dette kant-billede. Hvis de stærkeste linier optræder ved $0^{\circ}$ er rotation udført korrekt.

Da vi har lavet funktionen til rotation primært ved brug af en Matlab funktion (\textit{radon}), vil det ikke være relevant at teste funktionen. Derudover vil en test af funktionen betyde at alle pladernes rotation skulle bestemmes manuelt. Vi har i stedet besluttet at funktionen testes i forbindelse med test af de andre funktioner i dette afsnit, hvliket vil sige at rotation indgår som en del af separation og genkendelse af tegn og ikke er "sin egen" funktion. ER SKREVET TIDLIGER. OMSKRIVES.

\subsubsection*{Separation}
Metoderne til separation af tegn afprøves ved at se på om de finder syv objekter indenfor nummerpladens koordinater. Denne enkle afprøvning kan ikke være fyldestgørende. Det er rettere en indikation af ....

Vi vil desuden afprøve om der er forskel på succesraten i forhold til hvilket område der repræsenterer nummerpladen. Ved at definere forskellige størrelser for nummerpladeområdet kan afprøvningen indikere hvor robust metoderne er i forhold til størrelsen på udklipningen af nummerpladen fra originalbilledet.

%Parametre: størrelse på componenter der sorteres fra, skalering: kan vi tillade os at skalere ned (eller op, hvor vi måske får mere plads mellem komponenterne) eller mister vi for meget information VI KAN IKKE SKALERE

%OVERVEJ NEDENSTÅENDE SOM SAMLET GRAF I ET AFSNIT
Resultaterne fra afprøvningen af separation af tegn ses i figurerne \vref{fig:test:sep-traening-manuel} og \vref{fig:test:sep-kontrol-manuel}.

\begin{figure}[htp]
\centering
\begin{tabular}{|l|c|c|}\hline
\rowcolor[gray]{0.9} \multicolumn{3}{|>{\columncolor[gray]{0.9}}c|}{\textbf{Træningssæt}} \\ \hline
Antal pixels lagt til & Sammenhængende komponenter & Bjerg/dal \\\hline
10 & 96,5\% & 91,8\% \\\hline
20 & 95,8\% & 91,3\% \\\hline
30 & 92,5\% & 83,5\% \\\hline
40 & 87,5\% & 73,4\% \\\hline \end{tabular}
\caption{Afprøvning af metoderne til separation af tegn på træningssættet ved brug af manuelt udpegede plader.}
\label{fig:test:sep-traening-manuel}
\end{figure}

\begin{figure}[htp]
\centering
\begin{tabular}{|l|c|c|}\hline
\rowcolor[gray]{0.9} \multicolumn{3}{|>{\columncolor[gray]{0.9}}c|}{\textbf{Kontrolsæt}} \\ \hline
Antal pixels lagt til & Sammenhængende komponenter & Bjerg/dal \\\hline
10 & 90,0\% & 87,2\% \\\hline
20 & 86,8\% & 81,0\% \\\hline
30 & 84,7\% & 71,2\% \\\hline
40 & 79,0\% & 58,5\% \\\hline \end{tabular}
\caption{Afprøvning af metoderne til separation af tegn på kontrolsættet ved brug af manuelt udpegede plader.}
\label{fig:test:sep-kontrol-manuel}
\end{figure}

%\begin{tabular}{|l|c|c|}\hline
%\rowcolor[gray]{0.9} \multicolumn{3}{|>{\columncolor[gray]{0.9}}c|}{\textbf{Kontrolsæt}} \\ \hline
%Bedste udklip & Sammenhængende komponenter & Bjerg/dal \\\hline
% & 0\% & 0\% \\\hline
%\end{tabular}

Sammenhængende komponenter - Træningssæt

Denne metode giver de bedste resultater af de to præsenterede metoder. Tegnene klippes tæt og hvis nummerpladeområdet ikke er for stort, men derimod stadig indeholder nummerpladens kanter, sådan at en eventuelt rotation kan foretages korrekt, giver metoden sjældent fejl.

Fejlkilderne, når nummerpladeområdet er lille (det vil sige at området er udviddet med 10 eller 20 pixels), er primært skruerne i pladen og skæve plader. Skruerne bevirker enten at tegnene gror sammen med hinanden eller med pladens kant og derfor vælges fra, fordi de sammenhængende komponenter er for store. De skæve plader bevirker, at de komponenter der repræsenterer tegnene ikke står i samme højde og derfor bliver valgt fra.

EVT. EKSEMPEL PÅ PROBLEM MED SKRUER SAMT SKÆV PLADE.

Fejlkilderne når nummerpladeområdet er større (det vil sige udviddet med 30 eller 40 pixels) er at en anden komponent end den der indeholder nummerpladen vælges når billedet skal indskrænkes eller at indskrænkningen ikke er stor nok og at der derfor laves komponenter som er udenfor og er tilpas høje osv.

ETV. EKSEMPEL PÅ FORKERT INDSKRÆNKNING OG DÅRLIG INDSKRÆNKNING

Sammenhængende komponenter - kontrolsæt

Bjerg/dal - Træningssæt

Som tidligere beskrevet er bjerg/dal metoden meget følsom overfor støj, hvilket også ses i afprøvningsresultaterne. Derudover er afprøvningsmetoden et dårligt skøn på om metoden udklipper tegnene korrekt, da mange af de udklippede elementer enten er halve tegn eller store elementer, hvor tegnet kun er en del af.

EVT. EKSEMPEL PÅ SIGNATUR FØLSOM OVERFOR STØJ OG DÅRLIG UDKILP

Bjerg/dal - Kontrolsæt

Støt begrundelse af div. valg i koden med testresultater. Er det rimeligt at klippe de største område ud når vi finder nummerplader? 


\begin{comment}
Problemer ved 10: tegn der er smeltet sammen pga. søm skruer: OE26906 og ----290. Godt udskåret: UX33152, skæve plader et problem. go kontrast: YE39734.

Problemer ved 20: for stort område giver komponenter der ikke klippes bort, fordi kontrastforstærkning giver et andet resultat, her ville dynamisk kontrast-blok størrelse måske hjælpe. Dårlig indskrænk XK29750. TS57793: GOD(?) 

Problemer ve 30: stadig for stort område: giver komponenter udenfor der vælges. dårlig indskrænk, især foran? eksempler.

\end{comment}

%%% GENKENDELSE AF TEGN %%%

\subsection{Genkendelse af tegn}
I dette afsnit afprøves metoderne til genkendelse af tegn i en nummerplade. %Derudover vil syntaksanalysen blive afprøvet. VIL DEN?

Afprøvningen vil blive foretaget på alle tegn fra kontrolsættet, separeret med den bedste metode (sammenhængende komponenter hvor 10 pixels er lagt til). Disse tegn er sorteret så billeder af tegn der er udklippet forkert og hvor tegnet eksempelvis er blevet halveret under separationen er blevet sorteret fra. Afprøvningen vil derfor give en indikation af, hvor gode metoderne er til at genkende tegn i billeder som et menneske kan aflæse.

%Noget om at det er i forhold til alle plader, dvs. at procenterne nedenfor er afhængige af hvor gode vi er til separere tegn...


%Følgende tabeller viser resultaterne for at læse en hel plade, seks tegn i pladen osv. (med forskellige vektorlængder).

%Nedenstående tests er i forhold til manuelt udklippede plader - ikke automtisk lokalisering.

I figurerne \vref{fig:test:genkend_traening_tal}, \vref{fig:test:genkend_traening_bogstav}, \vref{fig:test:genkend_kontrol_tal} samt \vref{fig:test:genkend_kontrol_bogstav} ses resultaterne af afprøvningen af genkendelse af tegn. Resultaterne er delt op i tal- og bogstavertabeller.

\begin{comment}
\begin{figure}[htp]
\centering
\begin{tabular}{|l|c|c|c|}\hline
\rowcolor[gray]{0.9} \multicolumn{4}{|>{\columncolor[gray]{0.9}}c|}{\textbf{Træningssæt}} \\ \hline
Tegn & Middelvektor & Sum-billeder & And-billeder\\\hline
0 & 0\% & 0\% & 0\%\\\hline
1 & 0\%  & 0\% & 0\%\\\hline
2 & 0\% & 0\% & 0\%\\\hline
3 & 0\% & 0\% & 0\%\\\hline
4 & 0\% & 0\% & 0\%\\\hline
5 & 0\% & 0\% & 0\%\\\hline
6 & 0\% & 0\% & 0\%\\\hline
7 & 0\% & 0\% & 0\%\\\hline
8 & 0\% & 0\% & 0\%\\\hline
9 & 0\% & 0\% & 0\%\\\hline
\end{tabular}
\caption{Bla.}
\label{fig:test:and_tal}
\end{figure}

\begin{figure}[htp]
\centering
\begin{tabular}{|l|c|c|c|}\hline
\rowcolor[gray]{0.9} \multicolumn{4}{|>{\columncolor[gray]{0.9}}c|}{\textbf{Træningssæt}} \\ \hline
Tegn & Middelvektor & Sum-billeder & And-billeder\\\hline
A & 0\% & 0\% & 0\%\\\hline
B & 0\% & 0\% & 0\%\\\hline
C & 0\% & 0\% & 0\%\\\hline
D & 0\% & 0\% & 0\%\\\hline
E & 0\% & 0\% & 0\%\\\hline
H & 0\% & 0\% & 0\%\\\hline
J & 0\% & 0\% & 0\%\\\hline
K & 0\% & 0\% & 0\%\\\hline 
L & 0\% & 0\% & 0\%\\\hline
M & 0\% & 0\% & 0\%\\\hline
N & 0\% & 0\% & 0\%\\\hline
O & 0\% & 0\% & 0\%\\\hline
P & 0\% & 0\% & 0\%\\\hline
R & 0\% & 0\% & 0\%\\\hline
S & 0\% & 0\% & 0\%\\\hline
T & 0\% & 0\% & 0\%\\\hline
U & 0\% & 0\% & 0\%\\\hline
V & 0\% & 0\% & 0\%\\\hline
X & 0\% & 0\% & 0\%\\\hline
Y & 0\% & 0\% & 0\%\\\hline
Z & 0\% & 0\% & 0\%\\\hline
\end{tabular}
\caption{Bla.}
\label{fig:test:and_bogstav}
\end{figure}
\end{comment}

\begin{figure}[htp]
\centering
\begin{tabular}{|l|c|c|c|c|c|}\hline
\rowcolor[gray]{0.9} \multicolumn{6}{|>{\columncolor[gray]{0.9}}c|}{\textbf{Middelvektor}} \\ \hline
Tegn & 9 & 16 & 25 & 36 & 49\\\hline
0 & 21,0 & 77,1 & 74,8 & 56,1 & 91,1\\\hline
1 & 79,3 & 94,1 & 97,5 & 97,5 & 98,0\\\hline
2 & 92,7 & 93,3 & 98,0 & 98,9 & 99,2\\\hline
3 & 86,1 & 99,4 & 98,9 & 100,0 & 100,0\\\hline
4 & 84,1 & 99,1 & 99,1 & 99,7 & 99,7\\\hline
5 & 96,6 & 98,3 & 98,3 & 99,4 & 99,4\\\hline
6 & 42,1 & 99,5 & 100,0 & 99,5 & 100,0\\\hline
7 & 92,4 & 94,2 & 98,7 & 99,6 & 99,6\\\hline
8 & 13,0 & 45,2 & 90,4 & 93,5 & 97,0\\\hline
9 & 82,6 & 94,6 & 96,4 & 98,2 & 97,3\\\hline
\end{tabular}
\caption{Afprøvningsresultater for genkendelse af tal-tegn ved brug af middelvektorer. For hvert tegn er det angivet hvor mange procent af billederne af dette tegn, der er korrekt genkendt.}
\label{fig:test:middel_tal}
\end{figure}

\begin{figure}[htp]
\centering
\begin{tabular}{|l|c|c|c|c|c|}\hline
\rowcolor[gray]{0.9} \multicolumn{6}{|>{\columncolor[gray]{0.9}}c|}{\textbf{Middelvektor}} \\ \hline
Tegn & 9 & 16 & 25 & 36 & 49\\\hline
A & - & - & - & - & -\\\hline
B & 40,9 & 68,2 & 90,9 & 95,5 & 95,5\\\hline
C & 65,6 & 93,8 & 96,9 & 100,0 & 100,0\\\hline
D & 91,4 & 100,0 & 100,0 & 97,1 & 91,4\\\hline
E & 65,0 & 85,0 & 95,0 & 95,0 & 100,0\\\hline
H & 28,6 & 100,0 & 85,7 & 100,0 & 100,0\\\hline
J & 97,2 & 97,2 & 100,0 & 97,2 & 100,0\\\hline
K & 54,8 & 80,6 & 90,3 & 90,3 & 90,3\\\hline
L & 25,0 & 100,0 & 100,0 & 100,0 & 100,0\\\hline
M & 20,5 & 87,2 & 97,4 & 94,9 & 94,9\\\hline
N & 41,7 & 100,0 & 100,0 & 100,0 & 100,0\\\hline
O & 0,0 & 66,7 & 0,0    & 66,7 & 66,7\\\hline
P & 78,8 & 100,0 & 100,0 & 100,0 & 100,0\\\hline
R & 28,2 & 87,2 & 92,3 & 94,9 & 94,9\\\hline
S & 98,6 & 97,1 & 92,7 & 100,0 & 100,0\\\hline
T & 94,4 & 98,9 & 98,9 & 98,9 & 100,0\\\hline
U & 1,3 & 93,3 & 96,0 & 97,3 & 98,7\\\hline
V & 90,6 & 98,1 & 98,1 & 98,1 & 98,1\\\hline
X & 92,2 & 97,4 & 98,7 & 98,7 & 98,7\\\hline
Y & 95,8 & 98,9 & 98,9 & 98,4 & 98,9\\\hline
Z & 77,2 & 100,0 & 96,5 & 100,0 & 100,0\\\hline
\end{tabular}
\caption{Afprøvningsresultater for genkendelse af bogstav-tegn ved brug af middelvektorer. For hvert tegn er det angivet hvor mange procent af billederne af dette tegn, der er korrekt genkendt. Rækken med tegnet \textbf{A} er tom, da hverken trænings- eller kontrolsættet indeholder nogen tegn af denne type.}
\label{fig:test:middel_bogstav}
\end{figure}

\begin{figure}[htp]
\centering
\begin{tabular}{|l|c|c|c|c|c|}\hline
\rowcolor[gray]{0.9} \multicolumn{6}{|>{\columncolor[gray]{0.9}}c|}{\textbf{Sum-billeder}} \\ \hline
Tegn & 3 $\times$ 3 & 4 $\times$ 4 & 5 $\times$ 5 & 6 $\times$ 6 & 7 $\times$ 7\\\hline
0 & 0,0 & 64,0 & 73,4 & 54,2 & 81,3\\\hline
1 & 98,5  & 99,0 & 96,1 & 99,0 & 99,5\\\hline
2 & 81,6 & 56,7 & 90,5 & 99,7 & 99,7\\\hline
3 & 3,9 & 45,3 & 98,9 & 97,8 & 98,9\\\hline
4 & 22,8 & 98,8 & 98,8 & 99,4 & 99,4\\\hline
5 & 73,4 & 42,7 & 93,7 & 99,1 & 99,1\\\hline
6 & 31,6 & 57,4 & 96,3 & 98,4 & 97,9\\\hline
7 & 0,0 & 93,8 & 99,6 & 99,6 & 99,6\\\hline
8 & 1,3 & 0,7 & 42,6 & 80,0 & 98,3\\\hline
9 & 47,8 & 62,9 & 94,2 & 98,7 & 96,9\\\hline
\end{tabular}
\caption{Afprøvningsresultater for genkendelse af tal-tegn ved brug af sum-billeder. For hvert tegn er det angivet hvor mange procent af billederne af dette tegn, der er korrekt genkendt.}
\label{fig:test:sum_tal}
\end{figure}

\begin{figure}[htp]
\centering
\begin{tabular}{|l|c|c|c|c|c|}\hline
\rowcolor[gray]{0.9} \multicolumn{6}{|>{\columncolor[gray]{0.9}}c|}{\textbf{Sum-billeder}} \\ \hline
Tegn & 3 $\times$ 3 & 4 $\times$ 4 & 5 $\times$ 5 & 6 $\times$ 6 & 7 $\times$ \\\hline
A & - & - & - & - & -\\\hline
B & 27,3 & 40,9 & 95,5 & 100,0 & 95,5\\\hline
C & 0,0 & 84,4 & 28,1 & 3,1 & 53,1\\\hline
D & 20, & 100,0 & 100,0 & 91,4 & 80,0\\\hline
E & 65,0 & 90,0 & 100,0 & 100,0 & 100,0\\\hline
H & 100,0 & 85,7 & 85,7 & 85,7 & 85,7\\\hline
J & 100,0 & 97,2 & 97,2 & 97,2 & 97,2\\\hline
K & 54,8 & 80,6 & 87,1 & 90,3 & 87,1\\\hline 
L & 15,0 & 0,0 & 0,0 & 0,0 & 0,0\\\hline
M & 25,6 & 89,7 & 97,4 & 97,4 & 89,7\\\hline
N & 0,0 & 100,0 & 100,0 & 100,0 & 100,0\\\hline
O & 0,0 & 33,3 & 0,0 & 66,7 & 33,3\\\hline
P & 39,4 & 42,4 & 51,5 & 100,0 & 100,0\\\hline
R & 10,3 & 82,1 & 94,9 & 92,3 & 94,9\\\hline
S & 0,0 & 0,0 & 27,5 & 56,5 & 23,2\\\hline
T & 82,0 & 3,4 & 98,9 & 98,9 & 97,8\\\hline
U & 2,7 & 92,0 & 93,3 & 82,7 & 98,7\\\hline
V & 82,1 & 80,2 & 97,2 & 96,2 & 95,3\\\hline
X & 0,0 & 98,7 & 96,1 & 98,7 & 98,7\\\hline
Y & 97,9 & 54,5 & 93,1 & 97,9 & 97,4\\\hline
Z & 82,5 & 100,0 & 100,0 & 100,0 & 98,2\\\hline
Snit & \textbf{10} & \textbf{10} & \textbf{10} & \textbf{10} & \textbf{10}\\\hline
\end{tabular}
\caption{Afprøvningsresultater for genkendelse af bogstav-tegn ved brug af sum-billeder. For hvert tegn er det angivet hvor mange procent af billederne af dette tegn, der er korrekt genkendt. Rækken med tegnet \textbf{A} er tom, da hverken trænings- eller kontrolsættet indeholder nogen tegn af denne type.}
\label{fig:test:sum_bogstav}
\end{figure}

\begin{figure}[htp]
\centering
\begin{tabular}{|l|c|c|c|}\hline
\rowcolor[gray]{0.9} \multicolumn{4}{|>{\columncolor[gray]{0.9}}c|}{\textbf{And}} \\ \hline
Tegn & 9 & 16 & 25\\\hline
0 & 0 & 0 & 0\\\hline
1 & 0  & 0 & 0\\\hline
2 & 0 & 0 & 0\\\hline
3 & 0 & 0 & 0\\\hline
4 & 0 & 0 & 0\\\hline
5 & 0 & 0 & 0\\\hline
6 & 0 & 0 & 0\\\hline
7 & 0 & 0 & 0\\\hline
8 & 0 & 0 & 0\\\hline
9 & 0 & 0 & 0\\\hline
\end{tabular}
\caption{Bla.}
\label{fig:test:and_tal}
\end{figure}

\begin{figure}[htp]
\centering
\begin{tabular}{|l|c|c|c|}\hline
\rowcolor[gray]{0.9} \multicolumn{4}{|>{\columncolor[gray]{0.9}}c|}{\textbf{And}} \\ \hline
Tegn & 9 & 16 & 25\\\hline
A & - & - & -\\\hline
B & 0 & 0 & 0\\\hline
C & 0 & 0 & 0\\\hline
D & 0 & 0 & 0\\\hline
E & 0 & 0 & 0\\\hline
H & 0 & 0 & 0\\\hline
J & 0 & 0 & 0\\\hline
K & 0 & 0 & 0\\\hline 
L & 0 & 0 & 0\\\hline
M & 0 & 0 & 0\\\hline
N & 0 & 0 & 0\\\hline
O & 0 & 0 & 0\\\hline
P & 0 & 0 & 0\\\hline
R & 0 & 0 & 0\\\hline
S & 0 & 0 & 0\\\hline
T & 0 & 0 & 0\\\hline
U & 0 & 0 & 0\\\hline
V & 0 & 0 & 0\\\hline
X & 0 & 0 & 0\\\hline
Y & 0 & 0 & 0\\\hline
Z & 0 & 0 & 0\\\hline
\end{tabular}
\caption{Bla.}
\label{fig:test:and_bogstav}
\end{figure}


Noget med at der er få o'er og andre bogstaver. Noget med at man ikke kan bruge 2x2 billeder til noget. Noget med at der ikke er nogen Aer.

\begin{comment} %%%%%%%% COMMENT START

\subsubsection*{Middelvektor}

Ved at lave middelvektorer på træningssættet kan vi se at man ikke kan have vektorer af længden 4, da flere bogstavs vektorer så vil være de samme (nemlig 0 0 0 0).

\begin{tabular}{|l|c|c|c|c|c|c|c|c|}\hline
\rowcolor[gray]{0.9} \multicolumn{9}{|>{\columncolor[gray]{0.9}}c|}{\textbf{Træningssæt}} \\\hline
\multicolumn{9}{|c|}{Med syntaksanalyse}\\\hline
Vektorlængde & 7 & 6 & 5 & 4 & 3 & 2 & 1 & 0\\\hline
9 & 0\% & 0\% & 0\% & 0\% & 0\% & 0\% & 0\% & 0\% \\\hline
16 & 0\% & 0\% & 0\% & 0\% & 0\% & 0\% & 0\% & 0\%\\\hline
25 & 0\% & 0\% & 0\% & 0\% & 0\% & 0\% & 0\% & 0\%\\\hline 
\multicolumn{9}{|c|}{Uden syntaksanalyse}\\\hline
Vektorlængde & 7 & 6 & 5 & 4 & 3 & 2 & 1 & 0\\\hline
9 & 0\% & 0\% & 0\% & 0\% & 0\% & 0\% & 0\% & 0\% \\\hline
16 & 0\% & 0\% & 0\% & 0\% & 0\% & 0\% & 0\% & 0\%\\\hline
25 & 0\% & 0\% & 0\% & 0\% & 0\% & 0\% & 0\% & 0\%\\\hline 
\end{tabular}

\begin{tabular}{|l|c|c|c|c|c|c|c|c|}\hline
\rowcolor[gray]{0.9} \multicolumn{9}{|>{\columncolor[gray]{0.9}}c|}{\textbf{Kontrolsæt}} \\\hline
\multicolumn{9}{|c|}{Med syntaksanalyse}\\\hline
Vektorlængde & 7 & 6 & 5 & 4 & 3 & 2 & 1 & 0\\\hline
9 & 0\% & 0\% & 0\% & 0\% & 0\% & 0\% & 0\% & 0\% \\\hline
16 & 0\% & 0\% & 0\% & 0\% & 0\% & 0\% & 0\% & 0\%\\\hline
25 & 0\% & 0\% & 0\% & 0\% & 0\% & 0\% & 0\% & 0\%\\\hline 
\multicolumn{9}{|c|}{Uden syntaksanalyse}\\\hline
Vektorlængde & 7 & 6 & 5 & 4 & 3 & 2 & 1 & 0\\\hline
9 & 0\% & 0\% & 0\% & 0\% & 0\% & 0\% & 0\% & 0\% \\\hline
16 & 0\% & 0\% & 0\% & 0\% & 0\% & 0\% & 0\% & 0\%\\\hline
25 & 0\% & 0\% & 0\% & 0\% & 0\% & 0\% & 0\% & 0\%\\\hline 
\end{tabular}


\begin{tabular}{|l|c|c|c|}\hline
\rowcolor[gray]{0.9} \multicolumn{4}{|>{\columncolor[gray]{0.9}}c|}{\textbf{Træningssæt}} \\ \hline
Vektorlængde & Alle tegn & Bogstaver & Tal \\\hline
9 & 0\% & 0\% & 0\% \\\hline
16 & 0\% & 0\% & 0\%\\\hline
25 & 0\% & 0\% & 0\%\\\hline \end{tabular}

\begin{tabular}{|l|c|c|c|}\hline
\rowcolor[gray]{0.9} \multicolumn{4}{|>{\columncolor[gray]{0.9}}c|}{\textbf{Kontrolsæt}} \\ \hline
Vektorlængde & Alle tegn & Bogstaver & Tal \\\hline
9 & 0\% & 0\% & 0\% \\\hline
16 & 0\% & 0\% & 0\% \\\hline
25 & 0\% & 0\% & 0\% \\\hline \end{tabular}


\subsubsection*{Sum-billeder}

\begin{tabular}{|l|c|c|c|c|c|c|c|c|}\hline
\rowcolor[gray]{0.9} \multicolumn{9}{|>{\columncolor[gray]{0.9}}c|}{\textbf{Træningssæt}} \\\hline
\multicolumn{9}{|c|}{Med syntaksanalyse}\\\hline
Billedstørrelse & 7 & 6 & 5 & 4 & 3 & 2 & 1 & 0\\\hline
$5 \times 5$ & 0\% & 0\% & 0\% & 0\% & 0\% & 0\% & 0\% & 0\% \\\hline
$10 \times 10$ & 0\% & 0\% & 0\% & 0\% & 0\% & 0\% & 0\% & 0\%\\\hline
$20 \times 20$ & 0\% & 0\% & 0\% & 0\% & 0\% & 0\% & 0\% & 0\%\\\hline 
\multicolumn{9}{|c|}{Uden syntaksanalyse}\\\hline
Billedstørrelse & 7 & 6 & 5 & 4 & 3 & 2 & 1 & 0\\\hline
$5 \times 5$ & 0\% & 0\% & 0\% & 0\% & 0\% & 0\% & 0\% & 0\% \\\hline
$10 \times 10$ & 0\% & 0\% & 0\% & 0\% & 0\% & 0\% & 0\% & 0\%\\\hline
$20 \times 20$ & 0\% & 0\% & 0\% & 0\% & 0\% & 0\% & 0\% & 0\%\\\hline \end{tabular}

\begin{tabular}{|l|c|c|c|c|c|c|c|c|}\hline
\rowcolor[gray]{0.9} \multicolumn{9}{|>{\columncolor[gray]{0.9}}c|}{\textbf{Kontrolsæt}} \\\hline
\multicolumn{9}{|c|}{Med syntaksanalyse}\\\hline
Billedstørrelse & 7 & 6 & 5 & 4 & 3 & 2 & 1 & 0\\\hline
$5 \times 5$ & 0\% & 0\% & 0\% & 0\% & 0\% & 0\% & 0\% & 0\% \\\hline
$10 \times 10$ & 0\% & 0\% & 0\% & 0\% & 0\% & 0\% & 0\% & 0\%\\\hline
$20 \times 20$ & 0\% & 0\% & 0\% & 0\% & 0\% & 0\% & 0\% & 0\%\\\hline 
\multicolumn{9}{|c|}{Uden syntaksanalyse}\\\hline
Billedstørrelse & 7 & 6 & 5 & 4 & 3 & 2 & 1 & 0\\\hline
$5 \times 5$ & 0\% & 0\% & 0\% & 0\% & 0\% & 0\% & 0\% & 0\% \\\hline
$10 \times 10$ & 0\% & 0\% & 0\% & 0\% & 0\% & 0\% & 0\% & 0\%\\\hline
$20 \times 20$ & 0\% & 0\% & 0\% & 0\% & 0\% & 0\% & 0\% & 0\%\\\hline \end{tabular}

\begin{tabular}{|l|c|c|c|}\hline
\rowcolor[gray]{0.9} \multicolumn{4}{|>{\columncolor[gray]{0.9}}c|}{\textbf{Træningssæt}} \\ \hline
Billedstørrelse & Alle tegn & Bogstaver & Tal \\\hline
$5 \times 5$ & 0\% & 0\% & 0\% \\\hline
$10 \times 10$ & 0\% & 0\% & 0\%\\\hline
$20 \times 20$ & 0\% & 0\% & 0\%\\\hline \end{tabular}

\begin{tabular}{|l|c|c|c|}\hline
\rowcolor[gray]{0.9} \multicolumn{4}{|>{\columncolor[gray]{0.9}}c|}{\textbf{Kontrolsæt}} \\ \hline
Billedstørrelse & Alle tegn & Bogstaver & Tal \\\hline
$5 \times 5$ & 0\% & 0\% & 0\% \\\hline
$10 \times 10$ & 0\% & 0\% & 0\%\\\hline
$20 \times 20$ & 0\% & 0\% & 0\%\\\hline \end{tabular}


\subsubsection*{And-billeder}

\begin{tabular}{|l|c|c|c|c|c|c|c|c|}\hline
\rowcolor[gray]{0.9} \multicolumn{9}{|>{\columncolor[gray]{0.9}}c|}{\textbf{Træningssæt}} \\\hline
\multicolumn{9}{|c|}{Med syntaksanalyse}\\\hline
Billedstørrelse & 7 & 6 & 5 & 4 & 3 & 2 & 1 & 0\\\hline
$5 \times 5$ & 0\% & 0\% & 0\% & 0\% & 0\% & 0\% & 0\% & 0\% \\\hline
$10 \times 10$ & 0\% & 0\% & 0\% & 0\% & 0\% & 0\% & 0\% & 0\%\\\hline
$20 \times 20$ & 0\% & 0\% & 0\% & 0\% & 0\% & 0\% & 0\% & 0\%\\\hline 
\multicolumn{9}{|c|}{Uden syntaksanalyse}\\\hline
Billedstørrelse & 7 & 6 & 5 & 4 & 3 & 2 & 1 & 0\\\hline
$5 \times 5$ & 0\% & 0\% & 0\% & 0\% & 0\% & 0\% & 0\% & 0\% \\\hline
$10 \times 10$ & 0\% & 0\% & 0\% & 0\% & 0\% & 0\% & 0\% & 0\%\\\hline
$20 \times 20$ & 0\% & 0\% & 0\% & 0\% & 0\% & 0\% & 0\% & 0\%\\\hline \end{tabular}

\begin{tabular}{|l|c|c|c|c|c|c|c|c|}\hline
\rowcolor[gray]{0.9} \multicolumn{9}{|>{\columncolor[gray]{0.9}}c|}{\textbf{Kontrolsæt}} \\\hline
\multicolumn{9}{|c|}{Med syntaksanalyse}\\\hline
Billedstørrelse & 7 & 6 & 5 & 4 & 3 & 2 & 1 & 0\\\hline
$5 \times 5$ & 0\% & 0\% & 0\% & 0\% & 0\% & 0\% & 0\% & 0\% \\\hline
$10 \times 10$ & 0\% & 0\% & 0\% & 0\% & 0\% & 0\% & 0\% & 0\%\\\hline
$20 \times 20$ & 0\% & 0\% & 0\% & 0\% & 0\% & 0\% & 0\% & 0\%\\\hline 
\multicolumn{9}{|c|}{Uden syntaksanalyse}\\\hline
Billedstørrelse & 7 & 6 & 5 & 4 & 3 & 2 & 1 & 0\\\hline
$5 \times 5$ & 0\% & 0\% & 0\% & 0\% & 0\% & 0\% & 0\% & 0\% \\\hline
$10 \times 10$ & 0\% & 0\% & 0\% & 0\% & 0\% & 0\% & 0\% & 0\%\\\hline
$20 \times 20$ & 0\% & 0\% & 0\% & 0\% & 0\% & 0\% & 0\% & 0\%\\\hline \end{tabular}

\end{comment}

\begin{comment}
\begin{tabular}{|l|c|c|c|}\hline
\rowcolor[gray]{0.9} \multicolumn{4}{|>{\columncolor[gray]{0.9}}c|}{\textbf{Træningssæt}} \\ \hline
Billedstørrelse & Alle tegn & Bogstaver & Tal \\\hline
$5 \times 5$ & 0\% & 0\% & 0\% \\\hline
$10 \times 10$ & 0\% & 0\% & 0\%\\\hline
$20 \times 20$ & 0\% & 0\% & 0\%\\\hline \end{tabular}

\begin{tabular}{|l|c|c|c|}\hline
\rowcolor[gray]{0.9} \multicolumn{4}{|>{\columncolor[gray]{0.9}}c|}{\textbf{Kontrolsæt}} \\ \hline
Billedstørrelse & Alle tegn & Bogstaver & Tal \\\hline
$5 \times 5$ & 0\% & 0\% & 0\% \\\hline
$10 \times 10$ & 0\% & 0\% & 0\%\\\hline
$20 \times 20$ & 0\% & 0\% & 0\%\\\hline \end{tabular}
\end{comment}


%HVILKEN POSITION PÅ HITLISTEN TAGER DEN I GENNEMSNIT PR POSITION - TRÆNING?
%HVILKEN POSITION PÅ HITLISTEN TAGER DEN I GENNEMSNIT PR POSITION - TEST?
%MAXHITNO: HVAD GIVER DET JO LÆNGERE NED AD HITLISTEN VI KAN GÅ?

\begin{figure}[htp]
\centering
\begin{tabular}{|l|c|c|}\hline
\rowcolor[gray]{0.9} \multicolumn{3}{|>{\columncolor[gray]{0.9}}c|}{\textbf{Automatisk lokalisering}} \\ \hline
Sæt & Sammenhængende komponenter & Bjerg/dal \\\hline
Træningssæt & 0\% & 0\% \\\hline
Kontrolsæt & 0\% & 0\% \\\hline
\end{tabular}
\caption{Afprøvning af metoderne til separation af tegn på trænings- og kontrolsættet ved brug af automatisk lokalisering.}
\label{fig:test:sep-auto}
\end{figure}

\begin{comment}
Syntaks analyse: Hvilke hits bliver valgt på hitlisterne af syntaksanalysen (sæt maxhitno højt):

\begin{tabular}{|l|c|}\hline
\multicolumn{2}{|l|}{Træningssæt} \\\hline
Hitnr. & Valgt \\\hline
1 & 95,4\% \\\hline
2 & 3\% \\\hline
3 & 0\% \\\hline
4 & 0\% \\\hline
5 & 0\% \\\hline
6 & 0\% \\\hline \end{tabular}

\begin{tabular}{|l|c|}\hline
\multicolumn{2}{|l|}{Kontrolsæt} \\\hline
Hitnr. & Valgt \\\hline
1 & 92,9\% \\\hline
2 & 0\% \\\hline
3 & 0\% \\\hline
4 & 0\% \\\hline
5 & 0\% \\\hline
6 & 0\% \\\hline \end{tabular}

\end{comment} %%%%%%%% COMMENT END

\begin{comment} %%%%%%%% COMMENT START

\subsubsection*{Middelvektor}

Ved at lave middelvektorer på træningssættet kan vi se at man ikke kan have vektorer af længden 4, da flere bogstavs vektorer så vil være de samme (nemlig 0 0 0 0).

\begin{tabular}{|l|c|c|c|c|c|c|c|c|}\hline
\rowcolor[gray]{0.9} \multicolumn{9}{|>{\columncolor[gray]{0.9}}c|}{\textbf{Træningssæt}} \\\hline
\multicolumn{9}{|c|}{Med syntaksanalyse}\\\hline
Vektorlængde & 7 & 6 & 5 & 4 & 3 & 2 & 1 & 0\\\hline
9 & 0\% & 0\% & 0\% & 0\% & 0\% & 0\% & 0\% & 0\% \\\hline
16 & 0\% & 0\% & 0\% & 0\% & 0\% & 0\% & 0\% & 0\%\\\hline
25 & 0\% & 0\% & 0\% & 0\% & 0\% & 0\% & 0\% & 0\%\\\hline 
\multicolumn{9}{|c|}{Uden syntaksanalyse}\\\hline
Vektorlængde & 7 & 6 & 5 & 4 & 3 & 2 & 1 & 0\\\hline
9 & 0\% & 0\% & 0\% & 0\% & 0\% & 0\% & 0\% & 0\% \\\hline
16 & 0\% & 0\% & 0\% & 0\% & 0\% & 0\% & 0\% & 0\%\\\hline
25 & 0\% & 0\% & 0\% & 0\% & 0\% & 0\% & 0\% & 0\%\\\hline 
\end{tabular}

\begin{tabular}{|l|c|c|c|c|c|c|c|c|}\hline
\rowcolor[gray]{0.9} \multicolumn{9}{|>{\columncolor[gray]{0.9}}c|}{\textbf{Kontrolsæt}} \\\hline
\multicolumn{9}{|c|}{Med syntaksanalyse}\\\hline
Vektorlængde & 7 & 6 & 5 & 4 & 3 & 2 & 1 & 0\\\hline
9 & 0\% & 0\% & 0\% & 0\% & 0\% & 0\% & 0\% & 0\% \\\hline
16 & 0\% & 0\% & 0\% & 0\% & 0\% & 0\% & 0\% & 0\%\\\hline
25 & 0\% & 0\% & 0\% & 0\% & 0\% & 0\% & 0\% & 0\%\\\hline 
\multicolumn{9}{|c|}{Uden syntaksanalyse}\\\hline
Vektorlængde & 7 & 6 & 5 & 4 & 3 & 2 & 1 & 0\\\hline
9 & 0\% & 0\% & 0\% & 0\% & 0\% & 0\% & 0\% & 0\% \\\hline
16 & 0\% & 0\% & 0\% & 0\% & 0\% & 0\% & 0\% & 0\%\\\hline
25 & 0\% & 0\% & 0\% & 0\% & 0\% & 0\% & 0\% & 0\%\\\hline 
\end{tabular}



\begin{tabular}{|l|c|c|c|}\hline
\rowcolor[gray]{0.9} \multicolumn{4}{|>{\columncolor[gray]{0.9}}c|}{\textbf{Træningssæt}} \\ \hline
Vektorlængde & Alle tegn & Bogstaver & Tal \\\hline
9 & 0\% & 0\% & 0\% \\\hline
16 & 0\% & 0\% & 0\%\\\hline
25 & 0\% & 0\% & 0\%\\\hline \end{tabular}

\begin{tabular}{|l|c|c|c|}\hline
\rowcolor[gray]{0.9} \multicolumn{4}{|>{\columncolor[gray]{0.9}}c|}{\textbf{Kontrolsæt}} \\ \hline
Vektorlængde & Alle tegn & Bogstaver & Tal \\\hline
9 & 0\% & 0\% & 0\% \\\hline
16 & 0\% & 0\% & 0\% \\\hline
25 & 0\% & 0\% & 0\% \\\hline \end{tabular}

\subsubsection*{Sum-billeder}

\begin{tabular}{|l|c|c|c|c|c|c|c|c|}\hline
\rowcolor[gray]{0.9} \multicolumn{9}{|>{\columncolor[gray]{0.9}}c|}{\textbf{Træningssæt}} \\\hline
\multicolumn{9}{|c|}{Med syntaksanalyse}\\\hline
Billedstørrelse & 7 & 6 & 5 & 4 & 3 & 2 & 1 & 0\\\hline
$5 \times 5$ & 0\% & 0\% & 0\% & 0\% & 0\% & 0\% & 0\% & 0\% \\\hline
$10 \times 10$ & 0\% & 0\% & 0\% & 0\% & 0\% & 0\% & 0\% & 0\%\\\hline
$20 \times 20$ & 0\% & 0\% & 0\% & 0\% & 0\% & 0\% & 0\% & 0\%\\\hline 
\multicolumn{9}{|c|}{Uden syntaksanalyse}\\\hline
Billedstørrelse & 7 & 6 & 5 & 4 & 3 & 2 & 1 & 0\\\hline
$5 \times 5$ & 0\% & 0\% & 0\% & 0\% & 0\% & 0\% & 0\% & 0\% \\\hline
$10 \times 10$ & 0\% & 0\% & 0\% & 0\% & 0\% & 0\% & 0\% & 0\%\\\hline
$20 \times 20$ & 0\% & 0\% & 0\% & 0\% & 0\% & 0\% & 0\% & 0\%\\\hline \end{tabular}

\begin{tabular}{|l|c|c|c|c|c|c|c|c|}\hline
\rowcolor[gray]{0.9} \multicolumn{9}{|>{\columncolor[gray]{0.9}}c|}{\textbf{Kontrolsæt}} \\\hline
\multicolumn{9}{|c|}{Med syntaksanalyse}\\\hline
Billedstørrelse & 7 & 6 & 5 & 4 & 3 & 2 & 1 & 0\\\hline
$5 \times 5$ & 0\% & 0\% & 0\% & 0\% & 0\% & 0\% & 0\% & 0\% \\\hline
$10 \times 10$ & 0\% & 0\% & 0\% & 0\% & 0\% & 0\% & 0\% & 0\%\\\hline
$20 \times 20$ & 0\% & 0\% & 0\% & 0\% & 0\% & 0\% & 0\% & 0\%\\\hline 
\multicolumn{9}{|c|}{Uden syntaksanalyse}\\\hline
Billedstørrelse & 7 & 6 & 5 & 4 & 3 & 2 & 1 & 0\\\hline
$5 \times 5$ & 0\% & 0\% & 0\% & 0\% & 0\% & 0\% & 0\% & 0\% \\\hline
$10 \times 10$ & 0\% & 0\% & 0\% & 0\% & 0\% & 0\% & 0\% & 0\%\\\hline
$20 \times 20$ & 0\% & 0\% & 0\% & 0\% & 0\% & 0\% & 0\% & 0\%\\\hline \end{tabular}

\begin{tabular}{|l|c|c|c|}\hline
\rowcolor[gray]{0.9} \multicolumn{4}{|>{\columncolor[gray]{0.9}}c|}{\textbf{Træningssæt}} \\ \hline
Billedstørrelse & Alle tegn & Bogstaver & Tal \\\hline
$5 \times 5$ & 0\% & 0\% & 0\% \\\hline
$10 \times 10$ & 0\% & 0\% & 0\%\\\hline
$20 \times 20$ & 0\% & 0\% & 0\%\\\hline \end{tabular}

\begin{tabular}{|l|c|c|c|}\hline
\rowcolor[gray]{0.9} \multicolumn{4}{|>{\columncolor[gray]{0.9}}c|}{\textbf{Kontrolsæt}} \\ \hline
Billedstørrelse & Alle tegn & Bogstaver & Tal \\\hline
$5 \times 5$ & 0\% & 0\% & 0\% \\\hline
$10 \times 10$ & 0\% & 0\% & 0\%\\\hline
$20 \times 20$ & 0\% & 0\% & 0\%\\\hline \end{tabular}

\subsubsection*{And-billeder}

\begin{tabular}{|l|c|c|c|c|c|c|c|c|}\hline
\rowcolor[gray]{0.9} \multicolumn{9}{|>{\columncolor[gray]{0.9}}c|}{\textbf{Træningssæt}} \\\hline
\multicolumn{9}{|c|}{Med syntaksanalyse}\\\hline
Billedstørrelse & 7 & 6 & 5 & 4 & 3 & 2 & 1 & 0\\\hline
$5 \times 5$ & 0\% & 0\% & 0\% & 0\% & 0\% & 0\% & 0\% & 0\% \\\hline
$10 \times 10$ & 0\% & 0\% & 0\% & 0\% & 0\% & 0\% & 0\% & 0\%\\\hline
$20 \times 20$ & 0\% & 0\% & 0\% & 0\% & 0\% & 0\% & 0\% & 0\%\\\hline 
\multicolumn{9}{|c|}{Uden syntaksanalyse}\\\hline
Billedstørrelse & 7 & 6 & 5 & 4 & 3 & 2 & 1 & 0\\\hline
$5 \times 5$ & 0\% & 0\% & 0\% & 0\% & 0\% & 0\% & 0\% & 0\% \\\hline
$10 \times 10$ & 0\% & 0\% & 0\% & 0\% & 0\% & 0\% & 0\% & 0\%\\\hline
$20 \times 20$ & 0\% & 0\% & 0\% & 0\% & 0\% & 0\% & 0\% & 0\%\\\hline \end{tabular}

\begin{tabular}{|l|c|c|c|c|c|c|c|c|}\hline
\rowcolor[gray]{0.9} \multicolumn{9}{|>{\columncolor[gray]{0.9}}c|}{\textbf{Kontrolsæt}} \\\hline
\multicolumn{9}{|c|}{Med syntaksanalyse}\\\hline
Billedstørrelse & 7 & 6 & 5 & 4 & 3 & 2 & 1 & 0\\\hline
$5 \times 5$ & 0\% & 0\% & 0\% & 0\% & 0\% & 0\% & 0\% & 0\% \\\hline
$10 \times 10$ & 0\% & 0\% & 0\% & 0\% & 0\% & 0\% & 0\% & 0\%\\\hline
$20 \times 20$ & 0\% & 0\% & 0\% & 0\% & 0\% & 0\% & 0\% & 0\%\\\hline 
\multicolumn{9}{|c|}{Uden syntaksanalyse}\\\hline
Billedstørrelse & 7 & 6 & 5 & 4 & 3 & 2 & 1 & 0\\\hline
$5 \times 5$ & 0\% & 0\% & 0\% & 0\% & 0\% & 0\% & 0\% & 0\% \\\hline
$10 \times 10$ & 0\% & 0\% & 0\% & 0\% & 0\% & 0\% & 0\% & 0\%\\\hline
$20 \times 20$ & 0\% & 0\% & 0\% & 0\% & 0\% & 0\% & 0\% & 0\%\\\hline \end{tabular}

\begin{tabular}{|l|c|c|c|}\hline
\rowcolor[gray]{0.9} \multicolumn{4}{|>{\columncolor[gray]{0.9}}c|}{\textbf{Træningssæt}} \\ \hline
Billedstørrelse & Alle tegn & Bogstaver & Tal \\\hline
$5 \times 5$ & 0\% & 0\% & 0\% \\\hline
$10 \times 10$ & 0\% & 0\% & 0\%\\\hline
$20 \times 20$ & 0\% & 0\% & 0\%\\\hline \end{tabular}

\begin{tabular}{|l|c|c|c|}\hline
\rowcolor[gray]{0.9} \multicolumn{4}{|>{\columncolor[gray]{0.9}}c|}{\textbf{Kontrolsæt}} \\ \hline
Billedstørrelse & Alle tegn & Bogstaver & Tal \\\hline
$5 \times 5$ & 0\% & 0\% & 0\% \\\hline
$10 \times 10$ & 0\% & 0\% & 0\%\\\hline
$20 \times 20$ & 0\% & 0\% & 0\%\\\hline \end{tabular}

\end{comment} %%%%%%%% COMMENT END


Ovenstående med og uden syntaksanalyse? kun med? kun uden?

%Søren: Ved klassifikation bruges en eller anden grænseværdi som bestemmer hvad et element skal klassificeres som: denne grænseværdi kan reguleres og kan derfor testes.

Delkonklusion:

Vi kunne også have testet hvilke tegn der bliver valgt i stedet for, når et forkert tegn vælges. Vælges 8 f.eks. når B burde ha været valgt??

Billederne af tegnene er ret store i forhold til dem der bruges i litteraturen. Dette giver vores system en væsentlig fordel i forhold til "de andre".

\subsection{Det samlede system}

Her vil vi afprøve det samlede system... eller har vi gjort det "på vejen"?
KUN MED DEN BEDSTE KONFIGURATION:

pcT. AF ALLE TEGN
PCT. AF ALLE PLADER

\subsection{Sammenligning med andres resultater}

MÅSKE SKAL DETTE STÅ I KONKLUSION? ELLERS HER
