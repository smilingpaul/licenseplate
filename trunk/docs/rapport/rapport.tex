\documentclass[11pt,a4paper,final]{article}
\usepackage[utf8]{inputenc}
%\usepackage{ucs}
\usepackage[danish]{babel}
\usepackage{amsmath}
\usepackage{amsfonts}
\usepackage{amssymb}
\usepackage{verbatim}
\usepackage{listings} % Better source code listings
%\usepackage{hyperref}
\usepackage{fixme}
\usepackage[danish]{varioref}
\usepackage{graphicx}

\lstset{ %
language=Matlab,                % choose the language of the code
basicstyle=\scriptsize,       % the size of the fonts that are used for the code
numbers=left,                   % where to put the line-numbers
numberstyle=\scriptsize,      % the size of the fonts that are used for the line-numbers
stepnumber=5,                   % the step between two line-numbers. If it's 1 each line will be numbered
%numbersep=5pt,                  % how far the line-numbers are from the code
%backgroundcolor=\color{white},  % choose the background color. You must add \usepackage{color}
%showspaces=false,               % show spaces adding particular underscores
%showstringspaces=false,         % underline spaces within strings
%showtabs=false,                 % show tabs within strings adding particular underscores
%frame=single,                   % adds a frame around the code
%tabsize=2,                      % sets default tabsize to 2 spaces
%captionpos=b,                   % sets the caption-position to bottom
%breaklines=true,                % sets automatic line breaking
%breakatwhitespace=false,        % sets if automatic breaks should only happen at whitespace
%escapeinside={\%*}{*)}          % if you want to add a comment within your code
}



\author{Tobias Balle-Petersen og Esben Paul Bugge}
\title{Automatisk identifikation og læsning af nummerplader}

\parindent=0pt
\parskip=10pt
\usepackage[top=3cm, bottom=2.5cm, left=4cm, right=4cm]{geometry} 

%\lstset{language=python}
%\lstset{inputencoding=latin1}
%\lstset{extendedchars=true}
%\lstset{breaklines=true}
%\lstset{commentstyle=\textit}
%\lstset{showstringspaces=false}
%lstset{numbers=left, numberstyle=\tiny, stepnumber=2, numbersep=5pt,tabsize=3,basicstyle=\small}
%\lstset{numbers=left, numberstyle=\tiny, stepnumber=2, numbersep=5pt,stringstyle=\ttfamily, showstringspaces=false, basicstyle=\small, language={python}}

\hyphenation{ar-bej-de psy-ko-lo-gi-ske over-blik brug-en skit-se-re com-pu-ter reg-ning ud-ar-bej-de des-ud-en mu-lig-hed-en fjern-be-tje-ning ved-kom-men-de vur-de-ring ar-bejds-op-ga-ve knap bru-ger-e-va-lu-e-ring-en be-skre-vet vand-ret-te or-ga-ni-se-res hin-an-den pla-ce-ret han-dels-ud-dan-nel-se per-so-nen mang-len-de punkt-er-nes valg-mu-lig-he-der klas-sisk bru-ger-e-va-lu-e-ring op-ret-te ka-te-go-ri-en op-le-ve ge-stalt-psy-ko-lo-gi-ens af-snit knap-pen pro-blem-stil-ling-er pro-ble-mer fi-gu-ren eks-pe-ri-men-talt del-tag-el-se na-tur-vid-en-ska-be-lig med-ar-bej-de-re num-mer-plad-er par-ke-rings-an-læg kend-te sy-ste-met gen-kend-el-se bil-led-er-ne fo-to-gra-fi ma-te-ri-a-le le-ve-rer U-ni-ver-si-tet Der-i-mod mi-ni-mum num-mer-plad-er-ne lang-somt svar-en-de e-le-men-tæ-re a-na-ly-se-re bil-led-be-hand-ling pixels tids-plan}
	

\begin{document}

\maketitle
%Billedfilerne skal hedde "nummerpladens tekst"."filtype", eksempelvis EF12345.jpg.
\newpage
\tableofcontents
\newpage

%%%%%%%%%%%%%%
% INDLEDNING %
%%%%%%%%%%%%%%
\fixme{titel: genkendelse i stedet for læsning?}
\fixme{kan man bruge dansk tegnopdeling i tex?}
\section{Indledning}

\subsubsection*{TODO}
Overvej at skrive noget om hvordan man udregner gradienter.
Sig noget om at pladerne ofte er forsænkede når de sidder bag på bilerne.

Sproget skal være konsekvent vi gør/der foretages etc.

Sproget: datid/nutid

Ret headers i endelig version.

Sørg for at vi altid kaldet det LOKALISERING af nummerplader. p.t skifter vi mellem lokalisering og identifikation. OGSÅ I TITLEN!


I afsnit om metoder til lokalisering er det ikke konsekvent hvor det siges at metoden kommer fra litteraturen.

Nævn i system/implementation hvorfor vi har fravalgt nogle metoder som vi faktisk har implementeret

Indsæt figur som tidligere manglede i svn (implemenation),

\subsubsection*{Formalia}

TITEL: "GENKENDELSE" I STEDET FOR "LÆSNING"?
KAN MAN BRUGE DANSK TEGNOPDELING I TEX?

Denne rapport beskriver et system til automatisk identifikation og læsning af danske nummerplader. Systemet er udviklet som bachelorprojekt på Københavns Universitet i 2008.

Ved arbejdet med denne opgave har vi ønsket at få praktisk erfaring med grundlæggende metoder og teknikker indenfor billedbehandling og mønstergenkendelse som anvendt i et system til nummerpladegenkendelse. Findes der etablerede metoder? Kan vi selv udvikle teknikker der fungerer tilfredsstillende? I hvor høj grad vil et system vi selv bygger kunne identificere nummerplader og læse deres indhold?
OVENSTÅENDE ER LÆRINGSMÅL. ER DET FORMALIA?

Systemet er udviklet i Matlab. Vi har ikke haft nogen ambitioner om at kunne læse nummerplader i realtid, og har ikke foretaget os noget væsentligt for at optimere hastigheden på systemet.

HVEM ER LÆSEREN?

\subsubsection*{Baggrund}
I mange sammenhænge er det interessant, at have adgang til et system der automatisk kan genkende et køretøj ved at læse dets nummerplade. Et af de mest kendte områder hvor et sådant system er relevant, er registrering af køretøjer der overskrider hastighedsbegrænsninger. Et andet område er roadpricing. Et eksempel er det system der varetager at der skal betales en afgift når man kører ind i Stockholms indre by. Ved de veje der fører ind til midtbyen er der placeret kameraer der registrerer de kørertøjer der kører forbi og trækker (ER DET MED KAMERA??).

Skriv noget om england og folks bemymringer.
%On March 11, 2008, the Federal Constitutional Court of Germany ruled that the laws permitting the use of automated number plate recognition systems in Germany violated the right to privacy.

\subsubsection*{Rapportens opbygning}
Vi begynder i det følgende afsnit med at beskrive den type data vi arbejder med. Vi beskriver hvordan vi har delt materialet i forskellige sæt samt hvilke afgrænsninger vi har foretaget. Derefter følger afsnittet \textit{Vores system} hvor vi beskriver det system vi har udviklet. Vi forsøger at holde os fra tekniske detaljer og fokuserer på den mere intuitive forståelse af systemets virkemåde. I afsnittet \textit{Implementation} beskriver vi vores kildekodes virkemåde og uddyber emner som vi kun har forsøgt at give en intuitiv forståelse af i afsnittet \textit{Vores system}. Efter at have beskrevet systemet, præsenterer vi resultaterne af afprøvning af systemet i afsnittet \textit{Resultater}.

SKRIV OM APPENDIKS.

\section{Hvilken type inddata arbejder vi med?}
\label{sec:data}
Et system der skal genkende nummerplader har brug for at kunne observere køretøjer. Dette sker via stillbilleder eller videooptagelser. I denne opgave har vi valgt at arbejde med stilbilleder.

\subsection{Afgrænsninger}
For at begrænse opgavens omfang, har vi valgt en række afgrænsninger for de billeder vi ønsker at arbejde med. I det følgende beskriver vi de valg vi har truffet og baggrunden for dem.

\paragraph{Bilerne på billederne er ikke i bevægelse:}
For at forsøge at få så skarpe billeder som muligt, har vi valgt kun at tage billeder af parkerede biler. Vi vil dog stadig kunne få uskarpe billeder i situationer hvor kameraet ikke har stillet skarpt på køretøjet. Vi har taget alle billeder med autofokus, og har således ikke selv foretaget nogle valg i den forbindelse.

\paragraph{Alle billeder indeholder netop en nummerplade:}
Før at gøre opgaven begrænset, har vi valgt kun at kigge på billeder der indeholder en enkelt nummerplade. Hvis vi ved at et billede med garanti indeholder én nummerplade, kan vi kigge efter det område der ligner en nummerplade mest og være ret sikre på at dette område virkelig er en nummerplade. Vi behøver altså ikke vurdere, om det område der ligner mest ligner så meget at vi mener det er en nummerplade.

\paragraph{Amindelige danske nummerplader:}
For at kunne nøjes med at kigge efter en type nummerplader, har vi valgt kun at arbejde med de almindelige hvide nummerplader med en linie sort tekst på hvid baggrund og rød kant som vist i figur \vref{fig:typisk_nummerplade}. Endvidere arbejder vi ikke med de såkaldte personlige nummerplader hvor man selv kan bestemme teksten på nummerpladen. På de numerplader vi arbejder med, består teksten altså af syv tegn hvoraf de to første er bogstaver og de fem sidste er cifre mellem.

\begin{figure}[htp]
\centering
\framebox{\includegraphics[width=5cm]{illu/plate.jpg}} 
\caption{En illustration af den type nummerplader vi ønsker at genkende.}
\label{fig:typisk_nummerplade}
\end{figure}

\paragraph{Vandrette nummerplader:}
Vores billeder er taget så nummerpladerne er roteret i begrænset omfang. Denne begrænsning gør, at vi kan antage at de områder der indeholder nummerplader har et forhold mellem bredde og højde der ligger forholdsvis tæt på danske nummerplades officielle bredde/højde-forhold.

\paragraph{Ingen perspektivisk forvrængning:}
Vi har valgt at arbejde med billeder hvor den perspektivisk forvrængning af nummerpladerne er minimal. Uden væsentlig skævvridning af nummerpladerne, vil pladernes tegn stort set være lige høje og brede og dermed nemmere at læse. Vi har derfor bestræbt os på at tage vore billeder fra en position umiddelbart foran eller bagved den bil vi fotograferer. Denne afgrænsning gør, at vi ikke arbejder med opretning af perspektiviskt forvrængning\footnote{Affin transformation} i denne opgave. 
%Endvidere vil tegn stå på en vandret linie

\paragraph{Ensartede størrelser af nummerplader:}
For at kunne antage noget om størrelserne på de nummerplader vi ønsker at kunne leder efter i billederne, har vi taget alle billeder fra en afstand på mellem to og fire meter uden brug af zoomfunktion. Med denne afgrænsning, sikrer vi os også at tegnene på nummerpladerne aldrig bliver så små at vi må opgive at genkende dem.

\paragraph{Højtopløste billeder:}
Vores billeder er alle taget i en opløsning på $1024 \times 768$ pixels. Gode originalbilleder giver gode muligheder for at eksperimentere med billeder i forskellige opløsninger. 

\paragraph{Intet kunstigt lys:}
Vi har valgt at tage alle billeder uden brug af kunstigt lys som f.eks. blitz. Det er vores ønske at arbejde med billeder taget under forskellige lysforhold så nummerpladerne kan ligge helt eller delvist i skygge.  

Figur \vref{fig:typisk_billede} viser et eksempel på den type billeder vi arbejder med.

\begin{figure}[htp]
\centering
\framebox{\includegraphics[width=7cm]{illu/B_XC33139.jpg}} 
\caption{Et eksempel på den type billeder vi arbejder med. Nummerpladen er stort set vandret og den perspektivistiske forvrængning er minimal.}
\label{fig:typisk_billede}
\end{figure}

\subsection{To sæt billeder}
Vi har delt vores billeder op i to sæt. Det ene sæt, bestående af 400 billeder, kalder vi for vores \textit{træningsæt}. Det er billeder i dette sæt vi har analyseret under arbejdet med at udvikle vores software. F.eks. har vi på baggrund af dette sæt observeret nummerpladernes størrelser så vi har kunnet vurdere nummerpladestørrelser i den type billeder vi arbejder med.

Vores andet sæt, \textit{kontrolsættet}, består af 600 billeder. Dette sæt har vi brugt til at foretage en form for kontrol af vores færdige system. Det er vores tanke at denne kontrol vil kunne sige noget om i hvor høj grad vi har formået at lave et system der fungerer på ukendte billeder der overholder vores  tidligere beskrevne afgrænsninger. 

For at kunne automatisere afprøvning af vores system, har vi navngivet vores billedfiler så vi kan aflæse nummerpladens koordinater samt nummerpladens tekst. 

TAG 50 BILLEDER AF HVIDE PLADER M. BILTZ
TAG 50 BILLEDER AF GULE PLADER U. BILTZ


%Billederne blev navngivet i vores database, så det, i deres filnavn, indgik om billedet forestillede en bil set forfra eller bagfra samt hvilken nummerplade bilen på nummerpladen havde. Derudover udarbejdede vi et mindre program som hjalp os til at identificere nummerpladens fire hjørnekoordinater og indskrive disse i filnavnet. Denne sidste tilføjelse ville hjælpe os i testfasen, til at undersøge om de nummerpladekandidater vores system ville udvælge er de korrekte.

\begin{comment}
I mange sammenhænge vil det være relevant at automatisk kunne identificere og genkende køretøjers nummerplader. Et sådant system kunne eksempelvis være et system som bruges ved et parkeringsanlæg, hvor der f.eks. tages billeder af de biler der ankommer til anlægget, så man kan registrere hvilke biler der befinder sig på anlægget. Et sådant system kunne også bruges af parkeringsvagter, som eksempelvis manuelt tager billeder af parkerede biler hvorefter systemet identificere bilen.

I dette projekt vil vi arbejde med netop dette emne: automatisk identificering og genkendelse af nummerplader i billeder ved hjælp af et computersystem. Det system, vi ønsker at udvikle og afprøve, egner sig bedst til en situation, hvor bilen står stille eller bevæger sig meget langsomt.

Dette projekt laves som bachelorprojekt på Københavns Universitet. Vores forventning er ikke at systemet kan opnå en effektivitet svarende til etablerede systemer til genkendelse af nummerplader. Derimod ønsker vi via arbejdet at få erfaring med praktisk anvendelse af elementære teknikker indenfor billedbehandling og mønstergenkendelse.


\subsection{Problemformulering}
Opgavens problemformulering blev som følger:

\fixme{Skal det være spørgsmål}
\begin{itemize}
\item[-] Hvordan kan nummerplader på farvefotografier identificeres og læses af et computersystem?
\item[-] Hvilke kendte metoder til identifikation og læsning af nummerplader findes der?
\item[-] Hvor høj genkendelsesprocent kan et system, vi selv laver, opnå?
\end{itemize}
%\fixme{Søren: opdel sidste spørgsmål i flere}

\subsection{Afgrænsning}

\fixme{Søren: ikke sigende titel}
\fixme{Mere argumentation i dette afsnit}
\fixme{VI behøver ikke at lave afin transformation.}


På grund af projektets omfang var det nødvendigt at lave visse afgrænsninger. Disse afgrænsninger blev som følger:

\fixme{Bør ikke være punktopstilling når den er så lang}
\begin{itemize}
\item[-] Billederne skulle tages ved højlys dag uden brug af kunstig lys. På denne måde vurderede vi at man ville opnå fotografier hvor nummerpladerne havde nogenlunde ens farvemønstre. Ved brug af kunstigt lys ville man risikere at nummerpladen ville fremstå med en unartulig farve eller at det reflekterende materiale, som nummerplader er lavet af, ville skabe genskin og på denne måde unaturlige farver. \fixme{forklaring på farvemønster}
\item[-] Hvert fotografi skulle forestille en bil med netop én synlig nummerplade.
\item[-] Fotografierne skulle tages fra en position direkte foran bilen og i en højde på mellem 160 og 190 cm i en afstand på 3-6 meter fra bilen uden zoom. På denne måde må det forventes at nummerpladen bliver tilpas stor i billedet til at computersystem kan finde den, samt at man med det menneskelige øje kan aflæse nummerpladen udfra fotografierne. \fixme{evt. figur af hvordan vi stod da vi fotograferede}
\item[-] Nummerpladen skulle ikke nødvendigvis befinde sig midt i billedet. \fixme{midt i billede: er den en afgrænsning?} %Dette ville skabe lidt udfordring for vores system, som dermed ikke nødvendigvis kan udelukke objekter som ikke er i midten af fotografiet.
\item[-] Nummerpladen skulle fremstå vandret i billedet og skulle desuden være fri for urenheder eller lignende, der gjorde pladen sværere at læse. \fixme{hvad vil det sige at nummerpladen er vandret}
\item[-] Nummerpladerne skal alle være danske privatplader som de udstedtes i foråret 2008\footnote{Rød kant, hvid baggrund og sort tekst.} Formatet på nummerpladerne skal være \textit{AA XX XXX}, hvor \textit{A} er bogstaver i mængden \textit{A}-\textit{Z}\footnote{Enkelte tegn er muligvis ikke lovlige. Er dette tilfældet, skal systemet ikke tage højde for dem.}, og \textit{X} er heltal i mængden 0-9. Som eksemplet viser, skal der være mellemrum mellem de to bogstaver og de to første tal samt mellem de to første tal og de tre tal. Nummerpladerne må ikke indeholde andre tegn end de nævnte bogstaver og tal.
\end{itemize}

\fixme{har vi udelukket kvadratiske nummerplader?}





\subsection{Metoder fra litteraturen}

\fixme{Hvilken litteratur har vi og vi har vi den fra? Let at finde metoder? Hvorfor har vi kun beskrevet disse metoder? Mangler henvisninger til teoriafsnit. Argumentér for hvorfor der er fire elementer. Hvorfor er rotation et element når vi har vandrette billeder.}

I forbindelse med dette projekt har vi identificeret følgende fire dele, som et system der skal kunne identificere og aflæse nummerplade kunne indeholde: identificering af nummerpladen i billedet, eventuel rotation af nummerpladen, hvis denne ikke er vandret i billedet, opdeling af tegnene i nummerpladen samt aflæsning af de enkelte tegn. De første tre elementer har vi defineret som værende billedbehandling mens det fjerde element er mønstergenkendelse. Metoderne til billedbehandlingsdelene diskuteres i afsnit \ref{sec_billed} mens metoderne til mønstergenkendelse diskuteres i afsnit \ref{sec_monster}.

\end{comment}





%%%%%%%%%%%%%%%%%%%%%%%
% OPBYGNING AF SYSTEM %
%%%%%%%%%%%%%%%%%%%%%%%
\section{Vores system}
I vores system har vi opdelt arbejdet med at læse nummerplader i tre dele. Først forsøger vi at lokalisere nummerpladen i billedet. Derefter roterer vi nummerpladen til vandret position og prøver at separere tegnene. Til sidst forsøger vi at genkende nummerpladens bogstaver og tal. Figur \ref{fig:system_overblik} viser systemet som et diagram med data markeret af rektangler og databehandling markeret med ruder.

\begin{figure}[htp]
\centering
\includegraphics[width=10cm]{system/illu/overordnet_system.png} 
\caption{Vores system som diagram}
\label{fig:system_overblik}
\end{figure}

%%%%%%%%%%%%%%%%%%%%%%%%%%%%%%%%%%%%
% Hvordan lokaliserer vi pladerne? %
%%%%%%%%%%%%%%%%%%%%%%%%%%%%%%%%%%%%
%\section{Billedbehandling}


\subsubsection*{Noter fra møde med Søren 20/2:}
identifikation: Se på en pixel, har naboer en kontrast farve?
En scan-linie: hvordan varierer kontrasten henover linien?
Adaboost - godt

\subsubsection*{Matriculas2003}
Bruger kun gråtone info. Arbejder også med nummerplader som skal være læsbar for det menneskelige øje.

Metode:
Histogram - først normaliseres billedet
Sobel filter - fremhæver ikke-homogene områder
"A simple threshold and a sub-sampling" bruges til at vælge områder der kan være nummerpladen

Husker alle områder som kan være nummerplader så de forkerte først vælges fra i genkendelses-fasen. Bruger multi-hypothesis detection (ikke forklaret yderligere i teksten).

Feature vektorer: hver pixel i et træningsbilleder er blevet klassificeret som positiv (del af nummerplade) eller negativ (ikke del af nummerplade). Minimerer efterfølgende det negative sæt.

Bruger kd-træ data struktur og en "omtrent nærmeste nabo" søgeteknik.

\subsubsection*{A Real-time vehicle License Plate Recognition (LPR) System på http://visl.technion.ac.il/projects/2003w24/}

* Find de gule (hos os: hvide) områder i billedet
* Forstør disse områder
* Find vinklen på nummerpladen ved brug af "Radon transform"
* Justering af nummerpladens konturer
* Unødvendige dele af billedet fjernes (kun nummerpladen tilbage)
* Billedet i gråtone, herefter gøres det binært
* Billedet normaliseres
* Tegn-inddeling vha. peak-to-valley

De brugte Matlab. De havde følgende relevante problemer og løsningsforslag til disse:

* Udtrækning af det gule område giver ofte fejl. Man kunne supplere denne udtrækning med en algoritme der indberegner at nummerplader har en klar signatur idet der er stærke grå-tone variationer i regulære intervaller (henover nummerpladen, mener de vel?)

* Hvis der er flere nummerplade-kandidater i billedet skal hver af dem testes.

\subsubsection*{LicenseplateSydney.pdf}

Bruger regler for nummerplader i systemet

    * Starter med kontrast “udstrækning”, bortfiltrering af støj (der tages selvfølgelig højde for billedkvaliteten i denne del)
    * Lokalisering af nummerpladen: fuzzy clustering algoritme som bruger karakteristikker som “gul-hed” og teksturer
    * Gul-hed er defineret af frekvenstabel lavet fra manuelt udklippede nummerplader
    * Tekstur: her ser man på grå-værdien af de 8 nabopixels
    * “global threshold” baseret på gennemsnitsværdien af gråtone: fås binært billede
    * Udfra regler (højde, bredde m.m) findes potintielle tegn
    * Nummerpladen gives kun videre til mønstergenkendelse hvis den indeholder det rette antal tegn

Dette system godkender 75\% af billederne. Problemer med skruer i nummerpladen. Problemer med global threshold – burde gøres på pr-tegn-basis.

\subsubsection*{licence-plate-1996.pdf}

Der bruges en algoritme der først lokaliserer elementer der kan være tegn/bogstaver hvorefter den udvælger et område som nummerplade

* Konverter til gråtone billede
* 5x5 filter, fjerne støj
* Find kanter i billedet vha. Shen-Castan kant-detektor
* Gør billedet binært og del elementer i forgrunden fra hinanden
* Algoritme der finder bogstaver på baggrund af forskellen i gråtone værdien af bogstav/baggrund
* Områder hvor der ikke er (det rette antal) bogstaver udelukkes
* For at finde nummerplade bruges genetisk algoritme der bedømmer rektangler med tegn: har de den rette størrelse? er bogstaverne korrekt placeret i rektangel? osv.
* Algoritmen vægter hvert område og filtrerer til sidst i disse områder udfra deres vægt

Stort problem: svært at finde tegn.

\subsubsection{kwasnickawawrzyniak}
Billedet skiftes til farverummet YUV fra vilket luminans er det eneste der bevares. Herefter normaliseres billedet (Hele den diskerete "range" udnyttes). Kigger på skift i kontrast. Gælder alle nummerplader. Finder alle tekster. Den rigtige skal vælges.

Identifikation af plader:

1. "Connected components analysis" (der kigger på et binært billede?) vælger områder med høj kontrast (threshold). De fundne områder undersøges og områder elimineres efter regler i pdf.  Herefter, er der lignenede grupper i nærheden af funden gruppe? Måske er der en serie tegn dvs. en sætning = plade.

2. Searching for signatures of license plates. Et karakteristisk skift i luminans i en linie i billedet.

Potentielle plader roteres så de er vandrette.

Segmentering af tegn:
Scan af peaks og valleys samt analyse af sammenhængende grupper fra identifikationsprocessen sammenlignes og segmentering foretages.

Mønstergenkendelse foregår med neuraltnetværk.

\subsubsection{AdaptiveLicensePlateImageExtraction.pdf}
For at sætte hastigheden op foretages visse operationer på kraftigt nedsamplede billeder. Finder lodrette linier. Bruger Robert's edge detector til at fremhæve dem (Tegner den på billedet?). Dette efterlader en masse lodrette linier i området med nummerpladen. Et Rank filter? bruges på billedet. Efterlader en lys elipse i det område hvor pladen findes. Scanner billedet lodret for at finde det lyseste område og klipper ud (Klipper et noget større område end pladen ud på eksemplet i pdf'en). Der er formler i beskrivelsen. Roter pladen hvis skæv (Formler i pdf). Nohet med at finde linier i billedet og rotere stlsvarende (Hough og Radon transform). Det er først når vi skal genkende tegnene på pladen vi bruger billedet i sin originale opløsning.



%%%%%%%%%%%%%%%%%%%%%%%%%%%%%%%%%
% Hvordan separerer vi tegnene? %
%%%%%%%%%%%%%%%%%%%%%%%%%%%%%%%%%
\subsection{Separation af tegn}

Separation af tegnene i en nummerplade system består i vores system af to dele: Først må billedet indeholdende nummerpladen roteres så nummerpladen står vandret i billedet, dernæst skal tegnene fra nummerpladen identificeres og separeres.

\fixme{indsæt separations-del af diagram over system}

Rotationsmetoden i dette afsnit returnerer orginalbilledet af en bil med nummerplade mens separationsmetoderne returnerer syv binære billeder. Hvert billede forestiller et tegn fra en nummerplade, skåret ud fra pladen og omdannet til et binært billede med selve tegnet som forgrund og nummerpladens hvide område som baggrund.

%\fixme{Hvad beskriver vi i dette afsnit? Hvilke trin udgør processen?}

\subsubsection{Rotation}
%Nogle overvejelser om hvordan billedet skal være klippet til for at rotationen fungerer ordentligt...
I dette afsnit beskrives det hvordan et billede af en nummerplade kan roteres, så nummerpladen optræder vandret i billedet. Metoden er fundet i...


\subsubsection*{Radon transformation}
Et billede af en nummerplade, hvor nummerpladen er let fordrejet i forhold til vandret, skal drejes før separation og genkendelse af tegn kan foregå. En Radon transformation beregner projektionen af et billede fra flere forskellige steder og i flere forskellige retninger/vinkler. Denne type transformation kan bruges til at finde linier i billedet, og til at undersøge i hvilken retning disse linier går. Disse oplysninger er brugbare mht. rotation af et billede af en nummerplade, så denne plade står vandret i billedet.

Billedet nedenfor viser hvordan Radon transformationen findes i en enkelt vinkel, theta. Den grå firkant er billedet der projekteres. De grå pile er sensorer/radial linier. Resultatet af en Radon transformation er en matrice der angiver i hvor høj grad det er sandsynligt at der findes en linie i original billedet for hver vinkel samt radial linie.

%SKAL NOK BYTTES UD MED HJEMMELAVET BILLEDE:
%\begin{figure}[h]
%\includegraphics{billedbehandling/illu/transform74.jpg}
%\caption{Radon transformation}
%\end{figure}

Før Radon transformationen kan foretages skal billedet kant-detekteres (ANDEN OVERSÆTTELSE), da dette giver en større chance for at Radon transformationen opfanger linierne i billedet. Når nummerpladens hældning i billedet er fundet, kan billedet roteres.

I Figur ?? er vist et eksempel på et billede af en nummerplade, et binært billede udfra samme billede hvor kanterne i billedet er markeret med hvid samt det samme billede af nummerpladen, blot roteret. Bemærk at kun de horisontale kanter detekteres i kant-billedet (mere om dette i afsnit ??).

\includegraphics{system/illu/rotate_example_input.jpg}
\includegraphics{system/illu/rotate_example_edge.jpg}
\includegraphics{system/illu/rotate_example_output.jpg}


\subsubsection*{Andre metoder brugt i litteraturen}

\fixme{Hough?}

\subsubsection{Identifikation af tegn}

%Se nrpl.dk: Hvert af de op til 7 tegn på nummerpladen har et imaginært "felt" de kan brede sig i. Ikke alle tegn er lige brede, og generelt er bogstaver bredere end tal. "Feltet" til bogstaver er derfor bredere end feltet til tegn. Enkelte tegn er for smalle til at udfylde deres "felt", så designeren har fundet det hensigtsmæssigt at placere disse tegn visuelt centreret inden for deres "felt".

%"Således vil en nummerplade som MB 20 001 på grund af venstrestillingen af det sidste 1-tal have større mellemrum mellem højre kant og sidste tal end mellem venstre kant og første bogstav"


I det følgende beskrives to metoder til identifikation og separation af tegn i en nummerplade. Disse to metoder kaldes Sammenhængende komponenter og Peak-to-valley (da.: top-til-bund). Metoderne er fundet i....



\subsubsection*{Sammenhængende komponenter}
Denne metode er bygget op omkring sammenhængende komponenter. Ideén bygger på at hvert enkelt af de syv tegn som findes i billedet af nummerpladen er é sammenhængende komponent. I dette tilfælde vil hver sammenhængende komponent bestå af mørke pixels mens baggrunden er lyse pixels.

Med sandsynlighed vil tegnene ikke være de eneste sammenhængende komponenter i billedet. Eksempelvis kan nummerpladen være beskidt, lyset på billedet kan være dårligt, billedet kan være klippet 'for løst' sådan at elementer uden for nummerpladen vil være sammenhængende etc. For at tegnene skal udskille sig klart fra den lyse baggrund er det nødvendigt at forstærke kontrasten i billedet. Figur ? viser forskellen på et billede af en nummerplade uden forarbejdning og et med hvor kontrasten i billedet er forstærket.

EKSEMPEL PÅ BILLEDE MED FORSTÆRKEDE KONTRASTER.

Efterfølgende oprettes et binært billede med sammenhængende komponenter. I Figur ? ses et billede af en nummerplade hvor de sammen hængende komponenter er hvide.

EKSEMPEL PÅ BILLEDE MED SAMMENHÆNGENDE KOMPONENTER

Som det ses er det ikke kun tegnene der er sammenhængende komponenter. Derfor gennemføres en analyse på komponenterne for at frasortere de komponenter der ikke kan være tegn. Vi kan frasortere komponenter som ikke opfylder følgende regler:

\begin{itemize}
\item[-] Hvert tegn fylder maksimalt 1/7 del af billedet.
\item[-] Hvert tegn har en maksimal bredde på 1/7 af billedets bredde.
\item[-] Tegnenes højde er større end deres bredde
\item[-] Tegnenes minimale højde skal være end en vis konstant (her: 5 pixels)
\item[-] Tegnenes minimale størrelse skal være større end en vis konstant (her: 5 pixels)
\item[-] For hvert tegn findes der seks andre tegn som er ca. lige så højde som det pågældende tegn.
\item[-] For hvert tegn findes der seks andre tegn som befinder sig i samme højde som det pågældende.
\item[-] Afstanden mellem tegnene 2 og 3 samt 4 og 5 i en tegnfølge på syv tegn er større end afstanden mellem de resterende tegn der støder op til hinanden.
\end{itemize}

EKSEMPEL PÅ BILLEDE HVOR IKKE-TEGN KOMPONENTER ER SORTERET FRA

De resterende komponenter må forventes at være nummerpladens tegn.

EKSEMPEL PÅ UDKLIPPEDE TEGN

\fixme{beskrevet i kwas: man ku fjerne kanter før analysen}

\subsubsection*{peak-to-valley}

Denne metode baseres på en vertikal projektion af nummerpladen. Denne projektion vil, med en lys nummerplade med mørke tegn, give os en indikation af hvor der er dale (de lyse områder mellem tegnene) og bakker (tegnene), hvorefter vi kan udskære tegnene.

Som i metoden med sammenhængende komponenter er det en fordel at forstærke kontrasten i billedet. HVORFOR? Nummerpladens signatur fås ved at summere intensiteterne i alle kolonnerne i billedet. En graf over signaturen vil så give toppe i de kolonner hvor der er størst inte

\subsubsection{Metoder fra litteraturen}

Segmentering af nummerpladens tegn kan gøres ved brug af peak-to-valley (dansk: top-til-bund) \cite{ron}, \cite{kwas} eller ved brug af såkaldte sammenhængende komponenter (hvert enkelt tegn i nummerpladen må hver især forventes at være én sammenhængende figur) \cite{nijhuis}. I alle tilfælde er det en god idé at kombinere de enkelte elementer af systemet med viden om størrelsesforhold, antal (én nummerplade pr. billede, syv tegn pr. nummerplade osv.) og lign \cite{nijhuis}, \cite{parker}, \cite{kwas}. Til aflæsning af tegn kan man bl.a. bruge neurale netværk \cite{nijhuis}, \cite{kwas} eller klassifikations vektorer \cite{arth}.
\fixme{Søren: til Radon: han skriver "nej"??}
\fixme{skal vi ha noget om testdata her?}




%%%%%%%%%%%%%%%%%%%%%%%%%%%
% Note fra segmentering af tegn
%%%%%%%%%%%%%%%%%%%%%%%%%%%%


\begin{comment}
\subsubsection*{Skan-linie}

Bruges ikke da det er for mange felter som bliver valgt. Kan måske gøres bedre ved filtrering før???

Først gøres billedet sort-hvis med im2bw. Her kan grænseværdi bestemmes med greythresh. Virker måske bedre at sætte grænseværdien lavt, så meget af billedet bliver hvidt.

Billedet skal skæres foroven og forneden. Dette gøres simpelt ved at finde den største pixl-sum i toppen af billedet og den største sum i bunden. Det antages så at disse max-summer er dele af nummerpladerne hvor teksten ikke er startet.

Step igennem vertikale linier: hvad sker før tegn, i et tegn, i slutningen af et tegn og efter et tegn.
\end{comment}


%%%%%%%%%%%%%%%%%%%%%%%%%%%
% Beskrivelser af PDfer
%%%%%%%%%%%%%%%%%%%%%%%%%%%%


\begin{comment}
\subsubsection*{Noter fra møde med Søren 20/2:}
identifikation: Se på en pixel, har naboer en kontrast farve?
En scan-linie: hvordan varierer kontrasten henover linien?
Adaboost - godt










\subsubsection*{cano.pdf}
Bruger kun gråtone info. Arbejder også med nummerplader som skal være læsbar for det menneskelige øje.

Metode:
Histogram - først normaliseres billedet
Sobel filter - fremhæver ikke-homogene områder
"A simple threshold and a sub-sampling" bruges til at vælge områder der kan være nummerpladen

Husker alle områder som kan være nummerplader så de forkerte først vælges fra i genkendelses-fasen. Bruger multi-hypothesis detection (ikke forklaret yderligere i teksten).

Feature vektorer: hver pixel i et træningsbilleder er blevet klassificeret som positiv (del af nummerplade) eller negativ (ikke del af nummerplade). Minimerer efterfølgende det negative sæt.

Bruger kd-træ data struktur og en "omtrent nærmeste nabo" søgeteknik.

\subsubsection*{ron}

* Find de gule (hos os: hvide) områder i billedet
* Forstør disse områder
* Find vinklen på nummerpladen ved brug af "Radon transform"
* Justering af nummerpladens konturer
* Unødvendige dele af billedet fjernes (kun nummerpladen tilbage)
* Billedet i gråtone, herefter gøres det binært
* Billedet normaliseres
* Tegn-inddeling vha. peak-to-valley

De brugte Matlab. De havde følgende relevante problemer og løsningsforslag til disse:

* Udtrækning af det gule område giver ofte fejl. Man kunne supplere denne udtrækning med en algoritme der indberegner at nummerplader har en klar signatur idet der er stærke grå-tone variationer i regulære intervaller (henover nummerpladen, mener de vel?)

* Hvis der er flere nummerplade-kandidater i billedet skal hver af dem testes.

\subsubsection*{nijhuis.pdf}

Bruger regler for nummerplader i systemet

    * Starter med kontrast “udstrækning”, bortfiltrering af støj (der tages selvfølgelig højde for billedkvaliteten i denne del)
    * Lokalisering af nummerpladen: fuzzy clustering algoritme som bruger karakteristikker som “gul-hed” og teksturer
    * Gul-hed er defineret af frekvenstabel lavet fra manuelt udklippede nummerplader
    * Tekstur: her ser man på grå-værdien af de 8 nabopixels
    * “global threshold” baseret på gennemsnitsværdien af gråtone: fås binært billede
    * Udfra regler (højde, bredde m.m) findes potintielle tegn
    * Nummerpladen gives kun videre til mønstergenkendelse hvis den indeholder det rette antal tegn

Dette system godkender 75\% af billederne. Problemer med skruer i nummerpladen. Problemer med global threshold – burde gøres på pr-tegn-basis.

\subsubsection*{parker.pdf}

Der bruges en algoritme der først lokaliserer elementer der kan være tegn/bogstaver hvorefter den udvælger et område som nummerplade

* Konverter til gråtone billede
* 5x5 filter, fjerne støj
* Find kanter i billedet vha. Shen-Castan kant-detektor
* Gør billedet binært og del elementer i forgrunden fra hinanden
* Algoritme der finder bogstaver på baggrund af forskellen i gråtone værdien af bogstav/baggrund
* Områder hvor der ikke er (det rette antal) bogstaver udelukkes
* For at finde nummerplade bruges genetisk algoritme der bedømmer rektangler med tegn: har de den rette størrelse? er bogstaverne korrekt placeret i rektangel? osv.
* Algoritmen vægter hvert område og filtrerer til sidst i disse områder udfra deres vægt

Stort problem: svært at finde tegn.

\subsubsection{kwas.pdf}
Billedet skiftes til farverummet YUV fra vilket luminans er det eneste der bevares. Herefter normaliseres billedet (Hele den diskerete "range" udnyttes). Kigger på skift i kontrast. Gælder alle nummerplader. Finder alle tekster. Den rigtige skal vælges.

Identifikation af plader:

1. "Connected components analysis" (der kigger på et binært billede?) vælger områder med høj kontrast (threshold). De fundne områder undersøges og områder elimineres efter regler i pdf.  Herefter, er der lignenede grupper i nærheden af funden gruppe? Måske er der en serie tegn dvs. en sætning = plade.

2. Searching for signatures of license plates. Et karakteristisk skift i luminans i en linie i billedet.

Potentielle plader roteres så de er vandrette.

Segmentering af tegn:
Scan af peaks og valleys samt analyse af sammenhængende grupper fra identifikationsprocessen sammenlignes og segmentering foretages.

Mønstergenkendelse foregår med neuraltnetværk.

\subsubsection{shapiro.pdf}
For at sætte hastigheden op foretages visse operationer på kraftigt nedsamplede billeder. Finder lodrette linier. Bruger Robert's edge detector til at fremhæve dem (Tegner den på billedet?). Dette efterlader en masse lodrette linier i området med nummerpladen. Et Rank filter? bruges på billedet. Efterlader en lys elipse i det område hvor pladen findes. Scanner billedet lodret for at finde det lyseste område og klipper ud (Klipper et noget større område end pladen ud på eksemplet i pdf'en). Der er formler i beskrivelsen. Roter pladen hvis skæv (Formler i pdf). Nohet med at finde linier i billedet og rotere stlsvarende (Hough og Radon transform). Det er først når vi skal genkende tegnene på pladen vi bruger billedet i sin originale opløsning.
\end{comment}





%%%%%%%%%%%%%%%%%%%%%%%%%%%%%%%%%
% Hvordan genkender vi tegnene? %
%%%%%%%%%%%%%%%%%%%%%%%%%%%%%%%%%
\subsection{Genkendelse af tegn}

INDSÆT GENKENDELSESDEL AF DIAGRAM OVER SYSTEM
\label{sec_monster}

Til genkendelse af tegn vil vi bruge to forskellige metoder: Den første bruger informationer fra en feature vektor mens den anden analyserer enkelte pixels i hvert tegn-billede. I dette afsnit vil vi beskrive disse to metoder samt en metode til at udføre syntaks analyse på den tegnfølge der repræsenterer et gæt på nummerpladen.

De to metoder til genkendelse af tegn arbejder på de syv binære billeder som separationsdelen af systemet returnerede. Metoderne returnerer syv hitlister (én for hvert billede) med tegn. Det tegn der optræder øverst på den enkelte hitliste er metodens bedste gæt på det tegn billedet forestiller, nummer to på listen er det næstbedste gæt osv.

\subsubsection{Feature vektorer}
En feature vektor er en vektor der... Idéen i denne metode er at der oprettes en feature vektor for hvert muligt tegn der kan forekomme i en nummerplade. For et billede af et tegn oprettes der tillige en vektor, som herefter sammenlignes med tegn-vektorerne for at se hvilken vektor der ligger nærmest billedets vektor. Tegnet for den vektor der ligger nærmest vælges.

MAHAP AFSTAND SKAL I FØLGE SØREN BRUGE MANGE DATA. KAN IKKE HUSKE HVORFOR

EKSEMPEL PÅ VEKTOR?

\subsubsection{Simpel metode hvor træningsættet andes}

\subsubsection{Syntaks analyse}

Ved syntaks analyse analyseres tegn-hitlisterne fra de to ovenstående metoder. Analysen sker på baggrund af at et gæt fra en af de to metoder måske kan udelukkes på baggrund af overtrædelse af syntaktiske regler for hele tegnfølgen. Der kan forekomme følgende fejl:

\begin{itemize}
\item[-] Et tal er placeret på en af de første to pladser
\item[-] Et bogstav er placeret på en af de sidste fem pladser
\item[-] Bogstavkombinationen er ikke tilladt
\item[-] Talkombinationen er ikke tilladt (den samlede værdi af tallene er enten for høj eller for lav
\end{itemize}

Hvis en eller flere af disse muligheder er gældende itererer metoden ned igennem hitlisterne indtil et lovligt valg af tegn er fundet. Dette er illustreret i følgende eksempel: Tegnfølgen \textbf{DO 45 7B3} er retuneret fra mønstergenkendelse. I denne tegnfølge forekommer der to fejl: bogstavet \textbf{O} bruges ikke på 2. position og bogstavet \textbf{B} i 6. position burde have været et tal. På 2. position itererer metoden ned igennem hitlisten til den når et bogstav som sammen med \textbf{D} danner en lovlig bostavkombination og på 6. position itereres der indtil et tal findes.

I systemet sætter vi en øvre grænse for hvor langt ned ad hitlisten syntaks analysen kan vælge tegn. Hvis denne grænse ikke fandtes kunne det blive gætværk...

\subsubsection{Metoder fra litteraturen}





%%%%%%%%%%%%%%%%%%
% IMPLEMENTATION %
%%%%%%%%%%%%%%%%%%
\section{Implementering}
\label{sec:implementation}

I dette afsnit gennemgår vi implementeringssdetaljer i de funktioner som vores system består af. I flere af funktionerne har vi brugt Matlabs indbyggede komandoer, frem for at skrive egne implementationer. F.eks. har vi brugt Matlabs komandoer i forbindelse arbejdet med sammenhængende komponenter og filtrering af bileder. I det første underafsnit gennemgår vi nogle egenudviklede hjælpefunktioner som bruges flere steder i systemet. Dernæst gennemgår vi funktionerne til lokalisering af nummerplader, separation af tegn og til sidst funktionerne til genkendelse af tegn.

ORIGO: ER DET IKKE NÆRMERE AT VI ARBEJDER I ET ANDET KVARTIL(ELLER HVAD DET NU HEDDER).
Det skal bemærkes at koordinatsystemet der bruges i Matlab ikke har origo i det samme punkt som det man kender som et almindeligt koordinatsystem. Origo ligger derimod i øverste venstre hjørne (normalt ligger det i nederste venstre hjørne). Desuden er koordinaterne til origo i Matlab 1,1 og ikke 0,0. y-koordinaten til et punkt i Matlab angives normalt først, så en pixel i et billede refereres til med koordinaterne y,x i modsætning til den oftest anvendte rækkefølge x,y.

%%%%%%%%%%%%%%%%%%%%%%%%
%%% HJÆLPEFUNKTIONER %%%
%%%%%%%%%%%%%%%%%%%%%%%%

\subsection{Hjælpefunktioner}
De følgende funktioner bruges af flere andre funktioner i systemet hvorfor vi har valgt at beskrive dem først. Vi har valgt kun et beskrive de funktioner vi finde særligt interesante.

\subsubsection{Hjælpefunktion 1}
BinImgCleanup
\subsubsection{Hjælpefunktion 2}
GetBestCandidate
\subsubsection{Hjælpefunktion 3}
ContrastStretch
\subsubsection{Hjælpefunktion 4}
WhiteLine 
\subsubsection{Hjælpefunktion 3}
Distribution

%%%%%%%%%%%%%%%%%%%%%%%%%%%%%%%%%%%%
%%% LOKALISERING AF NUMMERPLADER %%%
%%%%%%%%%%%%%%%%%%%%%%%%%%%%%%%%%%%%

\subsection{Lokalisering af nummerplader}
I de følgende afsnit gennemgår vi væsentlige implementeringssdetaljer i de funktioner vi bruger til lokalisering af nummerplader. De tager alle stien til et \textit{JPEG}-billede som inddata og leverer koordinater til en enkelt nummerpladekandidat samt dennes point som uddata. Koordinaterne der beskriver nummerpladekandidatens position i inddata-billedet angives i rækkefølgen: Mindste \textit{x}-værdi, største \textit{x}-værdi, mindste \textit{y}-værdi og største \textit{y}-værdi. Hvis en funktion ikke kan finde en nummerpladekandidat returneres et koordinatsæt hvor alle værdier er $0$.

I metoderne til lokalisering af nummerplader, nedskalerer vi billederne til en fjerdedel. Det vil sige at opløsningen bliver $256 \times 192$ pixels i stedet for $1024 \times 768$ pixels. Den lavere opløsning gør billederne langt hurtigere at arbejde med. Da vi ikke arbejder med anaylse af teksten på nummerpladen i disse metoder, men blot forsøger at udpege områder der indeholder nummerplader, bekymrer vi os ikke om hvorvidt nummerpladens tegn kan læses i billeder med denne lave opløsning.


Ofte udvider vi kandidatområdet før vi returnerer det for at få hele pladen med. Det er vigtigt for at senere at kunne rotere da dette kræver tilstædeværelsen af markante linier der markerer pladens over- og underkant. Vi vil hellere returnere områder der er lidt for store end områder der er lidt for små.% (mere om dette i afsnit \vref{sec:implementation/sep/rotation}).

Når vi omdanner gråtonebilleder til binærebilleder har vi, med mindre andet er beskrevet, brugt Matlabs greythresh-kommando til at udregne den værdi der afgør hvilke pixels der bliver hvide og hvilke der bliver sorte i det binære billede.

\subsubsection{Analyse af interesseområder}
\label{sec:imp:BinImgCleanup}
Analysen af de binære billeder der markerer interesseområder varetages af funktionen \textit{BinImgCleanup} hvis kildekode findes i afsnit \vref{BinImgCleanup}. Denne funktion bliver brugt af alle de fem metoder vi bruger til at lokalisere nummerplader.

Funktionen tager et binært billede samt en skaleringsværdi som inddata, og returner et binært billede hvor de områder der sandsynligvis ikke er nummerpladeområder er fjernet som beskrevet i afsnit \vref{sec:BinImgCleanup}.

Det er nødvendigt at angive en skaleringsværdien da vi i programmet har defineret hvor høje og brede nummerplader maksimalt må være som heltalsvariable. Da disse værdier er baseret på analyse af billeder med en opløsning på $1024 \times 768$ pixels har vi behov for at gøre disse variable mindre når man arbejder på billeder med en lavere opløsning. Hvis man f.eks. giver funktionen et binært billede med en opløsning på $512 \times 384$ skal skaleringsværdien være $0.50$ da billedets højde og bredde er halvt så store, hvorimod den skal være $1.0$ hvis opløsningen er $1024 \times 768$ da variablene ikke skal ændres og derfor ganges med en.

DER SKAL STÅ NOGET OM CONNECTED COMPONENTS!!

\subsubsection{Valg af bedste nummerpladekandidat}
\label{sec:imp:GetBestCandidate}
DEN SKAL JO OGSÅ BESKRIVES FØR METODERNE.
NYT NAVN.


%%%%%%%%%%%%%%%%%%%%%%
\subsubsection{Metode: Områder domineret af lyse gråtoner}
Metoden hvor vi førsøger at finde nummerplader ved at kigge på grå områder i farvebilleder er implementeret i funktionen \textit{DetectSameness} hvis kildekode findes i afsnit \vref{code:DetectSameness}. Funktionens eneste argument er stien til et farvebillede i \textit{JPG}-format i en opløsning på $1024 \times 768$ pixels. Programmet udfører følgende trin:

\paragraph{Klargørelse af originalbillede}
Billedet indlæses og nedskaleres til en fjerdedel. 

\paragraph{Skab binært billede}
På baggrund formlen i afsnit \ref{sec:DetectSameness} danner vi et binært billede hvor de områder hvor lysere gråtoner dominerer i originalbilledet er markeret.

\paragraph{Manipuler binært billede}
For at forsøge at undgår at nummerpladeområdet er forbundet med andre områder, trækker vi alle områder i det binære billede en anelse sammen. Vi bruger Matlabs funktion \textit{imerode} med et kvadrat på $2 \times 2$ pixels som figur. Dette trin kan muligvis have delt nummerpladområdet op i flere områder. For igen at have et samlet nummerpladeområde, udvidder vi alle områder i det binære billede med Matlabs \textit{imdilate}-funktion. Denne gang bruger vi et liggende rektangel med højden 2 og bredden 4 som figur. Vi bruger denne liggende figur da vi ønsker at forbinde områder som ligger ved siden af hinanden, dele af nummerpladen, men vil undgå at forbinde områder der ligger over og under hinanden. 

OVERVEJ ILLU FRA: http://www.ph.tn.tudelft.nl/Courses/FIP/noframes/fip-Morpholo.html

\paragraph{Fjern uniteressante områder fra binært billede}
I det binære billede er der nu en masse områder som helt sikkert ikke kan være nummerpladen. Et helt oplagt eksempel er områder der kun består af ganske få pixels. Vi sletter alle de områder vi vurderer som uinteresant ved at kalde funktionen \textit{BinImgCleanup} som tidligere beskrevet i \vref{sec:imp:BinImgCleanup} med det binære billede som argument.


\paragraph{Vælg den bedste nummerpladekandidat}
For at finde den bedste nummerpladekandidate i det nu "rensede" binære billede kalder vi funktionen \textit{GetBestCandidate} som tidligere beskrevet i afsnit \vref{sec:imp:GetBestCandidate}. Funktionen returnerer koordinater og point for den nummerpladekandidat der har fået færrest point. 


\paragraph{Returner resultat}
For at forsøge at sikres os at koordinaterne dækker hele nummerpladeområdet, udvidder vi området en smule før vi returnerer det og de tilhørende point fra det tidligere trin. 


%%%%%%%%%%%%%%%%%%%%%%%%%%%%%%%%%%%%%%%%%%
\subsubsection{Metode: Områder med høj kontrast}
Metoden hvor vi forsøger at finde nummerplader ved at markere områder med høj kontrast er implementeret i funktionen \textit{DetectContrastAvg} hvis kildekode findes i afsnit \vref{code:DetectContrastAvg}. Funktionens eneste argument er stien til et farvebillede i \textit{JPG}-format i en opløsning på $1024 \times 768$ pixels. Programmet udfører følgende trin:

\paragraph{Klargørelse af originalbillede}
Billedet indlæses og konverteres fra \textit{RGB} til 256 gråtoner. Herefter nedskaleres det til en fjerdedel. DER ER LOG I KODEN. VI GØR BILLEDET LYSERE. BESKRIV!

\paragraph{Udregn gradienter}



\begin{description}
\item[5.] Udregn gradienter.
\item[6.] Udregn gradienters vinkler.
\item[7.] Udregn gradienters længder og normaliser dem så de største har længden 1,0.
\item[8.] Udvælg gradienter hvis normaliserede længde er 0,25 eller mere.
\item[9.] Skab et billede ved at indsætte intensitetet af punkter der har "liggende" gradienter. Det vil sige, gradienter der har en vinkel på mellem 0 og 30 i forhold til vandret. Hvis gradienten er en af de længste i billedet, indsættes den fordoblede intensitet.
\item[10.] Kør et filter som udtværer billedet og forbinder områder der ligger ved siden af hinanden som beskrevet i afsnit \ref{sec:DetectContrastAvg}.
\item[11.] Skab et binært billede udfra det filtrerede billede af de "liggende" gradienter. 
\item[12.] Slet områder i det binære billede der sandsynligvis ikke er nummerplader ved at bruge funktionen \textit{BinImgCleanup}.
\item[13.] Vælg nummerpladekandidat ved at bruge funktionen \textit{GetBestCandidate}.
\item[14.] Returner koordinater.
\end{description}



\paragraph{Manipuler binært billede}
For at forsøge at undgår at nummerpladeområdet er forbundet med andre områder, trækker vi alle områder i det binære billede en anelse sammen. Vi bruger Matlabs funktion \textit{imerode} med et kvadrat på $2 \times 2$ pixels som figur. Dette trin kan muligvis have delt nummerpladområdet op i flere områder. For igen at have et samlet nummerpladeområde, udvidder vi alle områder i det binære billede med Matlabs \textit{imdilate}-funktion. Denne gang bruger vi et liggende rektangel med højden 2 og bredden 4 som figur. Vi bruger denne liggende figur da vi ønsker at forbinde områder som ligger ved siden af hinanden, dele af nummerpladen, men vil undgå at forbinde områder der ligger over og under hinanden. 

OVERVEJ ILLU FRA: http://www.ph.tn.tudelft.nl/Courses/FIP/noframes/fip-Morpholo.html

\paragraph{Fjern uniteressante områder fra binært billede}
I det binære billede er der nu en masse områder som helt sikkert ikke kan være nummerpladen. Et helt oplagt eksempel er områder der kun består af ganske få pixels. Vi sletter alle de områder vi vurderer som uinteresant ved at kalde funktionen \textit{BinImgCleanup} som tidligere beskrevet i \vref{sec:imp:BinImgCleanup} med det binære billede som argument.


\paragraph{Vælg den bedste nummerpladekandidat}
For at finde den bedste nummerpladekandidate i det nu "rensede" binære billede kalder vi funktionen \textit{GetBestCandidate} som tidligere beskrevet i afsnit \vref{sec:imp:GetBestCandidate}. Funktionen returnerer koordinater og point for den nummerpladekandidat der har fået færrest point. 


\paragraph{Returner resultat}
For at forsøge at sikres os at koordinaterne dækker hele nummerpladeområdet, udvidder vi området en smule før vi returnerer det og de tilhørende point fra det tidligere trin. 


Vi bruger Matlabs funktioner til at udregne gradienterne samt anvende filteret.


Pladerne bliver meget lave da vi kun kigger på kontrasten. Vi får altså ikke den hvide over- og underkant med. Derfor udvider vi kandidaten med 15 pixels på over- og underkanten. 


\subsubsection{Metode: Frekvensanalyse}
\begin{description}
\item[1.] Indlæs billede.
\item[2.] Konverter til gråtoner.
\item[3.] Nedskaler billede.
\item[4.] Filtrer billede som beskrevet i \ref{sec:DetectPlateness}.
\item[5.] Skab binært billede ud fra det filtrerede billede.
\item[6.] Slet områder i det binære billede der sandsynligvis ikke er nummerplader ved at bruge funktionen \textit{BinImgCleanup}.
\item[7.] Vælg nummerpladekandidat ved at bruge funktionen \textit{GetBestCandidate}.
\item[8.] Returner koordinater.
\end{description}

I punkt 8 returnerer vi et kandidatområde der er gjort højere da vi ellers ofte ikke får hele pladen med, da frekvensen i det hvide område over og under tegnene i nummerpladen er lav og dermed ikke karateristisk for en nummerplade.

Metoden bliver dårligere jo mere man skalerer billedet ned da tegn i nummerpladerne ikke længere er klart separeret. Vi burde nok køre dette på større billeder.
Vi bruger funktionen plateness. Den kunne være mere robust.

FIGURER AF HISTOGRAMMER OG LIGN.


\subsubsection{Metode: Skru op for Kontrast}
\begin{description}
\item[1.] Indlæs billede.
\item[2.] Konverter til gråtoner.
\item[3.] Nedskaler billede.
\item[4.] Filtrer billede som beskrevet i \ref{sec:DetectCStretch}.
\item[5.] Skab binært billede ud fra det filtrerede billede.
\item[8.] Slet områder i det binære billede der sandsynligvis ikke er nummerplader ved at bruge funktionen \textit{BinImgCleanup}.
\item[7.] Vælg nummerpladekandidat ved at bruge funktionen \textit{GetBestCandidate}.
\item[8.] Returner koordinater.
\end{description}

I trin 4 bruger vi hjælpefunktionen \textit{ContrastStretc} i en dobbeltløkke der behandler billedet i kvadratiske blokke af 4x4 pixels. Når vi skaber det binære billede i trin 5 bruger vi en grænseværdi på 0,65 der gør at det hvide i det binære billede skal være pixels der er noget over middelværdien i det filtrerede billede. Vores erfaring er, at vi i mange tilfælde kan separere nummerpladerne bedre med denne grænseværdi.

ER DET BEDRE AT GANGE GRAYTHRESH LIDT OP??    
GØR VI KANDIDATEN STØRRE?
Måske kunne vi opnå bedre resultater ved at bruge filter frem for blok, men blok er meget hurtigere.


%%%%%%%%%%%%%%%%%%%%%%%%
\subsubsection{Metode: Kvantifisering}
\begin{description}
\item[1.] Indlæs billede.
\item[2.] Konverter til gråtoner.
\item[3.] Nedskaler billede.
\item[4.] Filtrer billede som beskrevet i \ref{sec:DetectQuant}.
\item[5.] Sæt antal forskellige intensiteter i billedet ned til 7.
\item[6.] Lav et binært billede for hver af intensiteterne.
\item[7.] For hver intensitet, slet områder i det binære billeder der sandsynligvis ikke er nummerplader ved at bruge funktionen \textit{BinImgCleanup}.
\item[8.] Gør områderne i de binære billeder lidt mindre.  
\item[9.] Slå de binære billeder sammen til et enkelt binært billede. 
\item[10.] Vælg nummerpladekandidat ved at bruge funktionen \textit{GetBestCandidate}.
\item[11.] Udvid koordinaterne for den valgte nummerpladekandidat.
\item[12.] Returner koordinater.
\end{description}

I trin 4 dividerer vi alle intensiteterne i billedet med 43 og runder af for at sætte antallet af mulige intensitetsværdier ned til 7 i intervallet 1 til 7. De binære billeder i trin 6 gemmer vi en matrix med dimensionerne \textit{[billedehøjde, billedbrede, antal intensiteter]}. I trin 8 gør vi områderne mindre så se ikke smelter sammen med andre områder i trin 9.

METODEN ER DÅRLIG PÅ HVIDE BILER



%\subsubsection{Metode: Histogram}

\subsubsection{Endeligt valg af nummerpladekandidat}
\begin{description}
\item[1.] Indlæs billede.
\end{description}


Illustrer hvordan vi afgør (u)enighed. 

VI RETURNERER DEN STØRST MULIGE PLADE

\subsubsection{Mulige forbedringer}
Vi kunne arbejde mere med manipulation af de binære billeder før vi renser dem. Vi kunne arbejde mere med støjfiltrering etc. altså preprocessing af billederne. Citer folk der gør det. Skriv også hvad de gør.
Man kunne have taget højde for at de forskelleige metoder returnerer områder af forskellig bredde og intensitet.


%%%%%%%%%%%%%%%%%%%%%%%%%%
%%% SEPARATION AF TEGN %%%
%%%%%%%%%%%%%%%%%%%%%%%%%%

\subsection{Separation af tegn}

I de følgende afsnit gennemgår vi væsentlige implementeringssdetaljer i de funktioner vi bruger til separation af tegn. 

\subsubsection{Rotation}
\label{sec:implementation/sep/rotation}

Metoden til rotation er implementeret i funktionen \textit{RotatePlateRadon}, hvis kildekode findes i afsnit \vref{code:RotatePlateRadon}. Metoden tager stien til et \textit{JPEG}-billede samt koordinaterne for nummerpladen i billedet som indata. Funktionen returnerer billedet af den roterede plade samt dennes (roterede) koordinater i original billedet. Funktionens forløb er som følger:

\paragraph{1. Billedet af nummerpladen "tages ud" af originalbilledet.}
\paragraph{2. Billedet af pladen kant-detekteres:}
For at finde pladens rotation er det nødvendigt at finde kanterne i billedet af nummerpladen. Matlab funktionen \textit{edge} bruges til at lave et binært billede hvor alle horisontale kanter er markeret. De vertikale kanter markeres ikke. På denne måde kan vi nøjes med at analysere én dimension ved brug af Radon transformationen. I denne sammenhæng er vi interesseret i de kanter der er (næsten) horisontale, da det med stor sandsynlighed er de "lange" kanter i nummerpladens rektangel. Det vil være et problem, hvis nummerpladen "klippes" så tæt at den øvre eller nedre kant af nummerpladen ikke kommer med i billedet. I dette tilfælde er det meget usikkert om de horisontale kanter i nummerpladens tegn vil være kraftige nok til at Radon transformationen vil "bruge" disse kanter til at finde den tydeligste kant.

\paragraph{3. Radon matricen for kant-billedet findes ved brug af Matlab funktionen \textit{radon}.}

\paragraph{4. Den maksimale værdi i Radon matricen findes og rotationsvinklen registres:}
Rotationsvinklen svarer til ...

\paragraph{5. Billedet af nummerpladen roteres ved brug af \textit{imrotate}.}
\paragraph{6. Koordinaterne til den roterede plade findes ved brug af en rotationsmatrix.}

\subsubsection{Separation}

Metoderne til separation af tegn tager et billede af en nummerplade samt koordinaterne til denne plade i orginalbilledet som inddata, og returnerer syv billeder af de syv tegn (hvis syv tegn er fundet), koordinaterne til disse tegn i orginalbilledet samt antallet af fundne tegn som uddata.

\subsubsection*{Sammenhængende komponenter}

Denne metode er implementeret i funktionen \textit{CharSeparationCC}, hvis kildekode findes i afsnit \vref{code:CharSeparationCC}. Funktionen forløber således:

\paragraph{1. Billedet af pladen omdannes til gråtone.}

\paragraph{2. Billedet laves om til sort-hvid, "roughly" og indskrænkes til den største komponent:}
De billeder der i denne del af systemet modtages fra metoderne til identificering af nummerplader er ikke nødvendigvis udskåret så nøjagtigt at nummerpladen dækker hele billedet. Billedet kan derfor indeholde elementer som ikke er en del af nummerpladen (eksempelvis et klistermærke som sidder på bilen). Derfor er det hensigtsmæssigt, når tegnene i nummerpladen skal separeres, at indskrænke billedet så det mest muligt kun forestiller nummerpladen. Da nummerpladen med sandsynlighed er det mest lyse element i billedet, gøres dette ved at gøre billedet binært og indskrænke det til det største hvide område i billedet. Dette gøres ved hjælp af Matlab funktionen im2bw, hvor grænsen for hvilke intensitetsværdier der skal være hvide i det binære billede sættes under middel. På denne måde sikrer vi os, at nummerpladen vil være hvid og at den ikke bliver sort (SKRIV OM).

\paragraph{3. Kontrast..:}
For at få nummerpladens tegn til at adskille sig mest muligt fra pladens hvide baggrund skal billedet kontrastforstærkes. Til at kontrastforstærke bruges hjælpefunktionen \textit{ContrastStretch}. Denne funktion udføres i blokke så man udover den forstærkede kontrast mellem tegnene og baggrunden også opnår forstærket kontrast mellem støj (f.eks. snavs på nummerpladen) og baggrunden. Hvis denne støj er tæt på et tegn vil der være chance for at tegnet "gror sammen" med støjen og derfor bliver sorteret fra når de sammenhængende komponenter analyseres. Hvis kontrasten forstærkes i blokke er der dog en mulighed for at snavsen og tegnets kant vil blive adskilt. Størrelsen af blokkene der arbejdes med er bestemt eksperimentielt.

\paragraph{4. Tynde komponenter i toppen og i siderne af billedet fjernes:}
Når det binære billede er dannet opstår der igen en mulighed for at minimere tilfælde hvor tegn er groet sammen med eksempelvis nummerpladens kant. Se et eksempel nedenfor... Når dette .... Derfor slettes alle pixels som har en tom plads på begge sider i den øverste 1/3 af billedet, samt i den yderste 1/10 i både højre og venstre side.

BILLEDE MED NUMMERPLADE SOM ER I SKYGGE FOROVEN SAMT BINÆRT BILLEDE HVOR TYNDE KOMPONENTER KAN FJERNES

Derudover slettes alle pixels i de rækker af billedet som ikke har mindst en 1/10 eller højst 2/3 af alle pixels i billedets bredde sat. Dette gøres fordi en linie som går i gennem alle tegn vil ligge i dette område.

BESKRIV PROBLEM MED TEGNET K

\paragraph{5. Horisontale linier med meget lidt hvid, eller meget hvid gøres helt sorte:}
Beskrivelse...

\paragraph{6. Det sort-hvide billede laves til et sammenhængende-komponenter-billede.}
\paragraph{7. For små og for store komponenter fjernes.}
I næste del af funktionen frasorteres alle de sammenhængende komponenter der er for små eller for store.

\paragraph{8. Grupper på syv komponenter samles hvor alle syv komponenter har samme højde og er i samme højde.}

Så findes der grupper af syv sammenhængende komponenter hvor elementerne i hver gruppe skal være ca. lige høje og befinde sig i ca. samme højde.

\paragraph{9. Afstandene mellem komponenter i hver gruppe udregnes og en "god" gruppe vælges:}
Til sidst analyseres disse grupper og hvis komponenterne i én af grupperne har de rette afstande i mellem sig, vælges elementerne i gruppen som tegnene.

NOGET MED AT TEGNENE SKÆRES "TÆT"	

\subsubsection*{Bjerg/dal}

Denne metode er implementeret i funktionen \textit{CharSeparationPTV}, hvis kildekode findes i afsnit \vref{code:CharSeparationPTV}.


Nummerpladens signatur fås ved at summere projektionerne af alle rækkerne i billedet. Hvis denne signatur præsenteres som en graf i et koordinatsystem vil der forekomme toppe i de kolonner hvor der høj intensitet. Idéen er så at udvælge de otte højeste toppe som de otte steder i x-planen der skal skæres ved.

KONTRAST?

\begin{enumerate}
\item funktion pkt 1
\item funktion pkt 2
\item funktion pkt n
\end{enumerate}

Pladens signatur findes ved at opsummere intensitetsværdierne i billedet af nummerpladen omdannet til gråtone. Hvis man betragter signaturen som en graf vil den være "hakket" på grund af støj i billedet (dvs. det er kun i teorien at billedet kun indeholder en hvid baggrund og nogle mørke bogstaver, hvilket ville give en perfekt graf). Signaturen skal derfor udglattes. Dette gøres ved at tildele det indeværende punkt på grafen middelværdien af punktets nærområde. EKSEMPEL PÅ UDREGNING? Et eksempel på en "rå" graf og samme graf, udglattet ses i Figur \vref{fig:smoothSig}.

EKSEMPEL PÅ HAKKET SIGNATUR OG UDGLATTET SIGNATUR

\begin{figure}[htp]
\label{fig:smoothSig}
\caption{bla}
\end{figure}

For at finde de toppe hvor der skal skæres, leder vi efter punkter, $x$ i signaturgrafen hvor de $n$ forrige punkter har haft stigende værdier op til $x$ og hvor de efterfølgende $n$ punkter har faldende værdier. DETTE GØR VI IKKE I PRAKSIS! SKAL LAVES OM. De otte $x$'er som har de otte højeste værdier vælges som de otte skæringspunkter.

NOGET OM HVORDAN VI FINDER TOP OG BUND BESKÆRING


\subsubsection{Mulige forbedringer}
DER MÅ VÆRE MANGE KOMMENTARER TIL HVORDAN VI KAN GØR BJERG/DAL BEDRE

I funktionen der bruger sammenhængende komponenter findes der flere statiske variable, som eksempelvis størrelsen af de blokke der kontrastforstærkes i, komponenternes minimum bredde osv. Disse variable kunne være lavet dynamisk, men de er ret gode nu og derfor har vi ladet dem være. SKRIV DET SIDSTE OM.

CONCOMP: BESKREVET I KWAS: MAN KU FJERNE KANTER FØR ANALYSE?

SKALERING KAN FORBEDRE HASTIGHED? PERFORMANCE?

MEDIAN FILTER ELLER ANDRE FILTRE? LYSFORSTÆRKNING?


%%%%%%%%%%%%%%%%%%%%%%%%%%
%%% GEKENDELSE AF TEGN %%%
%%%%%%%%%%%%%%%%%%%%%%%%%%

\subsection{Genkendelse af tegn}

\subsubsection{Middelvektorer}

\begin{enumerate}
\item funktion pkt 1
\item funktion pkt 2
\item funktion pkt n
\end{enumerate}

Implementationen af metoden til tegngenkendelse ved hjælp af vektor analyse findes i filerne \textit{GetMeanVectors.m} og \textit{ReadPlateFV.m}.

For at disse to funktioner/metoder skal fungere sammen er det nødvendigt at længden, $d$ af middelvektorerne er den samme i begge funktioner. Man bør eksperimentere med størrelsen af $d$.

\textit{GetMeanVectors.m} udarbejder en matrix bestående af middelvektorer for hvert lovligt tegn. Hver vektor vil have længden $d$. Denne matrix bruges af \textit{ReadPlateFV.m}: Hvert inputbillede omformes til en vektor af længden $d$ og herefter løbes middelvektor-matrixen igennem. Afstanden fra det omformede billede til hver middelvektor noteres og tegnhitlisten udarbejdes på baggrund af disse afstande. Sammen med denne hitliste returneres afstandene, så disse kan bruges i forbindelse med syntaks analyse.
Som input modtager funktionen en mængde billeder af manuelt sorterede tegn.

Hvert billede laves til et kvadrat. Ved at lave middelvektorer på træningssættet kan vi se at man ikke kan have vektorer af længden 4, da flere bogstavs vektorer så vil være de samme (nemlig 0 0 0 0). Afprøv med andre længder.

\textit{ReadPlateFV.m}...

\begin{enumerate}
\item funktion pkt 1
\item funktion pkt 2
\item funktion pkt n
\end{enumerate}

\subsubsection{Sum-billeder}

\begin{enumerate}
\item funktion pkt 1
\item funktion pkt 2
\item funktion pkt n
\end{enumerate}

\subsubsection{Syntaks analyse}

Metoden til syntaks analyse tager tegnhitlisterne for de syv tegn, listerne over afstandene til middelvektorerne samt et "maksimal-hit-nummer" som inddata. Uddata er en syntaksanalyseret tegnfølge samt en vektor indeholdende de hits der er blevet brugt til at lave denne tegnfølge. Implementeringen af metoden til syntaksanalyse findes i filen \textit{SyntaxAnalysis.m}.

Funktionens forløb er som følger:

\begin{description}
\item[1.] Hitlisterne for tegnene på de to første pldser gennemløbes indtil der står to bogstaver
\item[2.] De to tegn på de første to pladser analyseres og udskiftes om nødvendigt
\end{description}

Hvis der i på tegnfølgens to første positioner fremkommer en ulovlig bogstavkombination fås et problem: Hvilket tegn skal man forsøge at udskifte så en lovlig kombination forekommer? Dette gøres på baggrund af billedets (af tegnet) afstand til de to middelvektorer: Den middelvektor der er længst væk har størst sandsynlighed for at være et forkerte bogstav, hvorfor denne skiftes ud.

\begin{description}
\item[3.] Hitlisterne for tegnene på de fem sidste pladser gennemløbes indtil der står fem tal
\item[4.] De to tegn på plads nr. tre og fire analyseres og udskiftes om nødvendigt
\end{description}

Bla bla

\begin{description}
\item[5.] De valgte hitnumre analyseres. Hvis et nummer er for højt i forhold til det maksimale, returnes et "\_" tegn på den givne plads
\end{description}

\subsubsection{Mulige forbedringer}
Når der eksempelvis forekommer en ulovlig bogstavkombination i syntaksanalysen udskiftes det tegn som har den længste afstand til middelvektoren hver gang. Det vil sige at kun det ene tegn vil blive udskiftet (tegnets vektor er til at starte med den vektor der ligger længest væk og dette vil ikke ændre sig, da man hele tiden vælger vektorer der ligger længere og længere væk). En anden mulighed kunne være at man skifter mellem at udskifte det ene og det andet tegn Eksempel: tegnfølgen \textbf{AB} er fundet, men denne er ulovlig. Først udskiftes \textbf{B}, da denne ligger længst væk. En ny tegnfølge \textbf{AC} findes, men denne er også ulovlig. I stedet for at udskifte tegnet på 2. position igen (\textbf{C}) udskiftes \textbf{A} og \textbf{B} sættes tilbage på 2. position.

Man kunne bruge mahalanopis afstand


%%%%%%%%%%%%%%
% Afprøvning %
%%%%%%%%%%%%%%
\section{Resultater}

HVAD SKER DER MED GULE ETC. PLADER??
\subsection{Indsamling af testdata}

%Fra start var vi meget opmærksomme på at afgrænsningen af projektet skulle være klar, så det ikke blev for omfattende. Omkring valg af billedemateriale afholdt vi os eksempelvis fra at 

%Til at udarbejde og teste systemet havde vi brug for nogle fotografier af biler med nummerplader. Ved fotograferingen var vi opmærksomme på følgende afgrænsninger:


%\begin{figure}[h]
%\begin{center}
%\includegraphics[width=10cm]{illu/B_XC33139.jpg}
%\label{b_xc33139}
%\caption{Fotografieksempel}
%\end{center}
%\end{figure}

% vi holder dem adskilt - ikke noget med histo
Da vi bl.a. havde planer om udarbejdelse af en histogrambaseret metode til identificering af nummerplader (se afsnit \ref{histo}), holdt vi de to fotografisæt adskilte. På denne måde ville det f.eks. være muligt for os at teste om skift fra et kamera til et andet, vil give ændrede resultater. Billederne blev navngivet i vores database, så det, i deres filnavn, indgik om billedet forestillede en bil set forfra eller bagfra samt hvilken nummerplade bilen på nummerpladen havde. Derudover udarbejdede vi et mindre program som hjalp os til at identificere nummerpladens fire hjørnekoordinater og indskrive disse i filnavnet. Denne sidste tilføjelse ville hjælpe os i testfasen, til at undersøge om de nummerpladekandidater vores system ville udvælge er de korrekte.
 
Kontrolsæt: 600 billeder


%\subsection{Fotografering}
%Hvor mange og hvilke billeder har vi taget.
%At billederne er taget i naturligt lys for at undgå kunstige farver og genskin fra pladen som er malet med reflekterende materiale.
%Opdeling af billeder i træning- og testsæt
%til f.eks. histogram metode herunder adskillelse af fotos fra de to forskellige kameraer.

%%% LOKALISERING AF NUMMERPLADE %%%

\subsection{Lokalisering af nummerplade}
HVOR GODE ER VI PÅ TRÆNINGSSÆT?
HVOR GODE ER VI PÅ TESTSÆT?

\subsubsection{Områder domineret af lyse gråtoner}
HVOR GODE ER VI PÅ TRÆNINGSSÆT?
HVOR GODE ER VI PÅ TESTSÆT?

\begin{tabular}{|l|l|l|l|l|}
\hline
\multicolumn{5}{|c|}{DetectMain} \\ \hline
Param 1 & Param 2 & Skalering & Overordnet resultat & Sande positiver\\ \hline
1 & 1 & 1 & 19 \% & 19 \%\\ \hline
2 & 2 & 2 & 19 \% & 21 \% \\ \hline
3 & 3 & 3 & 19 \% & 22 \% \\
\hline
\end{tabular}

\subsubsection{Områder med høj kontrast}
HVOR GODE ER VI PÅ TRÆNINGSSÆT?
HVOR GODE ER VI PÅ TESTSÆT?

\begin{tabular}{|l|l|l|l|l|}
\hline
\multicolumn{5}{|c|}{DetectMain} \\ \hline
Param 1 & Param 2 & Skalering & Overordnet resultat & Sande positiver\\ \hline
1 & 1 & 1 & 19 \% & 19 \%\\ \hline
2 & 2 & 2 & 19 \% & 21 \% \\ \hline
3 & 3 & 3 & 19 \% & 22 \% \\
\hline
\end{tabular}

\subsubsection{Frekvensanalyse}
HVOR GODE ER VI PÅ TRÆNINGSSÆT?
HVOR GODE ER VI PÅ TESTSÆT?

\begin{tabular}{|l|l|l|l|l|}
\hline
\multicolumn{5}{|c|}{DetectMain} \\ \hline
Param 1 & Param 2 & Skalering & Overordnet resultat & Sande positiver\\ \hline
1 & 1 & 1 & 19 \% & 19 \%\\ \hline
2 & 2 & 2 & 19 \% & 21 \% \\ \hline
3 & 3 & 3 & 19 \% & 22 \% \\
\hline
\end{tabular}

\subsubsection{Maksimer lokal kontrast}
HVOR GODE ER VI PÅ TRÆNINGSSÆT?
HVOR GODE ER VI PÅ TESTSÆT?

\begin{tabular}{|l|l|l|l|l|}
\hline
\multicolumn{5}{|c|}{DetectMain} \\ \hline
Param 1 & Param 2 & Skalering & Overordnet resultat & Sande positiver\\ \hline
1 & 1 & 1 & 19 \% & 19 \%\\ \hline
2 & 2 & 2 & 19 \% & 21 \% \\ \hline
3 & 3 & 3 & 19 \% & 22 \% \\
\hline
\end{tabular}

\subsubsection{Kvantifisering}
HVOR GODE ER VI PÅ TRÆNINGSSÆT?
HVOR GODE ER VI PÅ TESTSÆT?

\begin{tabular}{|l|l|l|l|l|}
\hline
\multicolumn{5}{|c|}{DetectMain} \\ \hline
Param 1 & Param 2 & Skalering & Overordnet resultat & Sande positiver\\ \hline
1 & 1 & 1 & 19 \% & 19 \%\\ \hline
2 & 2 & 2 & 19 \% & 21 \% \\ \hline
3 & 3 & 3 & 19 \% & 22 \% \\
\hline
\end{tabular}

\subsubsection{endeligt valg af nummerpladekandidat}

Min antal enige 1:
HVOR GODE ER VI PÅ TRÆNINGSSÆT?
HVOR GODE ER VI PÅ TESTSÆT?


Min antal enige 2:
HVOR GODE ER VI PÅ TRÆNINGSSÆT?
HVOR GODE ER VI PÅ TESTSÆT?

Min antal enige 3:
HVOR GODE ER VI PÅ TRÆNINGSSÆT?
HVOR GODE ER VI PÅ TESTSÆT?

Min antal enige 5:
HVOR GODE ER VI PÅ TRÆNINGSSÆT?
HVOR GODE ER VI PÅ TESTSÆT?

Min antal enige 5:
HVOR GODE ER VI PÅ TRÆNINGSSÆT?
HVOR GODE ER VI PÅ TESTSÆT?

\begin{tabular}{|l|l|l|l|l|}
\hline
\multicolumn{5}{|c|}{DetectMain} \\ \hline
Param 1 & Param 2 & Skalering & Overordnet resultat & Sande positiver\\ \hline
1 & 1 & 1 & 19 \% & 19 \%\\ \hline
2 & 2 & 2 & 19 \% & 21 \% \\ \hline
3 & 3 & 3 & 19 \% & 22 \% \\
\hline
\end{tabular}


VI SKAL SE HVORDAN DETECTMAIN OPFØRER SIG NÅR F.EKS. ALLE METODER SKAL VÆRE ENIGE.
 Vi har skaleret ned for at spare tid.

\subsubsection*{Observeret sæt på 407 billeder}
Scale: 0.25
DetectMain: 96.6/99.24
DetectQuant: 67.8/75.4
DetectSameness: 56.8/95.5
DetectContrastAvg: 62.7/85.0
DetectPlateness: 50.4/65.5
DetectCStretch: 84.0/92.7

Scale: 0.50 (Ekstremt langsomt)
DetectPlateness: 29.7/56.5
DetectCStretch:

Der skal testet på hver metode som fritstående. Herefter på DetectMain der samler metoderne. Hvordan ændrer resultaterne sig når man ændrer opløsning af billederne?
Hvor gode er vi på det set vi har observeret? Hvor gode er vi på et set vi ikke har observeret. Er der et mønster i de billeder hvot vi ikke finder nummerpladen? Hvile udfordringer møder vi: Mørke plader... 

Confusion matrix: De elementer der ligger udenfor diagonalen er elementer der ikke er nummerplader.




Delkonklussion:

%%% SEPARATION AF TEGN %%%

\subsection{Separation af tegn}
I dette afsnit testes de to metoder til separation af tegn. Metoden til rotation vil ikke blive testes, blot kommenteret.

\subsubsection*{Rotation}
Om en nummerplade står vandret i et billede kan afgøres ved at kantdetektere billedet og udføre en Radon transformation på dette kant-billede. Hvis de stærkeste linier optræder ved $0^{\circ}$ er rotation udført korrekt.

Vi har ikke testet det fordi...

VI KUNNE MÅSKE GODT TESTE DET?

\subsubsection*{Separation}
De to metoder til separation af tegn testes ved at se på om de finder syv objekter som optræder indenfor pladens koordinater. Denne forholdsvis enkle test mener vi er fyldestgørende, da funktionerne hvori metoderne er implementeret, bør være "skrappe" nok.

Vi vil desuden se på om der er forskel på succesraten i forhold til hvilket område der repræsenterer nummerpladen: Her findes to muligheder: pladen er fundet med vores egne metoder eller defineret ved hjælp af pladekoordinaterne og et eller andet lagt til.
%Er det forskel på succesraten i forhold til om vi ser på pladerne vi manuelt har plottet (med evt. et tilfældigt yderområde lagt til) eller om vi ser på de nummerplader som vores indetifikationsmetoder finder?

Parametre: størrelse på componenter der sorteres fra, skalering: kan vi tillade os at skalere ned (eller op, hvor vi måske får mere plads mellem komponenterne) eller mister vi for meget information VI KAN IKKE SKALERE

VI VIL KUN SE PÅ TEST AF DETTE NIVEAU, IKKE? 

%OVERVEJ NEDENSTÅENDE SOM SAMLET GRAF I ET AFSNIT

%\subsubsection{Bjerg/Dal}
%HVOR GODE ER VI TIL AT FINDE 7 OMRÅDER INDEN FOR PLADEN I TRÆNINGSSÆT?
%HVOR GODE ER VI TIL AT FINDE 7 OMRÅDER INDEN FOR PLADEN I TESTSÆT
%HVOR GODE ER VI TIL AT FINDE 7 OMRÅDER INDEN FOR PLADEN I TRÆNINGSSÆT?
%HVOR GODE ER VI TIL AT FINDE 7 OMRÅDER INDEN FOR PLADEN I TESTSÆT
%HVAD SKER DER NÅR PLADEOMRÅDET UDVIDES MED 10PIXELS PÅ ALLE SIDER?
%HVAD SKER DER NÅR PLADEOMRÅDET UDVIDES MED 20PIXELS PÅ ALLE SIDER?
%HVAD SKER DER NÅR PLADEOMRÅDET UDVIDES MED 30PIXELS PÅ ALLE SIDER?


\begin{tabular}{|l|c|c|}\hline
\multicolumn{3}{|l|}{Træningssæt} \\\hline
Metode & Sammenhængende komponenter & Bjerg/dal \\\hline
Egne metoder & 96\% & 3\% \\\hline
10 & 96\% & 3\% \\\hline
20 & 96\% & 3\% \\\hline
30 & 96\% & 3\% \\\hline \end{tabular}

\begin{tabular}{|l|c|c|}\hline
\multicolumn{3}{|l|}{Kontrolsæt} \\\hline
Metode & Sammenhængende komponenter & Bjerg/dal \\\hline
Egne metoder & 96\% & 3\% \\\hline
10 & 96\% & 3\% \\\hline
20 & 96\% & 3\% \\\hline
30 & 96\% & 3\% \\\hline \end{tabular}

Delkonklusion:
Sammenhængende komponenter er go og bjerg/dal dårlig?

%%% GENKENDELSE AF TEGN %%%

\subsection{Genkendelse af tegn}
I dette afsnit testes de to metoder til genkendelse af tegn i en nummerplade. Derudover vil syntaksanalysen blive testet.

Noget om at det er i forhold til alle plader, dvs. at procenterne nedenfor er afhængige af hvor gode vi er til separere tegn.

\subsubsection{Egenskabsvektor}

%HVOR MANGE PLADER ER KORREKT LÆST I TRÆNINGSSÆT?
%HVOR MANGE PLADER ER KORREKT LÆST I TETSSÆT?
%HVOR MANGE PLADER ER KORREKT LÆST PÅ 6 POSITIONER I TRÆNINGSSÆT?
%HVOR MANGE PLADER ER KORREKT LÆST PÅ 6 POSITIONER I TETSSÆT?
%HVOR MANGE PLADER ER KORREKT LÆST PÅ 5 POSITIONER I TRÆNINGSSÆT?
%HVOR MANGE PLADER ER KORREKT LÆST PÅ 5 POSITIONER I TETSSÆT?

Følgende tabeller viser resultaterne for at læse en hel plade, seks tegn i pladen osv. (med forskellige vektorlængder).

\begin{tabular}{|l|c|c|c|c|c|c|}\hline
\multicolumn{7}{|l|}{Træningssæt} \\\hline
Vektorlængde & Hele pladen læst & 6 tegn læst & 5 tegn & 4 tegn & 3 tegn & 2 tegn \\\hline
9 & 0\% & 0\% & 0\% & 0\% & 0\% & 0\% \\\hline
16 & 0\% & 0\% & 0\% & 0\% & 0\% & 0\% \\\hline
25 & 0\% & 0\% & 0\% & 0\% & 0\% & 0\% \\\hline \end{tabular}

\begin{tabular}{|l|c|c|c|c|c|c|}\hline
\multicolumn{7}{|l|}{Kontrolsæt} \\\hline
Vektorlængde & Hele pladen læst & 6 tegn læst & 5 tegn & 4 tegn & 3 tegn & 2 tegn \\\hline
9 & 0\% & 0\% & 0\% & 0\% & 0\% & 0\% \\\hline
16 & 0\% & 0\% & 0\% & 0\% & 0\% & 0\% \\\hline
25 & 0\% & 0\% & 0\% & 0\% & 0\% & 0\% \\\hline \end{tabular}

%HVOR GODE ER VI PÅ TAL I TRÆNINGSSÆT?
%HVOR GODE ER VI PÅ TAL I TESTSÆT?
%HVOR GODE ER VI PÅ BOGSTAVER I TRÆNINGSSÆT?
%HVOR GODE ER VI PÅ BOGSTAVER I TESTSÆT?
%HVAD ER GENKENDELSESPROCENTEN PÅ TEGN PR PLADE I TRÆNINGSÆT?
%HVAD ER GENKENDELSESPROCENTEN PÅ TEGN PR PLADE I TESTSÆT?

Følgende tabeller viser hvor godt systemet er til at læse bogstaver hhv. tal. hhv. alle tegn

\begin{tabular}{|l|c|c|c|}\hline
\multicolumn{4}{|l|}{Træningssæt} \\\hline
Vektorlængde & Alle tegn & Bogstaver & Tal \\\hline
9 & 0\% & 0\% & 0\% \\\hline
16 & 0\% & 0\% & 0\%\\\hline
25 & 0\% & 0\% & 0\%\\\hline \end{tabular}

\begin{tabular}{|l|c|c|c|}\hline
\multicolumn{4}{|l|}{Kontrolsæt} \\\hline
Vektorlængde & Alle tegn & Bogstaver & Tal \\\hline
9 & 0\% & 0\% & 0\% \\\hline
16 & 0\% & 0\% & 0\% \\\hline
25 & 0\% & 0\% & 0\% \\\hline \end{tabular}

%HVILKEN POSITION PÅ HITLISTEN TAGER DEN I GENNEMSNIT PR POSITION - TRÆNING?
%HVILKEN POSITION PÅ HITLISTEN TAGER DEN I GENNEMSNIT PR POSITION - TEST?
%MAXHITNO: HVAD GIVER DET JO LÆNGERE NED AD HITLISTEN VI KAN GÅ?

Syntaks analyse: Hvilke hits bliver valgt på hitlisterne af syntaksanalysen (sæt maxhitno højt):

\begin{tabular}{|l|c|}\hline
\multicolumn{2}{|l|}{Træningssæt} \\\hline
Hitnr. & Valgt \\\hline
1 & 95,4\% \\\hline
2 & 3\% \\\hline
3 & 0\% \\\hline
4 & 0\% \\\hline
5 & 0\% \\\hline
6 & 0\% \\\hline \end{tabular}

\begin{tabular}{|l|c|}\hline
\multicolumn{2}{|l|}{Kontrolsæt} \\\hline
Hitnr. & Valgt \\\hline
1 & 92,9\% \\\hline
2 & 0\% \\\hline
3 & 0\% \\\hline
4 & 0\% \\\hline
5 & 0\% \\\hline
6 & 0\% \\\hline \end{tabular}

UDEN SYNTAKSANALYSE - TRÆNING?
UDEN SYNTAKSANALYSE - TEST?

Evt. en tabel over de enkelte tegn A, B, C osv. Hvor gode er vi til at genkende disse? Tabellen kunne laves udfra den bedste vektorstørrelse? Måske bare en tabel over de dårligste tegn, altså dem der er sværest at genkende? På denne måde bliver det ikke en LANG tabel

\begin{tabular}{|l|c|}\hline
\multicolumn{2}{|l|}{Træningsæt} \\\hline
Tegn & Valgt \\\hline
0 & 92,9\% \\\hline
1 & 0\% \\\hline
2 & 0\% \\\hline
3 & 0\% \\\hline
4 & 0\% \\\hline
5 & 0\% \\\hline
6 & 92,9\% \\\hline
7 & 0\% \\\hline
8 & 0\% \\\hline
9 & 0\% \\\hline
A & 0\% \\\hline
B & 0\% \\\hline
C & 92,9\% \\\hline
D & 0\% \\\hline
E & 0\% \\\hline
H & 0\% \\\hline
J & 0\% \\\hline
K & 0\% \\\hline 
L & 92,9\% \\\hline
M & 0\% \\\hline
N & 0\% \\\hline
O & 0\% \\\hline
P & 0\% \\\hline
R & 0\% \\\hline
S & 0\% \\\hline
T & 0\% \\\hline
U & 0\% \\\hline
V & 0\% \\\hline
X & 0\% \\\hline
Y & 0\% \\\hline
Z & 0\% \\\hline \end{tabular}



\subsubsection{Summerede billeder}
Samme tabeller som ovenfor?


%Søren: Ved klassifikation bruges en eller anden grænseværdi som bestemmer hvad et element skal klassificeres som: denne grænseværdi kan reguleres og kan derfor testes.

Delkonklusion:
Billederne af tegnene er ret store i forhold til dem der bruges i litteraturen. Dette giver vores system en væsentlig fordel i forhold til "de andre".





%\newpage
%\section{Evaluering}

% God bedømmelse af projektet fås hvis man kan:
%1. Opstille en arbejdsplan og afslutte en undersøgelse af problemet indenfor den tid der er til rådighed.
%2. Kunne kombinere relevant datalogisk og eventuelt anden viden i en beskrivelse af styrker og svagheder i tidligere forsøg på løsning af problemet.
%3. Kunne begrunde valget af en eller flere eksplicit beskrevne metoder, og anvende dem til løsning af problemet, eller til afprøvning af en mulig løsning.
%4. Kunne kombinere relevant datalogisk eller anden viden fra flere områder eller eventuelt viden fra et område og empiriske resultater, så de bidrager til løsning af problemet.
%5. Kunne give en sammenhængende, præcis og ikke-triviel beskrivelse af og begrundelse for væsentlige dele af den konkrete løsning, eller af de generelle muligheder for at løse problemet.
%6. Kunne vurdere på hvilke områder det er lykkedes at løse problemet, og på hvilke områder det ikke er lykkedes, og kunne påpege eventuelle svagheder i løsningen.
%7. Kunne reflektere over sin egen arbejdsproces og når det er relevant, komme med forslag til forbedring af den.
%8. Kunne skrive en sproglig og strukturel velskrevet rapport med relevante illustrationer og referencer som følger en etableret standard.

\begin{comment}

Skal vi have arbejdsopgaverne med i rapporten og så bedømmes de?

Blev spørgsmålene besvaret:

Hvordan kan nummerplader på farvefotografier identificeres og læses af et computersystem? Hvilke kendte metoder findes der, og hvor høj genkendelsesprocent kan et system vi selv laver opnå?

\end{comment}

%\subsection{Performance}
%\subsection{Problemer}
%\subsection{Resultater i forhold til læst litteratur}
\section{Konklusion}
Man skulle have taget alle billeder med et samme kamera. Så kunne man være sikker på at et kamera ikke var en faktor.

Vores testbilleder blev taget sent i forløbet. Vi er opmærksomme på, at vi nok burde have taget  billeder til både trænings- og testsæt fra starten, da vi ikke kan udelukke at arbejdet med opgaven kan have påvirket den måde vi har taget billederne i \textit{testsættet} på. Vi mener dog stadig, at dette andet sæt er relevant i forhold til afprøvning af vores system.

Noter fra møde med Søren: Konklusion skal indeholde "Kan man gå videre med dette system efter de påviste resultater?", "Hvad gik godt? : henvisninger til testene"

Vi har ikke så mange bogstaver i vores træningsæt.
%\end{comment} % remove this later

\newpage % new page before litterature
%\documentclass[10pt,a4paper,final]{report}
\usepackage[utf8]{inputenc}
\usepackage{ucs}
\usepackage[danish]{babel}
\usepackage{amsmath}
\usepackage{amsfonts}
\usepackage{amssymb}
\usepackage{verbatim}
\usepackage{listings} % Better source code listings
\usepackage{graphicx}
%\author{Tobias Balle-Petersen og Esben Paul Bugge}
\title{Litteraturliste}

\parindent=0pt
\parskip=8pt
 \usepackage[top=3cm, bottom=3cm, left=4cm, right=4cm]{geometry} 

\lstset{language=python}
\lstset{inputencoding=latin1}
\lstset{extendedchars=true}
\lstset{breaklines=true}
\lstset{commentstyle=\textit}
\lstset{showstringspaces=false}
\lstset{numbers=left, numberstyle=\tiny, stepnumber=2, numbersep=5pt,tabsize=3,basicstyle=\small}
%\lstset{numbers=left, numberstyle=\tiny, stepnumber=2, numbersep=5pt,stringstyle=\ttfamily, showstringspaces=false, basicstyle=\small, language={python}}


\begin{document}
\maketitle
%Hentet 11. februar 2008:
%http://www.it.lth.se/users/lambert/leftovers/LicenseplateSydney.pdf
%http://www.ci.pwr.wroc.pl/~kwasnick/download/kwasnickawawrzyniak.pdf
%http://www.icg.tu-graz.ac.at/pub/pdf/arth_-_real-time_license_plate_recognition_ecw07.pdf

%Andet:
%http://visl.technion.ac.il/projects/2003w24/ (11. februar 2008)
%www.retsinformation.dk

%Hentet 17. februar 2008:
%http://plutarco.disca.upv.es/~jcperez/Documentos/Matriculas2003.pdf
%http://ecet.ecs.ru.acad.bg/cst04/Docs/sIIIA/32.pdf (kaldet AdaptiveLicensePlateImageExtraction.pdf)
%http://pages.cpsc.ucalgary.ca/~federl/Publications/LicencePlate1996/licence-plate-1996.pdf
\end{document}
%\section{Litteraturliste}
\section{Litteratur}

\begin{thebibliography}{99}
%\bibliographystyle{plain}


%%%%% BELOW: DONE %%%%%

%http://visl.technion.ac.il/projects/2003w24/ (11. februar 2008)
\bibitem{ron} Ron, B.-H.: \textit{A Real-time vehicle License Plate Recognition (LPR) System}.

%Hentet 11. februar 2008:
\bibitem{nijhuis} Nijhuis, J. A. G., ter Brugge, M. H., Helmholt, K. A., Pluim, J. P. W., Spaanenburg, L., Venema, R. S., Westenberg, M. A.: \textit{Car License Plate Recognition with Neural Networks and Fuzzy Logic}.
%http://www.it.lth.se/users/lambert/leftovers/LicenseplateSydney.pdf

% 17. april
\bibitem{shapiro} Shapiro, V., Dimov, D., Bonchev, S., Velichkov, V., Gluhchev, G.: \textit{Adaptive License Plate Image Extraction}
% http://ecet.ecs.ru.acad.bg/cst04/Docs/sIIIA/32.pdf

%Hentet 17. februar 2008:
\bibitem{parker} Parker, J. R., Federl, P.: \textit{An Approach To Licence Plate Recognition}.
%http://pages.cpsc.ucalgary.ca/~federl/Publications/LicencePlate1996/licence-plate-1996.pdf

%Hentet 11. februar 2008:
\bibitem{kwas} Kwa\'snicka, H., Wawrzyniak, B.: \textit{License plate localization and recognition in camera pictures}.
%http://www.ci.pwr.wroc.pl/~kwasnick/download/kwasnickawawrzyniak.pdf

\bibitem{arth} Arth, C., Limberger, F., Bischof, H.: \textit{Real-Time License Plate Recognition on an Embedded DSP-Platform}.
%http://www.icg.tu-graz.ac.at/pub/pdf/arth_-_real-time_license_plate_recognition_ecw07.pdf

%Hentet 17. februar 2008:
\bibitem{cano} Cano, J., Pérez-Cortés, J.-C.: \textit{Vehicle License Plate Segmentation In Natural Images}.
%http://plutarco.disca.upv.es/~jcperez/Documentos/Matriculas2003.pdf



%%%%% NOT DONE %%%%%


%http://nrpl.dk/ - Nummerplader

%http://www.mathworks.com: matlab info, radon

\bibitem{murphy} Murphy-Chutorian, E., Trivedi, M.: \textit{N-Tree Disjoint-Set Forests for Maximally Stable Extremal Regions}.
\bibitem{vedaldi} Vedaldi, A.: \textit{An Implementation of Multi-Dimensional Maximally Stable Extremal Regions}.

\bibitem{matas} Matas, J., Chum, O., Urban, M., Pajdla, T.: \textit{Robust Wide Baseline Stereo from Maximally Stable Extremal Regions}.


\bibitem{wiki_gradienter} Ukendt forfatter: \textit{http://en.wikipedia.org/wiki/Gradient}.

\bibitem{wiki_radon} wikiradon%http://en.wikipedia.org/wiki/Radon_transform
\bibitem{toft_radon} toftradon %http://eivind.imm.dtu.dk/staff/ptoft/Radon/Radon.html


\end{thebibliography}

\appendix


\section{Kildekode}
Her er kode blah...

%%%%%%%%%%
\subsection{Identifikation af nummerplade}

\subsubsection{DetectSameness.m}
\label{code:DetectSameness}
\lstinputlisting{../../src/detection/DetectSameness.m}

\subsubsection{DetectContrastAvg.m}
\label{code:DetectContrastAvg}
\lstinputlisting{../../src/detection/DetectContrastAvg.m}

\subsubsection{DetectQuant.m}
\label{code:DetectQuant}
\lstinputlisting{../../src/detection/DetectQuant.m}


%%%%%%%%%%%%%
\subsection{Separation af tegn}
\subsubsection{RotatePlateRadon.m}

\subsubsection{CharSeparationCC.m}
\subsubsection{CharSeparationPTV.m}

\subsection{Genkendelse af tegn}
\subsubsection{GetMeanVectors.m}
\subsubsection{ReadPlateFV.m}


\section{Synopsis}
\subsection*{Titel}
Projektets titel: \textbf{Automatisk identifikation og læsning af nummerplader}.
\subsection*{Problemformulering}
I mange sammenhænge er der behov for automatisk at kunne læse nummerplader. Det er f.eks. identifikation af en bil der ankommer til et parkeringsanlæg, identifikation af køretøjer der overtræder færdselsloven og lignende.

Vi ønsker derfor at arbejde med spørgsmålene: Hvordan kan nummerplader på farvefotografier identificeres og læses af et computersystem? Hvilke kendte metoder findes der, og hvor høj genkendelsesprocent kan et system vi selv laver opnå?

Det system, vi ønsker at udvikle, egner sig bedst til en situation som i det første eksempel, hvor bilen står stille eller bevæger sig meget langsomt. Systemet laves som et bachelorprojekt. Vores forventning er ikke at systemet kan opnå en effektivitet svarende til etablerede systemer til genkendelse af nummerplader. Derimod ønsker vi via arbejdet at få erfaring med praktisk anvendelse af elementære teknikker indenfor billedbehandling og mønstergenkendelse.


\subsection*{Afgrænsning}
Systemet udvikles som et computerprogram. Systemet skal kunne analysere farvefotografier med en opløsning på minimum 1024 x 768 pixels taget ved højlys dag uden brug af kunstigt lys. Hvert fotografi skal forestille en bil med synlig nummerplade. Fotografierne skal tages fra en position direkte foran bilen og i en højde på mellem 160 og 190 cm i en afstand på 3 - 6 meter fra bilen uden zoom. Det er ikke nødvendigt at nummerpladen skal befinde sig i midten af billedet. Nummerpladen skal fremstå vandret på billedet og skal desuden være fri for urenheder eller lignende, der gør pladen sværere at læse. Ligeledes må der heller ikke være objekter såsom anhængertræk, der hindrer frit udsyn til nummerpladen.

Alle billeder skal indeholde en enkelt nummerplade og bilerne på billederne skal være parkerede. Nummerpladerne skal alle være danske privatplader som de udstedes i foråret 2008\footnote{Rød kant, hvid baggrund og sort tekst.} og af typen med en enkelt linie tekst. Formatet på nummerpladerne skal være \textit{AA XX XXX}, hvor \textit{A} er bogstaver i mængden \textit{A}-\textit{Z}\footnote{Enkelte tegn er muligvis ikke lovlige. Er dette tilfældet, skal systemet ikke tage højde for dem.}, og \textit{X} er heltal i mængden 0-9. Som eksemplet viser, skal der være mellemrum mellem de to bogstaver og de to første tal samt mellem de to første tal og de tre tal. Nummerpladerne må ikke indeholde andre tegn end de nævnte bogstaver og tal.

%Der skal tages 150 fotografier med to forskellige kameraer dvs. samlet 300 billeder. 

Der er intet krav til den tid det tager for systemet at analysere et fotografi.


%Evt. begrænsning i hvilke tegn vi ønsker at genkende? Man kunne evt. begrgnse sig til tallene 0-9 og udelade A-Z.

%Systemet skal ikke kunne genkende såkaldte ønskenummerplader hvor teksten er bestemt af ejeren af bilen. 


%Billedetype og -kvalitet, farvebilleder, nummerpladetype herunder form på nummerplade, kun biler Danske nummerplader, reglerne for danske nummerplader skal undersøges. Ingen ønskenummerplader, billeder taget af biler som ikke er i bevægelse, billeder kan være taget med skæv vinkel, evt. 45 grader. Der er nummerplader i alle billeder. Belysningen på billederne skal være fuld dagslys, uden kunstig belysning som eksempelvis blitz.

%Ingen real-time, brug af Matlab, programmet får et billede som inddata og returnerer en tekststreng med systemet læsning af nummerpladens tekst.

%Nummerpladerne indeholder alfanumeriske tegn, ingen bindestreg, punktum osv.

%Mønstergenkendelse

%Softwaren skal fungere ved slutningen af projektet, men det er ikke planen at den skal implementeres i et miljø hvor den faktisk kan anvendes.


%\subsection*{Begrundelse}

	
\subsection*{Arbejdsopgaver}
I det følgende har vi beskrevet de overordnede arbejdsopgaver, der skal udføres i løbet af dette projekt. Efter disse beskrivelser følger vores tidsplan.
%Et bachelorprojekt er beregnet til syv ugers fuldtidsarbejde, hvilket svarer til 250-300 timer pr. mand. I dette afsnit har vi afsat tid til de enkelte opgaver i forhold til denne tidsmængde. Følgende overordnede arbejdsopgaver skal udføres i projektforløbet (rækkefølgen af opgaverne skal ikke anses som den rækkefølge opgaverne skal udføres i; de skal i større eller mindre grad udføres parallelt):

\begin{enumerate}
\item Research: Læsning af artikler omkring nummerpladegenkendelse samt materiale omkring mønstergenkendelse og billedbehandling. %Denne opgave er estimeret til ca. 45 timer pr. mand.

\item Fotografering: Fotografering af biler med nummerplader. Disse fotografier skal bruges som inddata til systemet.%Denne opgave er estimeret til ca. 10 timer pr. mand.

%\item Design af system: Bestem et overordnet systemdesign. %Denne opgave er estimeret til ca. 15 timer pr. mand.

\item Implementering af system, herunder:

\begin{itemize}
\item[A:] Billedbehandling: Konstruktion af et program der tager et fotografi indeholdende en nummerplade som inddata og leverer 7 billeder indeholdende de enkelte tegn på nummerpladen som uddata. Programmet implementeres i MatLab som tilgås på DIKUs maskiner via SSH samt på vores egne computere. Vi ønsker at eksperimentere med forskellige metoder indenfor billedbehandling og opfatter denne del af projektet som den mest omfattende.%Denne opgave er estimeret til ca. 90 timer pr. mand.

%- Find nummerpladen i billedet
%- Afin transformation
%- Opdeling af tegn
\item[B:] Mønstergenkendelse: Konstruktion af et program der tager 7 billeder, som hver indeholder et individuelt tegn som inddata og leverer en tegnfølge på 7 karakterer svarende til analysen som uddata. Programmet skal som ovenfor implementeres i MatLab. I denne del af projektet vil vi som udgangspunkt benytte simplere metoder.%Denne opgave er estimeret til ca. 90 timer pr. mand.

% - Syntax analyse
\end{itemize}


%Søren skriver: Tilføj ovevejelse om vægtning af de to delopgaver under punkt 4. Skal der eksperimenteres med flere delmetoder, skal de sammenlignes, lægges der mere vægt på A) and på B) ????

%Søren: Altså, vi ved jo ikke noget om mønstergenkendelse, derfor er det jo oplagt at lægge mest vægt på billedbehandlingen. Dog mener vi at vi skal have et system der kan udskrive et gæt på hvad der står på en nummerplade på et fotografi. Derfor SKAL vi jo have noget mønstergenkendelse. Vi har svært ved at afgøre hvornår vi lægger et for stort brød op. Du skriver at vi skal skrive om vi vil eksperimentere med delmetoder. Vil det sige at du allerede nu vil have at vi skal have et kendskab til området der gør at vi fravælger visse netoder og tilvælger andre før vi rigtigt er begyndt på projektet?


\item Rapportskrivning: Udformning af en rapport som beskriver forløbet omkring projektet, implementering af systemet, evaluering af systemet m.m. %Denne opgave er estimeret til ca. 50 timer pr. mand.


\end{enumerate}
%\subsection*{Evt. metoder, information og informationskilder}

%De fire pdf'er henviser til etablerede metoder og vi vil bruge nogle af dem. Hvilke noteres her.

%Synopsisforsvar

%Det er en god ide at afholde forsvar af synopsis, da det kan give jer mulighed for at diskutere projektet med andre, så I kan få et bedre overblik over dets stærke og svage sider.

%Tal med jeres vejleder om synposisforsvar. Det kan enten ske overfor en af de andre grupper som han eller hun vejleder, eller jeres vejleder kan arrangere et synopsisforsvar overfor en anden gruppe i hans eller hendes forskningsgruppe.

%I kan godt være tre grupper sammen til forsvar af synopsis. I stedet for at en gruppe kommenterer en anden, vil det så være to grupper som giver kommentarer på hver af projekterne.

%Formålet med synopsisforsvar er ikke at overbevise de andre om at man har lavet et godt projekt. Formålet er at få kommentarer så kan kan finde de svage punkter i ens projektplan og få gennemført et godt projekt.

%Derfor er følgende det bedste forløb i gennemgangen af en synopsis:

%- Grupper introducerer kort deres projekt.
%- Opponentgruppen fortæller hvordan de har forstået emnet. Det er for at sikre at de ikke har misforstået det.
%- Opponentgruppen fortæller hvad de synes er specielt godt i synopsis. Det er vigtigt, da gruppen så ved hvad de kan bygge videre på og helst ikke skal pille ud af projektet.
%- Opponentgruppen fortæller hvad de synes er svage punkter og mulige problemer, og kommer gerne med ideer til hvordan de kan løses.
%- Derefter diskuterer man hvordan projektet kan gøres bedre.


\end{document}

\end{document}
