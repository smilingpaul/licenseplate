\section{Implementation}
\label{sec_implementation}
Noget med at vi har brugt Matlabs egne metoder hvis vi kendte til komandoer for det vi gerne ville opnå. F.eks. filter og blokprocess.

\fixme{beskriv hvordan koden er implementeret - dens opbygning og de fede ting ved den}
\subsection{Lokalisering af nummerplader}
I de følgende afsnit gennemgår vi væsentlige implementationsdetaljer i de funktioner vi bruger til lokalisering af nummerplader. Vi har skaleret ned for at spare tid.

\subsubsection{Metode: Områder domineret af lyse gråtoner}

\subsubsection{Metode: Områder med høj kontrast}
Pladerne bliver meget lave da vi kun kigger på kontrasten. Vi får altså ikke den hvide over- og underkant med. Derfor udvider vi kandidaten med 15 pixels på over- og underkanten. 

\subsubsection{Metode: Frekvensanalyse}
Metoden bliver dårligere jo mere man skalerer billedet ned da tegn i nummerpladerne ikke længere er klart separeret.

\subsubsection{Metode: Skru op for Kontrast}
Brugt matlabs metoder.

\subsubsection{Metode: Kvantifisering}


\subsubsection{Metode: Histogram}




\subsection{Separation}

\fixme{Der er problemer med rotation hvis pladen klippes så tæt at over/under eller venstre/højre kanter ikke kommer med.}

\subsection{Genkendelse af tegn}

Implementationen af metoden til tegngenkendelse ved hjælp af vektor analyse findes i filerne \textit{GetMeanVectors.m} og \textit{ReadPlateFV.m}.

\textit{GetMeanVectors.m} udarbejder en matrix bestående af middelvektorer for hvert lovligt tegn.

\textit{ReadPlateFV.m}...

\fixme{Beskriv problemer med tegnet K.}

\subsubsection{Syntaks analyse}

Implementationen af metoden til syntaks analyse findes i filen \textit{SyntaxAnalysis.m}.

Hvis der i på tegnfølgens to første positioner fremkommer en ulovlig bogstavkombination fås et problem: Hvilket tegn skal man forsøge at udskifte så en lovlig kombination forekommer? 