\subsection{Separation af tegn}

Separation af tegnene i en nummerplade system består i vores system af to dele: Først må billedet indeholdende nummerpladen roteres så nummerpladen står vandret i billedet, dernæst skal tegnene fra nummerpladen identificeres og separeres.

\fixme{indsæt separations-del af diagram over system}

Rotationsmetoden i dette afsnit returnerer orginalbilledet af en bil med nummerplade mens separationsmetoderne returnerer syv binære billeder. Hvert billede forestiller et tegn fra en nummerplade, skåret ud fra pladen og omdannet til et binært billede med selve tegnet som forgrund og nummerpladens hvide område som baggrund.

%\fixme{Hvad beskriver vi i dette afsnit? Hvilke trin udgør processen?}

\subsubsection{Rotation}
%Nogle overvejelser om hvordan billedet skal være klippet til for at rotationen fungerer ordentligt...
I dette afsnit beskrives det hvordan et billede af en nummerplade kan roteres, så nummerpladen optræder vandret i billedet. Metoden er fundet i...


\subsubsection*{Radon transformation}
Et billede af en nummerplade, hvor nummerpladen er let fordrejet i forhold til vandret, skal drejes før separation og genkendelse af tegn kan foregå. En Radon transformation beregner projektionen af et billede fra flere forskellige steder og i flere forskellige retninger/vinkler. Denne type transformation kan bruges til at finde linier i billedet, og til at undersøge i hvilken retning disse linier går. Disse oplysninger er brugbare mht. rotation af et billede af en nummerplade, så denne plade står vandret i billedet.

Billedet nedenfor viser hvordan Radon transformationen findes i en enkelt vinkel, theta. Den grå firkant er billedet der projekteres. De grå pile er sensorer/radial linier. Resultatet af en Radon transformation er en matrice der angiver i hvor høj grad det er sandsynligt at der findes en linie i original billedet for hver vinkel samt radial linie.

%SKAL NOK BYTTES UD MED HJEMMELAVET BILLEDE:
%\begin{figure}[h]
%\includegraphics{billedbehandling/illu/transform74.jpg}
%\caption{Radon transformation}
%\end{figure}

Før Radon transformationen kan foretages skal billedet kant-detekteres (ANDEN OVERSÆTTELSE), da dette giver en større chance for at Radon transformationen opfanger linierne i billedet. Når nummerpladens hældning i billedet er fundet, kan billedet roteres.

I Figur ?? er vist et eksempel på et billede af en nummerplade, et binært billede udfra samme billede hvor kanterne i billedet er markeret med hvid samt det samme billede af nummerpladen, blot roteret. Bemærk at kun de horisontale kanter detekteres i kant-billedet (mere om dette i afsnit ??).

\includegraphics{system/illu/rotate_example_input.jpg}
\includegraphics{system/illu/rotate_example_edge.jpg}
\includegraphics{system/illu/rotate_example_output.jpg}


\subsubsection*{Andre metoder brugt i litteraturen}

\fixme{Hough?}

\subsubsection{Identifikation af tegn}

%Se nrpl.dk: Hvert af de op til 7 tegn på nummerpladen har et imaginært "felt" de kan brede sig i. Ikke alle tegn er lige brede, og generelt er bogstaver bredere end tal. "Feltet" til bogstaver er derfor bredere end feltet til tegn. Enkelte tegn er for smalle til at udfylde deres "felt", så designeren har fundet det hensigtsmæssigt at placere disse tegn visuelt centreret inden for deres "felt".

%"Således vil en nummerplade som MB 20 001 på grund af venstrestillingen af det sidste 1-tal have større mellemrum mellem højre kant og sidste tal end mellem venstre kant og første bogstav"


I det følgende beskrives to metoder til identifikation og separation af tegn i en nummerplade. Disse to metoder kaldes Sammenhængende komponenter og Peak-to-valley (da.: top-til-bund). Metoderne er fundet i....



\subsubsection*{Sammenhængende komponenter}
Denne metode er bygget op omkring sammenhængende komponenter. Ideén bygger på at hvert enkelt af de syv tegn som findes i billedet af nummerpladen er é sammenhængende komponent. I dette tilfælde vil hver sammenhængende komponent bestå af mørke pixels mens baggrunden er lyse pixels.

Med sandsynlighed vil tegnene ikke være de eneste sammenhængende komponenter i billedet. Eksempelvis kan nummerpladen være beskidt, lyset på billedet kan være dårligt, billedet kan være klippet 'for løst' sådan at elementer uden for nummerpladen vil være sammenhængende etc. For at tegnene skal udskille sig klart fra den lyse baggrund er det nødvendigt at forstærke kontrasten i billedet. Figur ? viser forskellen på et billede af en nummerplade uden forarbejdning og et med hvor kontrasten i billedet er forstærket.

EKSEMPEL PÅ BILLEDE MED FORSTÆRKEDE KONTRASTER.

Efterfølgende oprettes et binært billede med sammenhængende komponenter. I Figur ? ses et billede af en nummerplade hvor de sammen hængende komponenter er hvide.

EKSEMPEL PÅ BILLEDE MED SAMMENHÆNGENDE KOMPONENTER

Som det ses er det ikke kun tegnene der er sammenhængende komponenter. Derfor gennemføres en analyse på komponenterne for at frasortere de komponenter der ikke kan være tegn. Vi kan frasortere komponenter som ikke opfylder følgende regler:

\begin{itemize}
\item[-] Hvert tegn fylder maksimalt 1/7 del af billedet.
\item[-] Hvert tegn har en maksimal bredde på 1/7 af billedets bredde.
\item[-] Tegnenes højde er større end deres bredde
\item[-] Tegnenes minimale højde skal være end en vis konstant (her: 5 pixels)
\item[-] Tegnenes minimale størrelse skal være større end en vis konstant (her: 5 pixels)
\item[-] For hvert tegn findes der seks andre tegn som er ca. lige så højde som det pågældende tegn.
\item[-] For hvert tegn findes der seks andre tegn som befinder sig i samme højde som det pågældende.
\item[-] Afstanden mellem tegnene 2 og 3 samt 4 og 5 i en tegnfølge på syv tegn er større end afstanden mellem de resterende tegn der støder op til hinanden.
\end{itemize}

EKSEMPEL PÅ BILLEDE HVOR IKKE-TEGN KOMPONENTER ER SORTERET FRA

De resterende komponenter må forventes at være nummerpladens tegn.

EKSEMPEL PÅ UDKLIPPEDE TEGN

\fixme{beskrevet i kwas: man ku fjerne kanter før analysen}

\subsubsection*{peak-to-valley}

Denne metode baseres på en vertikal projektion af nummerpladen. Denne projektion vil, med en lys nummerplade med mørke tegn, give os en indikation af hvor der er dale (de lyse områder mellem tegnene) og bakker (tegnene), hvorefter vi kan udskære tegnene.

Som i metoden med sammenhængende komponenter er det en fordel at forstærke kontrasten i billedet. HVORFOR? Nummerpladens signatur fås ved at summere intensiteterne i alle kolonnerne i billedet. En graf over signaturen vil så give toppe i de kolonner hvor der er størst inte

\subsubsection{Metoder fra litteraturen}

Segmentering af nummerpladens tegn kan gøres ved brug af peak-to-valley (dansk: top-til-bund) \cite{ron}, \cite{kwas} eller ved brug af såkaldte sammenhængende komponenter (hvert enkelt tegn i nummerpladen må hver især forventes at være én sammenhængende figur) \cite{nijhuis}. I alle tilfælde er det en god idé at kombinere de enkelte elementer af systemet med viden om størrelsesforhold, antal (én nummerplade pr. billede, syv tegn pr. nummerplade osv.) og lign \cite{nijhuis}, \cite{parker}, \cite{kwas}. Til aflæsning af tegn kan man bl.a. bruge neurale netværk \cite{nijhuis}, \cite{kwas} eller klassifikations vektorer \cite{arth}.
\fixme{Søren: til Radon: han skriver "nej"??}
\fixme{skal vi ha noget om testdata her?}




%%%%%%%%%%%%%%%%%%%%%%%%%%%
% Note fra segmentering af tegn
%%%%%%%%%%%%%%%%%%%%%%%%%%%%


\begin{comment}
\subsubsection*{Skan-linie}

Bruges ikke da det er for mange felter som bliver valgt. Kan måske gøres bedre ved filtrering før???

Først gøres billedet sort-hvis med im2bw. Her kan grænseværdi bestemmes med greythresh. Virker måske bedre at sætte grænseværdien lavt, så meget af billedet bliver hvidt.

Billedet skal skæres foroven og forneden. Dette gøres simpelt ved at finde den største pixl-sum i toppen af billedet og den største sum i bunden. Det antages så at disse max-summer er dele af nummerpladerne hvor teksten ikke er startet.

Step igennem vertikale linier: hvad sker før tegn, i et tegn, i slutningen af et tegn og efter et tegn.
\end{comment}


%%%%%%%%%%%%%%%%%%%%%%%%%%%
% Beskrivelser af PDfer
%%%%%%%%%%%%%%%%%%%%%%%%%%%%


\begin{comment}
\subsubsection*{Noter fra møde med Søren 20/2:}
identifikation: Se på en pixel, har naboer en kontrast farve?
En scan-linie: hvordan varierer kontrasten henover linien?
Adaboost - godt










\subsubsection*{cano.pdf}
Bruger kun gråtone info. Arbejder også med nummerplader som skal være læsbar for det menneskelige øje.

Metode:
Histogram - først normaliseres billedet
Sobel filter - fremhæver ikke-homogene områder
"A simple threshold and a sub-sampling" bruges til at vælge områder der kan være nummerpladen

Husker alle områder som kan være nummerplader så de forkerte først vælges fra i genkendelses-fasen. Bruger multi-hypothesis detection (ikke forklaret yderligere i teksten).

Feature vektorer: hver pixel i et træningsbilleder er blevet klassificeret som positiv (del af nummerplade) eller negativ (ikke del af nummerplade). Minimerer efterfølgende det negative sæt.

Bruger kd-træ data struktur og en "omtrent nærmeste nabo" søgeteknik.

\subsubsection*{ron}

* Find de gule (hos os: hvide) områder i billedet
* Forstør disse områder
* Find vinklen på nummerpladen ved brug af "Radon transform"
* Justering af nummerpladens konturer
* Unødvendige dele af billedet fjernes (kun nummerpladen tilbage)
* Billedet i gråtone, herefter gøres det binært
* Billedet normaliseres
* Tegn-inddeling vha. peak-to-valley

De brugte Matlab. De havde følgende relevante problemer og løsningsforslag til disse:

* Udtrækning af det gule område giver ofte fejl. Man kunne supplere denne udtrækning med en algoritme der indberegner at nummerplader har en klar signatur idet der er stærke grå-tone variationer i regulære intervaller (henover nummerpladen, mener de vel?)

* Hvis der er flere nummerplade-kandidater i billedet skal hver af dem testes.

\subsubsection*{nijhuis.pdf}

Bruger regler for nummerplader i systemet

    * Starter med kontrast “udstrækning”, bortfiltrering af støj (der tages selvfølgelig højde for billedkvaliteten i denne del)
    * Lokalisering af nummerpladen: fuzzy clustering algoritme som bruger karakteristikker som “gul-hed” og teksturer
    * Gul-hed er defineret af frekvenstabel lavet fra manuelt udklippede nummerplader
    * Tekstur: her ser man på grå-værdien af de 8 nabopixels
    * “global threshold” baseret på gennemsnitsværdien af gråtone: fås binært billede
    * Udfra regler (højde, bredde m.m) findes potintielle tegn
    * Nummerpladen gives kun videre til mønstergenkendelse hvis den indeholder det rette antal tegn

Dette system godkender 75\% af billederne. Problemer med skruer i nummerpladen. Problemer med global threshold – burde gøres på pr-tegn-basis.

\subsubsection*{parker.pdf}

Der bruges en algoritme der først lokaliserer elementer der kan være tegn/bogstaver hvorefter den udvælger et område som nummerplade

* Konverter til gråtone billede
* 5x5 filter, fjerne støj
* Find kanter i billedet vha. Shen-Castan kant-detektor
* Gør billedet binært og del elementer i forgrunden fra hinanden
* Algoritme der finder bogstaver på baggrund af forskellen i gråtone værdien af bogstav/baggrund
* Områder hvor der ikke er (det rette antal) bogstaver udelukkes
* For at finde nummerplade bruges genetisk algoritme der bedømmer rektangler med tegn: har de den rette størrelse? er bogstaverne korrekt placeret i rektangel? osv.
* Algoritmen vægter hvert område og filtrerer til sidst i disse områder udfra deres vægt

Stort problem: svært at finde tegn.

\subsubsection{kwas.pdf}
Billedet skiftes til farverummet YUV fra vilket luminans er det eneste der bevares. Herefter normaliseres billedet (Hele den diskerete "range" udnyttes). Kigger på skift i kontrast. Gælder alle nummerplader. Finder alle tekster. Den rigtige skal vælges.

Identifikation af plader:

1. "Connected components analysis" (der kigger på et binært billede?) vælger områder med høj kontrast (threshold). De fundne områder undersøges og områder elimineres efter regler i pdf.  Herefter, er der lignenede grupper i nærheden af funden gruppe? Måske er der en serie tegn dvs. en sætning = plade.

2. Searching for signatures of license plates. Et karakteristisk skift i luminans i en linie i billedet.

Potentielle plader roteres så de er vandrette.

Segmentering af tegn:
Scan af peaks og valleys samt analyse af sammenhængende grupper fra identifikationsprocessen sammenlignes og segmentering foretages.

Mønstergenkendelse foregår med neuraltnetværk.

\subsubsection{shapiro.pdf}
For at sætte hastigheden op foretages visse operationer på kraftigt nedsamplede billeder. Finder lodrette linier. Bruger Robert's edge detector til at fremhæve dem (Tegner den på billedet?). Dette efterlader en masse lodrette linier i området med nummerpladen. Et Rank filter? bruges på billedet. Efterlader en lys elipse i det område hvor pladen findes. Scanner billedet lodret for at finde det lyseste område og klipper ud (Klipper et noget større område end pladen ud på eksemplet i pdf'en). Der er formler i beskrivelsen. Roter pladen hvis skæv (Formler i pdf). Nohet med at finde linier i billedet og rotere stlsvarende (Hough og Radon transform). Det er først når vi skal genkende tegnene på pladen vi bruger billedet i sin originale opløsning.
\end{comment}


