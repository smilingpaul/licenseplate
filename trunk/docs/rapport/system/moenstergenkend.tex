\subsection{Genkendelse af tegn}

\label{sec_monster}
Til genkendelse af tegn har vi arbejdet med tre forskellige metoder: Den første bruger informationer fra en såkaldt middelvektor, den anden arbejder med såkaldte sum-billeder mens den tredje er en simpel metode. I dette afsnit vil vi beskrive disse tre metoder samt en metode til at udføre syntaksanalyse på den tegnfølge der repræsenterer nummerpladens indhold.

Metoderne til genkendelse af tegn arbejder på de syv binære billeder som separationsdelen af systemet returnerer. Metoderne returnerer syv hitlister (én for hvert billede) med tegn. Det tegn der optræder øverst på den enkelte hitliste er metodens bedste gæt på det tegn billedet forestiller, nummer to på listen er det næstbedste gæt osv. Syntaksanlysen arbejder med disse hitlister som inddata og returnerer en tegnfølge.

Til de tre genkendelsesmetoder har vi uarbejdet et træningssæt betående af billeder af tegn, sorteret efter tegnets type (\textbf{A}, \textbf{B}, \textbf{C} etc.). Udfra dette træningssæt laver vi de enkelte træningsbilleder.
%SKAL DER IKKE STÅ AT VI HAR LAVET ET TRÆNINGSÆT BESTÅENDE AF TEGN? VEKTORERNE I DET FØLGENDE BLIVER DANNET PÅ BAGGRUND AF DETTE SÆT.

\begin{figure}[htp]
\centering
\includegraphics[width=12cm]{system/illu/dia_trin3.png} 
\caption{Inddata, i form af syv billeder af syv tegn kommer ind i systemet fra venstre. ..... Uddata er .... De grå kasser...}
\label{fig:dia_trin3}
\end{figure}

\subsubsection{Metode: Middelvektorer}
En egenskabsvektor (\textit{eng.: feature vector}) er en vektor der repræsenterer et objekts egenskaber, i dette tilfælde er objektet et billede og egenskabsvektorne kaldes middelvektorer. Idéen i denne metode er, at hvert billede af et tegn sammenlignes med middelvektorer for at finde det tegn som er i billedet. Der oprettes altså en middelvektor for hvert tegn. Metoden er fundet i \cite{arth}.

Først oprettes en middelvektor for hvert tegn der kan forekomme i en nummerplade. Det vil sige at der oprettes en middelvektor for tegnet \textbf{A}, tegnet \textbf{B} og så videre. Dette vektorsæt er et træningssæt (systemet "trænes" til at kende de forskellige typer af tegn). Vi har defineret en middelvektor, $M$ for et antal billeder $(B_{1},B_{2},...,B_{n})$ i et træningssæt for et givent tegn således:

\begin{displaymath}
	M = (\sum_{i=1}^{n}B_i)/n
\end{displaymath}

ILLU AF 2D-AFSTANDE MELLEM VEKTORER.

hvor summen er en elementvis sum DVIVSION ER VEL OGSÅ ELEMENTVIS FODNOTE OM ELEMENTVISE. For et inddatabillede af et tegn oprettes der tillige en vektor, som herefter sammenlignes med vektorerne i træningssættet for at finde den vektor der ligger nærmest billedets vektor. Tegnet for den vektor der ligger nærmest vælges.

I vores system findes afstanden, $D$ fra én (middel)vektor, $M = (m_{1},m_{2},...,m_{n})$ til en anden vektor, $P = (p_{1},p_{2},...,p_{n})$ i et $n$-dimensionelt rum ved brug af den euklidiske afstand\cite{wiki_euclid}:

\begin{displaymath}
	D = \sqrt{\sum_{i=1}^{n}(m_{i}-p_{i})^{2}}
\end{displaymath}

\subsubsection{Metode: Sum-billeder}

ILLU AF ET SUM-BILLEDE FOR ET TEGN

Denne metode minder om metoden med egenskabsvektorer. Her er idéen dog, at der oprettes såkaldet "sum-billeder" for hvert tegn i stedet for middelvektorer. Metoden er opstået af egen idé.

SVARER TIL? DET ER VEL LIGE PRÆCIS DET VI GØR ER DET IKKE?
Hvert binære billede i træningssættet for et givent tegn skaleres til en defineret størrelse, $I \times J$ og summeres med de andre billeder i sættet. Et sum-billede, $S$ konstrueres af billederne $(B_{1},B_{2},...,B_{n})$ altså således:

\begin{displaymath}
	S = \sum_{x=1}^n{B_x}
\end{displaymath}

hvor summen er en elementvis sum. Efter denne summering normaliseres $S$, ved at dividere alle elementer i $S$ med $\max{(S)}$. Alle værdier i $S$ vil nu være mellem $0$ og $1$.

For hvert binære testbillede, $T$, der først skaleres til samme størrelse som $S$, multipliceres alle elementerne i $T$ med de tilsvarende elementer i $S$ og summen, $v$ tages af disse:

\begin{displaymath}
	v = \sum_{i=1}^I{\sum_{j=1}^J{T_{ij} \cdot S_{ij}}}
\end{displaymath}

$v$ svarer altså til summen af elementer fra $S$ hvor elementet med samme koordinater i $T$ er markeret. Når $v$ er fundet for alle tegn for et testbillede, vælges tegnet hvor $v$ er maksimal.

NOGET OMKRING HVORDAN IDEEN ER OPSTÅET?

\subsubsection{Metode: Simpel metode}

I denne metode tager man alle de binære billeder i et træningssæt for et givent tegn og ser på hvilke pixels der altid er markeret i billeder af det givne tegn. Metoden er hjemmelavet.

And-billedet, $A$ laves ud fra billederne $(B_{1},B_{2},...,B_{n})$ af et givent tegn som følger:

\begin{displaymath}
A = B_1 \wedge B_2 \wedge ... \wedge B_n
\end{displaymath}

hvor AND'INGEN er elementvis. Et testbillede, $T$ and'es så med hvert enkelt $A$:

\begin{displaymath}
K = T \wedge A
\end{displaymath}

og antallet af 1'taller i $K$ optælles. Det tegn, hvor antallet af 1'taller i $K$ er maksimalt blandt alle tegnene, vælges som tegnet der repræsenteres af testbilledet.

\subsubsection{Syntaksanalyse}

Ved syntaksanalyse analyseres tegn-hitlisterne fra de ovenstående metoder. Analysen er relevant da et gæt fra en af de to metoder måske kan udelukkes på baggrund af overtrædelse af syntaktiske regler for hele tegnfølgen. Metoden bruges i \cite{nijhuis} og \cite{kwas}.

Der kan forekomme følgende fejl i en tegnfølge fra en nummerplade i vores system:

\begin{itemize}
\item Et tal er placeret på en af de første to pladser.
\item Et bogstav er placeret på en af de sidste fem pladser.
\item Den fundne bogstavkombination er ikke tilladt\footnote{En oversigt over lovlige og ulovlige bogstavkombinationer er fundet i \cite{bogstav_komb}}.
\item Den fundne talkombination er ikke tilladt (den samlede værdi af tallene er enten for høj eller for lav)\footnote{En oversigt over tilladte talværdier er fundet i \cite{nrpl}}.
\end{itemize}

Hvis en eller flere af disse muligheder er gældende itererer (DEN KIGGER GENNEM LISTEN?) metoden ned igennem hitlisterne indtil et lovligt valg af en kombination af tegn er fundet. Dette kan illustreres med følgende eksempel: Tegnfølgen \textbf{DO 45 7B3} er retuneret fra mønstergenkendelse. I denne tegnfølge forekommer der to fejl: bogstavet \textbf{O} bruges ikke på 2. position og bogstavet \textbf{B} i 6. position burde have været et tal. På 2. position itererer metoden ned igennem hitlisten til den når et bogstav som sammen med \textbf{D} danner en lovlig bostavkombination og på 6. position itereres der indtil et tal findes.

I systemet sættes en øvre grænse for hvor langt ned ad hitlisten syntaksanalysen kan vælge tegn. Hvis denne grænse ikke fandtes, ville der blot kunne itereres indtil en lovlig tegnfølge er fundet, og så ville metoderne til genkendelse af tegn give mindre mening HVIS MAN KOMMER LANGT NED I LISTERNE ER USIKKERHEDEN HØJ, DER SKAL ALTSÅ VÆLGES NOGET DER LIGNER RET MEGET. ELLER NOGET!

\subsubsection{Metoder fra litteraturen}

\subsubsection*{Neurale netværk}
I \cite{kwas} arbejder man med et simpelt neuralt netværk hvor antallet af neuroner i inddata-laget er lig størrelsen på en matrix der repræsenterer et billede af et tegn. Antallet af neuroner i uddata-laget er lig antallet af tegn.

ANDRE?
