\label{sec_billed}

\subsection{Lokalisering af nummerplader}
\fixme{Hvor kommer metoderne fra?}
\fixme{nedenfor i billedbehandling: der mangler intro til valg af metoder}

%I dette afsnit beskrives de tre første elementer af vores system: identificering af nummerpladen i billedet, rotation af nummerpladen samt segmentering af tegn i nummerpladen.


%\subsection{Identifikation af nummerplader}
I arbejdet med at lokalisere nummerplader, har vi valgt at arbejde med flere forskellige metoder. Det er vores tanke, at vi ved at bruge metoderne sammen kan opnå et bedre resultat end ved at bruge metoderne hver for sig. F.eks. vil et område som udpeges som værende en nummerplade af flere metoder have en høj grad af troværdighed mens det i en situation hvor metoderne alle udpeger forskellige områder vil være meget usikkert hvor nummerpladen befinder sig. I dette afsnit beskriver vi de forskellige metoder vi benytter os af, samt hvordan disse fungerer som et samlet system til lokalisering af af nummerplader. 

%\subsubsection*{Interesseområder}
Metoderne i dette afsnit, fungerer alle på den måde at de returnerer et binært\footnote{Et billede der kun består af farverne hvid og sort} billede hvor områder der opfattes som interessante, det vil sige potentielle nummerplader, er markeret med farven hvid. I den bedst tænkelige, og meget urealistske, situation er nummerpladeområdet markeret med hvidt mens resten af billedet er sort som vist på figur \vref{fig:binary_ideal}. I den realistiske situation er der mange interesseområder markeret i det binære billede som det kan ses på figur \vref{fig:binary_real}. Vi kalder disse områder for kandidater, og beskriver hvordan vi vælger mellem dem i det senere afsnit \textbf{Analyse af kandidater} der begynder på side \pageref{sec:kandidater}.
\fixme{Skal der refereres til billedet af bilen?}

\begin{comment}
\begin{figure}[htp]
\centering
\framebox{\includegraphics[width=5cm]{illu/B_XC33139.jpg}} 
\caption{Et billede hvor vi ønsker at lokalisere nummerpladen}
\label{fig:input_billede}
\end{figure}


\begin{figure}[htp]
\centering
\includegraphics[width=5cm]{system/illu/binary_ideal.png} 
\caption{Det bedst tænkelige billede af nummerpladekandidater i billedet på figur \ref{fig:input_billede}}
\label{fig:binary_ideal}
\end{figure}


\begin{figure}[htp]
\centering
\includegraphics[width=5cm]{system/illu/binary_real.png} 
\caption{Et realistisk eksempel på nummerpladekandidater i billedet på figur \ref{fig:input_billede}}
\fixme{Eksempelbilledet skal være et der viser situationen før der sorteres fra?}
\label{fig:binary_real}
\end{figure}
\end{comment}

\subsubsection{Metode: Områder domineret af lyse gråtoner}
\fixme{Søren: Næppe realitisk denne metode, afhængig af belysning}
I denne metode forsøger vi at finde nummerpladen i et billede ved at lede efter områder der domineres af lyse gråtoner i et farvebillede. Vi håber at en af de markerede interesseområder vil være nummerpladen da dens baggrund består af lyse gråtoner. I et farvebillede vil det sige, at området vil være domineret af pixels hvor de tre farvekanaler, rød, grøn og blå har intensitetsværdier der ligger forholdsvis tæt på hinanden. F.eks. vil en helt rød pixel have værdien 256 i den røde farvekanal mens værdierne i den grønne og blå farvekanal være 0. I dette tilfælde er der altså tale om en meget skæv fordeling af intensistestvrædierne på de tre farvekanaler. Et eksempel på en jævn fordeling af intensitetsværdierne er en middelgrå pixel der vil have værdien 128 i alle tre farvekanaler.

For at markere interesseområder udregner vi en gennemsnitsværdi for hver pixel i farvebilledet ved at summere værdierne for hver af de tre farvekanaler og dividere med tre. Hvis alle tre kanaler har en værdi der ligger tæt\footnote{Et parameter i programmet} på middelværdien og denne samtidig er høj, det vil sige lys, \footnote{Et parameter i programmet} farves den indeværende pixel i det binære billede der viser interesseområderne hvid. Figur \ref{fig:binary_DetectSameness} viser et eksempel på interesemetoder markeret med denne metode. Denne metode er implementeret i funktionen \textit{DetectSameness} hvis kildekode findes i afsnit \vref{code:DetectSameness}. 

Hvis $X$ er en nedre grænseværdi for de intensiteter vi opfatter som lyse. $Y$ er en værdi der bestemmer hvor stor afstand der må være mellem intensiteterne i de tre farvekanaler og $I_{ij}$ er pixels i et farvebillede hvis nummerplade vi ønsker at finde, er pixels i det binære billede $B_{ij}$ defineret som:
\begin{equation}
B_{ij} = 
\begin{Bmatrix}
1 & \text{Hvis } (R(I_{ij})+G(I_{ij})+B(I_{ij})/3) > X\\
 & R(I_{ij}) > mean - Y & < Y + mean\\
0 & \delta
\end{Bmatrix}
\end{equation}
\fixme{jeg ved ikke hvordan dette skrives ind i latex}

\begin{figure}[htbp]
  \centering
  \begin{minipage}[b]{5 cm}
    \framebox{\includegraphics[width=4cm]{illu/B_XC33139.jpg}}
  \end{minipage}
  \begin{minipage}[b]{5 cm}
    \framebox{\includegraphics[width=4cm]{system/illu/binary_DetectSameness.png}}  
  \end{minipage}
  \caption{Det binære billede til højre viser viser de interesseområder metoden markerer når inddate er billedet til venstre}
  \label{fig:binary_DetectSameness}
\end{figure}


\subsubsection{Metode: Områder med høj kontrast}
Vi forventer at et område med en nummerplade vil indeholde områder med høj kontrast. Endvidere forventer vi at mange af disse kontrastområder vil være områder hvor den det ene intensitetsområde ligge ved siden af det andet i modsætning til over hinanden (beskriv nøjere? antagelse om vandrette plader). 

F.eks. har bogstavet I lyse områder på både venstre og højre side. Der er altså to lodrette linier der markerer overgangen mellem lyse og mørke områder. Ved at lave billedet om til gråtoner og derefter betragte intensiteterne i billedet som højder i et landskab kan vi beskrive hældninger i landskabet med såkaldte gradienter der beskriver hældningen i en pixel som en to dimensionel vektor(en illu). Et bredt område med en blød overgang i intensitet fra helt sort til helt hvid, vil have kortere vektorer i alle pixels mens en brat overgang fra sort til hvid vil resultere i meget lange vektorer for de pixels der ligger op ad overgangen. Ved brug af disse gradienter laver vi et nyt billeder der viser gradienternes vandrette længde. Peger en gradient f.eks. lodret op i billedet vil den ikke blive markeret, hvorimod den vil blive markeret hvis den er vandret og f.eks. peger mod højre. Lyse pixels i dette billede indikerer store gradienter. For at gøre områder med høje intensiteter mere sammenhængende filtrerer vi billedet med et ikke kvadratisk filter der er flere gange bredere end det er højt. BESKRIVELSE. Resultater et et udglattet billede hvor intensitetsområder der ligger ved siden af hinanden er blevet forbundet. Ud fra dette billede laver vi et binært billede med kandidatområder markeret med hvidt. (Henvisning til litteratur.)

ORIG\_IM BIN\_IM.

%\paragraph*{Analyse}
%\paragraph*{Implementation}
%detect6

\subsubsection{Metode: Frekvensanalyse}
\label{sec_frekvensanalyse} 
Ideen bag denne metode til markering af interesseområder er, at kigge på frekvensen af skift mellem lyse og mørke intensiteter i vandrette linier i et gråtonebillede. For at kunne markere områder der har en frekvens svarende til en nummerplade, skal vi først finde ud af hvad denne representative frekvens er. Derfor har vi analyseret nummerpladeområderne på vores testbilleder og udregnet den gennemsnitlige frekvens. Med denne viden kan vi skabe et binært billede hvor de markerede interesseområder er dem hvor frekvensen ligger tæt på den udregnede idealfrekvens. (ref til pdfer med signaturer)

ILLU\_SVINGNINGER\_PLADE ILLU\_SVINGNINGER\_IKKE\_PLADE

\fixme{figurer af histogrammer og lign.}

%\paragraph*{Analyse}
%\paragraph*{Implementation}

\subsubsection{Metode: Skru op for Kontrast}
%Lav et histogram. Lad et vindue køre hen over histogrammet. Vælg den position hvor der er mest info i vinduet. Stræk dette område så det fylder helle intensitetsspekter fra 0 til 255. Lav binært billede ud fra dette high-contrast-billede. Er det en metode eller en forberedelse af billedet før en af de andre metoder bruges?

%ORIG\_IM CONTRAST\_IM BIN\_IM.

%\paragraph*{Analyse}
%\paragraph*{Implementation}

\subsubsection{Metode: Kvantifisering}
Kun 8 farver i billedet.

\subsubsection{Metode: Histogram}
\label{sec_histo}

Idéen i denne metode er at hver pixel i et fotografi stilles op mod en frekvenstabel, indeholdende farvefrekvenser for en mængde nummerplader. Hvis pixelens farve svarer til en farve med høj frekvens i denne tabel, er det sandsynligt at den er en del af en nummerplade.

%Beskrevet i LicensePlateSydney.pdf

%Frekvenstabel, hvordan opbygges den?
%Frekvenstabellen indeholdende farvefrekvenserne for pixels i en række billeder af nummerplader blev lavet som følgende: Tabellen blev udformet vha. funktionen make\_freq\_table. I funktionen itereres der gennem alle pixels i et billede og deres RGB værdier noteres i en 255 x 255 x 255 matrice. Eksempelvis

Først oprettes en frekvenstabel udfra nogle billeder, hvor nummerpladens placering i billedet allerede er specificeret. Her itereres der gennem alle pixels i et billede og antallet af farver med en bestemt farvekombination summeres i en $255 \times 255 \times 255$ matrice, med plads til én frekvensværdi, $f$ for hver farvekombination. Når frekvenstabellen er udformet, normaliseres den. Det medfører at den RGB værdi der har den højeste frekvens, $f_{max}$ får værdien 1, mens de resterende RGB værdier får værdien $f$/$f_{max}$. Hvis RGB værdien 240, 240, 230 eksempelvis har den højeste frekvens, 55 og RGB værdien 230, 230, 220 har frekvensen $35$, får førstnævnte værdien 1 og sidstnævnte værdien $35/55 = 0,64$.


I systemet udarbejdede vi to frekvenstabeller. Én for billederne taget med et Canon digitalkamera og én for billederne taget med et Olympus digitalkamera. Grunden til at oprette disse to adskilte tabeller er, at vi får mulighed for at undersøge forskellen på brug af kamera. Eksempelvis kan det undersøges om en frekvenstabel lavet udfra billeder fra ét kamera kan bruges til at identificere nummerplader i billeder fra et andet kamera.


%Vi udvalgte tilfældigt hhv. 70 og 50 billeder fra de to grupper til de to frekvenstabeller og gemte disse tabeller i to seperate filer. Systemet kunne herefter bruge disse frekvenstabeller uden at skulle skabe dem først.

Ved hjælp af frekvenstabellerne skabes et billede hvor hver pixel gives en værdi fra 0 til 1 afhængig af om pixelens farve fremkommer sjældent (0) eller hyppigt (1) i en nummerplade. Disse værdier svarer altså til de værdier der står i frekvenstabellerne. I eksemplet ovenfor, ville en pixel med RGB værdien 230, 230, 220 få værdien 0,64 i tilhørsbilledet. Dette tilhørsbillede vil, med værdier fra 0 til 1, være et gråtone billede hvor de pixels, der har samme farve som de farver der oftest optræder i billeder af nummerplader, vil blive lyse og omvendt vil de resterende pixels blive mørke. De pixels der har de højeste frekvenser bliver altså interesseområder.

TO ILLUSTRATIONER: ORIGINAL BILLEDE OG TILHØRSBILLEDE

%\paragraph{Analyse}

%\paragraph{Implementation}

%Omdannes dette billede til et binært billede vil vi (ved succes) få forbundne komponenter bl.a. der hvor nummerpladen befinder sig i billedet. Disse komponenters størrelse, form osv. kan herefter analyseres og man kan give et bud på hvor nummerpladen befinder sig.

\subsubsection{Analyse af kandidater}
\label{sec:kandidater}
Ud fra de binære billeder der viser interesseområder, skal vi forsøget at finde de områder der mest sandsynligt er nummerplader. (noget med sammenhængende områder og dilate/erode). Vi begynder udvælgelsesprocessen med at frasortere de områdervi mener kan udelukkes som værende nummerplader. Vi sletter områder med følgende karateristika:
\begin{itemize}
\item Området er meget lille.
\item Området er meget stort.
\item Området er højere end det er bredt.
\item Området er for bredt i forhold til højden\footnote{I forhold til højde/bredde forholdet for en nummerplade}.
\item Området har få markerede pixels i forhold til det rektangel det udspænder.
\end{itemize}


I det resulterende binære billede giver vi de tilbageværende områder point efter deres højde/bredde forhold og deres frekvens. Et område der udspænder et rektangel med samme højde/bredde-forhold som en nummerplade får 0 point og et område der afviger meget fra det ideelle højde/bredde-forhold får et højt antal point.

Herefter kigger vi på en værdi der beskriver antallet af skift mellem meget lyse og meget mørke områder i kandidatområdet. Princippet er det samme som metoden beskrevet afsnittet \textbf{Frekvensanalyse} på side \pageref{sec_frekvensanalyse}. Forskellen er, at vi i det fåregående afsnit kiggede på linier der er indeholdt i nummerplader, mens vi her kigger på et helt områder der er større end selve nummerpladen. Der er en væsentlig forskel, da starten og slutningen på en nummerplade er meget karakteristisk. Udfra vores testdata har vi udregnet en gennemsnitsværdi for svingninger i nummerpladeområder. Vi sammenligner kandidaterne og giver dem point i forhold til hvor tæt de ligger på den på forhånd bestemte gennemsnitsværdi. Jo mere et område afviger fra det ideelle antal svingninger, jo flere point får det.

ILLU\_PLATESIG\_M\_AVG



%Signatur: avg plateness ganges op da plate sig ser ud som den gør.
ILLU MED TRIN I FRAVÆLGELSE AF KANDIDATER
%\paragraph*{Analyse}
%Ville være hurtigere at ikke slette men at fravælge. Størrelser. Må man det? Generelt vs. ikke generelt.


%\paragraph*{Analyse}
%Ville være hurtigere at ikke slette men at fravælge. Størrelser. Må man det? Generelt vs. ikke generelt.

%\paragraph*{Implementation}


%detect4
\subsubsection{Vælg nummerpladekandidat på baggrund af alle metoder.}


\subsubsection{Metoder fra litteraturen}
\cite{parker} bruger kant-detektion og lokalisering af tegn til at give et bud på hvor nummerpladen er.

Identificering af nummerpladen i billedet kan gøres ved blot at søge efter hvide områder i billedet \cite{ron}, \cite{nijhuis}, eller udfra viden om kanter i billedet \cite{shapiro} og eventuelt kombinere det med teksturanalyse \cite{parker}. Derudover kan man analysere kontrasterne i billedet \cite{kwas}, da en nummerplade må forventes af bestå af kontrastfyldte (sort/hvide) områder. Rotation af nummerpladen kan gøres ved brug af Radon transformation, som kan afgøre hvor der findes linier i billedet, samt hvilken vinkel disse linier har \cite{ron}, \cite{shapiro}. 

\subsubsection*{MSER}
Først sorteres alle pixels intensiteter (stigende). Dette gøres ved brug af bucket sort.


\begin{comment}
\subsubsection{Tobias' brainstorm}
Kan man kigge på højde bredde på components?
Scan linie. Man kan både kigge på components og "signatur" som beskrevet andetsteds(pdf).
Med gradienter kan man finde f.eks. lodrette men ikke vandrette streger. Der er mange lodrette i pladen.
Man kan kigge på en f.eks. 8x8 og se hvor mange komponenter der er tilstede. Der er mange komponenter i et 
lille område i nummerpladen(eller hvad?).

Kør en scanlinie. Noter kraftige gradienter. Hvis de er "tæt" på hinanden er det godt. giv "point"

Det her dropper jeg  indtil videre:
Kør en vertikal scanlinie. Hvis vi møder en gradient begynder vi at "tegne" en streg hvis intensitet 
falder. Den tegner altså en savtak når den møder en gradient. Hvis der er flere høje gradienter i træk får 
vi så en kasse med en skrå afslutning. Er det ikke en nummerplade? 


Nu gør jeg:
Find gradienter i billedet. Lav et binært billede hvor de steder hvor gradienten er større end (0.5 * den 
maximale gradient) er markeret med hvidt.

Der er nu mange markeringer i nummerpladeområdet.

Jeg tænker. Samme princip som med kun at vise maksimale gradienter:
Kig på summen (mængden af hvidt) af alle vandrette linier i billedet. "slet" de linier hvor der er mindre 
end (0.5 * summen af den linie med mest hvidt). Jeg burde nu have fjernet støj men stadig have pladen. Jeg 
statser på at pladen er på de linier med mest hvidt.

\end{comment}

\begin{comment}
Diskuter om vi skal vælge en kandidat når de forskellige metoder har forskellige kandidater. Parkering: man kunne bede bilen om at bakke og tage et nyt billeder hvis der ikke kan findes en fælles kandidat. Fartkontrol: ikke muligt med nyt billede, gær derimod på den kandidat der er returneret af den mest pålidelige metode. 

Regler:

- Kant efterfulgt af rød stribe (ovenover eller nedenunder)
- Pixel del af sammenhængende kæde som er længere end et vist antal pixels
- En linie hvorpå der er en anden retvinklet linie, og linierne har et vist forhold
- Længde af linie

\end{comment}
