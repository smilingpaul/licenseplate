\documentclass[10pt,a4paper,final]{report}
\usepackage[utf8]{inputenc}
\usepackage{ucs}
\usepackage[danish]{babel}
\usepackage{amsmath}
\usepackage{amsfonts}
\usepackage{amssymb}
\usepackage{verbatim}
\usepackage{listings} % Better source code listings
\usepackage{graphicx}
\author{Tobias Balle-Petersen og Esben Paul Bugge}
\title{Synopsis for Bachelorprojekt 2008}

\parindent=0pt
\parskip=8pt
 \usepackage[top=3cm, bottom=3cm, left=4cm, right=4cm]{geometry} 

\lstset{language=python}
\lstset{inputencoding=latin1}
\lstset{extendedchars=true}
\lstset{breaklines=true}
\lstset{commentstyle=\textit}
\lstset{showstringspaces=false}
\lstset{numbers=left, numberstyle=\tiny, stepnumber=2, numbersep=5pt,tabsize=3,basicstyle=\small}
%\lstset{numbers=left, numberstyle=\tiny, stepnumber=2, numbersep=5pt,stringstyle=\ttfamily, showstringspaces=false, basicstyle=\small, language={python}}


\begin{document}
\maketitle

\subsection*{Titel}
Projektets titel: Computer vision: Genkendelse/identifikation/læsning/ af nummerplader
På billeder
\subsection*{Problemformulering}
I diverse sammenhænge er der behov for at kunne identificere/læse nummerplader. Eksempelvis bruges det i politiets arbejde
\subsection*{Afgrænsning}
Billedetype, nummerpladetype herunder form på nummerplade, kun biler
Danske nummerplader, reglerne for danske nummerplader skal undersøges
Ingen personlig nummerplader

\subsection*{Begrundelse}
\subsection*{Arbejdsopgaver}
\subsection*{Evt. metoder, information og informationskilder}

Hvad skal der tages stilling til?

- Nummerpladetype
- Matlab vs. hurtigere sprog
- Billeder der skal bruges og kvalitet af disse

Synopsisforsvar

Det er en god ide at afholde forsvar af synopsis, da det kan give jer mulighed for at diskutere projektet med andre, så I kan få et bedre overblik over dets stærke og svage sider.

Tal med jeres vejleder om synposisforsvar. Det kan enten ske overfor en af de andre grupper som han eller hun vejleder, eller jeres vejleder kan arrangere et synopsisforsvar overfor en anden gruppe i hans eller hendes forskningsgruppe.

I kan godt være tre grupper sammen til forsvar af synopsis. I stedet for at en gruppe kommenterer en anden, vil det så være to grupper som giver kommentarer på hver af projekterne.

Formålet med synopsisforsvar er ikke at overbevise de andre om at man har lavet et godt projekt. Formålet er at få kommentarer så kan kan finde de svage punkter i ens projektplan og få gennemført et godt projekt.

Derfor er følgende det bedste forløb i gennemgangen af en synopsis:

- Grupper introducerer kort deres projekt.
- Opponentgruppen fortæller hvordan de har forstået emnet. Det er for at sikre at de ikke har misforstået det.
- Opponentgruppen fortæller hvad de synes er specielt godt i synopsis. Det er vigtigt, da gruppen så ved hvad de kan bygge videre på og helst ikke skal pille ud af projektet.
- Opponentgruppen fortæller hvad de synes er svage punkter og mulige problemer, og kommer gerne med ideer til hvordan de kan løses.
- Derefter diskuterer man hvordan projektet kan gøres bedre.

\end{document}