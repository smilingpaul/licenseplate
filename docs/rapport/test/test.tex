\section{Resultater}
\label{sec:resultater}
I dette afsnit beskriver vi afprøvningen af vores software. Vi afprøver alle trin i lokalisering og læsning af nummerplader individuelt før vi afprøver det samlede system. Som det første undersøger vi effektiviteten af metoderne til lokalisering af nummerplader. Derefter afprøver vi de to metoder til at separere tegn i nummerpladerne. Som den sidste individuelle afprøvning, undersøger vi forskellige metoder til genkendelse af tegn.

På baggrund af resultaterne af de individuelle afprøvninger, vælger vi den optimale konfiguration af systemet. Vi afprøver det samlede system på kontrolsættet med og uden syntaksanalyse af de nummerplader systemet læser.

Til sidst afprøver vi HVAD på billederne der er taget med blitz samt billederne af biler med gule nummerplader.

%%%%%%%%%%%%%%%%%%%%%%%%%%%%%%%%%%%
%%% LOKALISERING AF NUMMERPLADE %%%
%%%%%%%%%%%%%%%%%%%%%%%%%%%%%%%%%%%

\subsection{Lokalisering af nummerplader}
Vi undersøger vores systems evne til at lokalisere nummerplader. Vi kigger altså ikke på nummerpladens tegn, men blot på om vi kan afgøre dens position i inddata-billedet. Vi foretager afprøvning af metoderne både individuelt og som en samlet metode.

\subsubsection{Individuel lokalisering}
Vi afprøver effektiviteten af de fem metoder til lokalisering af nummerplader. Resultaterne af aprøvning på trænings- ok kontrolsættet er vist i figur \vref{fig:test:lokalisering_traening_kontrol}.

Alle metoderne giver dårligere resultater på kontrolsættet end på træningssættet. Størst forringelse observeres i metoden der kigger på områder domineret af lyse gråtoner. Dette kan skyldes væsentligt forskellige vejrforhold for de dage hvor billederne er taget. Billederne i kontrolsættet er taget på en dag med meget kraftigt sol, hvorimod billederne i træningssættet er taget på dage uden klar sol. Muligvis er farvesammensætningen på nummerpladerne væsentligt anderledes under disse forhold. Metoden der anvender frekvensanalyse er den mest robuste, og opnår næsten lige gode resultater på de to sæt. De øvrige metoder oplever et fald på mellem 9.5 og 6.1 procentpoint i effektivitet.

De to bedste metoder er dem der arbejder med kontrast. I træningsættet er metoden der kigger på områder i originalbilledet med høj kontrast den mest effektive med 83\%. I kontrolsættet er den mest effektive metode den der maksimerer lokal kontrast for at skabe et sammenhængende nummerpladeområde. 

HVAD ER HURTIGST?

\begin{figure}[htp]
\centering
\begin{tabular}{|l|c|c|c|}
\hline
\rowcolor[gray]{0.9} \multicolumn{4}{|>{\columncolor[gray]{0.9}}c|}{\textbf{Individuelle metoder til lokalisering}} \\ \hline
Metode & Træningsæt & Kontrolsæt & Forskel \\ \hline
Områder domineret af lyse gråtoner & 60.5\% & 32.2\% & 28.3\% \\ \hline
Områder med høj kontrast           & 83.0\% & 73.5\% &  9.5\% \\ \hline
Frekvensanalyse                    & 59.3\% & 58.3\% &  1.0\% \\ \hline
Maksimer lokal kontrast            & 82.8\% & 76.7\% &  6.1\% \\ \hline
Kvantifisering                     & 69.8\% & 60.5\% &  9.3\% \\ \hline
\end{tabular}
\caption{Resultaterne af afprøvning af de fem metoder til lokalisering af nummerplader på trænings- og kontrolsæt.}
\label{fig:test:lokalisering_traening_kontrol}
\end{figure}


\subsubsection{Samlet lokalisering}
Vi afprøver vi hvor god vores software er til at lokalisere nummerplader når alle fem metoder arbejder sammen. Vi har foretaget afprøvningen med flere forskellige indstillinger for minimal enighed. En enighed på for eksempel tre, vil sige at tre af de underliggende metoder til lokalisering skal have udpeget samme område for at systemet opfatter området som et nummerpladeområde og sender det videre i systemet til separation og genkendelse. Vi undersøger hvor høj enighed systemet skal konfigureres til for at give en korrekthed på 100\%. Resultaterne for afprøvning på træningsættet er vist i figur \vref{fig:test:lokalisering_traening_samlet} og resultaterne for kontrolsættet kan ses i figur \ref{fig:test:lokalisering_kontrol_samlet}.

I træningssættet er korrektheden høj selv ved en minimal enighed på en, men en korrekthed på 100\% opnås først når den minimale enighed sættes til fem. I denne konfiguration kan systemet kun lokalisere 35.3\% af nummerpladerne. For kontrolsættet opnås en korrekthed på 100\% med en minimal enighed på fire.

Selvom fire ud af de fem metoder til lokalisering giver dårligere resultater på kontrolsættet i de individuelle afprøvninger, falder den samlede effektivitet ikke tilsvarende når metoderne anvendes samlet. Faldet i evnen til at lokalisere nummerplader for den samlede metode er på 3 procentpoint, i den bedste konfiguration med minimal enighed på en, mod gennemsnitligt 10.8 procentpoint for de individuelle metoder. Den samlede metode er altså forholdsvis robust.

\begin{figure}[htp]
\centering
  \begin{tabular}{|l|c|c|}
    \hline
    \rowcolor[gray]{0.9} \multicolumn{3}{|>{\columncolor[gray]{0.9}}c|}{\textbf{Samlet metode til lokalisering - Træningssæt}} \\
    \hline
    Minimal enighed & Fundne nummerplader & Korrekthed \\ \hline
    1 &  97.5\% &  99.0\% \\ \hline
    2 &  96.0\% &  99.5\% \\ \hline
    3 &  88.0\% &  99.7\% \\ \hline
    4 &  68.2\% &  99.6\% \\ \hline
    5 &  35.3\% & 100.0\% \\ \hline
  \end{tabular}
\caption{Resultaterne af afprøvning af hvor effektivt vi kan lokalisere nummerplader i træningsættet når alle fem metoder arbejde sammen.}
\label{fig:test:lokalisering_traening_samlet}
\end{figure}


\begin{figure}[htp]
\centering
  \begin{tabular}{|l|c|c|}
    \hline
    \rowcolor[gray]{0.9} \multicolumn{3}{|>{\columncolor[gray]{0.9}}c|}{\textbf{Samlet metode til lokalisering - Kontrolsæt}} \\
    \hline
    Minimal enighed & Fundne nummerplader & Korrekthed\\ \hline
    1 &  94.5\% & 96.8\%\\ \hline
    2 &  93.0\% & 97.7\%\\ \hline
	3 &  78.7\% & 99.8\%\\ \hline
    4 &  49.0\% & 100.0\%\\ \hline
  \end{tabular}
\caption{Resultaterne af afprøvning af hvor effektivt vi kan lokalisere nummerplader i kontrolsættet når alle fem metoder arbejder sammen.}
\label{fig:test:lokalisering_kontrol_samlet}
\end{figure}

\subsubsection{Konklussion}
Ideen med at bruge flere metoder til lokalisering virker god. Vi er i stand til at lokalisere 94.5\% af nummerpladerne i ikke på forhånd observerede billeder. Metoden er forholdsvis robust overfor mindre forringelser i effektiviteten af de underliggende metoder.

Analyse af de billeder hvor vi ikke kan lokalisere nummerplader i træningsættet, afslører at meget mørke nummerplader er et problem. På grund af dette forhold må man overveje om man kan tale om en ideel intensitet for nummerpladeområder sådan som vi har gjort i vores pointsystem.

En anden fællesnævner for de billeder hvor vi ikke finder nummerpladerne er farven på bilen. Hvide biler er overrepræsenteret i gruppen af billeder hvor vi ikke kan finde nummerpladen. Metoderne \textit{Områder domineret af lyse gråtoner} og \textit{Kvantifisering} fungerer dårligt på hvide biler. Den ene fordi hvide bilers karosseri også er domineret af lyse gråtoner, og den anden fordi filteret der fjerner tegnenen på nummerpladerne også fjerner stregerne rundt om nummerpladerne og dermed skaber et stort sammenhængende område bestående af både nummerplade og bil. 

På figur \vref{fig:illu:wrong_candidates} i bilag \ref{sec:illu} viser vi eksempler på den type områder systemet fejlagtigt identificerer som nummerplader. Ofte sete fejl er klistermærker bag på biler, det rektangulere område der afgrænses af forsædernes nakkestøtter og bilens tag når man ser ind i gennem bagruden og områder med vinduer i bygninger i baggrunden af billedet.  


%%%%%%%%%%%%%%%%%%%%%%%%%%
%%% SEPARATION AF TEGN %%%
%%%%%%%%%%%%%%%%%%%%%%%%%%

\subsection{Separation af tegn, herunder rotation}

%DETTE AFSNIT SKRIVES SAMMEN MED GENKENDELSE.

I dette afsnit afprøves metoderne, der bruges til at separere tegn i en nummerplade. MERE?

\subsubsection*{Rotation}
%Om en nummerplade står vandret i et billede kan afgøres ved at kantdetektere billedet og udføre en Radon transformation på dette kant-billede. Hvis de stærkeste linier optræder ved $0^{\circ}$ er rotation udført korrekt.

Da vi har lavet funktionen til rotation primært ved brug af en Matlab funktion (\textit{radon}), vil det ikke være relevant at teste funktionen. Derudover vil en test af funktionen betyde at alle pladernes rotation skulle bestemmes manuelt. Vi har i stedet besluttet at funktionen testes i forbindelse med test af de andre funktioner i dette afsnit, hvliket vil sige at rotation indgår som en del af separation og genkendelse af tegn og ikke er ``sin egen'' funktion.

\subsubsection*{Separation}
Metoderne til separation af tegn afprøver vi ved at se på om de finder syv objekter indenfor nummerpladens koordinater. Denne enkle afprøvning er blot en en indikation af, hvor gode de enkelte metoder er.

Vi vil desuden afprøve om der er forskel på succesraten i forhold til hvor tæt nummerpladen er skåret til. Ved at definere forskellige størrelser for området omkring nummerpladen kan afprøvningen indikere, hvor robust metoderne er i forhold til størrelsen på udskæringen af pladen. Afprøvningsmetoden er dog følsom overfor koordinaterne til pladen.

%Parametre: størrelse på componenter der sorteres fra, skalering: kan vi tillade os at skalere ned (eller op, hvor vi måske får mere plads mellem komponenterne) eller mister vi for meget information VI KAN IKKE SKALERE

%OVERVEJ NEDENSTÅENDE SOM SAMLET GRAF I ET AFSNIT
Resultaterne fra afprøvningen af separation af tegn ses i figurerne \ref{fig:test:sep-traening-manuel} og \ref{fig:test:sep-kontrol-manuel}.

\begin{figure}[htp]
\centering
\begin{tabular}{|l|c|c|}\hline
\rowcolor[gray]{0.9} \multicolumn{3}{|>{\columncolor[gray]{0.9}}c|}{\textbf{Separation af tegn - Træningssæt}} \\ \hline
Antal pixels lagt til & Sammenhængende komponenter & Bjerg/dal \\\hline
10 & 96.5\% & 91.8\% \\\hline
20 & 95.8\% & 91.3\% \\\hline
30 & 92.5\% & 83.5\% \\\hline
40 & 87.5\% & 73.4\% \\\hline \end{tabular}
\caption{Afprøvning af metoderne til separation af tegn på træningssættet ved brug af definerede pladeområder. Værdierne i den første kolonne er den mængde pixels området er udviddet med.}
\label{fig:test:sep-traening-manuel}
\end{figure}

\begin{figure}[htp]
\centering
\begin{tabular}{|l|c|c|}\hline
\rowcolor[gray]{0.9} \multicolumn{3}{|>{\columncolor[gray]{0.9}}c|}{\textbf{Separation af tegn - Kontrolsæt}} \\ \hline
Antal pixels lagt til & Sammenhængende komponenter & Bjerg/dal \\\hline
10 & 90.0\% & 87.2\% \\\hline
20 & 86.8\% & 81.0\% \\\hline
30 & 84.7\% & 71.2\% \\\hline
40 & 79.0\% & 58.5\% \\\hline \end{tabular}
\caption{Afprøvning af metoderne til separation af tegn på kontrolsættet ved brug af manuelt udpegede plader. Værdierne i den første kolonne er den mængde pixels området er udviddet med.}
\label{fig:test:sep-kontrol-manuel}
\end{figure}

%\begin{tabular}{|l|c|c|}\hline
%\rowcolor[gray]{0.9} \multicolumn{3}{|>{\columncolor[gray]{0.9}}c|}{\textbf{Kontrolsæt}} \\ \hline
%Bedste udklip & Sammenhængende komponenter & Bjerg/dal \\\hline
% & 0\% & 0\% \\\hline
%\end{tabular}

Resultaterne af afprøvningen er generelt bedre ved brug af træningssættet end ved brug af kontrolsættet. Metoden der bruger sammenhængende komponenter giver de bedste resultater af de to præsenterede metoder. Tegnene udskæres desuden tæt og hvis nummerpladeområdet ikke er for stort, men derimod stadig indeholder nummerpladens kanter, sådan at en eventuelt rotation kan foretages korrekt, giver metoden sjældent fejl. Fejlkilderne, når nummerpladeområdet er lille (det vil sige at området er udviddet med 10 eller 20 pixels), er primært skruerne i pladen og skæve plader. Skruerne bevirker enten at tegnene gror sammen med hinanden eller med pladens kant og derfor vælges fra, fordi de sammenhængende komponenter er for store. De skæve plader bevirker, at de komponenter der repræsenterer tegnene ikke står i samme højde og derfor bliver valgt fra.

%EVT. EKSEMPEL PÅ PROBLEM MED SKRUER SAMT SKÆV PLADE.

Fejlkilderne når nummerpladeområdet er større (det vil sige udviddet med 30 eller 40 pixels) er at en anden komponent end den der indeholder nummerpladen vælges når billedet skal indskrænkes eller at indskrænkningen ikke er stor nok og at der derfor laves komponenter som er udenfor og er tilpas høje oog så videre.

%ETV. EKSEMPEL PÅ FORKERT INDSKRÆNKNING OG DÅRLIG INDSKRÆNKNING

I kontrolsættet tillægges fejlkilden hvor nummerpladerne er delvist dækket af skygge. Når dette er tilfældet opstår der nogle gange en komponent som går på tværs af flere tegn.

I APPENDIX ses en collage med nummerplader hvor separationen af tegn ved brug af sammenhængende komponenter er fejlet.

%Et eksempel på denne fejltype ses i figur \vref{fig:kraftig_skygge}.

%\begin{figure}[htp]
%\centering
%\begin{minipage}[c]{8 cm}
%\framebox{\includegraphics[width=8cm]{test/illu/kraftig_skygge.png}}
%\framebox{\includegraphics[width=8cm]{test/illu/kraftig_skygge_concomp.png}} 
%\end{minipage}
%\caption{Et eksempel på en nummerplade fra kontrolsættet, hvor der en kraftig skygge over nummerpladen. Det øverste billede er originalbilledet, mens det nederste billede er det binære billede der dannes af pladen. Det er tydeligt at skyggen har stor indflydelse på hvilke komponenter der er sammenhængende.}
%\label{fig:kraftig_skygge}
%\end{figure}

Som tidligere beskrevet er bjerg/dal-metoden meget følsom overfor støj, hvilket også ses i afprøvningsresultaterne. Derudover er afprøvningsmetoden et dårligt skøn på om tegnene udskæres korrekt, da mange af de udklippede elementer enten er halve tegn eller store elementer, hvor tegnet kun er en mindre del af. Et eksempel på syv tegn der er dårligt/forkert udklippet ses i figur \vref{fig:tegn-udklip-daarlig}. Resultaterne er på samme måde som ovenfor ringere ved kontrolsættet, men da afprøvningsmetoden giver en dårlig indikation for resultaterne med denne metode, er det svært at drage konklusioner udfra disse tal.

\begin{figure}[htp]
  \centering
  \begin{minipage}[c]{8 cm}
  	  \framebox{\includegraphics[height=1cm, width=0.8cm]{test/illu/char1_ptv.png}}
  	  \framebox{\includegraphics[height=1cm, width=0.8cm]{test/illu/char2_ptv.png}}
	  \framebox{\includegraphics[height=1cm, width=0.8cm]{test/illu/char3_ptv.png}}
  	  \framebox{\includegraphics[height=1cm, width=0.8cm]{test/illu/char4_ptv.png}}
	  \framebox{\includegraphics[height=1cm, width=0.8cm]{test/illu/char5_ptv.png}}
  	  \framebox{\includegraphics[height=1cm, width=0.8cm]{test/illu/char6_ptv.png}}
	  \framebox{\includegraphics[height=1cm, width=0.8cm]{test/illu/char7_ptv.png}}
	\end{minipage}
  \caption{Et eksempel på syv billeder af syv tegn, som er dårligt udklippede. Eksemplet viser udklipninger hvor området der udklippes er meget stort i forhold til det enkelte tegn. Derudover er det er også vist et eksempel på et tegn der klippes midt over fordi der er få markerede pixels i midten af tegnet.}
  \label{fig:tegn-udklip-daarlig}
\end{figure}

\subsubsection{Konklusion}

Den bedste separationsmetode er den der bruger sammenhængende komponenter. Ydermere er denne metode bedst på når pladerne er udskåret forholdsvis tæt, sådan at komponenter, der ligger udenfor pladen, ikke vil være forstyrrende elementer. Metoden er dog ret følsom overfor skygger der kastes på pladen, fordi pladen sidder nedsunket i bilen, især bagerst på bilen. Når tynde elementer frasorteres gøres dette kun i den øverste $\frac{1}{3}$ af pladen. For at eliminerer fejlene der opstår på grund af skygger, kunne man eksperimentere med at fjerne tynd elementer i den øverste halvdel af pladen.

For at afprøve bjerg/dal-metoden ordentlig er det nødvendigt at udvikle en bedre afprøvningsmetode.

\begin{comment}
Problemer ved 10: tegn der er smeltet sammen pga. søm skruer: OE26906 og ----290. Godt udskåret: UX33152, skæve plader et problem. go kontrast: YE39734.

Problemer ved 20: for stort område giver komponenter der ikke klippes bort, fordi kontrastforstærkning giver et andet resultat, her ville dynamisk kontrast-blok størrelse måske hjælpe. Dårlig indskrænk XK29750. TS57793: GOD(?) 

Problemer ve 30: stadig for stort område: giver komponenter udenfor der vælges. dårlig indskrænk, især foran? eksempler.

\end{comment}

%%%%%%%%%%%%%%%%%%%%%%%%%%%
%%% GENKENDELSE AF TEGN %%%
%%%%%%%%%%%%%%%%%%%%%%%%%%%

\subsection{Genkendelse af tegn}
I dette afsnit afprøves metoderne til genkendelse af tegn. Syntaksanalysen vil blive afprøvet i afprøvningen af det samlede system i afsnit \vref{sec:test:samlet_system}, da denne analyse ikke direkte arbejder med genkendelse men kun arbejder på en tegnfølge.


Afprøvningen vil blive foretaget på alle tegn fra kontrolsættet, separeret med den bedste metode (sammenhængende komponenter hvor 10 pixels er lagt til). Disse tegn er sorteret så billeder der tydeligvis ikke ligner et tegn (hvor tegnet eksempelvis er blevet halveret under separationen eller hvor en komponent der ikke er et tegn er blevet valgt) er blevet sorteret fra.

I træningssættet er der generelt færre bogstaver end tal. Vi vurderer at dette vil have indflydelse på afprøvningsresultaterne. Der er for eksempel kun tre billeder af tegnet \textbf{O}. Derudover er der ingen billeder af tegnet \textbf{A} i hverken træningssættet eller kontrolsættet. Derfor vil tabelrækker, hvor afprøvningsresultater for \textbf{A} skulle stå, være udfyldt med "-".

%Nedenstående tests er i forhold til manuelt udklippede plader - ikke automtisk lokalisering.

I figurerne \ref{fig:test:middel_tal} til \ref{fig:test:middel_alletegn} ses resultaterne af afprøvningen af genkendelse af tegn ved brug af egenskabsvektorer. Afprøvningen er udført med vektorer som er lavet ud fra billeder med størrelsen 3 $\times$ 3 og opefter. Vektorer der laves ud fra mindre billeder vil ikke være fornuftige at bruge til afprøvning, da flere af disse vektorer ligner hinanden og man derfor ikke kan skelne mellem genkendelse af to tegn med samme vektor. Den øvre grænse sættes ved vektorer der er lavet ud fra ? $\times$ ? billeder, da vi ikke forventer nogle væsentlige forbedringer af resultaterne udover denne grænse. Desuden forventer vi at hastigheden for systemet nedsættes jo større vektorer man arbejder med, hvilket vil være en væsentlig detalje hvis systemet skulle arbejde i realtid. MANGLER 8 x 8!!!

\begin{figure}[htp]
\centering
\begin{tabular}{|l|c|c|c|c|c|}\hline
\rowcolor[gray]{0.9} \multicolumn{6}{|>{\columncolor[gray]{0.9}}c|}{\textbf{Genkendelse af cifre - Egenskabsvektor}} \\ \hline
Tegn & 3 $\times$ 3 & 4 $\times$ 4 & 5 $\times$ 5 & 6 $\times$ 6 & 7 $\times$ 7\\\hline
0 & 21.0\% & 77.1\% & 74.8\% & 56.1\% & 91.1\%\\\hline
1 & 79.3\% & 94.1\% & 97.5\% & 97.5\% & 98.0\%\\\hline
2 & 92.7\% & 93.3\% & 98.0\% & 98.9\% & 99.2\%\\\hline
3 & 86.1\% & 99.4\% & 98.9\% & 100.0\% & 100.0\%\\\hline
4 & 84.1\% & 99.1\% & 99.1\% & 99.7\% & 99.7\%\\\hline
5 & 96.6\% & 98.3\% & 98.3\% & 99.4\% & 99.4\%\\\hline
6 & 42.1\% & 99.5\% & 100.0\% & 99.5\% & 100.0\%\\\hline
7 & 92.4\% & 94.2\% & 98.7\% & 99.6\% & 99.6\%\\\hline
8 & 13.0\% & 45.2\% & 90.4\% & 93.5\% & 97.0\%\\\hline
9 & 82.6\% & 94.6\% & 96.4\% & 98.2\% & 97.3\%\\\hline
\end{tabular}
\caption{Afprøvningsresultater for genkendelse af cifre ved brug af egenskabsvektorer. For hvert ciffer er det angivet hvor stor en del af billederne af dette ciffer, der er korrekt genkendt.}
\label{fig:test:middel_tal}
\end{figure}

\begin{figure}[htp]
\centering
\begin{tabular}{|l|c|c|c|c|c|}\hline
\rowcolor[gray]{0.9} \multicolumn{6}{|>{\columncolor[gray]{0.9}}c|}{\textbf{Genkendelse af bogstaver - Egenskabsvektor}} \\ \hline
Tegn & 3 $\times$ 3 & 4 $\times$ 4 & 5 $\times$ 5 & 6 $\times$ 6 & 7 $\times$ 7\\\hline
A & - & - & - & - & -\\\hline
B & 40.9\% & 68.2\% & 90.9\% & 95.5\% & 95.5\%\\\hline
C & 65.6\% & 93.8\% & 96.9\% & 100.0\% & 100.0\%\\\hline
D & 91.4\% & 100.0\% & 100.0\% & 97.1\% & 91.4\%\\\hline
E & 65.0\% & 85.0\% & 95.0\% & 95.0\% & 100.0\%\\\hline
H & 28.6\% & 100.0\% & 85.7\% & 100.0\% & 100.0\%\\\hline
J & 97.2\% & 97.2\% & 100.0\% & 97.2\% & 100.0\%\\\hline
K & 54.8\% & 80.6\% & 90.3\% & 90.3\% & 90.3\%\\\hline
L & 25.0\% & 100.0\% & 100.0\% & 100.0\% & 100.0\%\\\hline
M & 20.5\% & 87.2\% & 97.4\% & 94.9\% & 94.9\%\\\hline
N & 41.7\% & 100.0\% & 100.0\% & 100.0\% & 100.0\%\\\hline
O & 0.0\% & 66.7\% & 0.0\%  & 66.7\% & 66.7\%\\\hline
P & 78.8\% & 100.0\% & 100.0\% & 100.0\% & 100.0\%\\\hline
R & 28.2\% & 87.2\% & 92.3\% & 94.9\% & 94.9\%\\\hline
S & 98.6\% & 97.1\% & 92.7\% & 100.0\% & 100.0\%\\\hline
T & 94.4\% & 98.9\% & 98.9\% & 98.9\% & 100.0\%\\\hline
U & 1.3\% & 93.3\% & 96.0\% & 97.3\% & 98.7\%\\\hline
V & 90.6\% & 98.1\% & 98.1\% & 98.1\% & 98.1\%\\\hline
X & 92.2\% & 97.4\% & 98.7\% & 98.7\% & 98.7\%\\\hline
Y & 95.8\% & 98.9\% & 98.9\% & 98.4\% & 98.9\%\\\hline
Z & 77.2\% & 100.0\% & 96.5\% & 100.0\% & 100.0\%\\\hline
\end{tabular}
\caption{Afprøvningsresultater for genkendelse af bogstaver ved brug af egenskabsvektorer. For hvert bogstav er det angivet hvor stor en del af billederne af dette bogstav, der er korrekt genkendt. Rækken med bogstavet \textbf{A} er tom, da hverken trænings- eller kontrolsættet indeholder nogen tegn af denne type.}
\label{fig:test:middel_bogstav}
\end{figure}

\begin{figure}[htp]
\centering
\begin{tabular}{|l|c|c|c|c|c|c|c|}\hline
\rowcolor[gray]{0.9} \multicolumn{8}{|>{\columncolor[gray]{0.9}}c|}{\textbf{Genkendelse af tegn - Egenskabsvektor}} \\ \hline
Tegn & 3 $\times$ 3 & 4 $\times$ 4 & 5 $\times$ 5 & 6 $\times$ 6 & 7 $\times$ 7 & 8 $\times$ 8 & 9 $\times$ 9\\\hline
Alle tal & 73.3\% & 90.8\% & 95.9\% & 95.3\% & 98.4\% & 98.8\% & 97.0\%\\\hline
Alle bogstaver & 74.5\% & 95.7\% & 97.0\% & 98.0\% & 98.2\% & 98.6\% & 98.4\%\\\hline
\end{tabular}
\caption{Afprøvningsresultater for genkendelse af tegn ved brug af egenskabsvektorer, hvor tegnene er opdelt i en gruppe af alle cifre og en gruppe af alle bogstaver. For hver gruppe er det angivet hvor stor en del af tegnene i gruppen der er korrekt genkendt.}
\label{fig:test:middel_alletegn}
\end{figure}

Som det ses har metoden sværest ved at genkende tegnene \textbf{0} og \textbf{O} hviket skyldes, at middelvektorerne for disse tegn ligger tæt på hinanden (selv med det menneskelige øje, kan disse to tegn være svære at skelne fra hinanden). Derudover ses det at resultaterne for vektorer der er lavet på baggrund af 3 $\times$ 3 billeder, giver væsentligt dårlige resultater end de restende vektorer. Dette kommer især til udtryk i tabellen i figur \vref{fig:test:middel_alletegn}, hvor alle vektorer giver resultater over 90 \%, bortset fra den første.

I figurene \ref{fig:test:sum_tal} til \ref{fig:test:sum_alletegn} vises afprøvningsresultaterne for genkendelse af tegn ved brug af sum-billeder. Som ovenfor vil det ikke give nogen mening at teste på sum-billeder lavet ud fra 2 $\times$ 2 billeder.

\begin{figure}[htp]
\centering
\begin{tabular}{|l|c|c|c|c|c|}\hline
\rowcolor[gray]{0.9} \multicolumn{6}{|>{\columncolor[gray]{0.9}}c|}{\textbf{Genkendelse af cifre - Sum-billeder}} \\ \hline
Tegn & 3 $\times$ 3 & 4 $\times$ 4 & 5 $\times$ 5 & 6 $\times$ 6 & 7 $\times$ 7\\\hline
0 & 0.0\% & 64.0\% & 73.4\% & 54.2\% & 81.3\%\\\hline
1 & 98.5\%  & 99.0\% & 96.1\% & 99.0\% & 99.5\%\\\hline
2 & 81.6\% & 56.7\% & 90.5\% & 99.7\% & 99.7\%\\\hline
3 & 3.9\% & 45.3\% & 98.9\% & 97.8\% & 98.9\%\\\hline
4 & 22.8\% & 98.8\% & 98.8\% & 99.4\% & 99.4\%\\\hline
5 & 73.4\% & 42.7\% & 93.7\% & 99.1\% & 99.1\%\\\hline
6 & 31.6\% & 57.4\% & 96.3\% & 98.4\% & 97.9\%\\\hline
7 & 0.0\% & 93.8\% & 99.6\% & 99.6\% & 99.6\%\\\hline
8 & 1.3\% & 0.7\% & 42.6\% & 80.0\% & 98.3\%\\\hline
9 & 47.8\% & 62.9\% & 94.2\% & 98.7\% & 96.9\%\\\hline
\end{tabular}
\caption{Afprøvningsresultater for genkendelse af cifre ved brug af sum-billeder. For hvert ciffer er det angivet hvor stor en del af billederne af dette tegn, der er korrekt genkendt.}
\label{fig:test:sum_tal}
\end{figure}

\begin{figure}[htp]
\centering
\begin{tabular}{|l|c|c|c|c|c|}\hline
\rowcolor[gray]{0.9} \multicolumn{6}{|>{\columncolor[gray]{0.9}}c|}{\textbf{Genkendelse af bogstaver - Sum-billeder}} \\ \hline
Tegn & 3 $\times$ 3 & 4 $\times$ 4 & 5 $\times$ 5 & 6 $\times$ 6 & 7 $\times$ \\\hline
A & - & - & - & - & -\\\hline
B & 27.3\% & 40.9\% & 95.5\% & 100.0\% & 95.5\%\\\hline
C & 0.0\% & 84.4\% & 28.1\% & 3.1\% & 53.1\%\\\hline
D & 20.0\% & 100.0\% & 100.0\% & 91.4\% & 80.0\%\\\hline
E & 65.0\% & 90.0\% & 100.0\% & 100.0\% & 100.0\%\\\hline
H & 100.0\% & 85.7\% & 85.7\% & 85.7\% & 85.7\%\\\hline
J & 100.0\% & 97.2\% & 97.2\% & 97.2\% & 97.2\%\\\hline
K & 54.8\% & 80.6\% & 87.1\% & 90.3\% & 87.1\%\\\hline 
L & 15.0\% & 0.0\% & 0.0\% & 0.0\% & 0.0\%\\\hline
M & 25.6\% & 89.7\% & 97.4\% & 97.4\% & 89.7\%\\\hline
N & 0.0\% & 100.0\% & 100.0\% & 100.0\% & 100.0\%\\\hline
O & 0.0\% & 33.3\% & 0.0\% & 66.7\% & 33.3\%\\\hline
P & 39.4\% & 42.4\% & 51.5\% & 100.0\% & 100.0\%\\\hline
R & 10.3\% & 82.1\% & 94.9\% & 92.3\% & 94.9\%\\\hline
S & 0.0\% & 0.0\% & 27.5\% & 56.5\% & 23.2\%\\\hline
T & 82.0\% & 3.4\% & 98.9\% & 98.9\% & 97.8\%\\\hline
U & 2.7\% & 92.0\% & 93.3\% & 82.7\% & 98.7\%\\\hline
V & 82.1\% & 80.2\% & 97.2\% & 96.2\% & 95.3\%\\\hline
X & 0.0\% & 98.7\% & 96.1\% & 98.7\% & 98.7\%\\\hline
Y & 97.9\% & 54.5\% & 93.1\% & 97.9\% & 97.4\%\\\hline
Z & 82.5\% & 100.0\% & 100.0\% & 100.0\% & 98.2\%\\\hline
\end{tabular}
\caption{Afprøvningsresultater for genkendelse af bogstaver ved brug af sum-billeder. For hvert bogstav er det angivet hvor stor en del af billederne af dette tegn, der er korrekt genkendt. Rækken med tegnet \textbf{A} er tom, da hverken trænings- eller kontrolsættet indeholder nogen tegn af denne type.}
\label{fig:test:sum_bogstav}
\end{figure}

\begin{figure}[htp]
\centering
\begin{tabular}{|l|c|c|c|c|c|c|c|}\hline
\rowcolor[gray]{0.9} \multicolumn{8}{|>{\columncolor[gray]{0.9}}c|}{\textbf{Genkendelse af tegn - Sum-billeder}} \\ \hline
Tegn & 3 $\times$ 3 & 4 $\times$ 4 & 5 $\times$ 5 & 6 $\times$ 6 & 7 $\times$ 7 & 8 $\times$ 8 & 9 $\times$ 9 \\\hline
Alle tal & 37.4\% & 61.4\% & 89.6\% & 93.8\% & 97.5\% & 97.6\% & 91.1\%\\\hline
Alle bogstaver & 47.2\% & 67.6\% & 86.1\% & 89.1\% & 88.3\% & 95.9\% & 91.9\%\\\hline
\end{tabular}
\caption{Afprøvningsresultater for genkendelse af tegn ved brug af sum-billeder, hvor tegnene er opdelt i en gruppe af alle cifre og en gruppe af alle bogstaver. For hver gruppe er det angivet hvor mange procent af tegnene i gruppen der er korrekt genkendt.}
\label{fig:test:sum_alletegn}
\end{figure}


I \vref{fig:test:sum_alletegn} er det tydeligt at denne metode er generelt dårligere end den forrige. Resultaterne med de mindre vektorer er ikke nær så gode som med egenskabsvektorer. Igen har metoden svært ved at genkende tegnene \textbf{0} og \textbf{O} men også tegnet \textbf{8} har behov for store sum-billeder for at kunne genkendes, hvilket i følge vores observationer er fordi, at det forveksles med tegnet \textbf{B}. Også \textbf{L} og \textbf{S} er svære at genkende, hvilket sandsynligvis skyldes at disse tegns densitet er forholdsvis lav, og at de binære billeder derfor ikke altid har et tilstrækkeligt antal markerede pixels til at tegnene genkendes.

I figurene \ref{fig:test:and_tal} til \ref{fig:test:and_alletegn} vises afprøvningsresultaterne for genkendelse af tegn ved brug af forenings-billeder. Denne metode har brug for billeder med megen information, hvorfor den mindste billedstørrelse er 10 $\times$ 10 og intervallerne mellem billedstørrelserne er større end ved de andre metoder.

\begin{figure}[htp]
\centering
\begin{tabular}{|l|c|c|c|c|c|c|}\hline
\rowcolor[gray]{0.9} \multicolumn{7}{|>{\columncolor[gray]{0.9}}c|}{\textbf{Genkendelse af cifre - Forenings-billeder}} \\ \hline
Tegn & 10 $\times$ 10 & 20 $\times$ 20 & 30 $\times$ 30 & 40 $\times$ 40 & 50 $\times$ 50 & 100 $\times$ 100\\\hline
0 & 2.3\% & 0.0\% & 8.4\% & 15.4\% & 0.0\% & 0.0\%\\\hline
1 & 0.0\% & 15.3\% & 1.5\% & 25.1\% & 3.9\% & 11.3\%\\\hline
2 & 99.7\% & 98.9\% & 99.4\% & 99.7\% & 99.4\% & 99.4\%\\\hline
3 & 100.0\% & 100.0\% & 99.7\% & 100.0\% & 100.0\% & 99.7\%\\\hline
4 & 99.7\% & 99.7\% & 99.7\% & 99.7\% & 99.7\% & 99.7\%\\\hline
5 & 97.7\% & 91.4\% & 99.4\% & 99.4\% & 99.1\% & 99.1\%\\\hline
6 & 0.0\% & 0.0\% & 0.0\% & 0.0\% & 0.0\% & 0.0\%\\\hline
7 & 97.3\% & 99.1\% & 99.6\% & 98.7\% & 99.1\% & 99.1\%\\\hline
8 & 93.5\% & 98.7\% & 98.7\% & 99.6\% & 99.6\% & 99.1\%\\\hline
9 & 40.2\% & 69.6\% & 79.9\% & 78.1\% & 67.0\% & 72.8\%\\\hline
\end{tabular}
\caption{Afprøvningsresultater for genkendelse af cifre ved brug af forenings-billeder. For hvert ciffer er det angivet hvor stor en del af billederne af dette tegn, der er korrekt genkendt.}
\label{fig:test:and_tal}
\end{figure}

\begin{figure}[htp]
\centering
\begin{tabular}{|l|c|c|c|c|c|c|}\hline
\rowcolor[gray]{0.9} \multicolumn{7}{|>{\columncolor[gray]{0.9}}c|}{\textbf{Genkendelse af bogstaver - Forenings-billeder}} \\ \hline
Tegn & 10 $\times$ 10 & 20 $\times$ 20 & 30 $\times$ 30 & 40 $\times$ 40 & 50 $\times$ 50 & 100 $\times$ 100\\\hline
A & - & - & - & - & - & -\\\hline
B & 77.3\% & 36.4\% & 59.1\% & 50.0\% & 50.0\% & 63.6\%\\\hline
C & 0.0\% & 0.0\% & 0.0\% & 0.0\% & 0.0\% & 0.0\%\\\hline
D & 100.0\% & 2.9\% & 80.0\% & 62.9\% & 62.9\% & 48.6\%\\\hline
E & 100.0\% & 100.0\% & 100.0\% & 100.0\% & 100.0\% & 100.0\%\\\hline
H & 100.0\% & 100.0\% & 100.0\% & 100.0\% & 100.0\% & 100.0\%\\\hline
J & 0.0\% & 25.0\% & 83.3\% & 88.9\% & 58.3\% & 25.0\%\\\hline
K & 87.1\% & 90.3\% & 90.3\% & 90.3\% & 90.3\% & 90.3\%\\\hline
L & 0.0\% & 0.0\% & 0.0\% & 0.0\% & 0.0\% & 0.0\%\\\hline
M & 79.5\% & 38.5\% & 46.2\% & 94.9\% & 59.0\% & 74.4\%\\\hline
N & 12.5\% & 95.8\% & 95.8\% & 100.0\% & 100.0\% & 95.8\%\\\hline
O & 33.3\% & 100.0\% & 100.0\% & 100.0\% & 100.0\% & 100.0\%\\\hline
P & 81.8\% & 75.8\% & 9.1\% & 12.1\% & 12.1\% & 27.3\%\\\hline
R & 97.4\% & 89.7\% & 87.2\% & 94.9\% & 89.7\% & 89.7\%\\\hline
S & 0.0\% & 1.4\% & 0.0\% & 1.4\% & 0.0\% & 0.0\%\\\hline
T & 98.9\% & 98.9\% & 98.9\% & 98.9\% & 98.9\% & 98.9\%\\\hline
U & 2.7\% & 0.0\% & 0.0\% & 0.0\% & 0.0\% & 0.0\%\\\hline
V & 92.5\% & 97.2\% & 97.2\% & 98.1\% & 98.1\% & 97.2\%\\\hline
X & 94.2\% & 98.1\% & 97.4\% & 98.1\% & 97.4\% & 98.1\%\\\hline
Y & 87.3\% & 97.9\% & 96.8\% & 97.4\% & 97.4\% & 97.9\%\\\hline
Z & 100.0\% & 100.0\% & 100.0\% & 100.0\% & 100.0\% & 100.0\%\\\hline
\end{tabular}
\caption{Afprøvningsresultater for genkendelse af bogstaver ved brug af forenings-billeder. For hvert bogstaver er det angivet hvor stor en del af billederne af dette tegn, der er korrekt genkendt.}
\label{fig:test:and_bogstav}
\end{figure}

\begin{figure}[htp]
\centering
\begin{tabular}{|l|c|c|c|c|c|c|}\hline
\rowcolor[gray]{0.9} \multicolumn{7}{|>{\columncolor[gray]{0.9}}c|}{\textbf{Genkendelse af tegn - Forenings-billeder}} \\ \hline
Tegn & 10 $\times$ 10 & 20 $\times$ 20 & 30 $\times$ 30 & 40 $\times$ 40 & 50 $\times$ 50 & 100 $\times$ 100\\\hline
Alle tal & 71.6\% & 74.7\% & 76.3\% & 78.5\% & 74.7\% & 75.7\%\\\hline
Alle bogstaver & 70.7\% & 70.3\% & 73.0\% & 75.0\% & 72.3\% & 72.0\%\\\hline
\end{tabular}
\caption{Afprøvningsresultater for genkendelse af tegn ved brug af forenings-billeder, hvor tegnene er opdelt i en gruppe af alle cifre og en gruppe af alle bogstaver. For hver gruppe er det angivet hvor stor en del af tegnene i gruppen der er korrekt genkendt.}
\label{fig:test:and_alletegn}
\end{figure}

Generelt viser afprøvningen af metoden med forenings-billeder at metoden er meget dårlig til genkendelse af enkelte tegn. Det drejer sig i sær om \textbf{0}, \textbf{1}, \textbf{6}, \textbf{C}, \textbf{L}, \textbf{S} og \textbf{U}. Fælles for disse tegn er at tegnenes markerede pixels i det binære billeder ligger i yderkanten af billederne.

%HVILKEN POSITION PÅ HITLISTEN TAGER DEN I GENNEMSNIT PR POSITION - TRÆNING?
%HVILKEN POSITION PÅ HITLISTEN TAGER DEN I GENNEMSNIT PR POSITION - TEST?
%MAXHITNO: HVAD GIVER DET JO LÆNGERE NED AD HITLISTEN VI KAN GÅ?

\subsubsection{Konklusion}

Metoden der bruger egenskabsvektorer giver generelt de bedste resultater. De resultater metoden giver er til og med så gode, at det må formodes at den ikke har en negativ indflydelse på det samlede system. Metoden giver de bedste resultater hvor vektorerne er lavet ud fra billeder af størrelsen $8 \times 8$. Man kan synes at disse vektorer indeholder meget information, og det måske vil påvirke systemets hastighed. I følge vores egne observationer lader der dog ikke til at være nogen hastighedsmæssig forskel når man bruger vektorer med denne længde i forhold til hvis man bruger kortere vektorer.

Vi kunne også have testet hvilke tegn der bliver valgt i stedet for, når et forkert tegn vælges. Vælges 8 f.eks. når B burde ha været valgt??

Billederne af tegnene er ret store i forhold til dem der bruges i litteraturen. Dette giver vores system en væsentlig fordel i forhold til "de andre".




\begin{comment}
Syntaks analyse: Hvilke hits bliver valgt på hitlisterne af syntaksanalysen (sæt maxhitno højt):

\begin{tabular}{|l|c|}\hline
\multicolumn{2}{|l|}{Træningssæt} \\\hline
Hitnr. & Valgt \\\hline
1 & 95,4\% \\\hline
2 & 3\% \\\hline
3 & 0\% \\\hline
4 & 0\% \\\hline
5 & 0\% \\\hline
6 & 0\% \\\hline \end{tabular}

\begin{tabular}{|l|c|}\hline
\multicolumn{2}{|l|}{Kontrolsæt} \\\hline
Hitnr. & Valgt \\\hline
1 & 92,9\% \\\hline
2 & 0\% \\\hline
3 & 0\% \\\hline
4 & 0\% \\\hline
5 & 0\% \\\hline
6 & 0\% \\\hline \end{tabular}

\end{comment} %%%%%%%% COMMENT END


%Ovenstående med og uden syntaksanalyse? kun med? kun uden?

%Søren: Ved klassifikation bruges en eller anden grænseværdi som bestemmer hvad et element skal klassificeres som: denne grænseværdi kan reguleres og kan derfor testes.

%Delkonklusion:



\subsection{Det samlede system}
\label{sec:test:samlet_system}

I dette afsnit viser vi resultaterne af afprøvning af hele systemet på trænings- og kontrolsættet. Systemet giver kun svar når det mener at have identificeret alle syv karakterer i nummerpladen. Vi angiver for hvor mange billeder systemet giver at svar, samt fejlraten for svaret. Vi har afprøvet systemet med og uden syntaksanalyse og raporteret begge resultater.

Systemet er konfigureret så vi opnår de bedste resultater for hvert trin i processen. Til lokalisering bruger vi den samlede metode med en minimal enighed på 1. Til segmentering af tegn bruger vi metoden der benytter sig af sammenhængende komponenter. Til gendkendelse af tegn bruger vi egenskabsvektorer med længden 64.

\begin{figure}[htp]
\centering
\begin{tabular}{|l|l|l|}\hline
\rowcolor[gray]{0.9} \multicolumn{3}{|>{\columncolor[gray]{0.9}}c|}{\textbf{Samlet system}} \\ \hline
Sæt         & Med syntaxanalyse & Uden syntaxanalyse \\ \hline
Træningssæt & 93.0\% / 1.6\%    &                    \\ \hline
Kontrolsæt  & 86.7\% / 2.3\%    &                    \\ \hline
\end{tabular}
\caption{Resultaterne for afprøvning af det samlede systems evne til at læse alle syv karakterer i nummerpladerne i trænings- og kontrolsættet. I cellerne med procentsatser, angiver det første tal hvor i  hvor mange }
\label{fig:test:samlet_train_kontrol}
\end{figure}



\subsection{Billeder uden for afgrænsningerne}
I dette afsnit viser vi resultaterne for afprøvning af vores software på de to sæt billeder der ikke overholder vores afgrænsninger. Bemærk at de to sæt består af væsentligt færre billeder end vores trænings- og kontrolsæt, og resultaterne derfor er mindre troværdige end resultaterne for de større sæt.

\subsubsection{Lokalisering af nummerplader}
Resultaterne for afprøvning af de fem metoder til lokalisering af nummerplader på blitzsættet og de gule sæt i forhold til resultaterne for vores træningssæt er vist på figur \vref{fig:test:lokalisering_blitz_gul}.

Metoden der kigger efter lyse nummerplader er, ikke helt overraskende, ikke i stand til at finde nummerplader i det gule sæt. I blitzsætter er samme metode overraskende ineffektiv med et fald i effektivitet fra 60.5\% i træningsættet til 24.6\%. HVORFOR!. Det mest overraskende resultat er, at metoden der kigger efter områder med høj kontrast er stort set ubruelig på blitzsættet med et resultat på blot 5.4\%. Blitzlyset er så kraftigt, at det gør tegnene i nummerpladerne meget tynde, og i nogle tilfælde næsten helt udvisker dem. Figur \vref{fig:illu:badgrads} i bilag \ref{sec:illu} viser
gradienter i et udvalg af billeder fra blitzsættet. En enkelt af metoderne opnår bedre resultater i blitzsættet end i træningssættet. Netoden der sætter antallet af farver i billedet voldsomt ned, kvantifisering, stiger fra 69.8\% i træningsættet til 85.4\% i blitzsættet. De lysende nummerplader adskiller sig nu endnu mere fra det omkringliggende område end før. Selv på hvide biler, hvor metoden er svag, lykkes det i flere tilfælde for metoden at lokalisere nummerpladen i blitzsættet. Metoden der benytter sig af frekvensanalyse fungerer dårligt på blitzsættet men tilfredsstillende på det gule sæt med en tilbagegang i effektivitet på 7.3\% procentpoint. Den metode der fungerer bedst på tværs af de tre sæt er metoden der optimerer lokal kontrast.

\begin{figure}[htp]
\centering
\begin{tabular}{|l|l|l|l|}
\hline
\rowcolor[gray]{0.9} \multicolumn{4}{|>{\columncolor[gray]{0.9}}c|}{\textbf{Individuelle metoder til lokalisering}} \\ \hline
Metode & Træningssæt & Blitzsæt & Gult sæt\\ \hline
Områder domineret af lyse gråtoner & 60.5\% &  24.6\% & 0.0\%\\ \hline
Områder med høj kontrast & 83.0\% & 5.4\% & 76.0\%\\ \hline
Frekvensanalyse & 59.3\% &  20.8\% & 52.0\%\\ \hline
Maksimer lokal kontrast & 82.8\% &  70.6\% & 68.0\%\\ \hline
Kvantifisering & 69.8\% &  85.4\% & 48.0\%\\
\hline
\end{tabular}
\caption{Resultaterne af afprøvning af de individuelle metoder til lokalisering af nummerplader på blitzsættet og de gule sæt.}
\label{fig:test:lokalisering_blitz_gul}
\end{figure}


\begin{figure}[htp]
\centering
  \begin{tabular}{|l|l|l|}
    \hline
    \rowcolor[gray]{0.9} \multicolumn{3}{|>{\columncolor[gray]{0.9}}c|}{\textbf{Blitzsæt}} \\
    \hline
    Minimal enighed & Fundne nummerplader & Korrekthed\\ \hline
    1 &  83.1\% & 85.7\%\\ \hline
  \end{tabular}
\caption{Resultaterne af afprøvning af hvor effektivt vi kan lokalisere nummerplader i blitzsættet når alle fem metoder arbejde sammen.}
\label{fig:test:lokalisering_blitz_samlet}
\end{figure}


\begin{figure}[htp]
\centering
  \begin{tabular}{|l|l|l|}
    \hline
    \rowcolor[gray]{0.9} \multicolumn{3}{|>{\columncolor[gray]{0.9}}c|}{\textbf{Gult sæt}} \\
    \hline
    Minimal enighed & Fundne nummerplader & Korrekthed\\ \hline
    1 &  84.0\% & 89.4\%\\ \hline
  \end{tabular}
\caption{Resultaterne af afprøvning af hvor effektivt vi kan lokalisere nummerplader i det gule sæt når alle fem metoder arbejde sammen.}
\label{fig:test:lokalisering_gul_samlet}
\end{figure}


\subsubsection{Det samlede system}
\begin{figure}[htp]
\centering
\begin{tabular}{|l|l|l|}\hline
\rowcolor[gray]{0.9} \multicolumn{3}{|>{\columncolor[gray]{0.9}}c|}{\textbf{Samlet system}} \\ \hline
Sæt      &  Med syntaxanalyse & Uden syntaxanalyse \\ \hline
Blitzsæt & 0\% / 0\%          & 81.5\% / 22.6\%     \\ \hline
Gult sæt & -                  & 80.0\% / 15.0\%     \\ \hline

\end{tabular}
\caption{Resultaterne for afprøvning af det samlede systems evne til at læse alle syv karakterer i nummerpladerne i blitzsættet og det gule sæt. I cellerne med procentsatser, angiver det første tal hvor i  hvor mange }
\label{fig:test:samlet_blitz_gul}
\end{figure}


HVAD ER FEJLRATEN?

\subsection{Sammenligning med andres resultater}
Parker : Kan ikke finde resultater

Shapiro: Lokaliserer 90\% af pladerne. Arbejder kun med lokalisering.

Andre opgiver i korrekt læste karakterer. Det er ike godt da... \cite{nijhuis}

kwas får 49\% . Dog får de 99\% lokalisering.

dlagnekov 95.6/5.7

MÅSKE SKAL DETTE STÅ I KONKLUSION? ELLERS HER
