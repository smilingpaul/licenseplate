\section{Resultater}
\label{sec:resultater}

I dette afsnit vil vi beskrive afprøvningen af vores system og give resultaterne af denne afprøvning. Alle afprøvninger vil blive udført på både trænings- og kontrolsættet.

%%%%%%%%%%%%%%%%%%%%%%%%%%%%%%%%%%%
%%% LOKALISERING AF NUMMERPLADE %%%
%%%%%%%%%%%%%%%%%%%%%%%%%%%%%%%%%%%

\subsection{Lokalisering af nummerplader}
I dette afsnit afprøver vi vores systems evne til at lokalisere nummerplader. Vi kigger altså ikke på nummerpladens tegn, men blot på om vi kan afgøre dens position i inddata-billedet. Vi har foretaget afprøvning af det samlede system som beskrevet i \vref{sec:system:lokalisering} samt individuelle afprøvninger af de fem metoder til lokalisering som beskrevet i samme afsnit.

Hvis vi skal have test med flere forskellige parametre, kan vi jo også køre en gang hvor vi siger at mindst to metoder skal være enige før vi udpeger et område.s
Trainingset


\subsubsection{Individuel afprøvning af metoderne}
Vi afprøver metoderne individuelt. Korrekthed kan ikke bruges til noget da vi ikke opgiver nummerpladekandidater på det individuelle niveau.
\begin{center}
\begin{tabular}{|l|l|l|}
\hline
\rowcolor[gray]{0.9} \multicolumn{3}{|>{\columncolor[gray]{0.9}}c|}{\textbf{..}} \\ \hline
Metode & Træningsæt & Kontrolsæt\\ \hline
Områder domineret af lyse gråtoner &  0 \% & 0 \%\\ \hline
Områder med høj kontrast &  0 \% & 0 \%\\ \hline
Frekvensanalyse &  0 \% & 0 \%\\ \hline
Maksimer lokal kontrast &  0 \% & 0 \%\\ \hline
Kvantifisering &  0 \% & 0 \%\\
\hline
\end{tabular}
\end{center}


\subsubsection{Afprøvning af metoderne som et system}
Vi afprøver vi hvor god vores software er til at lokalisere nummerplader når alle fem metoder arbejder sammen. Vi har foretaget afprøvningen med flere forskellige indstillinger for minimal enighed. En enighed på f.eks. vil sige at tre af de underliggende metoder til lokalisering skal have udpeget samme område for at systemet opfatter området som et numerpladeområde og sender det videre i systemet til seperation og genkendelse. Resultaterne for afprøvningerne på træningsættet er vist på figur \vref{fig:test:lokalisering_traening_samlet} og de tilsvarende resultater for kontrolsættet er vist i figur \ref{fig:test:lokalisering_kontrol_samlet}

DER VAR EN FEJL HVIS KUN EN METODE RETURNEREDE EN KANDIDAT. KØR RESULTATER FOR ENIGHED 1 IGEN.

\begin{figure}[htp]
\centering
  \begin{tabular}{|l|l|l|}
    \hline
    \rowcolor[gray]{0.9} \multicolumn{3}{|>{\columncolor[gray]{0.9}}c|}{\textbf{Træningssæt}} \\
    \hline
    Minimal enighed & Fundne nummerplader & Korrekthed\\ \hline
    1 &  97.5\% & 99.2\%\\ \hline
    2 &  95.8\% & 99.5\%\\ \hline
    3 &  87.8\% & 99.4\%\\ \hline
  \end{tabular}
\caption{Resultaterne af afprøvning af hvor effektivt vi kan lokalisere nummerplader i træningsættet når alle fem metoder arbejde sammen.}
\label{fig:test:lokalisering_traening_samlet}
\end{figure}


\begin{figure}[htp]
\centering
  \begin{tabular}{|l|l|l|}
    \hline
    \rowcolor[gray]{0.9} \multicolumn{3}{|>{\columncolor[gray]{0.9}}c|}{\textbf{Kontrolsæt}} \\
    \hline
    Minimal enighed & Fundne nummerplader & Korrekthed\\ \hline
    1 &  93.5\% & 95.6\%\\ \hline
    2 &  92.0\% & 97.7\%\\ \hline
	3 &  78.3\% & 99.6\%\\ \hline
  \end{tabular}
\caption{Resultaterne af afprøvning af hvor effektivt vi kan lokalisere nummerplader i kontrolsættet når alle fem metoder arbejde sammen.}
\label{fig:test:lokalisering_kontrol_samlet}
\end{figure}



\begin{comment} % udkommenteret da vi vel ikke skal bruge resultater på sættet med 407 billeder?
\subsubsection*{Observeret sæt på 407 billeder}
Scale: 0.25
DetectMain: 96.6/99.24
DetectQuant: 67.8/75.4
DetectSameness: 56.8/95.5
DetectContrastAvg: 62.7/85.0
DetectPlateness: 50.4/65.5
DetectCStretch: 84.0/92.7

Scale: 0.50 (Ekstremt langsomt)
DetectPlateness: 29.7/56.5
DetectCStretch:
\end{comment}

%Confusion matrix: De elementer der ligger udenfor diagonalen er elementer der ikke er nummerplader.

Delkonklusion:

VI SKAL SE HVORDAN DETECTMAIN OPFØRER SIG NÅR F.EKS. ALLE METODER SKAL VÆRE ENIGE.

Vi har skaleret ned for at spare tid.

Hvordan ændrer resultaterne sig når man ændrer opløsning af billederne?
Er der et mønster i de billeder hvot vi ikke finder nummerpladen? Hvile udfordringer møder vi: Mørke plader... 
HVAD SKER DER MED GULE ETC. PLADER??


%%%%%%%%%%%%%%%%%%%%%%%%%%
%%% SEPARATION AF TEGN %%%
%%%%%%%%%%%%%%%%%%%%%%%%%%

\subsection{Separation}

%DETTE AFSNIT SKRIVES SAMMEN MED GENKENDELSE.

I dette afsnit testes alle metoderne der bruges til at separere og genkende tegn i sammenhæng. Det vil sige at vi opfatter samarbejdet mellem disse metoder som en sammenhængende metoder. OMFORMULERES.

\subsubsection{Rotation}
%Om en nummerplade står vandret i et billede kan afgøres ved at kantdetektere billedet og udføre en Radon transformation på dette kant-billede. Hvis de stærkeste linier optræder ved $0^{\circ}$ er rotation udført korrekt.

Da vi har lavet funktionen til rotation primært ved brug af en Matlab funktion (\textit{radon}), vil det ikke være relevant at teste funktionen. Derudover vil en test af funktionen betyde at alle pladernes rotation skulle bestemmes manuelt. Vi har i stedet besluttet at funktionen testes i forbindelse med test af de andre funktioner i dette afsnit, hvliket vil sige at rotation indgår som en del af separation og genkendelse af tegn og ikke er "sin egen" funktion. ER SKREVET TIDLIGER. OMSKRIVES.

\subsubsection{Separation}
De to metoder til separation af tegn testes ved at se på om de finder syv objekter som optræder indenfor pladens koordinater. Denne forholdsvis enkle test mener vi er fyldestgørende, da funktionerne hvori metoderne er implementeret, bør være "skrappe" nok. TESTEN ER EN INDIKATION!

Vi vil desuden se på om der er forskel på succesraten i forhold til hvilket område der repræsenterer nummerpladen. Her findes to muligheder: pladen er fundet ved brug af identifikationsmetoderne eller defineret ved hjælp af de manuelt registrerede pladekoordinater (med evt. et yderområde lagt til).

%Parametre: størrelse på componenter der sorteres fra, skalering: kan vi tillade os at skalere ned (eller op, hvor vi måske får mere plads mellem komponenterne) eller mister vi for meget information VI KAN IKKE SKALERE

%OVERVEJ NEDENSTÅENDE SOM SAMLET GRAF I ET AFSNIT

%\subsubsection{Bjerg/Dal}
%HVOR GODE ER VI TIL AT FINDE 7 OMRÅDER INDEN FOR PLADEN I TRÆNINGSSÆT?
%HVOR GODE ER VI TIL AT FINDE 7 OMRÅDER INDEN FOR PLADEN I TESTSÆT
%HVOR GODE ER VI TIL AT FINDE 7 OMRÅDER INDEN FOR PLADEN I TRÆNINGSSÆT?
%HVOR GODE ER VI TIL AT FINDE 7 OMRÅDER INDEN FOR PLADEN I TESTSÆT
%HVAD SKER DER NÅR PLADEOMRÅDET UDVIDES MED 10PIXELS PÅ ALLE SIDER?
%HVAD SKER DER NÅR PLADEOMRÅDET UDVIDES MED 20PIXELS PÅ ALLE SIDER?
%HVAD SKER DER NÅR PLADEOMRÅDET UDVIDES MED 30PIXELS PÅ ALLE SIDER?


\begin{tabular}{|l|c|c|}\hline
\rowcolor[gray]{0.9} \multicolumn{3}{|>{\columncolor[gray]{0.9}}c|}{\textbf{Træningssæt}} \\ \hline
Antal pixels lagt til & Sammenhængende komponenter & Bjerg/dal \\\hline
10 & 96,5\% & 0\% \\\hline
20 & 95,75\% & 0\% \\\hline
30 & 91\% & 0\% \\\hline \end{tabular}

\begin{tabular}{|l|c|c|}\hline
\rowcolor[gray]{0.9} \multicolumn{3}{|>{\columncolor[gray]{0.9}}c|}{\textbf{Kontrolsæt}} \\ \hline
Antal pixels lagt til & Sammenhængende komponenter & Bjerg/dal \\\hline
10 & 0\% & 0\% \\\hline
20 & 0\% & 0\% \\\hline
30 & 0\% & 0\% \\\hline \end{tabular}

\begin{tabular}{|l|c|c|}\hline
\rowcolor[gray]{0.9} \multicolumn{3}{|>{\columncolor[gray]{0.9}}c|}{\textbf{Automatisk lokalisering}} \\ \hline
Sæt & Sammenhængende komponenter & Bjerg/dal \\\hline
Træningssæt & 0\% & 0\% \\\hline
Kontrolsæt & 0\% & 0\% \\\hline
\end{tabular}

%\begin{tabular}{|l|c|c|}\hline
%\rowcolor[gray]{0.9} \multicolumn{3}{|>{\columncolor[gray]{0.9}}c|}{\textbf{Kontrolsæt}} \\ \hline
%Bedste udklip & Sammenhængende komponenter & Bjerg/dal \\\hline
% & 0\% & 0\% \\\hline
%\end{tabular}

Problemer ved 10: tegn der er smeltet sammen pga. søm skruer: OE26906 og ----290. Godt udskåret: UX33152, skæve plader et problem. go kontrast: YE39734.

Problemer ved 20: for stort område giver komponenter der ikke klippes bort, fordi kontrastforstærkning giver et andet resultat, her ville dynamisk kontrast-blok størrelse måske hjælpe. Dårlig indskrænk XK29750. TS57793: GOD(?) 

Problemer ve 30: stadig for stort område: giver komponenter udenfor der vælges. dårlig indskrænk, især foran? eksempler.



%Delkonklusion:
%Sammenhængende komponenter er go og bjerg/dal dårlig?

%%% GENKENDELSE AF TEGN %%%

\subsection{Genkendelse af tegn}
I dette afsnit testes de tre metoder til genkendelse af tegn i en nummerplade. Derudover vil syntaksanalysen blive testet.

Noget om at det er i forhold til alle plader, dvs. at procenterne nedenfor er afhængige af hvor gode vi er til separere tegn...


Følgende tabeller viser resultaterne for at læse en hel plade, seks tegn i pladen osv. (med forskellige vektorlængder).

Nedenstående tests er i forhold til manuelt udklippede plader - ikke automtisk lokalisering.

\subsubsection*{Middelvektor}

Ved at lave middelvektorer på træningssættet kan vi se at man ikke kan have vektorer af længden 4, da flere bogstavs vektorer så vil være de samme (nemlig 0 0 0 0).

\begin{tabular}{|l|c|c|c|c|c|c|c|c|}\hline
\rowcolor[gray]{0.9} \multicolumn{9}{|>{\columncolor[gray]{0.9}}c|}{\textbf{Træningssæt}} \\\hline
\multicolumn{9}{|c|}{Med syntaksanalyse}\\\hline
Vektorlængde & 7 & 6 & 5 & 4 & 3 & 2 & 1 & 0\\\hline
9 & 0\% & 0\% & 0\% & 0\% & 0\% & 0\% & 0\% & 0\% \\\hline
16 & 0\% & 0\% & 0\% & 0\% & 0\% & 0\% & 0\% & 0\%\\\hline
25 & 0\% & 0\% & 0\% & 0\% & 0\% & 0\% & 0\% & 0\%\\\hline 
\multicolumn{9}{|c|}{Uden syntaksanalyse}\\\hline
Vektorlængde & 7 & 6 & 5 & 4 & 3 & 2 & 1 & 0\\\hline
9 & 0\% & 0\% & 0\% & 0\% & 0\% & 0\% & 0\% & 0\% \\\hline
16 & 0\% & 0\% & 0\% & 0\% & 0\% & 0\% & 0\% & 0\%\\\hline
25 & 0\% & 0\% & 0\% & 0\% & 0\% & 0\% & 0\% & 0\%\\\hline 
\end{tabular}

\begin{tabular}{|l|c|c|c|c|c|c|c|c|}\hline
\rowcolor[gray]{0.9} \multicolumn{9}{|>{\columncolor[gray]{0.9}}c|}{\textbf{Kontrolsæt}} \\\hline
\multicolumn{9}{|c|}{Med syntaksanalyse}\\\hline
Vektorlængde & 7 & 6 & 5 & 4 & 3 & 2 & 1 & 0\\\hline
9 & 0\% & 0\% & 0\% & 0\% & 0\% & 0\% & 0\% & 0\% \\\hline
16 & 0\% & 0\% & 0\% & 0\% & 0\% & 0\% & 0\% & 0\%\\\hline
25 & 0\% & 0\% & 0\% & 0\% & 0\% & 0\% & 0\% & 0\%\\\hline 
\multicolumn{9}{|c|}{Uden syntaksanalyse}\\\hline
Vektorlængde & 7 & 6 & 5 & 4 & 3 & 2 & 1 & 0\\\hline
9 & 0\% & 0\% & 0\% & 0\% & 0\% & 0\% & 0\% & 0\% \\\hline
16 & 0\% & 0\% & 0\% & 0\% & 0\% & 0\% & 0\% & 0\%\\\hline
25 & 0\% & 0\% & 0\% & 0\% & 0\% & 0\% & 0\% & 0\%\\\hline 
\end{tabular}

\begin{comment}
\begin{tabular}{|l|c|c|c|}\hline
\rowcolor[gray]{0.9} \multicolumn{4}{|>{\columncolor[gray]{0.9}}c|}{\textbf{Træningssæt}} \\ \hline
Vektorlængde & Alle tegn & Bogstaver & Tal \\\hline
9 & 0\% & 0\% & 0\% \\\hline
16 & 0\% & 0\% & 0\%\\\hline
25 & 0\% & 0\% & 0\%\\\hline \end{tabular}

\begin{tabular}{|l|c|c|c|}\hline
\rowcolor[gray]{0.9} \multicolumn{4}{|>{\columncolor[gray]{0.9}}c|}{\textbf{Kontrolsæt}} \\ \hline
Vektorlængde & Alle tegn & Bogstaver & Tal \\\hline
9 & 0\% & 0\% & 0\% \\\hline
16 & 0\% & 0\% & 0\% \\\hline
25 & 0\% & 0\% & 0\% \\\hline \end{tabular}
\end{comment}

\subsubsection*{Sum-billeder}

\begin{tabular}{|l|c|c|c|c|c|c|c|c|}\hline
\rowcolor[gray]{0.9} \multicolumn{9}{|>{\columncolor[gray]{0.9}}c|}{\textbf{Træningssæt}} \\\hline
\multicolumn{9}{|c|}{Med syntaksanalyse}\\\hline
Billedstørrelse & 7 & 6 & 5 & 4 & 3 & 2 & 1 & 0\\\hline
$5 \times 5$ & 0\% & 0\% & 0\% & 0\% & 0\% & 0\% & 0\% & 0\% \\\hline
$10 \times 10$ & 0\% & 0\% & 0\% & 0\% & 0\% & 0\% & 0\% & 0\%\\\hline
$20 \times 20$ & 0\% & 0\% & 0\% & 0\% & 0\% & 0\% & 0\% & 0\%\\\hline 
\multicolumn{9}{|c|}{Uden syntaksanalyse}\\\hline
Billedstørrelse & 7 & 6 & 5 & 4 & 3 & 2 & 1 & 0\\\hline
$5 \times 5$ & 0\% & 0\% & 0\% & 0\% & 0\% & 0\% & 0\% & 0\% \\\hline
$10 \times 10$ & 0\% & 0\% & 0\% & 0\% & 0\% & 0\% & 0\% & 0\%\\\hline
$20 \times 20$ & 0\% & 0\% & 0\% & 0\% & 0\% & 0\% & 0\% & 0\%\\\hline \end{tabular}

\begin{tabular}{|l|c|c|c|c|c|c|c|c|}\hline
\rowcolor[gray]{0.9} \multicolumn{9}{|>{\columncolor[gray]{0.9}}c|}{\textbf{Kontrolsæt}} \\\hline
\multicolumn{9}{|c|}{Med syntaksanalyse}\\\hline
Billedstørrelse & 7 & 6 & 5 & 4 & 3 & 2 & 1 & 0\\\hline
$5 \times 5$ & 0\% & 0\% & 0\% & 0\% & 0\% & 0\% & 0\% & 0\% \\\hline
$10 \times 10$ & 0\% & 0\% & 0\% & 0\% & 0\% & 0\% & 0\% & 0\%\\\hline
$20 \times 20$ & 0\% & 0\% & 0\% & 0\% & 0\% & 0\% & 0\% & 0\%\\\hline 
\multicolumn{9}{|c|}{Uden syntaksanalyse}\\\hline
Billedstørrelse & 7 & 6 & 5 & 4 & 3 & 2 & 1 & 0\\\hline
$5 \times 5$ & 0\% & 0\% & 0\% & 0\% & 0\% & 0\% & 0\% & 0\% \\\hline
$10 \times 10$ & 0\% & 0\% & 0\% & 0\% & 0\% & 0\% & 0\% & 0\%\\\hline
$20 \times 20$ & 0\% & 0\% & 0\% & 0\% & 0\% & 0\% & 0\% & 0\%\\\hline \end{tabular}

\begin{comment}
\begin{tabular}{|l|c|c|c|}\hline
\rowcolor[gray]{0.9} \multicolumn{4}{|>{\columncolor[gray]{0.9}}c|}{\textbf{Træningssæt}} \\ \hline
Billedstørrelse & Alle tegn & Bogstaver & Tal \\\hline
$5 \times 5$ & 0\% & 0\% & 0\% \\\hline
$10 \times 10$ & 0\% & 0\% & 0\%\\\hline
$20 \times 20$ & 0\% & 0\% & 0\%\\\hline \end{tabular}

\begin{tabular}{|l|c|c|c|}\hline
\rowcolor[gray]{0.9} \multicolumn{4}{|>{\columncolor[gray]{0.9}}c|}{\textbf{Kontrolsæt}} \\ \hline
Billedstørrelse & Alle tegn & Bogstaver & Tal \\\hline
$5 \times 5$ & 0\% & 0\% & 0\% \\\hline
$10 \times 10$ & 0\% & 0\% & 0\%\\\hline
$20 \times 20$ & 0\% & 0\% & 0\%\\\hline \end{tabular}
\end{comment}

\subsubsection*{And-billeder}

\begin{tabular}{|l|c|c|c|c|c|c|c|c|}\hline
\rowcolor[gray]{0.9} \multicolumn{9}{|>{\columncolor[gray]{0.9}}c|}{\textbf{Træningssæt}} \\\hline
\multicolumn{9}{|c|}{Med syntaksanalyse}\\\hline
Billedstørrelse & 7 & 6 & 5 & 4 & 3 & 2 & 1 & 0\\\hline
$5 \times 5$ & 0\% & 0\% & 0\% & 0\% & 0\% & 0\% & 0\% & 0\% \\\hline
$10 \times 10$ & 0\% & 0\% & 0\% & 0\% & 0\% & 0\% & 0\% & 0\%\\\hline
$20 \times 20$ & 0\% & 0\% & 0\% & 0\% & 0\% & 0\% & 0\% & 0\%\\\hline 
\multicolumn{9}{|c|}{Uden syntaksanalyse}\\\hline
Billedstørrelse & 7 & 6 & 5 & 4 & 3 & 2 & 1 & 0\\\hline
$5 \times 5$ & 0\% & 0\% & 0\% & 0\% & 0\% & 0\% & 0\% & 0\% \\\hline
$10 \times 10$ & 0\% & 0\% & 0\% & 0\% & 0\% & 0\% & 0\% & 0\%\\\hline
$20 \times 20$ & 0\% & 0\% & 0\% & 0\% & 0\% & 0\% & 0\% & 0\%\\\hline \end{tabular}

\begin{tabular}{|l|c|c|c|c|c|c|c|c|}\hline
\rowcolor[gray]{0.9} \multicolumn{9}{|>{\columncolor[gray]{0.9}}c|}{\textbf{Kontrolsæt}} \\\hline
\multicolumn{9}{|c|}{Med syntaksanalyse}\\\hline
Billedstørrelse & 7 & 6 & 5 & 4 & 3 & 2 & 1 & 0\\\hline
$5 \times 5$ & 0\% & 0\% & 0\% & 0\% & 0\% & 0\% & 0\% & 0\% \\\hline
$10 \times 10$ & 0\% & 0\% & 0\% & 0\% & 0\% & 0\% & 0\% & 0\%\\\hline
$20 \times 20$ & 0\% & 0\% & 0\% & 0\% & 0\% & 0\% & 0\% & 0\%\\\hline 
\multicolumn{9}{|c|}{Uden syntaksanalyse}\\\hline
Billedstørrelse & 7 & 6 & 5 & 4 & 3 & 2 & 1 & 0\\\hline
$5 \times 5$ & 0\% & 0\% & 0\% & 0\% & 0\% & 0\% & 0\% & 0\% \\\hline
$10 \times 10$ & 0\% & 0\% & 0\% & 0\% & 0\% & 0\% & 0\% & 0\%\\\hline
$20 \times 20$ & 0\% & 0\% & 0\% & 0\% & 0\% & 0\% & 0\% & 0\%\\\hline \end{tabular}

\begin{comment}
\begin{tabular}{|l|c|c|c|}\hline
\rowcolor[gray]{0.9} \multicolumn{4}{|>{\columncolor[gray]{0.9}}c|}{\textbf{Træningssæt}} \\ \hline
Billedstørrelse & Alle tegn & Bogstaver & Tal \\\hline
$5 \times 5$ & 0\% & 0\% & 0\% \\\hline
$10 \times 10$ & 0\% & 0\% & 0\%\\\hline
$20 \times 20$ & 0\% & 0\% & 0\%\\\hline \end{tabular}

\begin{tabular}{|l|c|c|c|}\hline
\rowcolor[gray]{0.9} \multicolumn{4}{|>{\columncolor[gray]{0.9}}c|}{\textbf{Kontrolsæt}} \\ \hline
Billedstørrelse & Alle tegn & Bogstaver & Tal \\\hline
$5 \times 5$ & 0\% & 0\% & 0\% \\\hline
$10 \times 10$ & 0\% & 0\% & 0\%\\\hline
$20 \times 20$ & 0\% & 0\% & 0\%\\\hline \end{tabular}
\end{comment}


%HVILKEN POSITION PÅ HITLISTEN TAGER DEN I GENNEMSNIT PR POSITION - TRÆNING?
%HVILKEN POSITION PÅ HITLISTEN TAGER DEN I GENNEMSNIT PR POSITION - TEST?
%MAXHITNO: HVAD GIVER DET JO LÆNGERE NED AD HITLISTEN VI KAN GÅ?

\begin{comment}
Syntaks analyse: Hvilke hits bliver valgt på hitlisterne af syntaksanalysen (sæt maxhitno højt):

\begin{tabular}{|l|c|}\hline
\multicolumn{2}{|l|}{Træningssæt} \\\hline
Hitnr. & Valgt \\\hline
1 & 95,4\% \\\hline
2 & 3\% \\\hline
3 & 0\% \\\hline
4 & 0\% \\\hline
5 & 0\% \\\hline
6 & 0\% \\\hline \end{tabular}

\begin{tabular}{|l|c|}\hline
\multicolumn{2}{|l|}{Kontrolsæt} \\\hline
Hitnr. & Valgt \\\hline
1 & 92,9\% \\\hline
2 & 0\% \\\hline
3 & 0\% \\\hline
4 & 0\% \\\hline
5 & 0\% \\\hline
6 & 0\% \\\hline \end{tabular}

\end{comment}

\subsubsection*{De enkelte tegn}

Evt. en tabel over de enkelte tegn A, B, C osv. Hvor gode er vi til at genkende disse? Tabellen kunne laves udfra den bedste vektorstørrelse?

%Måske bare en tabel over de dårligste tegn, altså dem der er sværest at genkende? På denne måde bliver det ikke en LANG tabel

\begin{tabular}{|l|c|c|c|}\hline
\rowcolor[gray]{0.9} \multicolumn{4}{|>{\columncolor[gray]{0.9}}c|}{\textbf{Træningssæt}} \\ \hline
Tegn & Middelvektor & Sum-billeder & And-billeder\\\hline
0 & 0\% & 0\% & 0\%\\\hline
1 & 0\%  & 0\% & 0\%\\\hline
2 & 0\% & 0\% & 0\%\\\hline
3 & 0\% & 0\% & 0\%\\\hline
4 & 0\% & 0\% & 0\%\\\hline
5 & 0\% & 0\% & 0\%\\\hline
6 & 0\% & 0\% & 0\%\\\hline
7 & 0\% & 0\% & 0\%\\\hline
8 & 0\% & 0\% & 0\%\\\hline
9 & 0\% & 0\% & 0\%\\\hline
\end{tabular}

\begin{tabular}{|l|c|c|c|}\hline
\rowcolor[gray]{0.9} \multicolumn{4}{|>{\columncolor[gray]{0.9}}c|}{\textbf{Træningssæt}} \\ \hline
Tegn & Middelvektor & Sum-billeder & And-billeder\\\hline
A & 0\% & 0\% & 0\%\\\hline
B & 0\% & 0\% & 0\%\\\hline
C & 0\% & 0\% & 0\%\\\hline
D & 0\% & 0\% & 0\%\\\hline
E & 0\% & 0\% & 0\%\\\hline
H & 0\% & 0\% & 0\%\\\hline
J & 0\% & 0\% & 0\%\\\hline
K & 0\% & 0\% & 0\%\\\hline 
L & 0\% & 0\% & 0\%\\\hline
M & 0\% & 0\% & 0\%\\\hline
N & 0\% & 0\% & 0\%\\\hline
O & 0\% & 0\% & 0\%\\\hline
P & 0\% & 0\% & 0\%\\\hline
R & 0\% & 0\% & 0\%\\\hline
S & 0\% & 0\% & 0\%\\\hline
T & 0\% & 0\% & 0\%\\\hline
U & 0\% & 0\% & 0\%\\\hline
V & 0\% & 0\% & 0\%\\\hline
X & 0\% & 0\% & 0\%\\\hline
Y & 0\% & 0\% & 0\%\\\hline
Z & 0\% & 0\% & 0\%\\\hline
\end{tabular}


\begin{tabular}{|l|c|c|c|}\hline
\rowcolor[gray]{0.9} \multicolumn{4}{|>{\columncolor[gray]{0.9}}c|}{\textbf{Kontrolsæt}} \\ \hline
Tegn & Middelvektor & Sum-billeder & And-billeder\\\hline
0 & 0\% & 0\% & 0\%\\\hline
1 & 0\%  & 0\% & 0\%\\\hline
2 & 0\% & 0\% & 0\%\\\hline
3 & 0\% & 0\% & 0\%\\\hline
4 & 0\% & 0\% & 0\%\\\hline
5 & 0\% & 0\% & 0\%\\\hline
6 & 0\% & 0\% & 0\%\\\hline
7 & 0\% & 0\% & 0\%\\\hline
8 & 0\% & 0\% & 0\%\\\hline
9 & 0\% & 0\% & 0\%\\\hline
\end{tabular}

\begin{tabular}{|l|c|c|c|}\hline
\rowcolor[gray]{0.9} \multicolumn{4}{|>{\columncolor[gray]{0.9}}c|}{\textbf{Kontrolsæt}} \\ \hline
Tegn & Middelvektor & Sum-billeder & And-billeder\\\hline
A & 0\% & 0\% & 0\%\\\hline
B & 0\% & 0\% & 0\%\\\hline
C & 0\% & 0\% & 0\%\\\hline
D & 0\% & 0\% & 0\%\\\hline
E & 0\% & 0\% & 0\%\\\hline
H & 0\% & 0\% & 0\%\\\hline
J & 0\% & 0\% & 0\%\\\hline
K & 0\% & 0\% & 0\%\\\hline 
L & 0\% & 0\% & 0\%\\\hline
M & 0\% & 0\% & 0\%\\\hline
N & 0\% & 0\% & 0\%\\\hline
O & 0\% & 0\% & 0\%\\\hline
P & 0\% & 0\% & 0\%\\\hline
R & 0\% & 0\% & 0\%\\\hline
S & 0\% & 0\% & 0\%\\\hline
T & 0\% & 0\% & 0\%\\\hline
U & 0\% & 0\% & 0\%\\\hline
V & 0\% & 0\% & 0\%\\\hline
X & 0\% & 0\% & 0\%\\\hline
Y & 0\% & 0\% & 0\%\\\hline
Z & 0\% & 0\% & 0\%\\\hline
\end{tabular}

Ovenstående med og uden syntaksanalyse? kun med? kun uden?

%Søren: Ved klassifikation bruges en eller anden grænseværdi som bestemmer hvad et element skal klassificeres som: denne grænseværdi kan reguleres og kan derfor testes.

Delkonklusion:

Vi kunne også have testet hvilke tegn der bliver valgt i stedet for, når et forkert tegn vælges. Vælges 8 f.eks. når B burde ha været valgt??

Billederne af tegnene er ret store i forhold til dem der bruges i litteraturen. Dette giver vores system en væsentlig fordel i forhold til "de andre".

\subsection{Det samlede system}

Her vil vi afprøve det samlede system... eller har vi gjort det "på vejen"?
KUN MED DEN BEDSTE KONFIGURATION:

pcT. AF ALLE TEGN
PCT. AF ALLE PLADER

\subsection{Sammenligning med andres resultater}

MÅSKE SKAL DETTE STÅ I KONKLUSION? ELLERS HER
