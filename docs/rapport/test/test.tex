\section{Resultater}

HVAD SKER DER MED GULE ETC. PLADER??
\subsection{Indsamling af testdata}

%Fra start var vi meget opmærksomme på at afgrænsningen af projektet skulle være klar, så det ikke blev for omfattende. Omkring valg af billedemateriale afholdt vi os eksempelvis fra at 

%Til at udarbejde og teste systemet havde vi brug for nogle fotografier af biler med nummerplader. Ved fotograferingen var vi opmærksomme på følgende afgrænsninger:




%\begin{figure}[h]
%\begin{center}
%\includegraphics[width=10cm]{illu/B_XC33139.jpg}
%\label{b_xc33139}
%\caption{Fotografieksempel}
%\end{center}
%\end{figure}

% vi holder dem adskilt - ikke noget med histo
Da vi bl.a. havde planer om udarbejdelse af en histogrambaseret metode til identificering af nummerplader (se afsnit \ref{histo}), holdt vi de to fotografisæt adskilte. På denne måde ville det f.eks. være muligt for os at teste om skift fra et kamera til et andet, vil give ændrede resultater. Billederne blev navngivet i vores database, så det, i deres filnavn, indgik om billedet forestillede en bil set forfra eller bagfra samt hvilken nummerplade bilen på nummerpladen havde. Derudover udarbejdede vi et mindre program som hjalp os til at identificere nummerpladens fire hjørnekoordinater og indskrive disse i filnavnet. Denne sidste tilføjelse ville hjælpe os i testfasen, til at undersøge om de nummerpladekandidater vores system ville udvælge er de korrekte.
 
Vi skal ha' taget flere billeder! Skal vi have nogle billeder hvor der ikke er nummerplader i??


%\subsection{Fotografering}
%Hvor mange og hvilke billeder har vi taget.
%At billederne er taget i naturligt lys for at undgå kunstige farver og genskin fra pladen som er malet med reflekterende materiale.
%Opdeling af billeder i træning- og testsæt
%til f.eks. histogram metode herunder adskillelse af fotos fra de to forskellige kameraer.

\subsection{Lokalisering af nummerplade}
HVOR GODE ER VI PÅ TRÆNINGSSÆT?
HVOR GODE ER VI PÅ TESTSÆT?

\subsubsection{Områder domineret af lyse gråtoner}
HVOR GODE ER VI PÅ TRÆNINGSSÆT?
HVOR GODE ER VI PÅ TESTSÆT?

\subsubsection{Områder med høj kontrast}
HVOR GODE ER VI PÅ TRÆNINGSSÆT?
HVOR GODE ER VI PÅ TESTSÆT?

\subsubsection{Frekvensanalyse}
HVOR GODE ER VI PÅ TRÆNINGSSÆT?
HVOR GODE ER VI PÅ TESTSÆT?

\subsubsection{Maksimer lokal kontrast}
HVOR GODE ER VI PÅ TRÆNINGSSÆT?
HVOR GODE ER VI PÅ TESTSÆT?

\subsubsection{Kvantifisering}
HVOR GODE ER VI PÅ TRÆNINGSSÆT?
HVOR GODE ER VI PÅ TESTSÆT?

\subsubsection{endeligt valg af nummerpladekandidat}

Min antal enige 1:
HVOR GODE ER VI PÅ TRÆNINGSSÆT?
HVOR GODE ER VI PÅ TESTSÆT?


Min antal enige 2:
HVOR GODE ER VI PÅ TRÆNINGSSÆT?
HVOR GODE ER VI PÅ TESTSÆT?

Min antal enige 3:
HVOR GODE ER VI PÅ TRÆNINGSSÆT?
HVOR GODE ER VI PÅ TESTSÆT?

Min antal enige 5:
HVOR GODE ER VI PÅ TRÆNINGSSÆT?
HVOR GODE ER VI PÅ TESTSÆT?

Min antal enige 5:
HVOR GODE ER VI PÅ TRÆNINGSSÆT?
HVOR GODE ER VI PÅ TESTSÆT?


VI SKAL SE HVORDAN DETECTMAIN OPFØRER SIG NÅR F.EKS. ALLE METODER SKAL VÆRE ENIGE.
 Vi har skaleret ned for at spare tid.

\subsubsection*{Observeret sæt på 407 billeder}
Scale: 0.25
DetectMain: 96.6/99.24
DetectQuant: 67.8/75.4
DetectSameness: 56.8/95.5
DetectContrastAvg: 62.7/85.0
DetectPlateness: 50.4/65.5
DetectCStretch: 84.0/92.7

Scale: 0.50 (Ekstremt langsomt)
DetectPlateness: 29.7/56.5
DetectCStretch:

Der skal testet på hver metode som fritstående. Herefter på DetectMain der samler metoderne. Hvordan ændrer resultaterne sig når man ændrer opløsning af billederne?
Hvor gode er vi på det set vi har observeret? Hvor gode er vi på et set vi ikke har observeret. Er der et mønster i de billeder hvot vi ikke finder nummerpladen? Hvile udfordringer møder vi: Mørke plader... 

Confusion matrix: De elementer der ligger udenfor diagonalen er elementer der ikke er nummerplader.

Eksempel på testtabel for identifikation af nummerplader:

\begin{tabular}{|l|l|l|l|l|l|}
\hline
\multicolumn{6}{|c|}{Metode: Histogram} \\ \hline
Param 1 & Param 2 & Skalering & Overordnet resultat & Sande positiver & Bemærkninger\\ \hline
1 & 1 & 1 & 19 \% & 19 \% & skide godt\\ \hline
2 & 2 & 2 & 19 \% & 21 \% & vildt godt \\ \hline
3 & 3 & 3 & 19 \% & 22 \% & diku niveau \\
\hline
\end{tabular}

Konklussion:

\subsection{Separation af tegn}
Test af de to metoder.
Er alle karakteret indenfor pladen? Det er kun nødvendigt at se på om tegnene er indenfor pladen da blah blah. Er der forskel på succesraten i forhold til størrelsen på pladen? Er det forskel på succesraten i forhold til om vi ser på pladerne vi manuelt har plottet (med evt. et tilfældigt yderområde lagt til) eller om vi ser på de nummerplader som vores indetifikationsmetoder finder?

Parametre: størrelse på componenter der sorteres fra, skalering: kan vi tillade os at skalere ned (eller op, hvor vi måske får mere plads mellem komponenterne) eller mister vi for meget information

OVERVEJ NEDENSTÅENDE SOM SAMLET GRAF I ET AFSNIT
\subsubsection{Sammenhængende komponenter}
HVOR GODE ER VI TIL AT FINDE 7 OMRÅDER INDEN FOR PLADEN I TRÆNINGSSÆT?
HVOR GODE ER VI TIL AT FINDE 7 OMRÅDER INDEN FOR PLADEN I TESTSÆT

\subsubsection{Bjerg/Dal}
HVOR GODE ER VI TIL AT FINDE 7 OMRÅDER INDEN FOR PLADEN I TRÆNINGSSÆT?
HVOR GODE ER VI TIL AT FINDE 7 OMRÅDER INDEN FOR PLADEN I TESTSÆT



\subsection{Genkendelse af tegn}
Hvor gode er vi på tal. Hvor gode er vi på bogstaver. Hvor gode er vi på hele plader.


\subsubsection{Feature vektor}
HVOR GODE ER VI PÅ TAL I TRÆNINGSSÆT?
HVOR GODE ER VI PÅ TAL I TESTSÆT?

HVOR GODE ER VI PÅ BOGSTAVER I TRÆNINGSSÆT?
HVOR GODE ER VI PÅ BOGSTAVER I TESTSÆT?

HVAD ER GENKENDELSESPROCENTEN PÅ TEGN PR PLADE I TRÆNINGSÆT?
HVAD ER GENKENDELSESPROCENTEN PÅ TEGN PR PLADE I TESTSÆT?

HVOR MANGE PLADER ER KORREKT LÆST I TRÆNINGSSÆT?
HVOR MANGE PLADER ER KORREKT LÆST I TETSSÆT?

HVAD SKER DER NÅR PLADEOMRÅDET UDVIDES MED 10PIXELS PÅ ALLE SIDER?
HVAD SKER DER NÅR PLADEOMRÅDET UDVIDES MED 20PIXELS PÅ ALLE SIDER?
HVAD SKER DER NÅR PLADEOMRÅDET UDVIDES MED 30PIXELS PÅ ALLE SIDER?

HVILKEN POSITION PÅ HITLISTEN TAGER DEN I GENNEMSNIT PR POSITION - TRÆNING?
HVILKEN POSITION PÅ HITLISTEN TAGER DEN I GENNEMSNIT PR POSITION - TEST?

UDEN SYNTAKSANALYSE - TRÆNING?
UDEN SYNTAKSANALYSE - TEST?


\subsubsection{Feature vektor}



Ved klassifikation bruges en eller anden grænseværdi som bestemmer hvad et element skal klassificeres som: denne grænseværdi kan reguleres og kan derfor testes.

Konklusion:

