\section{Resultater}

HVAD SKER DER MED GULE ETC. PLADER??
\subsection{Indsamling af testdata}

%Fra start var vi meget opmærksomme på at afgrænsningen af projektet skulle være klar, så det ikke blev for omfattende. Omkring valg af billedemateriale afholdt vi os eksempelvis fra at 

%Til at udarbejde og teste systemet havde vi brug for nogle fotografier af biler med nummerplader. Ved fotograferingen var vi opmærksomme på følgende afgrænsninger:


%\begin{figure}[h]
%\begin{center}
%\includegraphics[width=10cm]{illu/B_XC33139.jpg}
%\label{b_xc33139}
%\caption{Fotografieksempel}
%\end{center}
%\end{figure}

% vi holder dem adskilt - ikke noget med histo
Da vi bl.a. havde planer om udarbejdelse af en histogrambaseret metode til identificering af nummerplader (se afsnit \ref{histo}), holdt vi de to fotografisæt adskilte. På denne måde ville det f.eks. være muligt for os at teste om skift fra et kamera til et andet, vil give ændrede resultater. Billederne blev navngivet i vores database, så det, i deres filnavn, indgik om billedet forestillede en bil set forfra eller bagfra samt hvilken nummerplade bilen på nummerpladen havde. Derudover udarbejdede vi et mindre program som hjalp os til at identificere nummerpladens fire hjørnekoordinater og indskrive disse i filnavnet. Denne sidste tilføjelse ville hjælpe os i testfasen, til at undersøge om de nummerpladekandidater vores system ville udvælge er de korrekte.
 
Kontrolsæt: 600 billeder


%\subsection{Fotografering}
%Hvor mange og hvilke billeder har vi taget.
%At billederne er taget i naturligt lys for at undgå kunstige farver og genskin fra pladen som er malet med reflekterende materiale.
%Opdeling af billeder i træning- og testsæt
%til f.eks. histogram metode herunder adskillelse af fotos fra de to forskellige kameraer.

%%% LOKALISERING AF NUMMERPLADE %%%

\subsection{Lokalisering af nummerplade}
HVOR GODE ER VI PÅ TRÆNINGSSÆT?
HVOR GODE ER VI PÅ TESTSÆT?

\subsubsection{Områder domineret af lyse gråtoner}
HVOR GODE ER VI PÅ TRÆNINGSSÆT?
HVOR GODE ER VI PÅ TESTSÆT?

\begin{tabular}{|l|l|l|l|l|}
\hline
\multicolumn{5}{|c|}{DetectMain} \\ \hline
Param 1 & Param 2 & Skalering & Overordnet resultat & Sande positiver\\ \hline
1 & 1 & 1 & 19 \% & 19 \%\\ \hline
2 & 2 & 2 & 19 \% & 21 \% \\ \hline
3 & 3 & 3 & 19 \% & 22 \% \\
\hline
\end{tabular}

\subsubsection{Områder med høj kontrast}
HVOR GODE ER VI PÅ TRÆNINGSSÆT?
HVOR GODE ER VI PÅ TESTSÆT?

\begin{tabular}{|l|l|l|l|l|}
\hline
\multicolumn{5}{|c|}{DetectMain} \\ \hline
Param 1 & Param 2 & Skalering & Overordnet resultat & Sande positiver\\ \hline
1 & 1 & 1 & 19 \% & 19 \%\\ \hline
2 & 2 & 2 & 19 \% & 21 \% \\ \hline
3 & 3 & 3 & 19 \% & 22 \% \\
\hline
\end{tabular}

\subsubsection{Frekvensanalyse}
HVOR GODE ER VI PÅ TRÆNINGSSÆT?
HVOR GODE ER VI PÅ TESTSÆT?

\begin{tabular}{|l|l|l|l|l|}
\hline
\multicolumn{5}{|c|}{DetectMain} \\ \hline
Param 1 & Param 2 & Skalering & Overordnet resultat & Sande positiver\\ \hline
1 & 1 & 1 & 19 \% & 19 \%\\ \hline
2 & 2 & 2 & 19 \% & 21 \% \\ \hline
3 & 3 & 3 & 19 \% & 22 \% \\
\hline
\end{tabular}

\subsubsection{Maksimer lokal kontrast}
HVOR GODE ER VI PÅ TRÆNINGSSÆT?
HVOR GODE ER VI PÅ TESTSÆT?

\begin{tabular}{|l|l|l|l|l|}
\hline
\multicolumn{5}{|c|}{DetectMain} \\ \hline
Param 1 & Param 2 & Skalering & Overordnet resultat & Sande positiver\\ \hline
1 & 1 & 1 & 19 \% & 19 \%\\ \hline
2 & 2 & 2 & 19 \% & 21 \% \\ \hline
3 & 3 & 3 & 19 \% & 22 \% \\
\hline
\end{tabular}

\subsubsection{Kvantifisering}
HVOR GODE ER VI PÅ TRÆNINGSSÆT?
HVOR GODE ER VI PÅ TESTSÆT?

\begin{tabular}{|l|l|l|l|l|}
\hline
\multicolumn{5}{|c|}{DetectMain} \\ \hline
Param 1 & Param 2 & Skalering & Overordnet resultat & Sande positiver\\ \hline
1 & 1 & 1 & 19 \% & 19 \%\\ \hline
2 & 2 & 2 & 19 \% & 21 \% \\ \hline
3 & 3 & 3 & 19 \% & 22 \% \\
\hline
\end{tabular}

\subsubsection{endeligt valg af nummerpladekandidat}

Min antal enige 1:
HVOR GODE ER VI PÅ TRÆNINGSSÆT?
HVOR GODE ER VI PÅ TESTSÆT?


Min antal enige 2:
HVOR GODE ER VI PÅ TRÆNINGSSÆT?
HVOR GODE ER VI PÅ TESTSÆT?

Min antal enige 3:
HVOR GODE ER VI PÅ TRÆNINGSSÆT?
HVOR GODE ER VI PÅ TESTSÆT?

Min antal enige 5:
HVOR GODE ER VI PÅ TRÆNINGSSÆT?
HVOR GODE ER VI PÅ TESTSÆT?

Min antal enige 5:
HVOR GODE ER VI PÅ TRÆNINGSSÆT?
HVOR GODE ER VI PÅ TESTSÆT?

\begin{tabular}{|l|l|l|l|l|}
\hline
\multicolumn{5}{|c|}{DetectMain} \\ \hline
Param 1 & Param 2 & Skalering & Overordnet resultat & Sande positiver\\ \hline
1 & 1 & 1 & 19 \% & 19 \%\\ \hline
2 & 2 & 2 & 19 \% & 21 \% \\ \hline
3 & 3 & 3 & 19 \% & 22 \% \\
\hline
\end{tabular}


VI SKAL SE HVORDAN DETECTMAIN OPFØRER SIG NÅR F.EKS. ALLE METODER SKAL VÆRE ENIGE.
 Vi har skaleret ned for at spare tid.

\subsubsection*{Observeret sæt på 407 billeder}
Scale: 0.25
DetectMain: 96.6/99.24
DetectQuant: 67.8/75.4
DetectSameness: 56.8/95.5
DetectContrastAvg: 62.7/85.0
DetectPlateness: 50.4/65.5
DetectCStretch: 84.0/92.7

Scale: 0.50 (Ekstremt langsomt)
DetectPlateness: 29.7/56.5
DetectCStretch:

Der skal testet på hver metode som fritstående. Herefter på DetectMain der samler metoderne. Hvordan ændrer resultaterne sig når man ændrer opløsning af billederne?
Hvor gode er vi på det set vi har observeret? Hvor gode er vi på et set vi ikke har observeret. Er der et mønster i de billeder hvot vi ikke finder nummerpladen? Hvile udfordringer møder vi: Mørke plader... 

Confusion matrix: De elementer der ligger udenfor diagonalen er elementer der ikke er nummerplader.




Delkonklussion:

%%% SEPARATION AF TEGN %%%

\subsection{Separation af tegn}
I dette afsnit testes de to metoder til separation af tegn. Metoden til rotation vil ikke blive testes, blot kommenteret.

\subsubsection*{Rotation}
Om en nummerplade står vandret i et billede kan afgøres ved at kantdetektere billedet og udføre en Radon transformation på dette kant-billede. Hvis de stærkeste linier optræder ved $0^{\circ}$ er rotation udført korrekt.

Vi har ikke testet det fordi...

VI KUNNE MÅSKE GODT TESTE DET?

\subsubsection*{Separation}
De to metoder til separation af tegn testes ved at se på om de finder syv objekter som optræder indenfor pladens koordinater. Denne forholdsvis enkle test mener vi er fyldestgørende, da funktionerne hvori metoderne er implementeret, bør være "skrappe" nok.

Vi vil desuden se på om der er forskel på succesraten i forhold til hvilket område der repræsenterer nummerpladen: Her findes to muligheder: pladen er fundet med vores egne metoder eller defineret ved hjælp af pladekoordinaterne og et eller andet lagt til.
%Er det forskel på succesraten i forhold til om vi ser på pladerne vi manuelt har plottet (med evt. et tilfældigt yderområde lagt til) eller om vi ser på de nummerplader som vores indetifikationsmetoder finder?

Parametre: størrelse på componenter der sorteres fra, skalering: kan vi tillade os at skalere ned (eller op, hvor vi måske får mere plads mellem komponenterne) eller mister vi for meget information VI KAN IKKE SKALERE

VI VIL KUN SE PÅ TEST AF DETTE NIVEAU, IKKE? 

%OVERVEJ NEDENSTÅENDE SOM SAMLET GRAF I ET AFSNIT

%\subsubsection{Bjerg/Dal}
%HVOR GODE ER VI TIL AT FINDE 7 OMRÅDER INDEN FOR PLADEN I TRÆNINGSSÆT?
%HVOR GODE ER VI TIL AT FINDE 7 OMRÅDER INDEN FOR PLADEN I TESTSÆT
%HVOR GODE ER VI TIL AT FINDE 7 OMRÅDER INDEN FOR PLADEN I TRÆNINGSSÆT?
%HVOR GODE ER VI TIL AT FINDE 7 OMRÅDER INDEN FOR PLADEN I TESTSÆT
%HVAD SKER DER NÅR PLADEOMRÅDET UDVIDES MED 10PIXELS PÅ ALLE SIDER?
%HVAD SKER DER NÅR PLADEOMRÅDET UDVIDES MED 20PIXELS PÅ ALLE SIDER?
%HVAD SKER DER NÅR PLADEOMRÅDET UDVIDES MED 30PIXELS PÅ ALLE SIDER?


\begin{tabular}{|l|c|c|}\hline
\multicolumn{3}{|l|}{Træningssæt} \\\hline
Metode & Sammenhængende komponenter & Bjerg/dal \\\hline
Egne metoder & 96\% & 3\% \\\hline
10 & 96\% & 3\% \\\hline
20 & 96\% & 3\% \\\hline
30 & 96\% & 3\% \\\hline \end{tabular}

\begin{tabular}{|l|c|c|}\hline
\multicolumn{3}{|l|}{Kontrolsæt} \\\hline
Metode & Sammenhængende komponenter & Bjerg/dal \\\hline
Egne metoder & 96\% & 3\% \\\hline
10 & 96\% & 3\% \\\hline
20 & 96\% & 3\% \\\hline
30 & 96\% & 3\% \\\hline \end{tabular}

Delkonklusion:
Sammenhængende komponenter er go og bjerg/dal dårlig?

%%% GENKENDELSE AF TEGN %%%

\subsection{Genkendelse af tegn}
I dette afsnit testes de to metoder til genkendelse af tegn i en nummerplade. Derudover vil syntaksanalysen blive testet.

Noget om at det er i forhold til alle plader, dvs. at procenterne nedenfor er afhængige af hvor gode vi er til separere tegn.

\subsubsection{Egenskabsvektor}

%HVOR MANGE PLADER ER KORREKT LÆST I TRÆNINGSSÆT?
%HVOR MANGE PLADER ER KORREKT LÆST I TETSSÆT?
%HVOR MANGE PLADER ER KORREKT LÆST PÅ 6 POSITIONER I TRÆNINGSSÆT?
%HVOR MANGE PLADER ER KORREKT LÆST PÅ 6 POSITIONER I TETSSÆT?
%HVOR MANGE PLADER ER KORREKT LÆST PÅ 5 POSITIONER I TRÆNINGSSÆT?
%HVOR MANGE PLADER ER KORREKT LÆST PÅ 5 POSITIONER I TETSSÆT?

Følgende tabeller viser resultaterne for at læse en hel plade, seks tegn i pladen osv. (med forskellige vektorlængder).

\begin{tabular}{|l|c|c|c|c|c|c|}\hline
\multicolumn{7}{|l|}{Træningssæt} \\\hline
Vektorlængde & Hele pladen læst & 6 tegn læst & 5 tegn & 4 tegn & 3 tegn & 2 tegn \\\hline
9 & 0\% & 0\% & 0\% & 0\% & 0\% & 0\% \\\hline
16 & 0\% & 0\% & 0\% & 0\% & 0\% & 0\% \\\hline
25 & 0\% & 0\% & 0\% & 0\% & 0\% & 0\% \\\hline \end{tabular}

\begin{tabular}{|l|c|c|c|c|c|c|}\hline
\multicolumn{7}{|l|}{Kontrolsæt} \\\hline
Vektorlængde & Hele pladen læst & 6 tegn læst & 5 tegn & 4 tegn & 3 tegn & 2 tegn \\\hline
9 & 0\% & 0\% & 0\% & 0\% & 0\% & 0\% \\\hline
16 & 0\% & 0\% & 0\% & 0\% & 0\% & 0\% \\\hline
25 & 0\% & 0\% & 0\% & 0\% & 0\% & 0\% \\\hline \end{tabular}

%HVOR GODE ER VI PÅ TAL I TRÆNINGSSÆT?
%HVOR GODE ER VI PÅ TAL I TESTSÆT?
%HVOR GODE ER VI PÅ BOGSTAVER I TRÆNINGSSÆT?
%HVOR GODE ER VI PÅ BOGSTAVER I TESTSÆT?
%HVAD ER GENKENDELSESPROCENTEN PÅ TEGN PR PLADE I TRÆNINGSÆT?
%HVAD ER GENKENDELSESPROCENTEN PÅ TEGN PR PLADE I TESTSÆT?

Følgende tabeller viser hvor godt systemet er til at læse bogstaver hhv. tal. hhv. alle tegn

\begin{tabular}{|l|c|c|c|}\hline
\multicolumn{4}{|l|}{Træningssæt} \\\hline
Vektorlængde & Alle tegn & Bogstaver & Tal \\\hline
9 & 0\% & 0\% & 0\% \\\hline
16 & 0\% & 0\% & 0\%\\\hline
25 & 0\% & 0\% & 0\%\\\hline \end{tabular}

\begin{tabular}{|l|c|c|c|}\hline
\multicolumn{4}{|l|}{Kontrolsæt} \\\hline
Vektorlængde & Alle tegn & Bogstaver & Tal \\\hline
9 & 0\% & 0\% & 0\% \\\hline
16 & 0\% & 0\% & 0\% \\\hline
25 & 0\% & 0\% & 0\% \\\hline \end{tabular}

%HVILKEN POSITION PÅ HITLISTEN TAGER DEN I GENNEMSNIT PR POSITION - TRÆNING?
%HVILKEN POSITION PÅ HITLISTEN TAGER DEN I GENNEMSNIT PR POSITION - TEST?
%MAXHITNO: HVAD GIVER DET JO LÆNGERE NED AD HITLISTEN VI KAN GÅ?

Syntaks analyse: Hvilke hits bliver valgt på hitlisterne af syntaksanalysen (sæt maxhitno højt):

\begin{tabular}{|l|c|}\hline
\multicolumn{2}{|l|}{Træningssæt} \\\hline
Hitnr. & Valgt \\\hline
1 & 95,4\% \\\hline
2 & 3\% \\\hline
3 & 0\% \\\hline
4 & 0\% \\\hline
5 & 0\% \\\hline
6 & 0\% \\\hline \end{tabular}

\begin{tabular}{|l|c|}\hline
\multicolumn{2}{|l|}{Kontrolsæt} \\\hline
Hitnr. & Valgt \\\hline
1 & 92,9\% \\\hline
2 & 0\% \\\hline
3 & 0\% \\\hline
4 & 0\% \\\hline
5 & 0\% \\\hline
6 & 0\% \\\hline \end{tabular}

UDEN SYNTAKSANALYSE - TRÆNING?
UDEN SYNTAKSANALYSE - TEST?

Evt. en tabel over de enkelte tegn A, B, C osv. Hvor gode er vi til at genkende disse? Tabellen kunne laves udfra den bedste vektorstørrelse? Måske bare en tabel over de dårligste tegn, altså dem der er sværest at genkende? På denne måde bliver det ikke en LANG tabel

\begin{tabular}{|l|c|}\hline
\multicolumn{2}{|l|}{Træningsæt} \\\hline
Tegn & Valgt \\\hline
0 & 92,9\% \\\hline
1 & 0\% \\\hline
2 & 0\% \\\hline
3 & 0\% \\\hline
4 & 0\% \\\hline
5 & 0\% \\\hline
6 & 92,9\% \\\hline
7 & 0\% \\\hline
8 & 0\% \\\hline
9 & 0\% \\\hline
A & 0\% \\\hline
B & 0\% \\\hline
C & 92,9\% \\\hline
D & 0\% \\\hline
E & 0\% \\\hline
H & 0\% \\\hline
J & 0\% \\\hline
K & 0\% \\\hline 
L & 92,9\% \\\hline
M & 0\% \\\hline
N & 0\% \\\hline
O & 0\% \\\hline
P & 0\% \\\hline
R & 0\% \\\hline
S & 0\% \\\hline
T & 0\% \\\hline
U & 0\% \\\hline
V & 0\% \\\hline
X & 0\% \\\hline
Y & 0\% \\\hline
Z & 0\% \\\hline \end{tabular}



\subsubsection{Summerede billeder}
Samme tabeller som ovenfor?


%Søren: Ved klassifikation bruges en eller anden grænseværdi som bestemmer hvad et element skal klassificeres som: denne grænseværdi kan reguleres og kan derfor testes.

Delkonklusion:
Billederne af tegnene er ret store i forhold til dem der bruges i litteraturen. Dette giver vores system en væsentlig fordel i forhold til "de andre".


