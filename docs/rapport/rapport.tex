\documentclass[11pt,a4paper,draft]{article}
\usepackage[utf8]{inputenc}
%\usepackage{ucs}
\usepackage[danish]{babel}
%\usepackage{amsmath}
%\usepackage{amsfonts}
%\usepackage{amssymb}
\usepackage{verbatim}
%\usepackage{listings} % Better source code listings
\usepackage{graphicx}
\usepackage{hyperref}
\usepackage{fixme}

\author{Tobias Balle-Petersen og Esben Paul Bugge}
\title{Bachelorprojekt: Automatisk identifikation og læsning af nummerplader}

\parindent=0pt
\parskip=10pt
\usepackage[top=3cm, bottom=2.5cm, left=4cm, right=4cm]{geometry} 

%\lstset{language=python}
%\lstset{inputencoding=latin1}
%\lstset{extendedchars=true}
%\lstset{breaklines=true}
%\lstset{commentstyle=\textit}
%\lstset{showstringspaces=false}
%lstset{numbers=left, numberstyle=\tiny, stepnumber=2, numbersep=5pt,tabsize=3,basicstyle=\small}
%\lstset{numbers=left, numberstyle=\tiny, stepnumber=2, numbersep=5pt,stringstyle=\ttfamily, showstringspaces=false, basicstyle=\small, language={python}}

\hyphenation{ar-bej-de psy-ko-lo-gi-ske over-blik brug-en skit-se-re com-pu-ter reg-ning ud-ar-bej-de des-ud-en mu-lig-hed-en fjern-be-tje-ning ved-kom-men-de vur-de-ring ar-bejds-op-ga-ve knap bru-ger-e-va-lu-e-ring-en be-skre-vet vand-ret-te or-ga-ni-se-res hin-an-den pla-ce-ret han-dels-ud-dan-nel-se per-so-nen mang-len-de punkt-er-nes valg-mu-lig-he-der klas-sisk bru-ger-e-va-lu-e-ring op-ret-te ka-te-go-ri-en op-le-ve ge-stalt-psy-ko-lo-gi-ens af-snit knap-pen pro-blem-stil-ling-er pro-ble-mer fi-gu-ren eks-pe-ri-men-talt del-tag-el-se na-tur-vid-en-ska-be-lig med-ar-bej-de-re num-mer-plad-er par-ke-rings-an-læg kend-te sy-ste-met gen-kend-el-se bil-led-er-ne fo-to-gra-fi ma-te-ri-a-le le-ve-rer U-ni-ver-si-tet Der-i-mod mi-ni-mum num-mer-plad-er-ne lang-somt svar-en-de e-le-men-tæ-re a-na-ly-se-re bil-led-be-hand-ling pixels tids-plan}


\begin{document}

\maketitle
%Billedfilerne skal hedde "nummerpladens tekst"."filtype", eksempelvis EF12345.jpg.
\newpage
\tableofcontents
\newpage

\section{Introduktion}

\fixme{Alt skal være enten datid eller nutid}
%\fixme{noget om læringsmål}
\fixme{evt. færre fodnoter}


%\fixme{Fjerne eksempel ikke gennemføres. Give andet eksempel.}

I mange sammenhænge vil det være relevant at automatisk kunne identificere og genkende køretøjers nummerplader. Et sådant system kunne eksempelvis være et system som bruges ved et parkeringsanlæg, hvor der f.eks. tages billeder af de biler der ankommer til anlægget, så man kan registrere hvilke biler der befinder sig på anlægget. Et sådant system kunne også bruges af parkeringsvagter, som eksempelvis manuelt tager billeder af parkerede biler hvorefter systemet identificere bilen.

I dette projekt vil vi arbejde med netop dette emne: automatisk identificering og genkendelse af nummerplader i billeder ved hjælp af et computersystem. Det system, vi ønsker at udvikle og afprøve, egner sig bedst til en situation, hvor bilen står stille eller bevæger sig meget langsomt.

Dette projekt laves som bachelorprojekt på Københavns Universitet. Vores forventning er ikke at systemet kan opnå en effektivitet svarende til etablerede systemer til genkendelse af nummerplader. Derimod ønsker vi via arbejdet at få erfaring med praktisk anvendelse af elementære teknikker indenfor billedbehandling og mønstergenkendelse.

\subsection{Problemformulering}
Opgavens problemformulering blev som følger:

\fixme{Bør det være spørgsmål}
\begin{itemize}
\item[-] Hvordan kan nummerplader på farvefotografier identificeres og læses af et computersystem?
\item[-] Hvilke kendte metoder til identifikation og læsning af nummerplader findes der?
\item[-] Hvor høj genkendelsesprocent kan et system, vi selv laver, opnå?
\end{itemize}
%\fixme{Søren: opdel sidste spørgsmål i flere}

\subsection{Afgrænsning}

\fixme{Søren: ikke sigende titel}

\fixme{Mere argumentation i dette afsnit}

På grund af projektets omfang var det nødvendigt at lave visse afgrænsninger. Disse afgrænsninger blev som følger:

\fixme{Bør ikke være punktopstilling når den er så lang}
\begin{itemize}
\item[-] Billederne skulle tages ved højlys dag uden brug af kunstig lys. På denne måde vurderede vi at man ville opnå fotografier hvor nummerpladerne havde nogenlunde ens farvemønstre. Ved brug af kunstigt lys ville man risikere at nummerpladen ville fremstå med en unartulig farve eller at det reflekterende materiale, som nummerplader er lavet af, ville skabe genskin og på denne måde unaturlige farver. \fixme{forklaring på farvemønster}
\item[-] Hvert fotografi skulle forestille en bil med netop én synlig nummerplade.
\item[-] Fotografierne skulle tages fra en position direkte foran bilen og i en højde på mellem 160 og 190 cm i en afstand på 3-6 meter fra bilen uden zoom \begin{comment}Søren: med hvilken optik\end{comment}. På denne måde må det forventes at nummerpladen bliver tilpas stor i billedet til at computersystem kan finde den, samt at man med det menneskelige øje kan aflæse nummerpladen udfra fotografierne. \fixme{evt. figur af hvordan vi stod da vi fotograferede}
\item[-] Nummerpladen skulle ikke nødvendigvis befinde sig midt i billedet. \fixme{midt i billede: er den en afgrænsning?} %Dette ville skabe lidt udfordring for vores system, som dermed ikke nødvendigvis kan udelukke objekter som ikke er i midten af fotografiet.
\item[-] Nummerpladen skulle fremstå vandret i billedet og skulle desuden være fri for urenheder eller lignende, der gjorde pladen sværere at læse. \fixme{hvad vil det sige at nummerpladen er vandret}
\item[-] Nummerpladerne skal alle være danske privatplader som de udstedtes i foråret 2008\footnote{Rød kant, hvid baggrund og sort tekst.} Formatet på nummerpladerne skal være \textit{AA XX XXX}, hvor \textit{A} er bogstaver i mængden \textit{A}-\textit{Z}\footnote{Enkelte tegn er muligvis ikke lovlige. Er dette tilfældet, skal systemet ikke tage højde for dem.}, og \textit{X} er heltal i mængden 0-9. Som eksemplet viser, skal der være mellemrum mellem de to bogstaver og de to første tal samt mellem de to første tal og de tre tal. Nummerpladerne må ikke indeholde andre tegn end de nævnte bogstaver og tal.
\end{itemize}

\fixme{har vi udelukket kvadratiske nummerplader?}

\subsection{Metoder fra litteraturen}

\fixme{Hvilken litteratur har vi og vi har vi den fra? Let at finde metoder? Hvorfor har vi kun beskrevet disse metoder? Mangler henvisninger til teoriafsnit. Argumentér for hvorfor der er fire elementer. Hvorfor er rotation et element når vi har vandrette billeder.}

I forbindelse med dette projekt har vi identificeret følgende fire dele, som et system der skal kunne identificere og aflæse nummerplade kunne indeholde: identificering af nummerpladen i billedet, eventuel rotation af nummerpladen, hvis denne ikke er vandret i billedet, opdeling af tegnene i nummerpladen samt aflæsning af de enkelte tegn. De første tre elementer har vi defineret som værende billedbehandling mens det fjerde element er mønstergenkendelse. Metoderne til billedbehandlingsdelene diskuteres i afsnit \ref{sec_billed} mens metoderne til mønstergenkendelse diskuteres i afsnit \ref{sec_monster}.

Identificering af nummerpladen i billedet kan gøres ved blot at søge efter hvide områder i billedet \cite{ron}, \cite{nijhuis}, eller udfra viden om kanter i billedet \cite{shapiro} og eventuelt kombinere det med teksturanalyse \cite{parker}. Derudover kan man analysere kontrasterne i billedet \cite{kwas}, da en nummerplade må forventes af bestå af kontrastfyldte (sort/hvide) områder. Rotation af nummerpladen kan gøres ved brug af Radon transformation, som kan afgøre hvor der findes linier i billedet, samt hvilken vinkel disse linier har \cite{ron}, \cite{shapiro}. Segmentering af nummerpladens tegn kan gøres ved brug af peak-to-valley (dansk: top-til-bund) \cite{ron}, \cite{kwas} eller ved brug af såkaldte sammenhængende komponenter (hvert enkelt tegn i nummerpladen må hver især forventes at være én sammenhængende figur) \cite{nijhuis}. I alle tilfælde er det en god idé at kombinere de enkelte elementer af systemet med viden om størrelsesforhold, antal (én nummerplade pr. billede, syv tegn pr. nummerplade osv.) og lign \cite{nijhuis}, \cite{parker}, \cite{kwas}. Til aflæsning af tegn kan man bl.a. bruge neurale netværk \cite{nijhuis}, \cite{kwas} eller klassifikations vektorer \cite{arth}.
\fixme{Søren: til Radon: han skriver "nej"??}
\fixme{skal vi ha noget om testdata her?}

%\newpage

\fixme{nedenfor i billedbehandling: der mangler intro til valg af metoder}
\section{Billedbehandling}


\subsubsection*{Noter fra møde med Søren 20/2:}
identifikation: Se på en pixel, har naboer en kontrast farve?
En scan-linie: hvordan varierer kontrasten henover linien?
Adaboost - godt

\subsubsection*{Matriculas2003}
Bruger kun gråtone info. Arbejder også med nummerplader som skal være læsbar for det menneskelige øje.

Metode:
Histogram - først normaliseres billedet
Sobel filter - fremhæver ikke-homogene områder
"A simple threshold and a sub-sampling" bruges til at vælge områder der kan være nummerpladen

Husker alle områder som kan være nummerplader så de forkerte først vælges fra i genkendelses-fasen. Bruger multi-hypothesis detection (ikke forklaret yderligere i teksten).

Feature vektorer: hver pixel i et træningsbilleder er blevet klassificeret som positiv (del af nummerplade) eller negativ (ikke del af nummerplade). Minimerer efterfølgende det negative sæt.

Bruger kd-træ data struktur og en "omtrent nærmeste nabo" søgeteknik.

\subsubsection*{A Real-time vehicle License Plate Recognition (LPR) System på http://visl.technion.ac.il/projects/2003w24/}

* Find de gule (hos os: hvide) områder i billedet
* Forstør disse områder
* Find vinklen på nummerpladen ved brug af "Radon transform"
* Justering af nummerpladens konturer
* Unødvendige dele af billedet fjernes (kun nummerpladen tilbage)
* Billedet i gråtone, herefter gøres det binært
* Billedet normaliseres
* Tegn-inddeling vha. peak-to-valley

De brugte Matlab. De havde følgende relevante problemer og løsningsforslag til disse:

* Udtrækning af det gule område giver ofte fejl. Man kunne supplere denne udtrækning med en algoritme der indberegner at nummerplader har en klar signatur idet der er stærke grå-tone variationer i regulære intervaller (henover nummerpladen, mener de vel?)

* Hvis der er flere nummerplade-kandidater i billedet skal hver af dem testes.

\subsubsection*{LicenseplateSydney.pdf}

Bruger regler for nummerplader i systemet

    * Starter med kontrast “udstrækning”, bortfiltrering af støj (der tages selvfølgelig højde for billedkvaliteten i denne del)
    * Lokalisering af nummerpladen: fuzzy clustering algoritme som bruger karakteristikker som “gul-hed” og teksturer
    * Gul-hed er defineret af frekvenstabel lavet fra manuelt udklippede nummerplader
    * Tekstur: her ser man på grå-værdien af de 8 nabopixels
    * “global threshold” baseret på gennemsnitsværdien af gråtone: fås binært billede
    * Udfra regler (højde, bredde m.m) findes potintielle tegn
    * Nummerpladen gives kun videre til mønstergenkendelse hvis den indeholder det rette antal tegn

Dette system godkender 75\% af billederne. Problemer med skruer i nummerpladen. Problemer med global threshold – burde gøres på pr-tegn-basis.

\subsubsection*{licence-plate-1996.pdf}

Der bruges en algoritme der først lokaliserer elementer der kan være tegn/bogstaver hvorefter den udvælger et område som nummerplade

* Konverter til gråtone billede
* 5x5 filter, fjerne støj
* Find kanter i billedet vha. Shen-Castan kant-detektor
* Gør billedet binært og del elementer i forgrunden fra hinanden
* Algoritme der finder bogstaver på baggrund af forskellen i gråtone værdien af bogstav/baggrund
* Områder hvor der ikke er (det rette antal) bogstaver udelukkes
* For at finde nummerplade bruges genetisk algoritme der bedømmer rektangler med tegn: har de den rette størrelse? er bogstaverne korrekt placeret i rektangel? osv.
* Algoritmen vægter hvert område og filtrerer til sidst i disse områder udfra deres vægt

Stort problem: svært at finde tegn.

\subsubsection{kwasnickawawrzyniak}
Billedet skiftes til farverummet YUV fra vilket luminans er det eneste der bevares. Herefter normaliseres billedet (Hele den diskerete "range" udnyttes). Kigger på skift i kontrast. Gælder alle nummerplader. Finder alle tekster. Den rigtige skal vælges.

Identifikation af plader:

1. "Connected components analysis" (der kigger på et binært billede?) vælger områder med høj kontrast (threshold). De fundne områder undersøges og områder elimineres efter regler i pdf.  Herefter, er der lignenede grupper i nærheden af funden gruppe? Måske er der en serie tegn dvs. en sætning = plade.

2. Searching for signatures of license plates. Et karakteristisk skift i luminans i en linie i billedet.

Potentielle plader roteres så de er vandrette.

Segmentering af tegn:
Scan af peaks og valleys samt analyse af sammenhængende grupper fra identifikationsprocessen sammenlignes og segmentering foretages.

Mønstergenkendelse foregår med neuraltnetværk.

\subsubsection{AdaptiveLicensePlateImageExtraction.pdf}
For at sætte hastigheden op foretages visse operationer på kraftigt nedsamplede billeder. Finder lodrette linier. Bruger Robert's edge detector til at fremhæve dem (Tegner den på billedet?). Dette efterlader en masse lodrette linier i området med nummerpladen. Et Rank filter? bruges på billedet. Efterlader en lys elipse i det område hvor pladen findes. Scanner billedet lodret for at finde det lyseste område og klipper ud (Klipper et noget større område end pladen ud på eksemplet i pdf'en). Der er formler i beskrivelsen. Roter pladen hvis skæv (Formler i pdf). Nohet med at finde linier i billedet og rotere stlsvarende (Hough og Radon transform). Det er først når vi skal genkende tegnene på pladen vi bruger billedet i sin originale opløsning.



%\begin{comment} % remove this later

%\newpage
\subsection{Genkendelse af tegn}

INDSÆT GENKENDELSESDEL AF DIAGRAM OVER SYSTEM
\label{sec_monster}

Til genkendelse af tegn vil vi bruge to forskellige metoder: Den første bruger informationer fra en feature vektor mens den anden analyserer enkelte pixels i hvert tegn-billede. I dette afsnit vil vi beskrive disse to metoder samt en metode til at udføre syntaks analyse på den tegnfølge der repræsenterer et gæt på nummerpladen.

De to metoder til genkendelse af tegn arbejder på de syv binære billeder som separationsdelen af systemet returnerede. Metoderne returnerer syv hitlister (én for hvert billede) med tegn. Det tegn der optræder øverst på den enkelte hitliste er metodens bedste gæt på det tegn billedet forestiller, nummer to på listen er det næstbedste gæt osv.

\subsubsection{Feature vektorer}
En feature vektor er en vektor der... Idéen i denne metode er at der oprettes en feature vektor for hvert muligt tegn der kan forekomme i en nummerplade. For et billede af et tegn oprettes der tillige en vektor, som herefter sammenlignes med tegn-vektorerne for at se hvilken vektor der ligger nærmest billedets vektor. Tegnet for den vektor der ligger nærmest vælges.

MAHAP AFSTAND SKAL I FØLGE SØREN BRUGE MANGE DATA. KAN IKKE HUSKE HVORFOR

EKSEMPEL PÅ VEKTOR?

\subsubsection{Simpel metode hvor træningsættet andes}

\subsubsection{Syntaks analyse}

Ved syntaks analyse analyseres tegn-hitlisterne fra de to ovenstående metoder. Analysen sker på baggrund af at et gæt fra en af de to metoder måske kan udelukkes på baggrund af overtrædelse af syntaktiske regler for hele tegnfølgen. Der kan forekomme følgende fejl:

\begin{itemize}
\item[-] Et tal er placeret på en af de første to pladser
\item[-] Et bogstav er placeret på en af de sidste fem pladser
\item[-] Bogstavkombinationen er ikke tilladt
\item[-] Talkombinationen er ikke tilladt (den samlede værdi af tallene er enten for høj eller for lav
\end{itemize}

Hvis en eller flere af disse muligheder er gældende itererer metoden ned igennem hitlisterne indtil et lovligt valg af tegn er fundet. Dette er illustreret i følgende eksempel: Tegnfølgen \textbf{DO 45 7B3} er retuneret fra mønstergenkendelse. I denne tegnfølge forekommer der to fejl: bogstavet \textbf{O} bruges ikke på 2. position og bogstavet \textbf{B} i 6. position burde have været et tal. På 2. position itererer metoden ned igennem hitlisten til den når et bogstav som sammen med \textbf{D} danner en lovlig bostavkombination og på 6. position itereres der indtil et tal findes.

I systemet sætter vi en øvre grænse for hvor langt ned ad hitlisten syntaks analysen kan vælge tegn. Hvis denne grænse ikke fandtes kunne det blive gætværk...

\subsubsection{Metoder fra litteraturen}

\newpage


\section{Implementation}

\fixme{beskriv hvordan koden er implementeret - dens opbygning og de fede ting ved den}

\subsection{}

\section{Resultater}

HVAD SKER DER MED GULE ETC. PLADER??
\subsection{Indsamling af testdata}

%Fra start var vi meget opmærksomme på at afgrænsningen af projektet skulle være klar, så det ikke blev for omfattende. Omkring valg af billedemateriale afholdt vi os eksempelvis fra at 

%Til at udarbejde og teste systemet havde vi brug for nogle fotografier af biler med nummerplader. Ved fotograferingen var vi opmærksomme på følgende afgrænsninger:


%\begin{figure}[h]
%\begin{center}
%\includegraphics[width=10cm]{illu/B_XC33139.jpg}
%\label{b_xc33139}
%\caption{Fotografieksempel}
%\end{center}
%\end{figure}

% vi holder dem adskilt - ikke noget med histo
Da vi bl.a. havde planer om udarbejdelse af en histogrambaseret metode til identificering af nummerplader (se afsnit \ref{histo}), holdt vi de to fotografisæt adskilte. På denne måde ville det f.eks. være muligt for os at teste om skift fra et kamera til et andet, vil give ændrede resultater. Billederne blev navngivet i vores database, så det, i deres filnavn, indgik om billedet forestillede en bil set forfra eller bagfra samt hvilken nummerplade bilen på nummerpladen havde. Derudover udarbejdede vi et mindre program som hjalp os til at identificere nummerpladens fire hjørnekoordinater og indskrive disse i filnavnet. Denne sidste tilføjelse ville hjælpe os i testfasen, til at undersøge om de nummerpladekandidater vores system ville udvælge er de korrekte.
 
Kontrolsæt: 600 billeder


%\subsection{Fotografering}
%Hvor mange og hvilke billeder har vi taget.
%At billederne er taget i naturligt lys for at undgå kunstige farver og genskin fra pladen som er malet med reflekterende materiale.
%Opdeling af billeder i træning- og testsæt
%til f.eks. histogram metode herunder adskillelse af fotos fra de to forskellige kameraer.

%%% LOKALISERING AF NUMMERPLADE %%%

\subsection{Lokalisering af nummerplade}
HVOR GODE ER VI PÅ TRÆNINGSSÆT?
HVOR GODE ER VI PÅ TESTSÆT?

\subsubsection{Områder domineret af lyse gråtoner}
HVOR GODE ER VI PÅ TRÆNINGSSÆT?
HVOR GODE ER VI PÅ TESTSÆT?

\begin{tabular}{|l|l|l|l|l|}
\hline
\multicolumn{5}{|c|}{DetectMain} \\ \hline
Param 1 & Param 2 & Skalering & Overordnet resultat & Sande positiver\\ \hline
1 & 1 & 1 & 19 \% & 19 \%\\ \hline
2 & 2 & 2 & 19 \% & 21 \% \\ \hline
3 & 3 & 3 & 19 \% & 22 \% \\
\hline
\end{tabular}

\subsubsection{Områder med høj kontrast}
HVOR GODE ER VI PÅ TRÆNINGSSÆT?
HVOR GODE ER VI PÅ TESTSÆT?

\begin{tabular}{|l|l|l|l|l|}
\hline
\multicolumn{5}{|c|}{DetectMain} \\ \hline
Param 1 & Param 2 & Skalering & Overordnet resultat & Sande positiver\\ \hline
1 & 1 & 1 & 19 \% & 19 \%\\ \hline
2 & 2 & 2 & 19 \% & 21 \% \\ \hline
3 & 3 & 3 & 19 \% & 22 \% \\
\hline
\end{tabular}

\subsubsection{Frekvensanalyse}
HVOR GODE ER VI PÅ TRÆNINGSSÆT?
HVOR GODE ER VI PÅ TESTSÆT?

\begin{tabular}{|l|l|l|l|l|}
\hline
\multicolumn{5}{|c|}{DetectMain} \\ \hline
Param 1 & Param 2 & Skalering & Overordnet resultat & Sande positiver\\ \hline
1 & 1 & 1 & 19 \% & 19 \%\\ \hline
2 & 2 & 2 & 19 \% & 21 \% \\ \hline
3 & 3 & 3 & 19 \% & 22 \% \\
\hline
\end{tabular}

\subsubsection{Maksimer lokal kontrast}
HVOR GODE ER VI PÅ TRÆNINGSSÆT?
HVOR GODE ER VI PÅ TESTSÆT?

\begin{tabular}{|l|l|l|l|l|}
\hline
\multicolumn{5}{|c|}{DetectMain} \\ \hline
Param 1 & Param 2 & Skalering & Overordnet resultat & Sande positiver\\ \hline
1 & 1 & 1 & 19 \% & 19 \%\\ \hline
2 & 2 & 2 & 19 \% & 21 \% \\ \hline
3 & 3 & 3 & 19 \% & 22 \% \\
\hline
\end{tabular}

\subsubsection{Kvantifisering}
HVOR GODE ER VI PÅ TRÆNINGSSÆT?
HVOR GODE ER VI PÅ TESTSÆT?

\begin{tabular}{|l|l|l|l|l|}
\hline
\multicolumn{5}{|c|}{DetectMain} \\ \hline
Param 1 & Param 2 & Skalering & Overordnet resultat & Sande positiver\\ \hline
1 & 1 & 1 & 19 \% & 19 \%\\ \hline
2 & 2 & 2 & 19 \% & 21 \% \\ \hline
3 & 3 & 3 & 19 \% & 22 \% \\
\hline
\end{tabular}

\subsubsection{endeligt valg af nummerpladekandidat}

Min antal enige 1:
HVOR GODE ER VI PÅ TRÆNINGSSÆT?
HVOR GODE ER VI PÅ TESTSÆT?


Min antal enige 2:
HVOR GODE ER VI PÅ TRÆNINGSSÆT?
HVOR GODE ER VI PÅ TESTSÆT?

Min antal enige 3:
HVOR GODE ER VI PÅ TRÆNINGSSÆT?
HVOR GODE ER VI PÅ TESTSÆT?

Min antal enige 5:
HVOR GODE ER VI PÅ TRÆNINGSSÆT?
HVOR GODE ER VI PÅ TESTSÆT?

Min antal enige 5:
HVOR GODE ER VI PÅ TRÆNINGSSÆT?
HVOR GODE ER VI PÅ TESTSÆT?

\begin{tabular}{|l|l|l|l|l|}
\hline
\multicolumn{5}{|c|}{DetectMain} \\ \hline
Param 1 & Param 2 & Skalering & Overordnet resultat & Sande positiver\\ \hline
1 & 1 & 1 & 19 \% & 19 \%\\ \hline
2 & 2 & 2 & 19 \% & 21 \% \\ \hline
3 & 3 & 3 & 19 \% & 22 \% \\
\hline
\end{tabular}


VI SKAL SE HVORDAN DETECTMAIN OPFØRER SIG NÅR F.EKS. ALLE METODER SKAL VÆRE ENIGE.
 Vi har skaleret ned for at spare tid.

\subsubsection*{Observeret sæt på 407 billeder}
Scale: 0.25
DetectMain: 96.6/99.24
DetectQuant: 67.8/75.4
DetectSameness: 56.8/95.5
DetectContrastAvg: 62.7/85.0
DetectPlateness: 50.4/65.5
DetectCStretch: 84.0/92.7

Scale: 0.50 (Ekstremt langsomt)
DetectPlateness: 29.7/56.5
DetectCStretch:

Der skal testet på hver metode som fritstående. Herefter på DetectMain der samler metoderne. Hvordan ændrer resultaterne sig når man ændrer opløsning af billederne?
Hvor gode er vi på det set vi har observeret? Hvor gode er vi på et set vi ikke har observeret. Er der et mønster i de billeder hvot vi ikke finder nummerpladen? Hvile udfordringer møder vi: Mørke plader... 

Confusion matrix: De elementer der ligger udenfor diagonalen er elementer der ikke er nummerplader.




Delkonklussion:

%%% SEPARATION AF TEGN %%%

\subsection{Separation af tegn}
I dette afsnit testes de to metoder til separation af tegn. Metoden til rotation vil ikke blive testes, blot kommenteret.

\subsubsection*{Rotation}
Om en nummerplade står vandret i et billede kan afgøres ved at kantdetektere billedet og udføre en Radon transformation på dette kant-billede. Hvis de stærkeste linier optræder ved $0^{\circ}$ er rotation udført korrekt.

Vi har ikke testet det fordi...

VI KUNNE MÅSKE GODT TESTE DET?

\subsubsection*{Separation}
De to metoder til separation af tegn testes ved at se på om de finder syv objekter som optræder indenfor pladens koordinater. Denne forholdsvis enkle test mener vi er fyldestgørende, da funktionerne hvori metoderne er implementeret, bør være "skrappe" nok.

Vi vil desuden se på om der er forskel på succesraten i forhold til hvilket område der repræsenterer nummerpladen: Her findes to muligheder: pladen er fundet med vores egne metoder eller defineret ved hjælp af pladekoordinaterne og et eller andet lagt til.
%Er det forskel på succesraten i forhold til om vi ser på pladerne vi manuelt har plottet (med evt. et tilfældigt yderområde lagt til) eller om vi ser på de nummerplader som vores indetifikationsmetoder finder?

Parametre: størrelse på componenter der sorteres fra, skalering: kan vi tillade os at skalere ned (eller op, hvor vi måske får mere plads mellem komponenterne) eller mister vi for meget information VI KAN IKKE SKALERE

VI VIL KUN SE PÅ TEST AF DETTE NIVEAU, IKKE? 

%OVERVEJ NEDENSTÅENDE SOM SAMLET GRAF I ET AFSNIT

%\subsubsection{Bjerg/Dal}
%HVOR GODE ER VI TIL AT FINDE 7 OMRÅDER INDEN FOR PLADEN I TRÆNINGSSÆT?
%HVOR GODE ER VI TIL AT FINDE 7 OMRÅDER INDEN FOR PLADEN I TESTSÆT
%HVOR GODE ER VI TIL AT FINDE 7 OMRÅDER INDEN FOR PLADEN I TRÆNINGSSÆT?
%HVOR GODE ER VI TIL AT FINDE 7 OMRÅDER INDEN FOR PLADEN I TESTSÆT
%HVAD SKER DER NÅR PLADEOMRÅDET UDVIDES MED 10PIXELS PÅ ALLE SIDER?
%HVAD SKER DER NÅR PLADEOMRÅDET UDVIDES MED 20PIXELS PÅ ALLE SIDER?
%HVAD SKER DER NÅR PLADEOMRÅDET UDVIDES MED 30PIXELS PÅ ALLE SIDER?


\begin{tabular}{|l|c|c|}\hline
\multicolumn{3}{|l|}{Træningssæt} \\\hline
Metode & Sammenhængende komponenter & Bjerg/dal \\\hline
Egne metoder & 96\% & 3\% \\\hline
10 & 96\% & 3\% \\\hline
20 & 96\% & 3\% \\\hline
30 & 96\% & 3\% \\\hline \end{tabular}

\begin{tabular}{|l|c|c|}\hline
\multicolumn{3}{|l|}{Kontrolsæt} \\\hline
Metode & Sammenhængende komponenter & Bjerg/dal \\\hline
Egne metoder & 96\% & 3\% \\\hline
10 & 96\% & 3\% \\\hline
20 & 96\% & 3\% \\\hline
30 & 96\% & 3\% \\\hline \end{tabular}

Delkonklusion:
Sammenhængende komponenter er go og bjerg/dal dårlig?

%%% GENKENDELSE AF TEGN %%%

\subsection{Genkendelse af tegn}
I dette afsnit testes de to metoder til genkendelse af tegn i en nummerplade. Derudover vil syntaksanalysen blive testet.

Noget om at det er i forhold til alle plader, dvs. at procenterne nedenfor er afhængige af hvor gode vi er til separere tegn.

\subsubsection{Egenskabsvektor}

%HVOR MANGE PLADER ER KORREKT LÆST I TRÆNINGSSÆT?
%HVOR MANGE PLADER ER KORREKT LÆST I TETSSÆT?
%HVOR MANGE PLADER ER KORREKT LÆST PÅ 6 POSITIONER I TRÆNINGSSÆT?
%HVOR MANGE PLADER ER KORREKT LÆST PÅ 6 POSITIONER I TETSSÆT?
%HVOR MANGE PLADER ER KORREKT LÆST PÅ 5 POSITIONER I TRÆNINGSSÆT?
%HVOR MANGE PLADER ER KORREKT LÆST PÅ 5 POSITIONER I TETSSÆT?

Følgende tabeller viser resultaterne for at læse en hel plade, seks tegn i pladen osv. (med forskellige vektorlængder).

\begin{tabular}{|l|c|c|c|c|c|c|}\hline
\multicolumn{7}{|l|}{Træningssæt} \\\hline
Vektorlængde & Hele pladen læst & 6 tegn læst & 5 tegn & 4 tegn & 3 tegn & 2 tegn \\\hline
9 & 0\% & 0\% & 0\% & 0\% & 0\% & 0\% \\\hline
16 & 0\% & 0\% & 0\% & 0\% & 0\% & 0\% \\\hline
25 & 0\% & 0\% & 0\% & 0\% & 0\% & 0\% \\\hline \end{tabular}

\begin{tabular}{|l|c|c|c|c|c|c|}\hline
\multicolumn{7}{|l|}{Kontrolsæt} \\\hline
Vektorlængde & Hele pladen læst & 6 tegn læst & 5 tegn & 4 tegn & 3 tegn & 2 tegn \\\hline
9 & 0\% & 0\% & 0\% & 0\% & 0\% & 0\% \\\hline
16 & 0\% & 0\% & 0\% & 0\% & 0\% & 0\% \\\hline
25 & 0\% & 0\% & 0\% & 0\% & 0\% & 0\% \\\hline \end{tabular}

%HVOR GODE ER VI PÅ TAL I TRÆNINGSSÆT?
%HVOR GODE ER VI PÅ TAL I TESTSÆT?
%HVOR GODE ER VI PÅ BOGSTAVER I TRÆNINGSSÆT?
%HVOR GODE ER VI PÅ BOGSTAVER I TESTSÆT?
%HVAD ER GENKENDELSESPROCENTEN PÅ TEGN PR PLADE I TRÆNINGSÆT?
%HVAD ER GENKENDELSESPROCENTEN PÅ TEGN PR PLADE I TESTSÆT?

Følgende tabeller viser hvor godt systemet er til at læse bogstaver hhv. tal. hhv. alle tegn

\begin{tabular}{|l|c|c|c|}\hline
\multicolumn{4}{|l|}{Træningssæt} \\\hline
Vektorlængde & Alle tegn & Bogstaver & Tal \\\hline
9 & 0\% & 0\% & 0\% \\\hline
16 & 0\% & 0\% & 0\%\\\hline
25 & 0\% & 0\% & 0\%\\\hline \end{tabular}

\begin{tabular}{|l|c|c|c|}\hline
\multicolumn{4}{|l|}{Kontrolsæt} \\\hline
Vektorlængde & Alle tegn & Bogstaver & Tal \\\hline
9 & 0\% & 0\% & 0\% \\\hline
16 & 0\% & 0\% & 0\% \\\hline
25 & 0\% & 0\% & 0\% \\\hline \end{tabular}

%HVILKEN POSITION PÅ HITLISTEN TAGER DEN I GENNEMSNIT PR POSITION - TRÆNING?
%HVILKEN POSITION PÅ HITLISTEN TAGER DEN I GENNEMSNIT PR POSITION - TEST?
%MAXHITNO: HVAD GIVER DET JO LÆNGERE NED AD HITLISTEN VI KAN GÅ?

Syntaks analyse: Hvilke hits bliver valgt på hitlisterne af syntaksanalysen (sæt maxhitno højt):

\begin{tabular}{|l|c|}\hline
\multicolumn{2}{|l|}{Træningssæt} \\\hline
Hitnr. & Valgt \\\hline
1 & 95,4\% \\\hline
2 & 3\% \\\hline
3 & 0\% \\\hline
4 & 0\% \\\hline
5 & 0\% \\\hline
6 & 0\% \\\hline \end{tabular}

\begin{tabular}{|l|c|}\hline
\multicolumn{2}{|l|}{Kontrolsæt} \\\hline
Hitnr. & Valgt \\\hline
1 & 92,9\% \\\hline
2 & 0\% \\\hline
3 & 0\% \\\hline
4 & 0\% \\\hline
5 & 0\% \\\hline
6 & 0\% \\\hline \end{tabular}

UDEN SYNTAKSANALYSE - TRÆNING?
UDEN SYNTAKSANALYSE - TEST?

Evt. en tabel over de enkelte tegn A, B, C osv. Hvor gode er vi til at genkende disse? Tabellen kunne laves udfra den bedste vektorstørrelse? Måske bare en tabel over de dårligste tegn, altså dem der er sværest at genkende? På denne måde bliver det ikke en LANG tabel

\begin{tabular}{|l|c|}\hline
\multicolumn{2}{|l|}{Træningsæt} \\\hline
Tegn & Valgt \\\hline
0 & 92,9\% \\\hline
1 & 0\% \\\hline
2 & 0\% \\\hline
3 & 0\% \\\hline
4 & 0\% \\\hline
5 & 0\% \\\hline
6 & 92,9\% \\\hline
7 & 0\% \\\hline
8 & 0\% \\\hline
9 & 0\% \\\hline
A & 0\% \\\hline
B & 0\% \\\hline
C & 92,9\% \\\hline
D & 0\% \\\hline
E & 0\% \\\hline
H & 0\% \\\hline
J & 0\% \\\hline
K & 0\% \\\hline 
L & 92,9\% \\\hline
M & 0\% \\\hline
N & 0\% \\\hline
O & 0\% \\\hline
P & 0\% \\\hline
R & 0\% \\\hline
S & 0\% \\\hline
T & 0\% \\\hline
U & 0\% \\\hline
V & 0\% \\\hline
X & 0\% \\\hline
Y & 0\% \\\hline
Z & 0\% \\\hline \end{tabular}



\subsubsection{Summerede billeder}
Samme tabeller som ovenfor?


%Søren: Ved klassifikation bruges en eller anden grænseværdi som bestemmer hvad et element skal klassificeres som: denne grænseværdi kan reguleres og kan derfor testes.

Delkonklusion:
Billederne af tegnene er ret store i forhold til dem der bruges i litteraturen. Dette giver vores system en væsentlig fordel i forhold til "de andre".



%\newpage
%\section{Evaluering}

% God bedømmelse af projektet fås hvis man kan:
%1. Opstille en arbejdsplan og afslutte en undersøgelse af problemet indenfor den tid der er til rådighed.
%2. Kunne kombinere relevant datalogisk og eventuelt anden viden i en beskrivelse af styrker og svagheder i tidligere forsøg på løsning af problemet.
%3. Kunne begrunde valget af en eller flere eksplicit beskrevne metoder, og anvende dem til løsning af problemet, eller til afprøvning af en mulig løsning.
%4. Kunne kombinere relevant datalogisk eller anden viden fra flere områder eller eventuelt viden fra et område og empiriske resultater, så de bidrager til løsning af problemet.
%5. Kunne give en sammenhængende, præcis og ikke-triviel beskrivelse af og begrundelse for væsentlige dele af den konkrete løsning, eller af de generelle muligheder for at løse problemet.
%6. Kunne vurdere på hvilke områder det er lykkedes at løse problemet, og på hvilke områder det ikke er lykkedes, og kunne påpege eventuelle svagheder i løsningen.
%7. Kunne reflektere over sin egen arbejdsproces og når det er relevant, komme med forslag til forbedring af den.
%8. Kunne skrive en sproglig og strukturel velskrevet rapport med relevante illustrationer og referencer som følger en etableret standard.

\begin{comment}

Skal vi have arbejdsopgaverne med i rapporten og så bedømmes de?

Blev spørgsmålene besvaret:

Hvordan kan nummerplader på farvefotografier identificeres og læses af et computersystem? Hvilke kendte metoder findes der, og hvor høj genkendelsesprocent kan et system vi selv laver opnå?

\end{comment}

%\subsection{Performance}
%\subsection{Problemer}
%\subsection{Resultater i forhold til læst litteratur}
\section{Konklusion}

%\end{comment} % remove this later

\newpage % new page before litterature
%\documentclass[10pt,a4paper,final]{report}
\usepackage[utf8]{inputenc}
\usepackage{ucs}
\usepackage[danish]{babel}
\usepackage{amsmath}
\usepackage{amsfonts}
\usepackage{amssymb}
\usepackage{verbatim}
\usepackage{listings} % Better source code listings
\usepackage{graphicx}
%\author{Tobias Balle-Petersen og Esben Paul Bugge}
\title{Litteraturliste}

\parindent=0pt
\parskip=8pt
 \usepackage[top=3cm, bottom=3cm, left=4cm, right=4cm]{geometry} 

\lstset{language=python}
\lstset{inputencoding=latin1}
\lstset{extendedchars=true}
\lstset{breaklines=true}
\lstset{commentstyle=\textit}
\lstset{showstringspaces=false}
\lstset{numbers=left, numberstyle=\tiny, stepnumber=2, numbersep=5pt,tabsize=3,basicstyle=\small}
%\lstset{numbers=left, numberstyle=\tiny, stepnumber=2, numbersep=5pt,stringstyle=\ttfamily, showstringspaces=false, basicstyle=\small, language={python}}


\begin{document}
\maketitle
%Hentet 11. februar 2008:
%http://www.it.lth.se/users/lambert/leftovers/LicenseplateSydney.pdf
%http://www.ci.pwr.wroc.pl/~kwasnick/download/kwasnickawawrzyniak.pdf
%http://www.icg.tu-graz.ac.at/pub/pdf/arth_-_real-time_license_plate_recognition_ecw07.pdf

%Andet:
%http://visl.technion.ac.il/projects/2003w24/ (11. februar 2008)
%www.retsinformation.dk

%Hentet 17. februar 2008:
%http://plutarco.disca.upv.es/~jcperez/Documentos/Matriculas2003.pdf
%http://ecet.ecs.ru.acad.bg/cst04/Docs/sIIIA/32.pdf (kaldet AdaptiveLicensePlateImageExtraction.pdf)
%http://pages.cpsc.ucalgary.ca/~federl/Publications/LicencePlate1996/licence-plate-1996.pdf
\end{document}
%\section{Litteraturliste}
\section{Litteratur}

\begin{thebibliography}{99}
%\bibliographystyle{plain}


%%%%% BELOW: DONE %%%%%

%http://visl.technion.ac.il/projects/2003w24/ (11. februar 2008)
\bibitem{ron} Ron, B.-H.: \textit{A Real-time vehicle License Plate Recognition (LPR) System}.

%Hentet 11. februar 2008:
\bibitem{nijhuis} Nijhuis, J. A. G., ter Brugge, M. H., Helmholt, K. A., Pluim, J. P. W., Spaanenburg, L., Venema, R. S., Westenberg, M. A.: \textit{Car License Plate Recognition with Neural Networks and Fuzzy Logic}.
%http://www.it.lth.se/users/lambert/leftovers/LicenseplateSydney.pdf

% 17. april
\bibitem{shapiro} Shapiro, V., Dimov, D., Bonchev, S., Velichkov, V., Gluhchev, G.: \textit{Adaptive License Plate Image Extraction}
% http://ecet.ecs.ru.acad.bg/cst04/Docs/sIIIA/32.pdf

%Hentet 17. februar 2008:
\bibitem{parker} Parker, J. R., Federl, P.: \textit{An Approach To Licence Plate Recognition}.
%http://pages.cpsc.ucalgary.ca/~federl/Publications/LicencePlate1996/licence-plate-1996.pdf

%Hentet 11. februar 2008:
\bibitem{kwas} Kwa\'snicka, H., Wawrzyniak, B.: \textit{License plate localization and recognition in camera pictures}.
%http://www.ci.pwr.wroc.pl/~kwasnick/download/kwasnickawawrzyniak.pdf

\bibitem{arth} Arth, C., Limberger, F., Bischof, H.: \textit{Real-Time License Plate Recognition on an Embedded DSP-Platform}.
%http://www.icg.tu-graz.ac.at/pub/pdf/arth_-_real-time_license_plate_recognition_ecw07.pdf

%Hentet 17. februar 2008:
\bibitem{cano} Cano, J., Pérez-Cortés, J.-C.: \textit{Vehicle License Plate Segmentation In Natural Images}.
%http://plutarco.disca.upv.es/~jcperez/Documentos/Matriculas2003.pdf



%%%%% NOT DONE %%%%%


%http://nrpl.dk/ - Nummerplader

%http://www.mathworks.com: matlab info, radon

\bibitem{murphy} Murphy-Chutorian, E., Trivedi, M.: \textit{N-Tree Disjoint-Set Forests for Maximally Stable Extremal Regions}.
\bibitem{vedaldi} Vedaldi, A.: \textit{An Implementation of Multi-Dimensional Maximally Stable Extremal Regions}.

\bibitem{matas} Matas, J., Chum, O., Urban, M., Pajdla, T.: \textit{Robust Wide Baseline Stereo from Maximally Stable Extremal Regions}.


\end{thebibliography}


\end{document}