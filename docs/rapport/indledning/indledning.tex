\section{Indledning}

\subsubsection*{TODO}
Overvej at skrive noget om hvordan man udregner gradienter.
Sig noget om at pladerne ofte er forsænkede når de sidder bag på bilerne.

Sproget skal være konsekvent vi gør/der foretages etc.

Sproget: datid/nutid

Ret headers i endelig version.

Sørg for at vi altid kaldet det LOKALISERING af nummerplader. p.t skifter vi mellem lokalisering og identifikation. OGSÅ I TITLEN!


I afsnit om metoder til lokalisering er det ikke konsekvent hvor det siges at metoden kommer fra litteraturen.

Nævn i system/implementation hvorfor vi har fravalgt nogle metoder som vi faktisk har implementeret

Indsæt figur som tidligere manglede i svn (implemenation),

\subsubsection*{Formalia}

TITEL: "GENKENDELSE" I STEDET FOR "LÆSNING"?
KAN MAN BRUGE DANSK TEGNOPDELING I TEX?

Denne rapport beskriver et system til automatisk identifikation og læsning af danske nummerplader. Systemet er udviklet som bachelorprojekt på Københavns Universitet i 2008.

Ved arbejdet med denne opgave har vi ønsket at få praktisk erfaring med grundlæggende metoder og teknikker indenfor billedbehandling og mønstergenkendelse som anvendt i et system til nummerpladegenkendelse. Findes der etablerede metoder? Kan vi selv udvikle teknikker der fungerer tilfredsstillende? I hvor høj grad vil et system vi selv bygger kunne identificere nummerplader og læse deres indhold?
OVENSTÅENDE ER LÆRINGSMÅL. ER DET FORMALIA?

Systemet er udviklet i Matlab. Vi har ikke haft nogen ambitioner om at kunne læse nummerplader i realtid, og har ikke foretaget os noget væsentligt for at optimere hastigheden på systemet.

HVEM ER LÆSEREN?

\subsubsection*{Baggrund}
I mange sammenhænge er det interessant, at have adgang til et system der automatisk kan genkende et køretøj ved at læse dets nummerplade. Et af de mest kendte områder hvor et sådant system er relevant, er registrering af køretøjer der overskrider hastighedsbegrænsninger. Et andet område er roadpricing. Et eksempel er det system der varetager at der skal betales en afgift når man kører ind i Stockholms indre by. Ved de veje der fører ind til midtbyen er der placeret kameraer der registrerer de kørertøjer der kører forbi og trækker (ER DET MED KAMERA??).

Skriv noget om england og folks bemymringer.
%On March 11, 2008, the Federal Constitutional Court of Germany ruled that the laws permitting the use of automated number plate recognition systems in Germany violated the right to privacy.

\subsubsection*{Rapportens opbygning}
Vi begynder i det følgende afsnit med at beskrive den type data vi arbejder med. Vi beskriver hvordan vi har delt materialet i forskellige sæt samt hvilke afgrænsninger vi har foretaget. Derefter følger afsnittet \textit{Vores system} hvor vi beskriver det system vi har udviklet. Vi forsøger at holde os fra tekniske detaljer og fokuserer på den mere intuitive forståelse af systemets virkemåde. I afsnittet \textit{Implementation} beskriver vi vores kildekodes virkemåde og uddyber emner som vi kun har forsøgt at give en intuitiv forståelse af i afsnittet \textit{Vores system}. Efter at have beskrevet systemet, præsenterer vi resultaterne af afprøvning af systemet i afsnittet \textit{Resultater}.

SKRIV OM APPENDIKS.

\section{Hvilken type inddata arbejder vi med?}
\label{sec:data}
Et system der skal genkende nummerplader har brug for at kunne observere køretøjer. Dette sker via stillbilleder eller videooptagelser. I denne opgave har vi valgt at arbejde med stilbilleder.

\subsection{Afgrænsninger}
For at begrænse opgavens omfang, har vi valgt en række afgrænsninger for de billeder vi ønsker at arbejde med. I det følgende beskriver vi de valg vi har truffet og baggrunden for dem.

\paragraph{Bilerne på billederne er ikke i bevægelse:}
For at forsøge at få så skarpe billeder som muligt, har vi valgt kun at tage billeder af parkerede biler. Vi vil dog stadig kunne få uskarpe billeder i situationer hvor kameraet ikke har stillet skarpt på køretøjet. Vi har taget alle billeder med autofokus, og har således ikke selv foretaget nogle valg i den forbindelse.

\paragraph{Alle billeder indeholder netop en nummerplade:}
Før at gøre opgaven begrænset, har vi valgt kun at kigge på billeder der indeholder en enkelt nummerplade. Hvis vi ved at et billede med garanti indeholder én nummerplade, kan vi kigge efter det område der ligner en nummerplade mest og være ret sikre på at dette område virkelig er en nummerplade. Vi behøver altså ikke vurdere, om det område der ligner mest ligner så meget at vi mener det er en nummerplade.

\paragraph{Amindelige danske nummerplader:}
For at kunne nøjes med at kigge efter en type nummerplader, har vi valgt kun at arbejde med de almindelige hvide nummerplader med en linie sort tekst på hvid baggrund og rød kant som vist i figur \vref{fig:typisk_nummerplade}. Endvidere arbejder vi ikke med de såkaldte personlige nummerplader hvor man selv kan bestemme teksten på nummerpladen. På de numerplader vi arbejder med, består teksten altså af syv tegn hvoraf de to første er bogstaver og de fem sidste er cifre mellem.

\begin{figure}[htp]
\centering
\framebox{\includegraphics[width=5cm]{illu/plate.jpg}} 
\caption{En illustration af den type nummerplader vi ønsker at genkende.}
\label{fig:typisk_nummerplade}
\end{figure}

\paragraph{Vandrette nummerplader:}
Vores billeder er taget så nummerpladerne er roteret i begrænset omfang. Denne begrænsning gør, at vi kan antage at de områder der indeholder nummerplader har et forhold mellem bredde og højde der ligger forholdsvis tæt på danske nummerplades officielle bredde/højde-forhold.

\paragraph{Ingen perspektivisk forvrængning:}
Vi har valgt at arbejde med billeder hvor den perspektivisk forvrængning af nummerpladerne er minimal. Uden væsentlig skævvridning af nummerpladerne, vil pladernes tegn stort set være lige høje og brede og dermed nemmere at læse. Vi har derfor bestræbt os på at tage vore billeder fra en position umiddelbart foran eller bagved den bil vi fotograferer. Denne afgrænsning gør, at vi ikke arbejder med opretning af perspektiviskt forvrængning\footnote{Affin transformation} i denne opgave. 
%Endvidere vil tegn stå på en vandret linie

\paragraph{Ensartede størrelser af nummerplader:}
For at kunne antage noget om størrelserne på de nummerplader vi ønsker at kunne leder efter i billederne, har vi taget alle billeder fra en afstand på mellem to og fire meter uden brug af zoomfunktion. Med denne afgrænsning, sikrer vi os også at tegnene på nummerpladerne aldrig bliver så små at vi må opgive at genkende dem.

\paragraph{Højtopløste billeder:}
Vores billeder er alle taget i en opløsning på $1024 \times 768$ pixels. Gode originalbilleder giver gode muligheder for at eksperimentere med billeder i forskellige opløsninger. 

\paragraph{Intet kunstigt lys:}
Vi har valgt at tage alle billeder uden brug af kunstigt lys som f.eks. blitz. Det er vores ønske at arbejde med billeder taget under forskellige lysforhold så nummerpladerne kan ligge helt eller delvist i skygge.  

Figur \vref{fig:typisk_billede} viser et eksempel på den type billeder vi arbejder med.

\begin{figure}[htp]
\centering
\framebox{\includegraphics[width=7cm]{illu/B_XC33139.jpg}} 
\caption{Et eksempel på den type billeder vi arbejder med. Nummerpladen er stort set vandret og den perspektivistiske forvrængning er minimal.}
\label{fig:typisk_billede}
\end{figure}

\subsection{To sæt billeder}
Vi har delt vores billeder op i to sæt. Det ene sæt, bestående af 400 billeder, kalder vi for vores \textit{træningsæt}. Det er billeder i dette sæt vi har analyseret under arbejdet med at udvikle vores software. F.eks. har vi på baggrund af dette sæt observeret nummerpladernes størrelser så vi har kunnet vurdere nummerpladestørrelser i den type billeder vi arbejder med.

Vores andet sæt, \textit{kontrolsættet}, består af 600 billeder. Dette sæt har vi brugt til at foretage en form for kontrol af vores færdige system. Det er vores tanke at denne kontrol vil kunne sige noget om i hvor høj grad vi har formået at lave et system der fungerer på ukendte billeder der overholder vores  tidligere beskrevne afgrænsninger. 

For at kunne automatisere afprøvning af vores system, har vi navngivet vores billedfiler så vi kan aflæse nummerpladens koordinater samt nummerpladens tekst. 

TAG 50 BILLEDER AF HVIDE PLADER M. BILTZ
TAG 50 BILLEDER AF GULE PLADER U. BILTZ


%Billederne blev navngivet i vores database, så det, i deres filnavn, indgik om billedet forestillede en bil set forfra eller bagfra samt hvilken nummerplade bilen på nummerpladen havde. Derudover udarbejdede vi et mindre program som hjalp os til at identificere nummerpladens fire hjørnekoordinater og indskrive disse i filnavnet. Denne sidste tilføjelse ville hjælpe os i testfasen, til at undersøge om de nummerpladekandidater vores system ville udvælge er de korrekte.

\begin{comment}
I mange sammenhænge vil det være relevant at automatisk kunne identificere og genkende køretøjers nummerplader. Et sådant system kunne eksempelvis være et system som bruges ved et parkeringsanlæg, hvor der f.eks. tages billeder af de biler der ankommer til anlægget, så man kan registrere hvilke biler der befinder sig på anlægget. Et sådant system kunne også bruges af parkeringsvagter, som eksempelvis manuelt tager billeder af parkerede biler hvorefter systemet identificere bilen.

I dette projekt vil vi arbejde med netop dette emne: automatisk identificering og genkendelse af nummerplader i billeder ved hjælp af et computersystem. Det system, vi ønsker at udvikle og afprøve, egner sig bedst til en situation, hvor bilen står stille eller bevæger sig meget langsomt.

Dette projekt laves som bachelorprojekt på Københavns Universitet. Vores forventning er ikke at systemet kan opnå en effektivitet svarende til etablerede systemer til genkendelse af nummerplader. Derimod ønsker vi via arbejdet at få erfaring med praktisk anvendelse af elementære teknikker indenfor billedbehandling og mønstergenkendelse.


\subsection{Problemformulering}
Opgavens problemformulering blev som følger:

\fixme{Skal det være spørgsmål}
\begin{itemize}
\item[-] Hvordan kan nummerplader på farvefotografier identificeres og læses af et computersystem?
\item[-] Hvilke kendte metoder til identifikation og læsning af nummerplader findes der?
\item[-] Hvor høj genkendelsesprocent kan et system, vi selv laver, opnå?
\end{itemize}
%\fixme{Søren: opdel sidste spørgsmål i flere}

\subsection{Afgrænsning}

\fixme{Søren: ikke sigende titel}
\fixme{Mere argumentation i dette afsnit}
\fixme{VI behøver ikke at lave afin transformation.}


På grund af projektets omfang var det nødvendigt at lave visse afgrænsninger. Disse afgrænsninger blev som følger:

\fixme{Bør ikke være punktopstilling når den er så lang}
\begin{itemize}
\item[-] Billederne skulle tages ved højlys dag uden brug af kunstig lys. På denne måde vurderede vi at man ville opnå fotografier hvor nummerpladerne havde nogenlunde ens farvemønstre. Ved brug af kunstigt lys ville man risikere at nummerpladen ville fremstå med en unartulig farve eller at det reflekterende materiale, som nummerplader er lavet af, ville skabe genskin og på denne måde unaturlige farver. \fixme{forklaring på farvemønster}
\item[-] Hvert fotografi skulle forestille en bil med netop én synlig nummerplade.
\item[-] Fotografierne skulle tages fra en position direkte foran bilen og i en højde på mellem 160 og 190 cm i en afstand på 3-6 meter fra bilen uden zoom. På denne måde må det forventes at nummerpladen bliver tilpas stor i billedet til at computersystem kan finde den, samt at man med det menneskelige øje kan aflæse nummerpladen udfra fotografierne. \fixme{evt. figur af hvordan vi stod da vi fotograferede}
\item[-] Nummerpladen skulle ikke nødvendigvis befinde sig midt i billedet. \fixme{midt i billede: er den en afgrænsning?} %Dette ville skabe lidt udfordring for vores system, som dermed ikke nødvendigvis kan udelukke objekter som ikke er i midten af fotografiet.
\item[-] Nummerpladen skulle fremstå vandret i billedet og skulle desuden være fri for urenheder eller lignende, der gjorde pladen sværere at læse. \fixme{hvad vil det sige at nummerpladen er vandret}
\item[-] Nummerpladerne skal alle være danske privatplader som de udstedtes i foråret 2008\footnote{Rød kant, hvid baggrund og sort tekst.} Formatet på nummerpladerne skal være \textit{AA XX XXX}, hvor \textit{A} er bogstaver i mængden \textit{A}-\textit{Z}\footnote{Enkelte tegn er muligvis ikke lovlige. Er dette tilfældet, skal systemet ikke tage højde for dem.}, og \textit{X} er heltal i mængden 0-9. Som eksemplet viser, skal der være mellemrum mellem de to bogstaver og de to første tal samt mellem de to første tal og de tre tal. Nummerpladerne må ikke indeholde andre tegn end de nævnte bogstaver og tal.
\end{itemize}

\fixme{har vi udelukket kvadratiske nummerplader?}





\subsection{Metoder fra litteraturen}

\fixme{Hvilken litteratur har vi og vi har vi den fra? Let at finde metoder? Hvorfor har vi kun beskrevet disse metoder? Mangler henvisninger til teoriafsnit. Argumentér for hvorfor der er fire elementer. Hvorfor er rotation et element når vi har vandrette billeder.}

I forbindelse med dette projekt har vi identificeret følgende fire dele, som et system der skal kunne identificere og aflæse nummerplade kunne indeholde: identificering af nummerpladen i billedet, eventuel rotation af nummerpladen, hvis denne ikke er vandret i billedet, opdeling af tegnene i nummerpladen samt aflæsning af de enkelte tegn. De første tre elementer har vi defineret som værende billedbehandling mens det fjerde element er mønstergenkendelse. Metoderne til billedbehandlingsdelene diskuteres i afsnit \ref{sec_billed} mens metoderne til mønstergenkendelse diskuteres i afsnit \ref{sec_monster}.

\end{comment}


