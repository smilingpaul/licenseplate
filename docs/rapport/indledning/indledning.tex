\section{Indledning}

\subsubsection*{TODO}

Støt begrundelse af div. valg i koden med testresultater. Er det rimeligt at klippe de største område ud når vi finder nummerplader? 

Overvej at skrive noget om hvordan man udregner gradienter.
Sig noget om at pladerne ofte er forsænkede når de sidder bag på bilerne.

Sproget skal være konsekvent vi gør/der foretages etc.

Sproget: datid/nutid

Ret headers i endelig version.

Sørg for at vi altid kaldet det LOKALISERING af nummerplader. p.t skifter vi mellem lokalisering og identifikation. OGSÅ I TITLEN!

I system/sep: Hvis at du rent faktisk får klippet de små broer over i toppen af nummerpladen i dit eksempel.

I afsnit om metoder til lokalisering er det ikke konsekvent hvor det siges at metoden kommer fra litteraturen.

Nævn i system/implementation hvorfor vi har fravalgt nogle metoder som vi faktisk har implementeret

Metoderne i system-diagrammerne skal nogle steder navngives (hedder nu eksempelvis Genkendelse af tegn 1.

Slå conjunction op.

KAN MAN BRUGE DANSK TEGNOPDELING I TEX?

\subsubsection*{Formalia}

Denne rapport beskriver et system til automatisk lokalisering og læsning af danske nummerplader. Systemet og rapporten er udviklet som bachelorprojekt på Københavns Universitet i 2008.

Systemet er udviklet i \textit{Matlab}. Vi har ikke haft nogen ambitioner om at kunne læse nummerplader i realtid, og har ikke foretaget os noget væsentligt for at optimere hastigheden på systemet.

HVEM ER LÆSEREN?

HER SKAL VEL STÅ AT TOBIAS HAR ARBEJDET MEST MED LOKALISERING, ESBEN MEST MED SEP. OG GENKEND.

\subsubsection*{Læringsmål}
Ved arbejdet med denne opgave har vi ønsket at få praktisk erfaring med grundlæggende metoder og teknikker indenfor billedbehandling og mønstergenkendelse som anvendt i et system til nummerpladegenkendelse. Findes der etablerede metoder? Kan vi selv udvikle teknikker der fungerer tilfredsstillende? I hvor høj grad vil et system vi selv bygger kunne lokalisere nummerplader og læse deres indhold?

\subsubsection*{Rapportens opbygning}
Vi begynder i det følgende afsnit rapporten med en kort introduktion til opgavens emne. I det næste afsnit beskriver vi den type data vi arbejder med. Vi beskriver hvordan vi har delt materialet i forskellige sæt samt hvilke afgrænsninger vi har foretaget. Derefter følger afsnittet \textit{Vores system} hvor vi beskriver det system vi har udviklet. Vi forsøger at holde os fra detaljer og fokuserer på den mere intuitive forståelse af systemets virkemåde. I afsnittet \textit{Implementation} beskriver vi mere detaljeret systemets virkemåde og uddyber emner som vi kun har forsøgt at give en intuitiv forståelse af i afsnittet \textit{Vores system}. Efter at have beskrevet systemet, præsenterer vi resultaterne af afprøvning af systemet i afsnittet \textit{Resultater}.

SKRIV OM APPENDIKS.


\section{Hvilken type billeder arbejder vi med?}
\label{sec:data}
Et system der skal genkende nummerplader har brug for at kunne observere køretøjer. Dette sker via stillbilleder eller videooptagelser. I denne opgave har vi valgt at arbejde med stilbilleder.

\subsection{Afgrænsninger}
For at begrænse opgavens omfang, har vi valgt en række afgrænsninger for de billeder vi ønsker at arbejde med. I det følgende beskriver vi de valg vi har truffet og baggrunden for dem.

\paragraph{Bilerne på billederne er ikke i bevægelse:}
For at forsøge at få så skarpe billeder som muligt, har vi valgt kun at tage billeder af parkerede biler. Vi vil dog stadig kunne få uskarpe billeder i situationer hvor kameraet ikke har stillet skarpt på køretøjet. Vi har taget alle billeder med autofokus, og har således ikke selv foretaget nogle valg i den forbindelse.

\paragraph{Alle billeder indeholder netop en nummerplade:}
Før at gøre opgaven begrænset, har vi valgt kun at kigge på billeder der indeholder en enkelt nummerplade. Hvis vi ved at et billede med garanti indeholder én nummerplade, kan vi kigge efter det område der ligner en nummerplade mest og være ret sikre på at dette område virkelig er en nummerplade. Vi behøver altså ikke vurdere, om det område der ligner mest ligner så meget at vi mener det er en nummerplade.

\paragraph{Amindelige danske nummerplader:}
For at kunne nøjes med at kigge efter en type nummerplader, har vi valgt kun at arbejde med de almindelige hvide nummerplader med en linie sort tekst på hvid baggrund og rød kant som vist i figur \vref{fig:typisk_nummerplade}. Endvidere arbejder vi ikke med de såkaldte personlige nummerplader hvor man selv kan bestemme teksten på nummerpladen. På de numerplader vi arbejder med, består teksten altså af syv tegn hvoraf de to første er bogstaver og de fem sidste er cifre mellem.

\begin{figure}[htp]
\centering
\framebox{\includegraphics[width=5cm]{illu/plate.jpg}} 
\caption{En illustration af den type nummerplader vi ønsker at genkende.}
\label{fig:typisk_nummerplade}
\end{figure}

\paragraph{Vandrette nummerplader:}
Vores billeder er taget så nummerpladerne er roteret i begrænset omfang. Denne begrænsning gør, at vi kan antage at de områder der indeholder nummerplader har et forhold mellem bredde og højde der ligger forholdsvis tæt på danske nummerplades officielle bredde/højde-forhold.

\paragraph{Ingen perspektivisk forvrængning:}
Vi har valgt at arbejde med billeder hvor den perspektivisk forvrængning af nummerpladerne er minimal. Uden væsentlig skævvridning af nummerpladerne, vil pladernes tegn stort set være lige høje og brede og dermed nemmere at læse. Vi har derfor bestræbt os på at tage vore billeder fra en position umiddelbart foran eller bagved den bil vi fotograferer. Denne afgrænsning gør, at vi ikke arbejder med opretning af perspektiviskt forvrængning\footnote{Affin transformation} i denne opgave. 
%Endvidere vil tegn stå på en vandret linie

\paragraph{Ensartede størrelser af nummerplader:}
For at kunne antage noget om størrelserne på de nummerplader vi ønsker at kunne leder efter i billederne, har vi taget alle billeder fra en afstand på mellem to og fire meter uden brug af zoomfunktion. Med denne afgrænsning, sikrer vi os også at tegnene på nummerpladerne aldrig bliver så små at vi må opgive at genkende dem.

\paragraph{Højtopløste billeder:}
Vores billeder er alle taget i en opløsning på $1024 \times 768$ pixels. Gode originalbilleder giver gode muligheder for at eksperimentere med billeder i forskellige opløsninger. 

\paragraph{Intet kunstigt lys:}
Vi har valgt at tage alle billeder uden brug af kunstigt lys som f.eks. blitz. Det er vores ønske at arbejde med billeder taget under forskellige lysforhold så nummerpladerne kan ligge helt eller delvist i skygge.  

Figur \vref{fig:typisk_billede} viser et eksempel på den type billeder vi arbejder med.

\begin{figure}[htp]
\centering
\framebox{\includegraphics[width=7cm]{illu/B_XC33139.jpg}} 
\caption{Et eksempel på den type billeder vi arbejder med. Nummerpladen er stort set vandret og den perspektivistiske forvrængning er minimal.}
\label{fig:typisk_billede}
\end{figure}

\subsection{Opdeling af billeder}
Vi har delt vores billeder op i flere sæt. Vi har to primære sæt og to sekundære set. Det første primære sæt, der består af 400 billeder, kalder vi for vores \textit{træningsæt}. Det er billeder i dette sæt vi har analyseret under arbejdet med at udvikle vores software. F.eks. har vi på baggrund af dette sæt observeret størrelserne på nummerpladerne så vi har kunnet vurdere nummerpladestørrelser i den type billeder vi arbejder med.

Vores andet primære sæt, \textit{kontrolsættet}, består af 600 billeder. Dette sæt har vi brugt til at foretage en form for kontrol af vores færdige system. Det er vores tanke at denne kontrol vil kunne sige noget om i hvor høj grad vi har formået at lave et system der fungerer på ukendte billeder der overholder vores tidligere beskrevne afgrænsninger. 

De to sekundære sæt består af billeder der ikke overholder vores afgrænsninger, men dog kun på et enkelt område. I det første sæt, der består af 130 billeder, er alle billeder taget med blitz. Det andet sekundære sæt indeholder billeder af biler med gule numerplader. Sætter der består af XX billeder overholder alle øvrige afgrænsninger. Vi kalder disse to sæt for henholdsvis \textit{blitzsættet} og det \textit{gule sæt}.

TAG 50 BILLEDER AF GULE PLADER U. BLITZ

Vi har valgt at medtage de to sekundære set for at få en indikation af hvordan vores system opfører sig når det arbejder på data der ikke er taget højde for. Vil den gule farve gøre at vi slet ikke kan finde nummerpladerne? Vil blitzlyset gøre det lettere at finde nummerpladerne? Vil bogstaverne på de blitzoplyste nummerplader kunne genkendes af vores system?

For at kunne automatisere afprøvning af vores system, har vi navngivet vores billedfiler så vi kan aflæse nummerpladens koordinater samt nummerpladens tekst. Vi har angiver koordinaterne så hele nummerpladen er indeholdt i det specificerede område.