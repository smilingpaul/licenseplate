\section{Billedbehandling}

\begin{comment}
\subsubsection*{Noter fra møde med Søren 20/2:}
identifikation: Se på en pixel, har naboer en kontrast farve?
En scan-linie: hvordan varierer kontrasten henover linien?
Adaboost - godt

\subsubsection*{Matriculas2003}
Bruger kun gråtone info. Arbejder også med nummerplader som skal være læsbar for det menneskelige øje.

Metode:
Histogram - først normaliseres billedet
Sobel filter - fremhæver ikke-homogene områder
"A simple threshold and a sub-sampling" bruges til at vælge områder der kan være nummerpladen

Husker alle områder som kan være nummerplader så de forkerte først vælges fra i genkendelses-fasen. Bruger multi-hypothesis detection (ikke forklaret yderligere i teksten).

Feature vektorer: hver pixel i et træningsbilleder er blevet klassificeret som positiv (del af nummerplade) eller negativ (ikke del af nummerplade). Minimerer efterfølgende det negative sæt.

Bruger kd-træ data struktur og en "omtrent nærmeste nabo" søgeteknik.

\subsubsection*{A Real-time vehicle License Plate Recognition (LPR) System på http://visl.technion.ac.il/projects/2003w24/}

* Find de gule (hos os: hvide) områder i billedet
* Forstør disse områder
* Find vinklen på nummerpladen ved brug af "Radon transform"
* Justering af nummerpladens konturer
* Unødvendige dele af billedet fjernes (kun nummerpladen tilbage)
* Billedet i gråtone, herefter gøres det binært
* Billedet normaliseres
* Tegn-inddeling vha. peak-to-valley

De brugte Matlab. De havde følgende relevante problemer og løsningsforslag til disse:

* Udtrækning af det gule område giver ofte fejl. Man kunne supplere denne udtrækning med en algoritme der indberegner at nummerplader har en klar signatur idet der er stærke grå-tone variationer i regulære intervaller (henover nummerpladen, mener de vel?)

* Hvis der er flere nummerplade-kandidater i billedet skal hver af dem testes.

\subsubsection*{LicenseplateSydney.pdf}

Bruger regler for nummerplader i systemet

    * Starter med kontrast “udstrækning”, bortfiltrering af støj (der tages selvfølgelig højde for billedkvaliteten i denne del)
    * Lokalisering af nummerpladen: fuzzy clustering algoritme som bruger karakteristikker som “gul-hed” og teksturer
    * Gul-hed er defineret af frekvenstabel lavet fra manuelt udklippede nummerplader
    * Tekstur: her ser man på grå-værdien af de 8 nabopixels
    * “global threshold” baseret på gennemsnitsværdien af gråtone: fås binært billede
    * Udfra regler (højde, bredde m.m) findes potintielle tegn
    * Nummerpladen gives kun videre til mønstergenkendelse hvis den indeholder det rette antal tegn

Dette system godkender 75\% af billederne. Problemer med skruer i nummerpladen. Problemer med global threshold – burde gøres på pr-tegn-basis.

\subsubsection*{licence-plate-1996.pdf}

Der bruges en algoritme der først lokaliserer elementer der kan være tegn/bogstaver hvorefter den udvælger et område som nummerplade

* Konverter til gråtone billede
* 5x5 filter, fjerne støj
* Find kanter i billedet vha. Shen-Castan kant-detektor
* Gør billedet binært og del elementer i forgrunden fra hinanden
* Algoritme der finder bogstaver på baggrund af forskellen i gråtone værdien af bogstav/baggrund
* Områder hvor der ikke er (det rette antal) bogstaver udelukkes
* For at finde nummerplade bruges genetisk algoritme der bedømmer rektangler med tegn: har de den rette størrelse? er bogstaverne korrekt placeret i rektangel? osv.
* Algoritmen vægter hvert område og filtrerer til sidst i disse områder udfra deres vægt

Stort problem: svært at finde tegn.

\subsubsection{kwasnickawawrzyniak}
Billedet skiftes til farverummet YUV fra vilket luminans er det eneste der bevares. Herefter normaliseres billedet (Hele den diskerete "range" udnyttes). Kigger på skift i kontrast. Gælder alle nummerplader. Finder alle tekster. Den rigtige skal vælges.

Identifikation af plader:

1. "Connected components analysis" (der kigger på et binært billede?) vælger områder med høj kontrast (threshold). De fundne områder undersøges og områder elimineres efter regler i pdf.  Herefter, er der lignenede grupper i nærheden af funden gruppe? Måske er der en serie tegn dvs. en sætning = plade.

2. Searching for signatures of license plates. Et karakteristisk skift i luminans i en linie i billedet.

Potentielle plader roteres så de er vandrette.

Segmentering af tegn:
Scan af peaks og valleys samt analyse af sammenhængende grupper fra identifikationsprocessen sammenlignes og segmentering foretages.

Mønstergenkendelse foregår med neuraltnetværk.

\subsubsection{AdaptiveLicensePlateImageExtraction.pdf}
For at sætte hastigheden op foretages visse operationer på kraftigt nedsamplede billeder. Finder lodrette linier. Bruger Robert's edge detector til at fremhæve dem (Tegner den på billedet?). Dette efterlader en masse lodrette linier i området med nummerpladen. Et Rank filter? bruges på billedet. Efterlader en lys elipse i det område hvor pladen findes. Scanner billedet lodret for at finde det lyseste område og klipper ud (Klipper et noget større område end pladen ud på eksemplet i pdf'en). Der er formler i beskrivelsen. Roter pladen hvis skæv (Formler i pdf). Nohet med at finde linier i billedet og rotere stlsvarende (Hough og Radon transform). Det er først når vi skal genkende tegnene på pladen vi bruger billedet i sin originale opløsning.
\end{comment}

\subsection{Nummerplade identificering}

\subsubsection{Tobias' brainstorm}
Kan man kigge på højde bredde på components?
Scan linie. Man kan både kigge på components og "signatur" som beskrevet andetsteds(pdf).
Med gradienter kan man finde f.eks. lodrette men ikke vandrette streger. Der er mange lodrette i pladen.
Man kan kigge på en f.eks. 8x8 og se hvor mange komponenter der er tilstede. Der er mange komponenter i et 
lille område i nummerpladen(eller hvad?).

Kør en scanlinie. Noter kraftige gradienter. Hvis de er "tæt" på hinanden er det godt. giv "point"

Det her dropper jeg  indtil videre:
Kør en vertikal scanlinie. Hvis vi møder en gradient begynder vi at "tegne" en streg hvis intensitet 
falder. Den tegner altså en savtak når den møder en gradient. Hvis der er flere høje gradienter i træk får 
vi så en kasse med en skrå afslutning. Er det ikke en nummerplade? 


Nu gør jeg:
Find gradienter i billedet. Lav et binært billede hvor de steder hvor gradienten er større end (0.5 * den 
maximale gradient) er markeret med hvidt.

Der er nu mange markeringer i nummerpladeområdet.

Jeg tænker. Samme princip som med kun at vise maksimale gradienter:
Kig på summen (mængden af hvidt) af alle vandrette linier i billedet. "slet" de linier hvor der er mindre 
end (0.5 * summen af den linie med mest hvidt). Jeg burde nu have fjernet støj men stadig have pladen. Jeg 
statser på at pladen er på de linier med mest hvidt.

\subsubsection{Vægtet metode}

Regler:

- Kant efterfulgt af rød stribe (ovenover eller nedenunder)
- Pixel del af sammenhængende kæde som er længere end et vist antal pixels
- En linie hvorpå der er en anden retvinklet linie, og linierne har et vist forhold
- Længde af linie

\subsubsection{Histogram}

Beskrevet i LicensePlateSydney.pdf

Frekvenstabel, hvordan opbygges den?
Til historgram metoden skal der bruges en frekvenstabel indeholdende farvefrekvenserne for pixels i en række billeder af nummerplader. Denne tabel udformes vha. funktionen make\_freq\_table. I funktionen itereres der gennem alle pixels i et billede og deres RGB værdier noteres i en 255 x 255 x 255 matrice. Eksempelvis

Når frekvenstabellen er udformet, normaliseres den så den RGB værdi der har den højeste frekvens får værdien 1, mens de resterende RGB værdier får værdien frekvens/højeste frekvens. Hvis RGB værdien 240, 240, 230 eksempelvis har den højeste frekvens, 55 og RGB værdien 230, 230, 220 har frekvensen 35, får førstnævnte værdien 1 og sidstnævnte værdien 35/55 = 0,64.

Hvordan bruges den?



\subsubsection{AdaBoost?}



\subsection{Rotation}

Nogle overvejelser om hvordan billedet skal være klippet til for at rotationen fungerer ordentligt...

I dette afsnit beskrives det hvordan et billede af en nummerplade kan roteres, så nummerpladen optræder vandret i billedet.

%\subsubsection{Hough transformation}

%Hough transformation

\subsubsection{Radon transformation}

En Radon transformation beregner projektionen af et billede fra flere forskellige steder og i flere forskellige retninger/vinkler. Denne type transformation kan bruges til at finde linier i billedet, og til at undersøge i hvilken retning disse linier går. Disse oplysninger er meget brugbare mht. rotation af et billede af en nummerplade, så denne plade står vandret i billedet.

Billedet nedenfor viser hvordan Radon transformationen findes i en enkelt vinkel, theta. Den grå firkant er billedet der projekteres. De grå pile er sensorer/radial linier. Resultatet af en Radon transformation er en matrice der angiver i hvor høj grad det er sandsynligt at der findes en linie i original billedet for hver vinkel samt radial linie.

%SKAL NOK BYTTES UD MED HJEMMELAVET BILLEDE:
%\begin{figure}[h]
%\includegraphics{billedbehandling/illu/transform74.jpg}
%\caption{Radon transformation}
%\end{figure}

Betragt billedet i Figur \ref{xc33139} af en nummerplade. Nummerpladen i billedet er let fordrejet i forhold til vandret og skal drejes før segmenteringen af tegn i nummerpladen kan foregå.

BILLEDE AF NUMMERPLADE, LET FORDREJET

%\begin{figure}[h]
%\includegraphics{billedbehandling/illu/P_XC33139.jpg}
%\label{xc33139}
%\caption{En nummerplade}
%\end{figure}

Når billedet i Figur \ref{xc33139}

Før Radon transformationen kan foretages skal billedet kant-detekteres (ANDEN OVERSÆTTELSE), da dette giver en større chance for at Radon transformationen opfanger linierne i billedet. Dette gøres ved hjælp af MatLab funktionen edge, som returnerer et binært billede hvor de fundne kanter har værdien 1 og resten af punkterne i billedet har værdien 0. Dette er illustreret i ...

BILLEDE AF EDGE-BILLEDE AF TIDLIGERE VISTE PLADE

%\begin{figure}[h]
%\includegraphics[width=15cm]{billedbehandling/illu/R_XC33139.jpg}
%\label{r_xc33139}
%\caption{En nummerplade}
%\end{figure}

Når nummerpladens hældning i billedet er fundet, skal billedet roteres. Dette gøres ved brug af imrotate. 

BILLEDE AF NUMMERPLADE, ROTERET.

Ulemper:

\subsection{Segmentering af bogstaver}

\subsubsection{Test af segmentering}

Se nrpl.dk: %Hvert af de op til 7 tegn på nummerpladen har et imaginært "felt" de kan brede sig i. Ikke alle tegn er lige brede, og generelt er bogstaver bredere end tal. "Feltet" til bogstaver er derfor bredere end feltet til tegn. Enkelte tegn er for smalle til at udfylde deres "felt", så designeren har fundet det hensigtsmæssigt at placere disse tegn visuelt centreret inden for deres "felt".

"Således vil en nummerplade som MB 20 001 på grund af venstrestillingen af det sidste 1-tal have større mellemrum mellem højre kant og sidste tal end mellem venstre kant og første bogstav"

\subsubsection{Sammenhængende komponenter}

Denne metode er bygget op omkring sammenhængende komponenter. Ideén bygger på at de 7 tegn som findes i billedet af nummerpladen alle er forbunde komponenter og derfor vil fremstå efter en forbundne komponenter-analyse. Andre dele af billedet vil højst sandsynligt også være ..

%Metodens procedure er som følger: Først gøres billedet af nummerpladen binært ved brug af metoden im2bw. Det resulterende binære billede har værdien 1 (hvid) bl.a. i de pixels som forestiller pladens hvide baggrund, mens pixels i pladens tegn har værdien 0 (sort).

%EKSEMPEL PÅ BINÆRT BILLEDE AF NUMMERPLADE

For at tegnene i nummerpladen skal udskille sig klart fra den hvide baggrund (pladen kan være beskidt, lyset på billedet kan være dårligt etc.) er det nødvendigt at filtrere billedet af pladen. Først omdannes billedet til gråtone hvorefter der udføres henholdsvis en top-hat filtrering og en bottom-hat filtrering MERE OM HVAD DET ER. For at forstærke billedets kontraster lægges det top-hat filtrerede billede til original billedet hvorefter det bottom-hat filtrerede billede trækkes fra.

EKSEMPEL PÅ BILLEDE MED FORSTÆRKEDE KONTRASTER.

Herefter findes billedets sammenhængende komponenter ved brug af metoden bwLabel. Denne metode finder forbudne komponenter med værdien 1 i et binært billede, hvorfor metoden i denne situation bruges på det negerede binære billede.

EKSEMPEL PÅ OMVENDT BINÆRT BILLEDE

Som det ses på billedet i Figur ? er nummerpladens tegn ikke altid de eneste sammenhængende komponenter i billedet. Derfor er det nødvendigt at sortere flere af de sammenhængende komponenter fra udfra følgende overvejelser:

\begin{itemize}
\item[-] Hvert tegn fylder maksimalt 1/7 del af billedet.
\item[-] Hvert tegn har en maksimal bredde på 1/7 af billedets bredde.
\item[-] Tegnenes højde er større end deres bredde
\item[-] Tegnenes minimale højde skal være end en vis konstant (her: 5 pixels)
\item[-] Tegnenes minimale størrelse skal være større end en vis konstant (her: 5 pixels)
\end{itemize}

EKSEMPEL PÅ BILLEDE HVOR IKKE-TEGN KOMPONENTER ER SORTERET FRA

Til sidst udklippes de syv tegn og returneres. Eksempel på en fuldendt og succesfuld segmentering er vist i Figur ?.

EKSEMPEL PÅ UDKLIPPEDE TEGN

Som beskrevet i kwasnickawawrzyniak kunne man eventuelt fjerne kanter før man starter analysen?

\subsubsection{Skan-linie}
Bruges ikke da det er for mange felter som bliver valgt. Kan måske gøres bedre ved filtrering før???

Først gøres billedet sort-hvis med im2bw. Her kan grænseværdi bestemmes med greythresh. Virker måske bedre at sætte grænseværdien lavt, så meget af billedet bliver hvidt.

Billedet skal skæres foroven og forneden. Dette gøres simpelt ved at finde den største pixl-sum i toppen af billedet og den største sum i bunden. Det antages så at disse max-summer er dele af nummerpladerne hvor teksten ikke er startet.

Step igennem vertikale linier: hvad sker før tegn, i et tegn, i slutningen af et tegn og efter et tegn.

\subsubsection{peak-to-valley}

Beskrevet i kwasnickawawrzyniak og bør måske afprøves? Måske i samarbejde med første metode?

