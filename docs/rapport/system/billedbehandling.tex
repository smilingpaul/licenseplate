\label{sec_billed}

\subsection{Lokalisering af nummerplader}
!Lav om så eksempelbilledet kun vises en gang, og man forstår at det er dette billede alle eksempler bruger so inddata.

SKRIV AT VI IKKE BEHØVER AFFIN TRANSFORMATION PÅ GRUND AF VORES AFGRÆNSNINGER.
%I dette afsnit beskrives de tre første elementer af vores system: identificering af nummerpladen i billedet, rotation af nummerpladen samt segmentering af tegn i nummerpladen.


%\subsection{Identifikation af nummerplader}
I arbejdet med at lokalisere nummerplader, har vi valgt at arbejde med flere forskellige metoder. Det er vores tanke, at vi ved at bruge metoderne sammen kan opnå et bedre resultat end ved at bruge metoderne hver for sig. F.eks. vil et område som udpeges som værende en nummerplade af flere metoder have en høj grad af troværdighed mens det i en situation hvor metoderne alle udpeger forskellige områder vil være meget usikkert hvor nummerpladen befinder sig. I dette afsnit beskriver vi de forskellige metoder vi benytter os af, samt hvordan disse fungerer som et samlet system til lokalisering af af nummerplader. 

%\subsubsection*{Interesseområder}
Metoderne i dette afsnit, fungerer alle på den måde at de returnerer et binært\footnote{Et billede der kun består af farverne hvid og sort} billede hvor områder der opfattes som interessante, det vil sige potentielle nummerplader, er markeret med farven hvid. I den bedst tænkelige, og meget urealistske, situation er nummerpladeområdet markeret med hvidt mens resten af billedet er sort som vist på figur \vref{fig:binary_ideal}. I den realistiske situation er der mange interesseområder markeret i det binære billede som det kan ses på figur \vref{fig:binary_real}. Vi kalder disse områder for kandidater, og beskriver hvordan vi vælger mellem dem i det senere afsnit \textbf{Analyse af kandidater} der begynder på side \pageref{sec:kandidater}.

Nogle metoder er baseret på materiale vi har læst, mens andre er baseret på ideer der er opstået på baggrund af vores egen anlyse af nummerpladebillederne. 

SKAL DER REFERES TIL BILLEDET AF BILEN?

FJERNET FRAMEBOX - SKAL IND IGEN

\begin{figure}[htp]
\centering
\framebox{\includegraphics[width=5cm]{illu/B_XC33139.jpg}}
\caption{Et billede hvor vi ønsker at lokalisere nummerpladen}
\label{fig:input_billede}
\end{figure}

\begin{figure}[htp]
\centering
\includegraphics[width=5cm]{system/illu/binary_ideal.png} 
\caption{Det bedst tænkelige billede af nummerpladekandidater i billedet på figur \ref{fig:input_billede}}
\label{fig:binary_ideal}
\end{figure}

SKAL DER VÆRE ET BILLEDE DER VISER SITUATIONEN FØR DER SORTERES FRA?

\begin{figure}[htp]
\centering
\includegraphics[width=5cm]{system/illu/binary_real.png} 
\caption{Et mere realistisk eksempel på nummerpladekandidater for billedet på figur \ref{fig:input_billede}}
\label{fig:binary_real}
\end{figure}



\subsubsection{Metode: Områder domineret af lyse gråtoner}
\label{sec:DetectSameness}
I denne metode forsøger vi at finde nummerpladen i et billede ved at lede efter områder der domineres af lyse gråtoner i et farvebillede. Vi håber at en af de markerede interesseområder vil være nummerpladen da dens baggrund består af lyse gråtoner. I et farvebillede vil det sige, at området vil være domineret af pixels hvor de tre farvekanaler, rød, grøn og blå har intensitetsværdier der ligger forholdsvis tæt på hinanden. F.eks. vil en helt rød pixel have værdien 256 i den røde farvekanal mens værdierne i den grønne og blå farvekanal være 0. I dette tilfælde er der altså tale om en meget skæv fordeling af intensistestvrædierne på de tre farvekanaler. Et eksempel på en jævn fordeling af intensitetsværdierne er en middelgrå pixel der vil have værdien 128 i alle tre farvekanaler. Denne metode er inspireret af metoder beskrevet af \cite{ron} og \cite{nijhuis}, der dog begge forsøger at lokalisere nummerplader med en mere karakteristisk, gul, baggrund.

For at markere interesseområder udregner vi en gennemsnitsværdi, $M$ for hver pixel i farvebilledet ved at summere værdierne for hver af de tre farvekanaler og dividere med tre. Hvis alle tre kanaler har en værdi der ligger tæt \footnote{Et parameter i programmet} på gennemsnitsværdien og denne samtidig er høj, det vil sige lys, \footnote{Et parameter i programmet} farves den indeværende pixel, i det binære billede der viser interesseområderne, hvid. Figur \ref{fig:DetectSameness-binary} viser et eksempel på interesemetoder markeret med denne metode. 
BESKRIV DILATE ERODE
\begin{comment}
Hvis $X$ er en nedre grænseværdi for de intensiteter vi opfatter som lyse. $Y$ er en værdi der bestemmer hvor stor afstand der må være mellem intensiteterne i de tre farvekanaler og $I_{ij}$ er pixels i et farvebillede hvis nummerplade vi ønsker at finde, er pixels i det binære billede $B_{ij}$ defineret som:
\begin{equation}
B_{ij} = 
\begin{Bmatrix}
1 & \text{Hvis } (R(I_{ij})+G(I_{ij})+B(I_{ij})/3) > X\\
 & R(I_{ij}) > mean - Y & < Y + mean\\
0 & \delta
\end{Bmatrix}
\end{equation}

HVORDAN SKRIVES DETTE I LATEX?


\end{comment}


For hver pixel $x$ i et farvebillede, hvor vi ønsker at finde nummerpladen, udregnes værdien for samme pixel $x_{b}$ i det binære billede således:

\begin{equation}
x_{b} = 
\begin{Bmatrix}
1 & \text{Hvis } M > L\\
 & M-D > R(x) < M+D\\
  & M-D > G(x) < M+D\\
   & M-D > B(x) < M+D\\
0 & \text{ellers..}
\end{Bmatrix}
\end{equation}


hvor $M = ((R(x)+G(x)+B(x))/3)$ og $R(x)$, $G(x)$ og $B(x)$ er intensitetsværdierne for henholdsvis den røde, den grønne og den blå farvekanal i pixelen $x$, mens $L$ er den nedre grænse for middelværdien og $D$ er den maksimale afstand fra værdien i de tre farvekanaler til $M$.




\begin{figure}[htbp]
  \centering
  \begin{minipage}[b]{5 cm}
    \framebox{\includegraphics[width=4cm]{illu/B_XC33139.jpg}}
  \end{minipage}
  \begin{minipage}[b]{5 cm}
    \framebox{\includegraphics[width=4cm]{system/illu/DetectSameness-binary.png}}  
  \end{minipage}
  \caption{Det binære billede til højre viser viser de interesseområder metoden markerer når inddate er billedet til venstre}
  \label{fig:DetectSameness-binary}
\end{figure}



\subsubsection{Metode: Områder med høj kontrast}
\label{sec:DetectContrastAvg}
OVERVEJ PARKERS MEDIAN FILTER
Vi forventer at et område med en nummerplade vil indeholde områder med høj kontrast på grund af de mange skarpe kanter mellem de mørke tegn og den lyse baggrund. Ved at lave billedet om til gråtoner og derefter opfatte intensiteterne i billedet som højder i et landskab kan vi beskrive hældninger i landskabet med såkaldte gradienter. Gradienter beskriver hældningen i en pixel som en todimensionel vektor. Et område med en blød overgang i intensitet fra sort til hvid, har korte gradienter der alle peger den samme vej for alle pixels i området, mens et område med en brat overgang fra sort til hvid (høj kontrast) vil resultere i lange gradienter der peger mod det sorte område. Figur \vref{fig:gradienter} viser to simple eksempler på intensitetslandskaber og de tilhørende gradienter.

\begin{figure}[htp]
\centering
\includegraphics[width=10cm]{system/illu/gradienter.png} 
\caption{Illustration af konceptet gradienter. Pilene viser hældninger i intensitetslandskabet. Da der er tale om glidende overgange fra hvid til sort, har pilene alle samme længde. Illustrationen er fra \cite{wiki_gradienter}.}.
\label{fig:gradienter}
\end{figure}

Efter at have udregnet billedets gradienter som vist på figur \vref{fig:DetectContrastAvg-grads}, kan vi udregne deres vinkler. Vi vælger at lave et nyt billede der viser gradienter med en vinkel på mellem 0 og 30 mellem sig selv og en vandret linie. Et eksempel på et sådant billede er vist på figur \vref{fig:DetectContrastAvg-hgrads}. HREF HER??? Grunden til at vi kun tager disse "liggende" gradienter med, er at vi på den måde undgår at markere evt. vandrette kanter i billedet der måtte forbinde nummerpladen med bilens lygter eller andet. Vores håb er, at der er nok lodrette kanter i nummerpladeområdet til at det stadig bliver markeret på billedet der viser de vandrette gradienter. F.eks. ville en plade med to forekomster af bogstavet I og seks forekomster af tallet 1 være meget synlig på vores gradientbillede medens en nummerplade hvor begge bogstaver er et O og alle seks cifre er tallet 0 være mindre synlig på grund af det lavere antal lodrette kanter og deraf følgende lave antal vandrette gradienter.

% Input image & gradients
\begin{figure}[htbp]
  \centering
  \begin{minipage}[b]{5 cm}
    \framebox{\includegraphics[width=4cm]{illu/example_car_gray.jpg}}
  \end{minipage}
  \begin{minipage}[b]{5 cm}
    \framebox{\includegraphics[width=4cm]{system/illu/DetectContrastAvg-grads.png}}  
  \end{minipage}
  \caption{Billedet til højre viser gradienterne i billedet til venstre. Området med nummerpladen er tydeligt markeret på grund af de store kontraster i området.}
  \label{fig:DetectContrastAvg-grads}
  \end{figure}

% Horizontal gradients
\begin{figure}[htp]
  \centering
  \framebox{\includegraphics[width=10cm]{system/illu/DetectContrastAvg-hgrads.png}}
  \caption{Gradienterne fra figur \vref{fig:DetectContrastAvg-grads} med vinkler mellem 0 og 30 grader. Bemærk at de vandrette linier der er synlige over og under nummerpladen på figur \ref{fig:DetectContrastAvg-grads} ikke er synlige.}
  \label{fig:DetectContrastAvg-hgrads}  
\end{figure}

HUSK AT DER SKAL STÅ AT VI BOOSTER DE LÆNGSTE GRADIENTER


Før vi laver det binære billede med nummerpladekandidater, forsøger vi gøre nummerpladeområdet til en sammenhængende figur ved at udtvære\footnote{Eng.: Blur} billedet med et filter der giver hver pixel en intensitet svarende til middelværdien for dens nærområde. Da vi primært er interesserede i at forbinde pixels der ligger ved siden af hinanden (tegnene i nummerpladen) og ønsker at undgå at nummerpladeområdet bliver forbundet med områder der ligger tæt på dens over- og underkant, definerer vi nærområdet som et liggende rektangel. Resultatet bliver et sammenhængende område med en højde svarende til højden på tegnene i nummerpladen og en bredde der er noget længere end nummerpladens bredde. Et eksempel er vist på figur \vref{fig:DetectContrastAvg-blurredGrads}. 

% Blured horizontal gradients
\begin{figure}[htp]
  \centering
  \framebox{\includegraphics[width=10cm]{system/illu/DetectContrastAvg-blurredGrads.png}}
  \caption{De markerede områder i nummerpladen på figur \vref{fig:DetectContrastAvg-hgrads} forbindes ved at give hver pixel en intensitet der er proportional med gennemsnittet i nærområdet.}
  \label{fig:DetectContrastAvg-blurredGrads}  
\end{figure}


Som det sidste trin laver vi et binært billede med nummerpladekandidater som vist på figur \vref{fig:DetectContrastAvg-binary}. Vi bruger en forholdsvis lav grænseværdi når vi laver dette billede. Det vil sige, vi opfatter forholdsvis mørke områder som potentielle nummerpladeområder.

%Binary image
\begin{figure}[htp]
  \centering
  \framebox{\includegraphics[width=10cm]{system/illu/DetectContrastAvg-binary.png}}  
  \caption{Det binære billede med nummerpladekandidater der er afledt af billedet på figur \vref{fig:DetectContrastAvg-blurredGrads}.}
  \label{fig:DetectContrastAvg-binary}
\end{figure}

Metoden beskrevet i dette afsnit er baseret på systemet beskrevet i \cite{shapiro}. Vores implementation er lavet som funktionen \textit{DetectContrastAvg} hvis kildekode findes i afsnit \vref{code:DetectContrastAvg}.

\subsubsection{Metode: Frekvensanalyse}
\label{sec:DetectPlateness}
Ved at analysere svingningen mellem lyse og mørke intensiteter i de nummerplader der indgår i vores træningsæt, forsøger vi med denne metode at lokalisere nummerplader ved at markere de områder der har en frekvens der ligger tæt på den målte middelfrekvens for nummerpladeområder.  For hver pixel i billedet analyserer vi frekvensen af en linie der begynder et givent antal pixels før den indeværende pixel og slutter det samme antal pixels efter. Metoden er inspireret af \cite{kwas} der også arbejder med frekvensanalyse for at lokalisere nummerplader. Figur \vref{fig:DetectPlateness-binary} viser et eksempel på den type binære billeder frekvensanalysen resulterer i. Denne metode er implementeret i funktionen \textit{DetectPlateness} hvis kildekode findes i afsnit \vref{code:DetectPlateness}.
ILLU\_SVINGNINGER\_PLADE
ILLU\_SVINGNINGER\_IKKE\_PLADE
%Binary image
\begin{figure}[htp]
  \centering
  \framebox{\includegraphics[width=10cm]{system/illu/DetectPlateness-binary.png}}  
  \caption{Det binære billede med nummerpladekandidater der er afledt af billedet på figur \vref{fig:DetectPlateness-binary}.}
  \label{fig:DetectPlateness-binary}
\end{figure}

\subsubsection{Metode: Maksimer lokal kontrast}
Ideen bag denne metode er, at vi ved at dele billedet op i små blokke og uafhængigt af intensiteterne i resten af billedet maksimere kontrasten i disse små områder kan lave et binært billede hvor nummerpladen er et af få sammenhængende områder. Vi maksimerer kontrasten for hver blok med såkaldt \textit{Contrast streching}. Da spredningen af intensiteter i blok oftest vil være forholdsvis begrænset, kan vi øge kontrasten i området ved at strække intensiteterne så de udnytter hele spektret på 256 intensitetsværdier. På denne måde bliver de lyseste pixels i blokken helt hvide  mens de mørkeste bliver helt sorte når intensitetsværdierne blive strakt.   

Vi maksimerer kontrasten for hver blok med såkaldt \textit{Contrast streching}. Da spredningen af intensiteter i en blok oftest vil være forholdsvis begrænset, kan vi øge kontrasten i området ved at strække intensiteterne så de udnytter hele spektret på 256 intensitetsværdier. På denne måde bliver de lyseste pixels i blokken helt hvide  mens de mørkeste bliver helt sorte når intensitetsværdierne blive strakt. Figur \vref{fig:DetectCStretch-illu1} viser hvordan vi ved at strække intensiteterne deler et område med en blød overgang fra sort til hvid over i to forskellige områder der hver har intensitetsværdier i hele spektret. En nummerplade vil derimod fortsat være en sammenhængende figur efter intensiteterne er blevet strakt, da de lyseste områder, baggrunden, i nummerpladeblokkene fortsat vil støde op til de lyseste områder i de nummerpladeblokke der ligger ved siden af. Dette forhold er illustreret på figur \vref{fig:DetectCStretch-illu2}.


% Illu af contrast strech
\begin{figure}[htbp]
  \centering
  \begin{minipage}[b]{5 cm}
    \framebox{\includegraphics[width=4cm]{system/illu/DetectCStretch-illu1_1.png}}
  \end{minipage}
  \begin{minipage}[b]{5 cm}
    \framebox{\includegraphics[width=4cm]{system/illu/DetectCStretch-illu1_2.png}}  
  \end{minipage}
  \caption{Billedet til højre viser resultatet af at dele billedet til venstre op i to kvadratiske blokke og strække deres intensiteter. I hver blok bliver de mørkeste toner helt sorte, og de lyseste helt hvide når intensiteterne strækkes.}
  \label{fig:DetectCStretch-illu1}
\end{figure}

% Illu af contrast strech på plade
\begin{figure}[htbp]
  \centering
  \begin{minipage}[b]{5 cm}
    \framebox{\includegraphics[width=4cm]{system/illu/DetectCStretch-illu2_1.png}}
  \end{minipage}
  \begin{minipage}[b]{5 cm}
    \framebox{\includegraphics[width=4cm]{system/illu/DetectCStretch-illu2_2.png}}  
  \end{minipage}
  \caption{Den underbelyste nummerplade til venstre deles op i to kvadratiske blokke, og intensiteterne i dem strækkes uafhængigt af hinanden. Resultatet til højre viser, at hele nummerpladen stadig er et sammenhængende lyst område i modsætning til eksemplet på figur \vref{fig:DetectCStretch-illu1} hvor billedet bliver delt.}
  \label{fig:DetectCStretch-illu2}
\end{figure}

Med denne metode får vi det binære billede af interesseområder der er vist på figur \ref{fig:DetectCStretch-binary}. Kildekoden til metoden er implementeret i funktionen \textit{DetectCStretch} hvis kildekode findes i afsnit \vref{code:DetectCStretch}.

%Binary image
\begin{figure}[htp]
  \centering
  \framebox{\includegraphics[width=10cm]{system/illu/DetectCStretch-binary.png}}  
  \caption{Det binære billede af interesseområder skabt af metoden der maksimerer lokal kontrast. Bemærk, at nummerpladen er et fritstående sammenhængende område.}
  \label{fig:DetectCStretch-binary}
\end{figure}

\subsubsection{Metode: Kvantifisering}
Med denne metode forsøger vi at lokalisere nummerpladen ved at sætte antallet af gråtoner i vores inddatabillede ned fra 256 til 8. Med kun 8 intensitetsværdier vil billedet indeholde større sammenhængende områder der repræsenterer de områder i originalbilledet hvor intensiteterne ligge forholdsvis tæt på hinanden. Det er vores håb at et af disse områder er nummerpladen.

Som det første trin i denne metode, kører vi et filter der giver en pixel den maksimale intensitet fra dens nærområde. Vi forventer at dette vil udviske tegnene på nummerpladen og efterlade et lyst rektangel i billedet der hvor nummerpladen findes. Formen på vores filter er et liggende rektangel da vi vil forsøge at undgå at nummerpladen forbindes med lyse områder der ligger over og under nummerpladen. Et eksempel på hvordan resultatet af at anvende dette filter er vist på figur \vref{fig:DetectQuant-filteredImage}.

% Filtered image
\begin{figure}[htp]
  \centering
  \framebox{\includegraphics[width=10cm]{system/illu/DetectQuant-filteredImage.png}}
  \caption{Resultatet af at køre et filter der giver hver pixel samme intensitet som den maksimale intensitet i nærområdet.}
  \label{fig:DetectQuant-filteredImage}  
\end{figure}

Som næste trin sætter vi antallet af intensiteter i det filtrerede billede ned fra 256 til 8. Det giver os større sammenhængende områder som vist på figur \vref{fig:DetectQuant-quantImage}. På baggrund af dette billede danner vi et binært billede der viser vores interresseområder. Et eksempel er vist på figur \vref{fig:DetectQuant-binary}.

% Quant image
\begin{figure}[htp]
  \centering
  \framebox{\includegraphics[width=10cm]{system/illu/DetectQuant-quantImage.png}}
  \caption{Resultatet af at sætte antallet af intensitetet i billedet på figur \vref{fig:DetectQuant-filteredImage} ned til otte.}
  \label{fig:DetectQuant-quantImage}  
\end{figure}

% Quant binary image
\begin{figure}[htp]
  \centering
  \framebox{\includegraphics[width=10cm]{system/illu/DetectQuant-binary.png}}
  \caption{Interesseområde valgt på baggrund af billedet på figur \vref{fig:DetectQuant-quantImage}}
  \label{fig:DetectQuant-binary}  
\end{figure}

Metoden beskrevet i dette afsnit er implementeret i funktionen \textit{DetectQuant} hvis kildekode findes i afsnit \vref{code:DetectQuant}.




\subsubsection{Analyse af interesseområder}
\label{sec:kandidater}
Ud fra de binære billeder der viser interesseområder, skal vi forsøge at udvælge de områder der mest sandsynligt er nummerplader. Vi begynder denne udvælgelsesproces med at frasortere de områder vi mener at kunne udelukke som værende nummerplader. Vi sletter interesseområder med følgende karateristika:
\begin{itemize}
\item Området er meget lille.
\item Området er meget stort.
\item Området er højere end det er bredt.
\item Området er for bredt i forhold til højden\footnote{I forhold til højde/bredde forholdet for en nummerplade}.
\item Området har få markerede pixels i forhold til det rektangel det udspænder\footnote{F.eks. en streg med en vinkel på 45 grader}.
\end{itemize}

Denne frasortering af interesseområder er implementeret i funktionen \textit{BinImgCleanup} hvis kildekode findes i afsnit \ref{code:BinImgCleanup}. Figurene \ref{fig:DetectSameness-cleanup} til \ref{fig:DetectQuant-cleanup} viser frasorteringen anvendt på de binære billeder med interesseområder som vi brugte som eksempler i beskrivelsen af de forskellige metoder til lokalisering af nummerplader. 

% SAMENESS
\begin{figure}[htbp]
  \centering
  \begin{minipage}[b]{5 cm}
    \framebox{\includegraphics[width=4cm]{system/illu/DetectSameness-binary.png}}
  \end{minipage}
  \begin{minipage}[b]{5 cm}
    \framebox{\includegraphics[width=4cm]{system/illu/DetectSameness-cleaned.png}}  
  \end{minipage}
  \caption{Frasortering af interesseområder udvalgt af metoden der markerer områder domineret af lyse gråtoner i farvebilleder}
  \label{fig:DetectSameness-cleanup}
\end{figure}

% CONTRAST AVG
\begin{figure}[htbp]
  \centering
  \begin{minipage}[b]{5 cm}
    \framebox{\includegraphics[width=4cm]{system/illu/DetectContrastAvg-binary.png}}
  \end{minipage}
  \begin{minipage}[b]{5 cm}
    \framebox{\includegraphics[width=4cm]{system/illu/DetectContrastAvg-cleaned.png}}  
  \end{minipage}
  \caption{Frasortering af interesseområder udvalgt af metoden der markerer områder med høj kontrast.}
   \label{fig:DetectContrastAvg-cleanup}
\end{figure}

% PLATENESS
\begin{figure}[htbp]
  \centering
  \begin{minipage}[b]{5 cm}
    \framebox{\includegraphics[width=4cm]{system/illu/DetectPlateness-binary.png}}
  \end{minipage}
  \begin{minipage}[b]{5 cm}
    \framebox{\includegraphics[width=4cm]{system/illu/DetectPlateness-cleaned.png}}  
  \end{minipage}
  \caption{Frasortering af interesseområder udvalgt af metoden der markerer områder på baggrund af frekvensanalyse.}
   \label{fig:DetectPlateness-cleanup}
\end{figure}

% CONTRAST STRETCH
\begin{figure}[htbp]
  \centering
  \begin{minipage}[b]{5 cm}
    \framebox{\includegraphics[width=4cm]{system/illu/DetectCStretch-binary.png}}
  \end{minipage}
  \begin{minipage}[b]{5 cm}
    \framebox{\includegraphics[width=4cm]{system/illu/DetectCStretch-cleaned.png}}  
  \end{minipage}
  \caption{Frasortering af interesseområder udvalgt af metoden der markerer områder på baggrund af maksimeret lokal kontrast.}
   \label{fig:DetectCStretch-cleanup}
\end{figure}

% QUANT
\begin{figure}[htbp]
  \centering
  \begin{minipage}[b]{5 cm}
    \framebox{\includegraphics[width=4cm]{system/illu/DetectQuant-binary.png}}
  \end{minipage}
  \begin{minipage}[b]{5 cm}
    \framebox{\includegraphics[width=4cm]{system/illu/DetectQuant-cleaned.png}}  
  \end{minipage}
  \caption{Frasortering af interesseområder udvalgt af metoden der markerer områder på baggrund af kvantifisering.}
   \label{fig:DetectQuant-cleanup}
\end{figure}


OVERVEJ OMSKRIVNING AF GETBESTCANDIDATE SÅ DEN SVARER TIL DENNE BESKRIVELSE.

I det resulterende binære billede giver vi de tilbageværende områder point efter deres højde/bredde forhold og deres frekvens. Et område der udspænder et rektangel med samme shøjde/bredde-forhold som en nummerplade får 0 point og et område der afviger meget fra det ideelle højde/bredde-forhold får et højt antal point.

Herefter kigger vi på en værdi der beskriver antallet af skift mellem meget lyse og meget mørke områder i kandidatområdet. Princippet er det samme som metoden beskrevet afsnittet \textbf{Frekvensanalyse} på side \pageref{sec:DetectPlateness}. Forskellen er, at vi i det fåregående afsnit kiggede på linier der er indeholdt i nummerplader, mens vi her kigger på et helt områder der er større end selve nummerpladen. Der er en væsentlig forskel, da starten og slutningen på en nummerplade er meget karakteristisk. Udfra vores testdata har vi udregnet en gennemsnitsværdi for svingninger i nummerpladeområder. Vi sammenligner kandidaterne og giver dem point i forhold til hvor tæt de ligger på den på forhånd bestemte gennemsnitsværdi. Jo mere et område afviger fra det ideelle antal svingninger, jo flere point får det.

.... er implementeret i funktionen \textit{GetBestCandidate} hvis kildekode findes i afsnit \ref{code:GetBestCandidate}.

ILLU\_PLATESIG\_M\_AVG

%%%%%%%%%%%%%%%%%%%%%%%%%
\subsubsection{Endeligt valg af nummerpladekandidat}
Vores system vælger en endelig nummerpladekandidat ved at sammenligne nummerpladekandidaterne fra samtlige metoder beskrevet i dette afsnit. Vi undersøger om metoderne er enige ved at se om de udpeger områder der ligger tæt på hinanden. Det område som flest metoder er enige om udpeges som den endelige nummerpladekandidat. Hvis der er større uenighed, det vil sige at lige mange metoder er enige om flere forskellige områder, returnerer systemet ingen nummepladekandidat. For eksempel vil situationer hvor kun to metoder returnerer en nummerpladekandidat, men disse to områder er forskellige resultere i at der ikke udpeges en endelig nummerpladekandidat. I denne situation opgiver systemet altså at finde en nummerplade i billedet. Figur \ref{fig:DetectMain-result} viser valget af endelig nummerpladekandidat på baggrund af eksemplerne på figurene \ref{fig:DetectSameness-cleanup} til \vref{fig:DetectQuant-cleanup}. Udvælgelsen er implementeret i funktionen \textit{DetectMain} hvis kildekode findes i afsnit \ref{code:DetectMain}.

% Billede af kandidater i MainDetect. 
\begin{figure}[htp]
  \centering
  \framebox{\includegraphics[width=10cm]{system/illu/DetectMain-result.jpg}}
  \caption{Det endelige nummerpladeområde som udvalgt på baggrund af nummerpladekandidater fra samtlige metoder. I dette eksempel er alle metoderne enige om at udpege det område der er markeret af det blå rektangel.}
  \label{fig:DetectMain-result}  
\end{figure}


%%%%%%%%%%%%%%%%%%%%%%%%%%%%%%%%%%%%%%%%
\subsubsection{Metoder fra litteraturen}
I litteraturen er der mange andre bud på hvordan man kan lokalisere end dem vi har arbejdet med. Et eksempel er \cite{parker} der forsøger at lokalisere områder der har et højde/bredde-forhold der svarer til tegn i en nummerplade. Hvis flere af sådanne områder står efter hinanden er det en nummerpladekandidat.

En anden metode til at lokalisere områder i billeder er \textit{Maximally stable extremal regions (MSER)}. En god intuitiv beskrivelse af ideen bag findes i \cite{murphy} som vi her citerer:
\begin{quote}
Fundamentally, a grayscale image is a two-dimensional function, mapping an (x, y) coordinate to an intensity value. Similarly, a watershed can be represented as a function assigning a depth to every 2-D position. To understand MSERs, you must first imagine that the the watershed is initially dry and then slowly filled with water. Initially, puddles would begin to form in the deepest crevasses. As the water level increases, the puddles would become ponds and lakes, and occasionally two of these would merge to form a larger body of water. This step can be viewed as the termination of the smaller lake, and the addition of all of its water into the larger lake. When this occurs, the volume of water in the lake is highly unstable as a tiny increase in the water level changed the volume dramatically. With Maximally Stable Extremal Regions, the focus is to discover water levels that are instead local minima in the rate of change of the water volume.
\end{quote}


%%%%%%%%%%%%
% COMMENTS %
%%%%%%%%%%%%

\begin{comment}
\subsubsection{Tobias' brainstorm}
Kan man kigge på højde bredde på components?
Scan linie. Man kan både kigge på components og "signatur" som beskrevet andetsteds(pdf).
Med gradienter kan man finde f.eks. lodrette men ikke vandrette streger. Der er mange lodrette i pladen.
Man kan kigge på en f.eks. 8x8 og se hvor mange komponenter der er tilstede. Der er mange komponenter i et 
lille område i nummerpladen(eller hvad?).

Kør en scanlinie. Noter kraftige gradienter. Hvis de er "tæt" på hinanden er det godt. giv "point"

Det her dropper jeg  indtil videre:
Kør en vertikal scanlinie. Hvis vi møder en gradient begynder vi at "tegne" en streg hvis intensitet 
falder. Den tegner altså en savtak når den møder en gradient. Hvis der er flere høje gradienter i træk får 
vi så en kasse med en skrå afslutning. Er det ikke en nummerplade? 


Nu gør jeg:
Find gradienter i billedet. Lav et binært billede hvor de steder hvor gradienten er større end (0.5 * den 
maximale gradient) er markeret med hvidt.

Der er nu mange markeringer i nummerpladeområdet.

Jeg tænker. Samme princip som med kun at vise maksimale gradienter:
Kig på summen (mængden af hvidt) af alle vandrette linier i billedet. "slet" de linier hvor der er mindre 
end (0.5 * summen af den linie med mest hvidt). Jeg burde nu have fjernet støj men stadig have pladen. Jeg 
statser på at pladen er på de linier med mest hvidt.

\end{comment}

\begin{comment}
Diskuter om vi skal vælge en kandidat når de forskellige metoder har forskellige kandidater. Parkering: man kunne bede bilen om at bakke og tage et nyt billeder hvis der ikke kan findes en fælles kandidat. Fartkontrol: ikke muligt med nyt billede, gær derimod på den kandidat der er returneret af den mest pålidelige metode. 

Regler:

- Kant efterfulgt af rød stribe (ovenover eller nedenunder)
- Pixel del af sammenhængende kæde som er længere end et vist antal pixels
- En linie hvorpå der er en anden retvinklet linie, og linierne har et vist forhold
- Længde af linie

\end{comment}

\begin{comment}
\subsubsection{Metode: Histogram}
\label{sec_histo}

SKAL DENNE METODE BESKRIVES?

Idéen i denne metode er at hver pixel i et fotografi stilles op mod en frekvenstabel, indeholdende farvefrekvenser for en mængde nummerplader. Hvis pixelens farve svarer til en farve med høj frekvens i denne tabel, er det sandsynligt at den er en del af en nummerplade.

%Beskrevet i LicensePlateSydney.pdf

%Frekvenstabel, hvordan opbygges den?
%Frekvenstabellen indeholdende farvefrekvenserne for pixels i en række billeder af nummerplader blev lavet som følgende: Tabellen blev udformet vha. funktionen make\_freq\_table. I funktionen itereres der gennem alle pixels i et billede og deres RGB værdier noteres i en 255 x 255 x 255 matrice. Eksempelvis

Først oprettes en frekvenstabel udfra nogle billeder, hvor nummerpladens placering i billedet allerede er specificeret. Her itereres der gennem alle pixels i et billede og antallet af farver med en bestemt farvekombination summeres i en $255 \times 255 \times 255$ matrice, med plads til én frekvensværdi, $f$ for hver farvekombination. Når frekvenstabellen er udformet, normaliseres den. Det medfører at den RGB værdi der har den højeste frekvens, $f_{max}$ får værdien 1, mens de resterende RGB værdier får værdien $f$/$f_{max}$. Hvis RGB værdien 240, 240, 230 eksempelvis har den højeste frekvens, 55 og RGB værdien 230, 230, 220 har frekvensen $35$, får førstnævnte værdien 1 og sidstnævnte værdien $35/55 = 0,64$.


I systemet udarbejdede vi to frekvenstabeller. Én for billederne taget med et Canon digitalkamera og én for billederne taget med et Olympus digitalkamera. Grunden til at oprette disse to adskilte tabeller er, at vi får mulighed for at undersøge forskellen på brug af kamera. Eksempelvis kan det undersøges om en frekvenstabel lavet udfra billeder fra ét kamera kan bruges til at identificere nummerplader i billeder fra et andet kamera.


%Vi udvalgte tilfældigt hhv. 70 og 50 billeder fra de to grupper til de to frekvenstabeller og gemte disse tabeller i to seperate filer. Systemet kunne herefter bruge disse frekvenstabeller uden at skulle skabe dem først.

Ved hjælp af frekvenstabellerne skabes et billede hvor hver pixel gives en værdi fra 0 til 1 afhængig af om pixelens farve fremkommer sjældent (0) eller hyppigt (1) i en nummerplade. Disse værdier svarer altså til de værdier der står i frekvenstabellerne. I eksemplet ovenfor, ville en pixel med RGB værdien 230, 230, 220 få værdien 0,64 i tilhørsbilledet. Dette tilhørsbillede vil, med værdier fra 0 til 1, være et gråtone billede hvor de pixels, der har samme farve som de farver der oftest optræder i billeder af nummerplader, vil blive lyse og omvendt vil de resterende pixels blive mørke. De pixels der har de højeste frekvenser bliver altså interesseområder.

TO ILLUSTRATIONER: ORIGINAL BILLEDE OG TILHØRSBILLEDE
%\paragraph{Analyse}
%\paragraph{Implementation}
%Omdannes dette billede til et binært billede vil vi (ved succes) få forbundne komponenter bl.a. der hvor nummerpladen befinder sig i billedet. Disse komponenters størrelse, form osv. kan herefter analyseres og man kan give et bud på hvor nummerpladen befinder sig.
\end{comment}

