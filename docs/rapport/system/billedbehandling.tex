\section{Billedbehandling}
\label{sec_billed}

\fixme{Hvor kommer metoderne fra?}
\fixme{Et eller andet sted skal der stå, at der er godt hvis pladerne ikke er klippet helt tæt, da rotation så er nemmere da der vil være en over- og underkant.}

I dette afsnit beskrives de tre første elementer af vores system: identificering af nummerpladen i billedet, rotation af nummerpladen samt segmentering af tegn i nummerpladen.


\subsection{Identifikation af nummerplader}
I arbejdet med at udvælge nummerpladekandidater i billeder, har vi valgt at bruge flere forskellige metoder. Det er vores tanke, at vi ved at bruge metoderne sammen kan opnå et bedre resultat end ved at bruge metoderne hver for sig. F.eks. vil et område som udpeges som værende en nummerplade af flere metoder have en høj grad af troværdighed mens det i en situation hvor metoderne alle udpeger forskellige områder vil være meget usikkert hvor nummerpladen befinder sig. I dette afsnit beskriver vi de forskellige metoder vi benytter os af, samt hvordan disse fungerer som et samlet system til identifikation af nummerplader. 


\subsubsection*{Interesseområder}
Metoderne i dette afsnit, har alle det til fælles at de returnerer et binært\footnote{Et billede der kun består af farverne hvid og sort} billede hvor områder der opfattes som interessante ar markeret med farven hvid. I den bedst tænkelige, og meget urealistske, situation er nummerpladeområdet markeret med hvidt mens resten af billedet er sort. I den realistiske situation er der mange interesseområder markeret i det binære billede. Vi kalder disse områder for kandidater, og beskriver hvordan vi vælger mellem det senere afsnit \textbf{Analyse af kandidater} der begynder på side \pageref{sec_kandidater}.

TRE ILLUSTRATIONER ORIGINAL BILLEDE, IDEAL ILLU, REAL ILLU



%skal forbudne komponenter der ikke er kandidater slettes eller bare fravælges. Det første giver bedre illu samt debugging mens det sidste nok er hurtigere.

\subsubsection*{Metode: Områder domineret af lyse gråtoner}
\fixme{Søren: Næppe realitisk denne metode, afhængig af belysning}
Med denne metode forsøger vi at finde nummerpladen i et billede ved at lede efter områder der domineres af gråtoner i et farvebillede. Det er vores forventning at en nummerplade vil være domineret af pixels hvor de tre farvekanaler, rød, grøn og blå har intensitetsværdier der ligger meget tæt på hinanden. F.eks. vil en helt rød pixel have værdien 256 i den røde farvekanal mens værdierne i den grønne og blå farvekanal være 0 mens en middelgrå pixel vil derimod have værdien 128 i alle tre farvekanaler. For hver pixel i farvebilledet udregner vi gennemsnitsværdien ved at summere værdierne for de tre kanaler og dividere med tre. Hvis alle tre kanaler har en værdi der ligger tæt\footnote{Et parameter i programmet} på middelværdien og samtidig er lys\footnote{Et parameter i programmet} markeres den indeværende pixel i det binære billede der er resultatet af gennemgangen af alle pixels i farvebilledet.

ORIG\_IM BIN\_IM.
%detect5
%\paragraph*{Analyse}
%\paragraph*{Implementation}
 

\subsubsection*{Metode: Områder med høj kontrast}
Vi forventer at et område med en nummerplade vil indeholde områder med høj kontrast. Endvidere forventer vi at mange af disse kontrastområder vil være områder hvor den det ene intensitetsområde ligge ved siden af det andet i modsætning til over hinanden (beskriv nøjere? antagelse om vandrette plader). F.eks. har bogstavet I lyse områder på både venstre og højre side. Der er altså to lodrette linier der markerer overgangen mellem lyse og mørke områder. Ved at lave billedet om til gråtoner og derefter betragte intensiteterne i billedet som højder i et landskab kan vi beskrive hældninger i landskabet med såkaldte gradienter der beskriver hældningen i en pixel son en to dimensionel vektor(en illu). Et bredt område med en blød overgang i intensitet fra helt sort til helt hvid, hvil have korte vektorer i alle pixels mens en brat overgang fra sort til hvid vil resultere i meget lange gradienter for de pixels der ligger op ad overgangen. Ved brug af disse gradienter laver vi et nyt billeder der viser gradienternes vandrette længde. Peger en gradient f.eks. lodret op i billedet vil den ikke blive markeret, hvorimod den vil blive markeret hvis den f.eks. peger fra venstre mod højre. Lyse pixels i dette billede indikerer lange længder. For at gøre områder med høje intensiteter mere sammenhængende filtrerer vi billedet med et ikke kvadratisk filter der er flere gange bredere end det er højt. BESKRIVELSE. Resultater et et udglattet billede hvor intensitetsområder der ligger ved siden af hinanden er blevet forbundet. Ud fra dette billede laver vi et binært billede med kandidatområder markeret med hvidt. (Henvisning til litteratur.)

ORIG\_IM BIN\_IM.

%\paragraph*{Analyse}
%\paragraph*{Implementation}
%detect6

\subsubsection*{Metode: Frekvensanalyse}
\label{sec_frekvensanalyse} 
Ideen bag denne metode til markering af interesseområder er, at kigge på frekvensen af skift mellem lyse og mørke intensiteter i vandrette linier i et gråtonebillede. For at kunne markere områder der har en frekvens svarende til en nummerplade, skal vi først finde ud af hvad denne representative frekvens er. Derfor har vi analyseret nummerpladeområderne på vores testbilleder og udregnet den gennemsnitlige frekvens. Med denne viden kan vi skabe et binært billede hvor de markerede interesseområder er dem hvor frekvensen ligger tæt på den udregnede idealfrekvens. (ref til pdfer med signaturer)

ILLU\_SVINGNINGER\_PLADE ILLU\_SVINGNINGER\_IKKE\_PLADE

\fixme{figurer af histogrammer og lign.}

%\paragraph*{Analyse}
%\paragraph*{Implementation}

\subsubsection*{Metode: Skru op for Kontrast}
%Lav et histogram. Lad et vindue køre hen over histogrammet. Vælg den position hvor der er mest info i vinduet. Stræk dette område så det fylder helle intensitetsspekter fra 0 til 255. Lav binært billede ud fra dette high-contrast-billede. Er det en metode eller en forberedelse af billedet før en af de andre metoder bruges?

%ORIG\_IM CONTRAST\_IM BIN\_IM.

%\paragraph*{Analyse}
%\paragraph*{Implementation}

\subsubsection*{Metode: Kvantifisering}
Kun 8 farver i billedet.

\subsubsection*{Metode: Histogram}
\label{sec_histo}

Idéen i denne metode er at hver pixel i et fotografi stilles op mod en frekvenstabel, indeholdende farvefrekvenser for en mængde nummerplader. Hvis pixelens farve svarer til en farve med høj frekvens i denne tabel, er det sandsynligt at den er en del af en nummerplade.

%Beskrevet i LicensePlateSydney.pdf

%Frekvenstabel, hvordan opbygges den?
%Frekvenstabellen indeholdende farvefrekvenserne for pixels i en række billeder af nummerplader blev lavet som følgende: Tabellen blev udformet vha. funktionen make\_freq\_table. I funktionen itereres der gennem alle pixels i et billede og deres RGB værdier noteres i en 255 x 255 x 255 matrice. Eksempelvis

Først oprettes en frekvenstabel udfra nogle billeder, hvor nummerpladens placering i billedet allerede er specificeret. Her itereres der gennem alle pixels i et billede og antallet af farver med en bestemt farvekombination summeres i en $255 \times 255 \times 255$ matrice, med plads til én frekvensværdi, $f$ for hver farvekombination. Når frekvenstabellen er udformet, normaliseres den. Det medfører at den RGB værdi der har den højeste frekvens, $f_{max}$ får værdien 1, mens de resterende RGB værdier får værdien $f$/$f_{max}$. Hvis RGB værdien 240, 240, 230 eksempelvis har den højeste frekvens, 55 og RGB værdien 230, 230, 220 har frekvensen $35$, får førstnævnte værdien 1 og sidstnævnte værdien $35/55 = 0,64$.


I systemet udarbejdede vi to frekvenstabeller. Én for billederne taget med et Canon digitalkamera og én for billederne taget med et Olympus digitalkamera. Grunden til at oprette disse to adskilte tabeller er, at vi får mulighed for at undersøge forskellen på brug af kamera. Eksempelvis kan det undersøges om en frekvenstabel lavet udfra billeder fra ét kamera kan bruges til at identificere nummerplader i billeder fra et andet kamera.


%Vi udvalgte tilfældigt hhv. 70 og 50 billeder fra de to grupper til de to frekvenstabeller og gemte disse tabeller i to seperate filer. Systemet kunne herefter bruge disse frekvenstabeller uden at skulle skabe dem først.

Ved hjælp af frekvenstabellerne skabes et billede hvor hver pixel gives en værdi fra 0 til 1 afhængig af om pixelens farve fremkommer sjældent (0) eller hyppigt (1) i en nummerplade. Disse værdier svarer altså til de værdier der står i frekvenstabellerne. I eksemplet ovenfor, ville en pixel med RGB værdien 230, 230, 220 få værdien 0,64 i tilhørsbilledet. Dette tilhørsbillede vil, med værdier fra 0 til 1, være et gråtone billede hvor de pixels, der har samme farve som de farver der oftest optræder i billeder af nummerplader, vil blive lyse og omvendt vil de resterende pixels blive mørke. De pixels der har de højeste frekvenser bliver altså interesseområder.

TO ILLUSTRATIONER: ORIGINAL BILLEDE OG TILHØRSBILLEDE

%\paragraph{Analyse}

%\paragraph{Implementation}

%Omdannes dette billede til et binært billede vil vi (ved succes) få forbundne komponenter bl.a. der hvor nummerpladen befinder sig i billedet. Disse komponenters størrelse, form osv. kan herefter analyseres og man kan give et bud på hvor nummerpladen befinder sig.

\subsubsection*{Analyse af kandidater}
\label{sec_kandidater}
Ud fra de binære billeder der viser interesseområder, skal vi forsøget at finde de områder der mest sandsynligt er nummerplader. (noget med sammenhængende områder og dilate/erode). Vi begynder udvælgelsesprocessen med at frasortere de områdervi mener kan udelukkes som værende nummerplader. Vi sletter områder med følgende karateristika:
\begin{itemize}
\item Området er meget lille.
\item Området er meget stort.
\item Området er højere end det er bredt.
\item Området er for bredt i forhold til højden\footnote{I forhold til højde/bredde forholdet for en nummerplade}.
\item Området har få markerede pixels i forhold til det rektangel det udspænder.
\end{itemize}


I det resulterende binære billede giver vi de tilbageværende områder point efter deres højde/bredde forhold og deres frekvens. Et område der udspænder et rektangel med samme højde/bredde-forhold som en nummerplade får 0 point og et område der afviger meget fra det ideelle højde/bredde-forhold får et højt antal point.

Herefter kigger vi på en værdi der beskriver antallet af skift mellem meget lyse og meget mørke områder i kandidatområdet. Princippet er det samme som metoden beskrevet afsnittet \textbf{Frekvensanalyse} på side \pageref{sec_frekvensanalyse}. Forskellen er, at vi i det fåregående afsnit kiggede på linier der er indeholdt i nummerplader, mens vi her kigger på et helt områder der er større end selve nummerpladen. Der er en væsentlig forskel, da starten og slutningen på en nummerplade er meget karakteristisk. Udfra vores testdata har vi udregnet en gennemsnitsværdi for svingninger i nummerpladeområder. Vi sammenligner kandidaterne og giver dem point i forhold til hvor tæt de ligger på den på forhånd bestemte gennemsnitsværdi. Jo mere et område afviger fra det ideelle antal svingninger, jo flere point får det.

ILLU\_PLATESIG\_M\_AVG



%Signatur: avg plateness ganges op da plate sig ser ud som den gør.
ILLU MED TRIN I FRAVÆLGELSE AF KANDIDATER
%\paragraph*{Analyse}
%Ville være hurtigere at ikke slette men at fravælge. Størrelser. Må man det? Generelt vs. ikke generelt.


%\paragraph*{Analyse}
%Ville være hurtigere at ikke slette men at fravælge. Størrelser. Må man det? Generelt vs. ikke generelt.

%\paragraph*{Implementation}


%detect4


\subsubsection{Vælg nummerpladekadidt på baggrund af alle metoder.}



\begin{comment}
\subsubsection{Tobias' brainstorm}
Kan man kigge på højde bredde på components?
Scan linie. Man kan både kigge på components og "signatur" som beskrevet andetsteds(pdf).
Med gradienter kan man finde f.eks. lodrette men ikke vandrette streger. Der er mange lodrette i pladen.
Man kan kigge på en f.eks. 8x8 og se hvor mange komponenter der er tilstede. Der er mange komponenter i et 
lille område i nummerpladen(eller hvad?).

Kør en scanlinie. Noter kraftige gradienter. Hvis de er "tæt" på hinanden er det godt. giv "point"

Det her dropper jeg  indtil videre:
Kør en vertikal scanlinie. Hvis vi møder en gradient begynder vi at "tegne" en streg hvis intensitet 
falder. Den tegner altså en savtak når den møder en gradient. Hvis der er flere høje gradienter i træk får 
vi så en kasse med en skrå afslutning. Er det ikke en nummerplade? 


Nu gør jeg:
Find gradienter i billedet. Lav et binært billede hvor de steder hvor gradienten er større end (0.5 * den 
maximale gradient) er markeret med hvidt.

Der er nu mange markeringer i nummerpladeområdet.

Jeg tænker. Samme princip som med kun at vise maksimale gradienter:
Kig på summen (mængden af hvidt) af alle vandrette linier i billedet. "slet" de linier hvor der er mindre 
end (0.5 * summen af den linie med mest hvidt). Jeg burde nu have fjernet støj men stadig have pladen. Jeg 
statser på at pladen er på de linier med mest hvidt.

\end{comment}

\begin{comment}
Diskuter om vi skal vælge en kandidat når de forskellige metoder har forskellige kandidater. Parkering: man kunne bede bilen om at bakke og tage et nyt billeder hvis der ikke kan findes en fælles kandidat. Fartkontrol: ikke muligt med nyt billede, gær derimod på den kandidat der er returneret af den mest pålidelige metode. 

Regler:

- Kant efterfulgt af rød stribe (ovenover eller nedenunder)
- Pixel del af sammenhængende kæde som er længere end et vist antal pixels
- En linie hvorpå der er en anden retvinklet linie, og linierne har et vist forhold
- Længde af linie

\end{comment}


\subsubsection{Andre metoder brugt i litteraturen, som der skal diskuteres?}

\cite{parker} bruger kant-detektion og localisering af tegn til at give et bud på hvor nummerpladen er.


\subsubsection*{MSER}

Først sorteres alle pixels intensiteter (stigende). Dette gøres ved brug af bucket sort.

\subsection{Rotation}

%Nogle overvejelser om hvordan billedet skal være klippet til for at rotationen fungerer ordentligt...

I dette afsnit beskrives det hvordan et billede af en nummerplade kan roteres, så nummerpladen optræder vandret i billedet. Metoden er fundet i...

%\subsubsection{Hough transformation}

%Hough transformation


\subsubsection*{Radon transformation}

En Radon transformation beregner projektionen af et billede fra flere forskellige steder og i flere forskellige retninger/vinkler. Denne type transformation kan bruges til at finde linier i billedet, og til at undersøge i hvilken retning disse linier går. Disse oplysninger er brugbare mht. rotation af et billede af en nummerplade, så denne plade står vandret i billedet.

Billedet nedenfor viser hvordan Radon transformationen findes i en enkelt vinkel, theta. Den grå firkant er billedet der projekteres. De grå pile er sensorer/radial linier. Resultatet af en Radon transformation er en matrice der angiver i hvor høj grad det er sandsynligt at der findes en linie i original billedet for hver vinkel samt radial linie.

%SKAL NOK BYTTES UD MED HJEMMELAVET BILLEDE:
%\begin{figure}[h]
%\includegraphics{billedbehandling/illu/transform74.jpg}
%\caption{Radon transformation}
%\end{figure}

Betragt billedet i Figur \ref{xc33139} af en nummerplade. Nummerpladen i billedet er let fordrejet i forhold til vandret og skal drejes før segmenteringen af tegn i nummerpladen kan foregå.

BILLEDE AF NUMMERPLADE, LET FORDREJET

%\begin{figure}[h]
%\includegraphics{billedbehandling/illu/P_XC33139.jpg}
%\label{xc33139}
%\caption{En nummerplade}
%\end{figure}

Når billedet i Figur \ref{xc33139}

Før Radon transformationen kan foretages skal billedet kant-detekteres (ANDEN OVERSÆTTELSE), da dette giver en større chance for at Radon transformationen opfanger linierne i billedet. Dette gøres ved hjælp af MatLab funktionen edge, som returnerer et binært billede hvor de fundne kanter har værdien 1 og resten af punkterne i billedet har værdien 0. Dette er illustreret i ...

BILLEDE AF EDGE-BILLEDE AF TIDLIGERE VISTE PLADE

%\begin{figure}[h]
%\includegraphics[width=15cm]{billedbehandling/illu/R_XC33139.jpg}
%\label{r_xc33139}
%\caption{En nummerplade}
%\end{figure}

Når nummerpladens hældning i billedet er fundet, skal billedet roteres. Dette gøres ved brug af imrotate. Billedet roteres kun hvis ...

BILLEDE AF NUMMERPLADE, ROTERET.

Udregner nye koordinater for roteret plade vha. rotationsmatrix....

Ulemper:

\subsubsection*{Andre metoder brugt i litteraturen}

Tror ikke der er andre?

\subsection{Opdeling af tegn}

\begin{comment}
Se nrpl.dk: Hvert af de op til 7 tegn på nummerpladen har et imaginært "felt" de kan brede sig i. Ikke alle tegn er lige brede, og generelt er bogstaver bredere end tal. "Feltet" til bogstaver er derfor bredere end feltet til tegn. Enkelte tegn er for smalle til at udfylde deres "felt", så designeren har fundet det hensigtsmæssigt at placere disse tegn visuelt centreret inden for deres "felt".

"Således vil en nummerplade som MB 20 001 på grund af venstrestillingen af det sidste 1-tal have større mellemrum mellem højre kant og sidste tal end mellem venstre kant og første bogstav"
\end{comment}

I det følgende beskrives to metoder til opdeling af tegn i en nummerplade. Med henblik på det komplette system er dette trin, trin nr. tre hvilket udføres efter at nummerpladen er blevet identificeret i billedet og eventuelt roteret. De to metoder til opdeling af tegn kaldes Sammenhængende komponenter og Peak-to-valley (da.: top-til-bund). Metoderne er fundet i....



\subsubsection*{Sammenhængende komponenter}
Denne metode er bygget op omkring sammenhængende komponenter. Ideén bygger på at de syv tegn som findes i billedet af nummerpladen er syv sammenhængende komponenter. I vores tilfælde vil hver sammenhængende komponent bestå af mørke pixels mens baggrunden er lyse pixels.

Sandsynligvis vil tegnene ikke være de eneste sammenhængende komponenter i billedet, hvis ikke billedet forarbejdes (nummerpladen kan være beskidt, lyset på billedet kan være dårligt, billedet kan være klippet 'for løst' sådan at elementer uden for nummerpladen vil være sammenhængende og mørke etc.). For at tegnene skal udskille sig klart fra den lyse baggrund er det nødvendigt at forstærke både lys og kontrast i billedet. Lyset forstærkes pga. muligheden for dårligt lys på nummerpladen mens kontrasten forstærkes for at adskille tegnene fra den lyse baggrund. I Figur ? ses forskellen på et billede af en nummerplade uden forarbejdning og et med.

EKSEMPEL PÅ BILLEDE MED FORSTÆRKEDE KONTRASTER.

Efterfølgende oprettes et billede med sammenhængende komponenter. I Figur ? ses et billede af en nummerplade hvor de sammen hængende komponenter er hvide.

EKSEMPEL PÅ BILLEDE MED SAMMENHÆNGENDE KOMPONENTER

Som det ses er det ikke kun tegnene der er sammenhængende komponenter. Derfor gennemføres en analyse på komponenterne for at frasortere de komponenter der ikke kan være tegn. Vi kan frasortere komponenter som ikke opfylder følgende regler:

\begin{itemize}
\item[-] Hvert tegn fylder maksimalt 1/7 del af billedet.
\item[-] Hvert tegn har en maksimal bredde på 1/7 af billedets bredde.
\item[-] Tegnenes højde er større end deres bredde
\item[-] Tegnenes minimale højde skal være end en vis konstant (her: 5 pixels)
\item[-] Tegnenes minimale størrelse skal være større end en vis konstant (her: 5 pixels)
\item[-] For hvert tegn findes der seks andre tegn som er ca. lige så højde som det pågældende tegn.
\end{itemize}

EKSEMPEL PÅ BILLEDE HVOR IKKE-TEGN KOMPONENTER ER SORTERET FRA

De resterende komponenter må forventes at være nummerpladens tegn.

EKSEMPEL PÅ UDKLIPPEDE TEGN

\fixme{beskrevet i kwas: man ku fjerne kanter før analysen}

\subsubsection*{peak-to-valley}

Denne metode baseres på en vertikal projektion af nummerpladen. Denne projektion vil, med en lys nummerplade med mørke tegn, give os en indikation af hvor der er dale (de lyse områder mellem tegnene) og bakker (tegnene), hvorefter vi kan udskære tegnene.

Som i metoden med sammenhængende komponenter er det en fordel at forstærke kontrasten i billedet. HVORFOR? Nummerpladens signatur fås ved at summere intensiteterne i alle kolonnerne i billedet. En graf over signaturen vil så give toppe i de kolonner hvor der er størst inte

\subsubsection*{Andre metoder brugt i litteraturen}



%%%%%%%%%%%%%%%%%%%%%%%%%%%
% Note fra segmentering af tegn
%%%%%%%%%%%%%%%%%%%%%%%%%%%%


\begin{comment}
\subsubsection*{Skan-linie}

Bruges ikke da det er for mange felter som bliver valgt. Kan måske gøres bedre ved filtrering før???

Først gøres billedet sort-hvis med im2bw. Her kan grænseværdi bestemmes med greythresh. Virker måske bedre at sætte grænseværdien lavt, så meget af billedet bliver hvidt.

Billedet skal skæres foroven og forneden. Dette gøres simpelt ved at finde den største pixl-sum i toppen af billedet og den største sum i bunden. Det antages så at disse max-summer er dele af nummerpladerne hvor teksten ikke er startet.

Step igennem vertikale linier: hvad sker før tegn, i et tegn, i slutningen af et tegn og efter et tegn.
\end{comment}


%%%%%%%%%%%%%%%%%%%%%%%%%%%
% Beskrivelser af PDfer
%%%%%%%%%%%%%%%%%%%%%%%%%%%%


\begin{comment}
\subsubsection*{Noter fra møde med Søren 20/2:}
identifikation: Se på en pixel, har naboer en kontrast farve?
En scan-linie: hvordan varierer kontrasten henover linien?
Adaboost - godt










\subsubsection*{cano.pdf}
Bruger kun gråtone info. Arbejder også med nummerplader som skal være læsbar for det menneskelige øje.

Metode:
Histogram - først normaliseres billedet
Sobel filter - fremhæver ikke-homogene områder
"A simple threshold and a sub-sampling" bruges til at vælge områder der kan være nummerpladen

Husker alle områder som kan være nummerplader så de forkerte først vælges fra i genkendelses-fasen. Bruger multi-hypothesis detection (ikke forklaret yderligere i teksten).

Feature vektorer: hver pixel i et træningsbilleder er blevet klassificeret som positiv (del af nummerplade) eller negativ (ikke del af nummerplade). Minimerer efterfølgende det negative sæt.

Bruger kd-træ data struktur og en "omtrent nærmeste nabo" søgeteknik.

\subsubsection*{ron}

* Find de gule (hos os: hvide) områder i billedet
* Forstør disse områder
* Find vinklen på nummerpladen ved brug af "Radon transform"
* Justering af nummerpladens konturer
* Unødvendige dele af billedet fjernes (kun nummerpladen tilbage)
* Billedet i gråtone, herefter gøres det binært
* Billedet normaliseres
* Tegn-inddeling vha. peak-to-valley

De brugte Matlab. De havde følgende relevante problemer og løsningsforslag til disse:

* Udtrækning af det gule område giver ofte fejl. Man kunne supplere denne udtrækning med en algoritme der indberegner at nummerplader har en klar signatur idet der er stærke grå-tone variationer i regulære intervaller (henover nummerpladen, mener de vel?)

* Hvis der er flere nummerplade-kandidater i billedet skal hver af dem testes.

\subsubsection*{nijhuis.pdf}

Bruger regler for nummerplader i systemet

    * Starter med kontrast “udstrækning”, bortfiltrering af støj (der tages selvfølgelig højde for billedkvaliteten i denne del)
    * Lokalisering af nummerpladen: fuzzy clustering algoritme som bruger karakteristikker som “gul-hed” og teksturer
    * Gul-hed er defineret af frekvenstabel lavet fra manuelt udklippede nummerplader
    * Tekstur: her ser man på grå-værdien af de 8 nabopixels
    * “global threshold” baseret på gennemsnitsværdien af gråtone: fås binært billede
    * Udfra regler (højde, bredde m.m) findes potintielle tegn
    * Nummerpladen gives kun videre til mønstergenkendelse hvis den indeholder det rette antal tegn

Dette system godkender 75\% af billederne. Problemer med skruer i nummerpladen. Problemer med global threshold – burde gøres på pr-tegn-basis.

\subsubsection*{parker.pdf}

Der bruges en algoritme der først lokaliserer elementer der kan være tegn/bogstaver hvorefter den udvælger et område som nummerplade

* Konverter til gråtone billede
* 5x5 filter, fjerne støj
* Find kanter i billedet vha. Shen-Castan kant-detektor
* Gør billedet binært og del elementer i forgrunden fra hinanden
* Algoritme der finder bogstaver på baggrund af forskellen i gråtone værdien af bogstav/baggrund
* Områder hvor der ikke er (det rette antal) bogstaver udelukkes
* For at finde nummerplade bruges genetisk algoritme der bedømmer rektangler med tegn: har de den rette størrelse? er bogstaverne korrekt placeret i rektangel? osv.
* Algoritmen vægter hvert område og filtrerer til sidst i disse områder udfra deres vægt

Stort problem: svært at finde tegn.

\subsubsection{kwas.pdf}
Billedet skiftes til farverummet YUV fra vilket luminans er det eneste der bevares. Herefter normaliseres billedet (Hele den diskerete "range" udnyttes). Kigger på skift i kontrast. Gælder alle nummerplader. Finder alle tekster. Den rigtige skal vælges.

Identifikation af plader:

1. "Connected components analysis" (der kigger på et binært billede?) vælger områder med høj kontrast (threshold). De fundne områder undersøges og områder elimineres efter regler i pdf.  Herefter, er der lignenede grupper i nærheden af funden gruppe? Måske er der en serie tegn dvs. en sætning = plade.

2. Searching for signatures of license plates. Et karakteristisk skift i luminans i en linie i billedet.

Potentielle plader roteres så de er vandrette.

Segmentering af tegn:
Scan af peaks og valleys samt analyse af sammenhængende grupper fra identifikationsprocessen sammenlignes og segmentering foretages.

Mønstergenkendelse foregår med neuraltnetværk.

\subsubsection{shapiro.pdf}
For at sætte hastigheden op foretages visse operationer på kraftigt nedsamplede billeder. Finder lodrette linier. Bruger Robert's edge detector til at fremhæve dem (Tegner den på billedet?). Dette efterlader en masse lodrette linier i området med nummerpladen. Et Rank filter? bruges på billedet. Efterlader en lys elipse i det område hvor pladen findes. Scanner billedet lodret for at finde det lyseste område og klipper ud (Klipper et noget større område end pladen ud på eksemplet i pdf'en). Der er formler i beskrivelsen. Roter pladen hvis skæv (Formler i pdf). Nohet med at finde linier i billedet og rotere stlsvarende (Hough og Radon transform). Det er først når vi skal genkende tegnene på pladen vi bruger billedet i sin originale opløsning.
\end{comment}


