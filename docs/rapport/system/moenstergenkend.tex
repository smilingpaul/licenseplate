\subsection{Genkendelse af tegn}
Hvordan genkender vi de tegn vi har klippet ud?
\label{sec_monster}

\begin{comment}
Noter fra møde med Søren 20/2:
Opret feature vektor $f$ for hvert bogstav $\omega$
Finde middelværdivektor for hvert $\omega$.
Afstandsfunktion: den afstand til en middelværdivektor der er mindst, vælges. Dvs. det bogstav vælges.
Featurevektorafstand udregnes hver gang

Euklidisk afstand
En-eller-anden mahap afstand
\end{comment}

%\subsection{Læsning af tegn}

Til læsning af tegn vil vi bruge to forskellige metoder: Den første bruger informationer fra en feature vektor mens den anden analyserer enkelte pixels i hvert tegn-billede. I dette afsnit vil vi beskrive disse to metoder.

\subsubsection{Feature vektorer}
En feature vektor er en vektor der... Idéen i denne metode er at der oprettes en feature vektor for hvert muligt tegn der kan forekomme i en nummerplade. For et billede af et tegn oprettes der tillige en vektor, som herefter sammenlignes med tegn-vektorerne for at se hvilken vektor der ligger nærmest billedets vektor. Tegnet for den vektor der ligger nærmest vælges.

Euklidisk afstand/mahap afstand.. mahap afstand skal (i følge søren) bruge mange data? Kan dog ikke huske hvorfor :)

\subsubsection{Simpel metode hvor træningsættet andes}

\subsubsection{Metoder fra litteraturen}
