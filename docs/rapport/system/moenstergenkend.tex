\subsection{Genkendelse af tegn}
Hvordan genkender vi de tegn vi har klippet ud?
\label{sec_monster}

\begin{comment}
Noter fra møde med Søren 20/2:
Opret feature vektor $f$ for hvert bogstav $\omega$
Finde middelværdivektor for hvert $\omega$.
Afstandsfunktion: den afstand til en middelværdivektor der er mindst, vælges. Dvs. det bogstav vælges.
Featurevektorafstand udregnes hver gang

Euklidisk afstand
En-eller-anden mahap afstand
\end{comment}

%\subsection{Læsning af tegn}

Til læsning af tegn vil vi bruge to forskellige metoder: Den første bruger informationer fra en feature vektor mens den anden analyserer enkelte pixels i hvert tegn-billede. I dette afsnit vil vi beskrive disse to metoder.

\subsubsection{Feature vektorer}
En feature vektor er en vektor der... Idéen i denne metode er at der oprettes en feature vektor for hvert muligt tegn der kan forekomme i en nummerplade. For et billede af et tegn oprettes der tillige en vektor, som herefter sammenlignes med tegn-vektorerne for at se hvilken vektor der ligger nærmest billedets vektor. Tegnet for den vektor der ligger nærmest vælges.

\fixme{mahap afstand skal (i følge søren) bruge mange data? Kan dog ikke huske hvorfor}

\fixme{Eksempel på vektor?}

\subsubsection{Simpel metode hvor træningsættet andes}

\subsubsection{Syntaks analyse}

Ved syntaks analyse analyseres tegn-hitlisterne fra de to ovenstående metoder. Her ledes efter følgende fejl:

\begin{itemize}
\item[-] Et tal er placeret på en af de første to pladser
\item[-] Et bogstav er placeret på en af de sidste fem pladser
\item[-] Bogstavkombinationen er ikke tilladt
\item[-] Talkombinationen er ikke tilladt (den samlede værdi af tallene er enten for høj eller for lav
\end{itemize}

Hvis en af disse muligheder forekommer itererer syntaks analysen ned igennem hitlisterne til et lovligt valg af tegn er fundet. Dette er illustreret i følgende eksempel: Tegnfølgen \textbf{DO 45 7B3} er retuneret fra mønstergenkendelse. I denne tegnfølge forekommer der to fejl: bogstavet \textbf{O} bruges ikke på 2. position og bogstavet \textbf{B} i 6. position burde have været et tal. På 2. position itererer syntaks analysen ned igennem hitlisten til den når et bogstav som sammen med \textbf{D} danner en lovlig bostavkombination og på 6. position itereres der indtil et tal findes.

I systemet sætter vi en øvre grænse for hvor langt ned ad hitlisten syntaks analysen kan vælge tegn. Hvis denne grænse ikke fandtes kunne det blive gætværk...

\subsubsection{Metoder fra litteraturen}
